%
% PREAMBOLO
%
\RequirePackage[l2tabu, orthodox]{nag} % Pacchetto nag per controllo errori sorgente
\documentclass[11pt,UdineBachThesis,italian]{PhdThesis}  % laurea triennale

\usepackage[latin1]{inputenc}
\usepackage[italian]{babel}
\usepackage{microtype}
\usepackage{graphicx}
\usepackage{amsthm}
\usepackage{lipsum}

% Teoremi
\newtheorem{teo}{Teorema}
\newtheorem*{nota}{Notazione}
\newtheorem*{term}{Terminologia}
\newtheorem*{dimo}{Dimostrazione}
% Traduzioni

% Direttive
\makeindex

% % % % % % % % % % % % % % % % % % % % % % % % % % % % % % % % % % % % % % % % % % % % % %
%			INIZIO DOC
% % % % % % % % % % % % % % % % % % % % % % % % % % % % % % % % % % % % % % % % % % % % % %
\title{Simulazione di Sistemi Reattivi Bigrafici}
\author{Elia Calligaris}
\email{eliac_mail@libero.it}
\supervisor{Prof. Marino Miculan}
\cosupervisor{Dott. Marco Peressotti}
\date{2014-2015}

\begin{document}
\pagestyle{empty}
\maketitle
\partstyle{serifbig}
\chaptertitlestyle{serifbig}
\pagestyle{serif}
\frontmatter
\begin{dedication}
\emph{A coloro che mi hanno donato la parola,\newline ed a coloro che mi hanno insegnato ad usarla.}
\end{dedication}

\tableofcontents

%\chapter*{Prefazione}
%\addcontentsline{toc}{chapter}{Prefazione}
%\lipsum

\mainmatter
\chapter{Introduzione}
%\pagestyle{empty}
%\addcontentsline{toc}{chapter}{Introduzione}

\chapter{Teoria dei bigrafi}
\pagestyle{plain}
\section{Segnatura}
\section{Place graph}
\section{Link graph}
\section{Bigrafo}
\section{Operazioni sui bigrafi}

% % % % % % % % % % % % % % % % % % % % % % % 
%			CAP 2							%
% % % % % % % % % % % % % % % % % % % % % % % 
\chapter{Simulazione di BRS}
Si immagini di avere un sistema che si vuole rappresentare usando il formalismo dei bigrafi. Si immagini di aver definito un'adeguata segnatura, un bigrafo rappresentante lo stato iniziale, e un insieme adeguato di regole di reazione. Si vuole ora costruire un programma che permetta di \emph{simulare} l'evoluzione del sistema definito sopra.
Innanzitutto bisogna implementare la segnatura, i bigrafi e le regole: per questo ci si appoggia alla libreria JLibBig.
Poi si pone il problema di come far evolvere il sistema, dato che le regole di reazione vengono applicate in modo non-deterministico: non esiste un solo percorso lineare di evoluzione, bens� ve ne sono molteplici, alcuni dei quali magari non hanno senso concretamente; possono addirittura formarsi cicli nell'evoluzione degli stati, portando un eventuale processo di simulazione a non terminare. In buona sostanza, pu� risultare necessario tenere traccia di tutte le applicazioni delle regole di reazione e di tutti gli stati generati, in modo da poter decidere in seguito quali ci interessano e quali no. 

\`E evidente che un un simulatore � un costrutto complesso: al fine di semplificarne la comprensione, nonch� l'implementazione, si propone di suddividerlo in moduli, che verranno discussi uno ad uno. Saranno altre-s� discusse le basi implementative necessarie.

\section{La libreria JLibBig}

\section{Rappresentare i Bigrafi}

\section{Rappresentare le Regole di Riscrittura}

\section{Modellare il BRS}
Il modulo BRS ha un semplice compito: dato un bigrafo ed un insieme di regole di riscrittura, esso applica queste ultime al bigrafo, producendo una lista di bigrafi che rappresentano tutti i possibili stati successivi; Non viene fatto alcun controllo di sorta sul risultato. 
%Si tenga conto che ogni regola pu� essere applicata pi� volte, in quanto il \emph{redex} pu� avere pi� di un match: questo significa che, anche con poche regole di riscrittura, lo stato del sistema pu� ramificarsi significativamente.
\paragraph{Implementazione} Nello specifico, BRS produce una lista di coppie \newline$(bigrafo,regolaRiscrittura)$, in modo che gli altri moduli del simulatore possano capire che regola ha prodotto una data evoluzione del sistema.
\paragraph{Strategie} Il modulo BRS � stato implementato in modo da poter definire con che criterio vengono applicate le regole di reazione: per esempio, � possibile definire delle regole di riscrittura con priorit�, e far s� che BRS le rispetti. Tuttavia, nella maggior parte dei casi pratici, questa versatilit� ha un impatto trascurabile.


\section{Grafo degli stati}
\lipsum[1]
\section{Comporre il Simulatore}
\lipsum[1]
\section{Visualizzare i Bigrafi}
\lipsum[1]
\section{Esempi e applicazioni}
\subsection{Il Gioco della Vita di Conway}
\lipsum[1-2]
% altro...
\section{Conclusioni}
\lipsum[1-2]
\end{document}