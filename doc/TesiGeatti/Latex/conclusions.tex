\conclusions\label{ch:concl}
L'obbiettivo principale di questa tesi � la creazione di uno strumento per la verifica di propriet� all'interno di un BRS, che � stato implementato risolvendo due problemi principali: l'isomorfismo tra bigrafi e la creazioni di un model checker con una logica generale.

Il requisito principale su questo strumento di verifica � la sua \emph{generalit�}: esso deve permettere di esprimere le propriet� su qualsiasi dominio il BRS rappresenti. Questa tesi ha rispettato il requisito appena citato, grazie soprattutto alla semantica della logica creata, che risulta quindi sufficientemente generale. D'altra parte, potrebbe non risultare molto espressiva: essa � principalmente una logica spaziale, e quindi non tiene conto di fattori temporali che potrebbero aumentare la sua espressibilit�. Si � per� dato spazio a future logiche che potranno facilmente estendere $MC_{big}$.

L'isomorfismo tra bigrafi ha permesse invece la costruzione dello scheletro per il model checker: qualsiasi strumento di verifica deve permettere di evitare esecuzioni infinite quando � possibile. L'isomorfismo ha permesso quindi il raggiungimento di questo obbiettivo. Il problema � stato risolto tramite la \emph{programmazione a vincoli}, che ha ridotto notevolmente il tempo di sviluppo dell'algoritmo ma che potrebbe non essere la soluzioni migliore in termini di complessit�.

Il contributo principale di questa tesi � stato il model checker $MC_{big}$: i bigrafi sono considerati un \emph{meta-modello} con cui si possono esprimere vari formalismi, come per esempio gli automi a stati finiti (che sono stati implementati in questa sede nell'esempio \ref{sec:nfa}), il $\pi$-calcolo, il $\lambda$-calcolo e molti altri, come le Reti di Petri. Inoltre � possibile esprimere anche un bigrafo appartenente a un dominio specifico (come l'esempio \ref{sec:secureBuilding}). Avere a disposizione uno strumento come $MC_{big}$ permette di verificare propriet� su tutti questi formalismi, sia \emph{general-purpose} (come gli NFA) sia a \emph{domain-specific} (come la rappresentazione di un edificio), risultando quindi \textbf{universale e flessibile} come lo sono i bigrafi.




























