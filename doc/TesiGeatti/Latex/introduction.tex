\chapter{Introduzione}\label{ch:intro}
Il lavoro riportato in questa tesi nasce dal problema di verificare delle propriet� in un Sistema Reattivo Bigrafico (BRS). In particolare, si � studiato il modo di controllare il sistema durante la sua evoluzione e dunque di fermare quest'ultima appena le propriet� desiderate siano state raggiunte. Questo tipo di verifica va sotto il nome di ``Model Checking''.

I Sistemi Reattivi Bigrafici (BRS) sono un nuovo formalismo con il quale si possono rappresentare sistemi distribuiti, di qualsiasi tipo essi siano: da un sistema di smartphones
ad un sistema biologico \cite{Damgaard08ageneric}. I BRS sono basati su un' importante struttura matematica: i bigrafi. Sono questi che permettono una facile trattazione dei vari ``oggetti distribuiti" che compongono il sistema, e di come essi interagiscono tra di loro.

L'importanza dei bigrafi si pu� riscontrare nella loro flessibilit�: essi costituiscono un \emph{meta-modello}, con cui � possibile rappresentare sistemi di qualsiasi dominio si voglia. Di recente i bigrafi sono stati usati per creare delle Reti di Petri \cite{DBLP:conf/ac/Milner03}, come anche per controllare un sistema di robot mobili \cite{pereiranetworked}.

Un altro punto di forza dei bigrafi sta nella loro capacit� di evolversi, potendo cos� rappresentare lo stato del sistema anche quando questo cambia. Si ha cos� a 
disposizione un Sistema Reattivo Bigrafico. 

Questa tesi tratta il problema di come poter sapere se un dato BRS rispetti certe propriet�. Per esempio: se rappresentiamo una rete con un BRS, ci possiamo chiedere se, 
dato uno stato iniziale in cui il pacchetto parte dal mittente A, esso arrivi o meno al destinatario B che si trova a vari router di distanza da A.
Oppure, cambiando dominio del problema, ci possiamo domandare se un automa rappresentato tramite bigrafi riconosca o meno una data stringa.

Il problema affrontato in questa sede prescinde quindi dal particolare dominio del problema, ed offre una soluzione generale, valida per qualsiasi BRS.
Per fare questo, si sono dovute affrontare varie problematiche. Tra le pi� importanti figurano:
\begin{itemize}
  \item
  quando due bigrafi sono uguali? Un BRS evolve senza memoria degli stati precedenti in cui si � trovato. Questo problema, in concreto, pu� potenzialmente
  causare evoluzioni infinite del BRS: per esempio, il pacchetto nella rete pu� girare all'infinito tra due router, perch� il BRS si "dimentica" da dove il pacchetto
  � arrivato.
  \item
  come rappresentare le propriet� da verificare nel BRS? In particolare, posso rappresentare con un solo formalismo vari tipi di propriet�, dall'arrivo a destinazione di
  un pacchetto al riconoscimento di una stringa?
  Il problema maggiore � il fatto che il modo per rappresentarle deve essere generale tanto quanto i BRS. In
  sostanza si � scelto un modo che astraesse ancora una volta dal dominio scelto.
\end{itemize}



La struttura della tesi rispetta dunque queste problematiche:

Nel capitolo \ref{ch:bigraphs} verranno presentate formalmente le nozioni di Bigrafo e di BRS. Con esse, verr� anche descritta un'algebra per creare nuovi bigrafi a partire da
bigrafi base.

Nel capitolo \ref{ch:isoChapter} si affronta il primo dei due principali problemi, che va sotto il nome di "isomorfismo tra bigrafi". La risoluzione di questo problema ci permetter� di poter affermare quando due bigrafi sono in pratica uguali o meno, e quindi di evitare evoluzioni infinite del BRS.

Si affronter� nel capitolo \ref{ch:modelCheckerChapter} il secondo problema, cio� quello delle propriet�. Esse verranno espresse sul calcolatore tramite una semplice logica a predicati. Si potranno cos� esprimere tutte le propriet� desiderate, indipendentemente dal dominio del sistema. Grazie a queste propriet�, si arriver� all'implementazione di un Model Checker per i bigrafi.

Nel capitolo \ref{ch:examples} verranno presentati alcuni esempi, presi da vari domini. Si potr� apprezzare l'importanza di avere un Model Checker e della semplicit� con cui si possono esprimere le propriet� da verificare.

Infine si sono tratte le conclusioni sull'intero lavoro. Verranno presentate alternative per l'implementazione dell'isomorfismo e delle propriet�.


















