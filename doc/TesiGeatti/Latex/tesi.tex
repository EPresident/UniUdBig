%&PDFLaTeX
\documentclass[11pt,UdineBachThesis,american,italian,draft]{PhdThesis}[2007/09/20]
\StudyCourse{Corso di Laurea in Tecnologie Web e Multimediali}

\usepackage[final]{graphicx}

% here more packages
\usepackage[latin1]{inputenc}
\usepackage[italian]{babel}
%Package aggiuntivi:
\usepackage{amsmath}
\usepackage{amssymb}
\usepackage{amsthm}
\usepackage{tikz}
\usepackage{subfigure}
\usepackage[final]{listings}
\usepackage{graphicx}
\usepackage{tabularx}
\usepackage{setspace}
\usepackage{pbox}
\usepackage{verbatim}
\usepackage{pgfplots}
\usepackage{rotating}
\usepackage{booktabs, longtable, tabu}
\usepackage{url} % hyperref works too
\urlstyle{same}
\usetikzlibrary{shapes, positioning, trees, arrows, shadows, calc, matrix}

%Settings for the Listings package
\lstset{frame=tb,
  language=Java,
  aboveskip=3mm,
  belowskip=3mm,
  showstringspaces=false,
  columns=flexible,
  basicstyle={\small\ttfamily},
  numbers=none,
  numberstyle=\tiny\color{gray},
  keywordstyle=\color{blue},
  commentstyle=\color{dkgreen},
  stringstyle=\color{mauve},
  breaklines=true,
  breakatwhitespace=true,
  tabsize=3
}

%this only if you like to use some of its features (like smartrefs)
%\usepackage{theorems}[2003/12/05]

% consider to use hyperref
%\usepackage[bookmarks=true,bookmarksopen=true,pdfhighlight=/I,pdfpagemode=UseOutlines]{hyperref}%if you use pdflatex
%\usepackage[hypertex, colorlinks=true, backref]{hyperref}%if you use latex

% learn to use includes! 
%\includeonly{introduction,preliminaries, }

% here macros
%Comandi
\newtheoremstyle{mystyle}{}{}{\normalfont}{}{\bfseries}{.}{ }{}
\newtheorem{teor}{Teorema}[section]
\newtheorem{define}{Definizione}[section]
\newtheorem*{notaz}{Notazione}
\newtheorem*{term}{Terminologia}
\newtheorem{prop}{Proposizione}[section]
\theoremstyle{mystyle}
\newtheorem*{dimos}{Dimostrazione}
\newtheorem{corol}{Corollario}[section]

%Tikz style
\definecolor{myGreen}{RGB}{18,142,21}
\definecolor{myGrey}{RGB}{222,223,222}
\definecolor{dkgreen}{rgb}{0,0.6,0}
\definecolor{gray}{rgb}{0.5,0.5,0.5}
\definecolor{mauve}{rgb}{0.58,0,0.82}

%\newcommand{\wario}{
%\begin{rotate}{180}M\end{rotate}
%}

\newcommand{\wario}{
\raisebox{\depth}{\rotatebox{180}{M}}
}


\title{Verifica di propriet� locali su BRS}
\author{Luca Geatti}
\email{geatti.luca@spes.uniud.it}
\supervisor{Prof.\ Marino Miculan}
\cosupervisor{Dott.\ Marco Peressotti}
\date{2014-2015}



%%%%%%%%%%%%%%%%%%%%%%%%%%%%%%%%%%%%%%%%%%%%%%%%%%%%%%%%%%%%%%%%%%%%%%%
%Inizio del documento
\begin{document}
\pagestyle{empty}
\maketitle

%Dedica
\begin{dedication}
	Ai miei genitori
\end{dedication}

%Sommario
%\begin{abstract}
%    Qui ci va una "fotografia" del lavoro svolto nella tesi: un estremo riassunto per fare capire al lettore se � quello che cerca.
%\end{abstract}
%

%Ringraziamenti
%\begin{acknowledgments}
%	Here the acknowledgments
%\end{acknowledgments}

\frontmatter %numeri romani per l'indice e l'introduzione
\partstyle{serifbig}
\chaptertitlestyle{serifbig}
\pagestyle{serif}
\tableofcontents
\listoffigures
%\listoftables

%\begin{preface}
%    Here the preface
%\end{preface}

\mainmatter %numeri arabi per il vero contenuto
%otherwise use
% \mainmatter
% \introduction\label{ch:intro}
Il lavoro riportato in questa tesi nasce dal problema di verificare delle propriet� in un Sistema Reattivo Bigrafico (BRS). In particolare, si � studiato il modo di controllare il sistema durante la sua evoluzione e dunque di fermare quest'ultima appena le propriet� desiderate siano state raggiunte. Questo tipo di verifica va sotto il nome di ``Model Checking''.\\

I Sistemi Reattivi Bigrafici (BRS) sono un nuovo formalismo con il quale si possono rappresentare sistemi distribuiti, di qualsiasi tipo essi siano: da un sistema di smartphones
ad un sistema biologico \cite{Damgaard08ageneric}. I BRS sono basati su un' importante struttura matematica: i bigrafi. Sono questi che permettono una facile trattazione dei vari ``oggetti distribuiti" che compongono il sistema, e di come essi interagiscono tra di loro.\\

L' importanza dei bigrafi la si pu� riscontrare nella loro flessibilit�: essi costituiscono un \emph{meta-modello}, con cui � possibile rappresentare sistemi di qualsiasi dominio si voglia. Di recente i bigrafi sono stati usati per creare delle Reti di Petri \cite{DBLP:conf/ac/Milner03}, come anche per controllare un sistema mobile di robot \cite{pereiranetworked}.

Un altro punto di forza dei bigrafi sta nella loro capacit� di evolversi, potendo cos� rappresentare lo stato del sistema anche quando questo cambia. Si ha cos� a 
disposizione un Sistema Reattivo Bigrafico. \\

Questa tesi tratta il problema di come poter sapere se un dato BRS rispetti certe propriet�. Per esempio: se rappresentiamo una rete con un BRS, ci possiamo chiedere se, 
dato uno stato iniziale in cui il pacchetto parte dal mittente A, esso arrivi o meno al destinatario B che si trova a vari router di distanza da A.
Oppure, cambiando dominio del problema, ci possiamo domandare se un automa rappresentato tramite bigrafi riconosca o meno una data stringa.\\

Il problema affrontato in questa sede prescinde quindi dal particolare dominio del problema, ed offre una soluzione generale, valida per qualsiasi BRS.
Per fare questo, si sono dovute affrontare varie problematiche. Tra le pi� importanti figurano:
\begin{itemize}
  \item
  quando due bigrafi sono uguali? Un BRS evolve senza memoria degli stati precedenti in cui si � trovato. Questo problema, in concreto, pu� potenzialmente
  causare evoluzioni infinite del BRS: per esempio, il pacchetto nella rete pu� girare all'infinito tra due router, perch� il BRS si "dimentica" da dove il pacchetto
  � arrivato.
  \item
  come rappresentare le propriet� da verificare nel BRS? In particolare, posso rappresentare con un solo formalismo vari tipi di propriet�, dall'arrivo a destinazione di
  un pacchetto al riconoscimento di una stringa?
  Il problema maggiore � il fatto che il modo per rappresentarle deve essere generale tanto quanto i BRS. In
  sostanza si � scelto un modo che astraesse ancora una volta dal dominio scelto. \\ \\
\end{itemize}



La struttura della tesi rispetta dunque queste problematiche:\\

Nel capitolo \ref{ch:bigraphs} verranno presentate formalmente le nozioni di Bigrafo e di BRS. Con esse, verr� anche descritta un'algebra per creare nuovi bigrafi a partire da
bigrafi base.\\

Nel capitolo 2 si affronta il primo dei due principali problemi, che va sotto il nome di "isomorfismo tra bigrafi". La risoluzione di questo problema ci permetter� di poter affermare
quando due bigrafi sono uguali o meno, e quindi di evitare evoluzioni infinite del BRS.\\

Si affronter� nel capitolo 3 il secondo problema, cio� quello delle propriet�. Esse verranno espresse sul calcolatore tramite una semplice logica a predicati. Si potranno cos�
esprimere tutte le propriet� desiderate, indipendentemente dal dominio del sistema. Grazie a queste propriet�, si arriver� all'implementazione di un Model Checker per i bigrafi.
\\

Nel capitolo 4 verranno presentati alcuni esempi, presi da vari domini. Si potr� apprezzare l'importanza di avere un Model Checker e della semplicit� con cui si possono
esprimere le propriet� da verificare.\\

Infine (capitolo 5), si sono tratte le conclusioni sull'intero lavoro. Verranno presentate alternative per l'implementazione dell'isomorfismo e delle propriet�.


















%with \chapter{\introductionname} instead of \introduction

% \part{First Part}
\introduction\label{ch:intro}
Il lavoro riportato in questa tesi nasce dal problema di verificare delle propriet� in un Sistema Reattivo Bigrafico (BRS). In particolare, si � studiato il modo di controllare il sistema durante la sua evoluzione e dunque di fermare quest'ultima appena le propriet� desiderate siano state raggiunte. Questo tipo di verifica va sotto il nome di ``Model Checking''.\\

I Sistemi Reattivi Bigrafici (BRS) sono un nuovo formalismo con il quale si possono rappresentare sistemi distribuiti, di qualsiasi tipo essi siano: da un sistema di smartphones
ad un sistema biologico \cite{Damgaard08ageneric}. I BRS sono basati su un' importante struttura matematica: i bigrafi. Sono questi che permettono una facile trattazione dei vari ``oggetti distribuiti" che compongono il sistema, e di come essi interagiscono tra di loro.\\

L' importanza dei bigrafi la si pu� riscontrare nella loro flessibilit�: essi costituiscono un \emph{meta-modello}, con cui � possibile rappresentare sistemi di qualsiasi dominio si voglia. Di recente i bigrafi sono stati usati per creare delle Reti di Petri \cite{DBLP:conf/ac/Milner03}, come anche per controllare un sistema mobile di robot \cite{pereiranetworked}.

Un altro punto di forza dei bigrafi sta nella loro capacit� di evolversi, potendo cos� rappresentare lo stato del sistema anche quando questo cambia. Si ha cos� a 
disposizione un Sistema Reattivo Bigrafico. \\

Questa tesi tratta il problema di come poter sapere se un dato BRS rispetti certe propriet�. Per esempio: se rappresentiamo una rete con un BRS, ci possiamo chiedere se, 
dato uno stato iniziale in cui il pacchetto parte dal mittente A, esso arrivi o meno al destinatario B che si trova a vari router di distanza da A.
Oppure, cambiando dominio del problema, ci possiamo domandare se un automa rappresentato tramite bigrafi riconosca o meno una data stringa.\\

Il problema affrontato in questa sede prescinde quindi dal particolare dominio del problema, ed offre una soluzione generale, valida per qualsiasi BRS.
Per fare questo, si sono dovute affrontare varie problematiche. Tra le pi� importanti figurano:
\begin{itemize}
  \item
  quando due bigrafi sono uguali? Un BRS evolve senza memoria degli stati precedenti in cui si � trovato. Questo problema, in concreto, pu� potenzialmente
  causare evoluzioni infinite del BRS: per esempio, il pacchetto nella rete pu� girare all'infinito tra due router, perch� il BRS si "dimentica" da dove il pacchetto
  � arrivato.
  \item
  come rappresentare le propriet� da verificare nel BRS? In particolare, posso rappresentare con un solo formalismo vari tipi di propriet�, dall'arrivo a destinazione di
  un pacchetto al riconoscimento di una stringa?
  Il problema maggiore � il fatto che il modo per rappresentarle deve essere generale tanto quanto i BRS. In
  sostanza si � scelto un modo che astraesse ancora una volta dal dominio scelto. \\ \\
\end{itemize}



La struttura della tesi rispetta dunque queste problematiche:\\

Nel capitolo \ref{ch:bigraphs} verranno presentate formalmente le nozioni di Bigrafo e di BRS. Con esse, verr� anche descritta un'algebra per creare nuovi bigrafi a partire da
bigrafi base.\\

Nel capitolo 2 si affronta il primo dei due principali problemi, che va sotto il nome di "isomorfismo tra bigrafi". La risoluzione di questo problema ci permetter� di poter affermare
quando due bigrafi sono uguali o meno, e quindi di evitare evoluzioni infinite del BRS.\\

Si affronter� nel capitolo 3 il secondo problema, cio� quello delle propriet�. Esse verranno espresse sul calcolatore tramite una semplice logica a predicati. Si potranno cos�
esprimere tutte le propriet� desiderate, indipendentemente dal dominio del sistema. Grazie a queste propriet�, si arriver� all'implementazione di un Model Checker per i bigrafi.
\\

Nel capitolo 4 verranno presentati alcuni esempi, presi da vari domini. Si potr� apprezzare l'importanza di avere un Model Checker e della semplicit� con cui si possono
esprimere le propriet� da verificare.\\

Infine (capitolo 5), si sono tratte le conclusioni sull'intero lavoro. Verranno presentate alternative per l'implementazione dell'isomorfismo e delle propriet�.



















\chapter{Bigrafi e BRS}\label{ch:bigraphs}

In questo capitolo vengono presentate le descrizioni formali di bigrafo e di Sistema Reattivo Bigrafico. Si vedr� come l'importanza dei bigrafi risieda nel fatto di rappresentare
contemporaneamente i concetti di \emph{localit�} e \emph{connessione}. 

L'algebra dei bigrafi � stata per gran parte costruita sulla base della Teoria delle Categorie, che in questa sede non verr� introdotta. Le definizioni ed i teoremi sono stati presi da \cite{DBLP:books/daglib/0022395}, a cui si rimanda per i dettagli sulla Teoria delle Categorie.
Infine si rimanda all'Appendice A per la descrizione della terminologia usata.


\section{Definizione informale di Bigrafo}
L'idea fondamentale alla base della loro teoria, � che ogni bigrafo sia composto da due strutture del tutto \emph{indipendenti} sullo \emph{stesso} insieme di nodi.
Queste due strutture si chiamano \emph{Place Graph} e \emph{Link Graph} e modellano rispettivamente la localit� e la connessione. 

Nell' introduzione, si � accennato al fatto che i bigrafi sono flessibili e adatti a rappresentare ogni dominio. Questo � possibile grazie al concetto di 
\emph{segnatura}, che � l'analogo ad una grammatica per un linguaggio.

\begin{define}[Segnatura e Controllo]
Una segnatura � una coppia $(K , ar)$, dove $K$ � un insieme di tipi di nodi chiamati \emph{controlli}, e $ar: K \to \mathbb{N}$ � una mappa che associa ad ogni tipo
di nodo (cio� ad ogni controllo) un numero naturale chiamato ariet�.
\end{define}

Quindi, dare una segnatura ad un bigrafo significa associare ad ogni nodo sia un tipo sia il suo numero di porte. L'equivalente grafico consiste nel disegnare nodi diversi con
simboli diversi.

\begin{notaz}[Segnatura]\label{def:sign}
Da qui in avanti, una segnatura $(K, ar)$ verr� indicata nel seguente modo: \\
\begin{center}
$K = \{K_1:a_1, \dots, K_n:a_n\}$
\end{center}
dove ogni nodo di tipo (controllo) $K_i$ ha ariet� $a_i$.
\end{notaz}


\subsection{Esempio}\label{sub:esempioInit}
Diamo un primo esempio informale di bigrafo. 

\begin{figure}[h]
\centering
\begin{tikzpicture}
%nodes
%\draw[help lines] (0,0) grid (8,5);
\draw[thick] (2.5,1.5) ellipse (2.5 and 1.5);%v0
\node[left] at (1.0,3.0) {$v_0$};
\draw[thick] (1.5,1.5) ellipse (0.5 and 0.5);%v1
\node[left] at (1.0,1.5) {$v_1$};
\draw[thick] (3.5,1.5) ellipse (1.0 and 1.0);%v2
\node[left] at (3.0,2.5) {$v_2$};
\draw[thick] (3.7,1.7) ellipse (0.3 and 0.3);%v3
\node[below right] at (3.5,1.5) {$v_3$};
\draw[thick] (6.5,3.0) ellipse (1.5 and 1.0);%v4
\node[above right] at (6.5,4.0) {$v_4$};
\draw[thick] (6.0,3.0) ellipse (0.5 and 0.5);%v5
\node[above right] at (6.5,3.0) {$v_5$};
%links
\draw [myGreen, thick] (1.5,1.0) to [out=320,in=230] (3.5,1.5);
\draw [fill] (1.5,1.0) circle [radius=0.05];
\draw [fill] (3.5,1.5) circle [radius=0.05];
\draw [myGreen, thick] (4.0,1.7) to [out=320,in=270] (5.6,2.7);
\draw [fill] (4.0,1.7) circle [radius=0.05];
\draw [fill] (5.6,2.7) circle [radius=0.05];
\draw [myGreen, thick] (5.0,3.0) to [out=270,in=180] (5.3,2.0);
\draw [fill] (5.0,3.0) circle [radius=0.05];

\draw [myGreen, thick] (1.5,2.0) to [out=90,in=230] (4.0,5.0);
\draw [fill] (1.5,2.0) circle [radius=0.05];
\draw [myGreen, thick] (3.0,3.0) to [out=90,in=230] (4.0,5.0);
\draw [fill] (3.0,3.0) circle [radius=0.05];
\draw [myGreen, thick] (6.0,4.0) to [out=90,in=230] (4.0,5.0);
\draw [fill] (6.0,4.0) circle [radius=0.05];

\end{tikzpicture}
\caption{Un semplice bigrafo.\label{fig:simpleBig}}
\end{figure}

In figura \ref{fig:simpleBig} vediamo un bigrafo B, con una data segnatura. Si vede subito come ci siano dei controlli diversi. Per esempio, il nodo $v_0$ ha una sola porta
mentre il nodo $v_1$ ne ha due. L'informazione che quest'ultimo sia contenuto in $v_0$ � rappresentata nel place graph di figura \ref{fig:placeLink}.a. Le interconnessioni dei vari nodi sono invece riportate nel rispettivo link graph di figura \ref{fig:placeLink}.b.\\

\begin{figure}[h]
 \centering
 \subfigure[Place Graph]
 {
   	\begin{tikzpicture}
	%nodes
	\draw [thick] (1.5,4.5) circle [radius=0.2];%v0
	\node[above right] at (1.5,4.5) {$v_0$};
	\draw  [thick] (0.5,3.5) circle [radius=0.2];%v1
	\node[above right] at (0.5,3.5) {$v_1$};
	\draw [thick]  (2.5,3.5) circle [radius=0.2];%v2
	\node[above right] at (2.5,3.5) {$v_2$};
	\draw  [thick] (2.5,1.5) circle [radius=0.2];%v3
	\node[above right] at (2.5,1.5) {$v_3$};
	\draw  [thick] (4.5,4.5) circle [radius=0.2];%v4
	\node[above right] at (4.5,4.5) {$v_4$};
	\draw  [thick] (4.5,2.5) circle [radius=0.2];%v5
	\node[above right] at (4.5,2.5) {$v_5$};

	%links
	\draw (1.3,4.5) -- (0.5,3.7);
	\draw (1.7,4.5) -- (2.5,3.7);
	\draw (2.5,3.3) -- (2.5,1.7);
	\draw (4.5,4.3) -- (4.5,2.7);

	\end{tikzpicture}
}
\hspace{5mm}
\subfigure[Link Graph]
{
  	\begin{tikzpicture}
	%nodes
	\draw[thick] (2.0,0.2) circle [radius=0.2];%v3
	\node[above left] at (2.0,0.2) {$v_3$};
	\draw[thick] (1.0,3.0) circle [radius=0.2];%v1
	\node[above left] at (1.0,3.0) {$v_1$};
	\draw[thick] (3.0,3.5) circle [radius=0.2];%v0
	\node[above right] at (3.0,3.5) {$v_0$};
	\draw[thick] (3.0,2.0) circle [radius=0.2];%v2
	\node[above left] at (3.0,2.0) {$v_2$};
	\draw[thick] (4.0,0.2) circle [radius=0.2];%v5
	\node[above right] at (4.0,0.2) {$v_5$};
	\draw[thick] (4.5,2.0) circle [radius=0.2];%v4
	\node[above right] at (4.5,2.0) {$v_4$};
	
	%links
	\draw [myGreen, thick] (4.3,2.0) to [out=180,in=0] (2.0,3.12);
	\draw [myGreen, thick] (1.0,2.8) to [out=270,in=90] (2.0,0.4);
	\draw [fill] (1.0,2.8) circle [radius=0.05];
	\draw [fill] (2.0,0.4) circle [radius=0.05];
	\draw [myGreen, thick] (1.2,3.0) to [out=0,in=180] (3.0,3.3);
	\draw [fill] (1.2,3.0) circle [radius=0.05];
	\draw [fill] (3.0,3.3) circle [radius=0.05];
	\draw [myGreen, thick] (2.2,0.2) to [out=0,in=180] (4.5,1.8);
	\draw [fill] (2.2,0.2) circle [radius=0.05];
	\draw [fill] (4.3,2.0) circle [radius=0.05];
	\draw [myGreen, thick] (4.0,0.4) to [out=90,in=0] (3.35,1.0);
	\draw [fill] (4.0,0.4) circle [radius=0.05];
	\draw [fill] (4.5,1.8) circle [radius=0.05];	
	
	\end{tikzpicture}
}
 \caption{Le due strutture ortogonali del bigrafo B.\label{fig:placeLink}}
 \end{figure}


Un'importante caratteristica dei bigrafi e della loro algebra � la possibilit� di essere composti, cio� di formare un nuovo bigrafo da due bigrafi di partenza. Questo � equivalente
al problema di considerare un bigrafo \emph{parte} di un altro. Si vedr� che questa operazione va sotto il nome di \emph{composizione}. Se prendiamo il bigrafo B di cui sopra,
allora � possibile scrivere un'equazione di questo tipo:

$\qquad \qquad {  B = A \circ C  }$

Per renderla possibile dobbiamo aggiungere struttura ai bigrafi. Si introducono quindi le interfacce interne ed esterne, sia per il place graph sia per il link graph:
\begin{itemize}
	\item
	L'interfaccia esterna ed interna del place graph sono un numero naturale \emph{n} che possiamo trattare come un ordinale: l'interfaccia $n$ indica l'insieme\\
	$\{0, \dots, n-1 \}$. Se l'interfaccia esterna � $k$, allora si dice che ci sono \emph{k radici}. Se l'interfaccia interna � $h$, allora si dice che ci sono 
	\emph{h siti}.
	
	\item
	Per il link graph, invece, le interfaccie sono \emph{insiemi} di nomi, come per esempio $\{x,y\}$. Se un link graph ha un interfaccia esterna del tipo $\{a,b,c\}$, allora
	diciamo che ci sono tre \emph{outername} chiamati \emph{a,b} e \emph{c}. Se l'interfaccia interna � del tipo $\{x,y\}$, allora diciamo che il bigrafo ha due 
	\emph{innername} chiamati \emph{x e y}.
\end{itemize}

Il concetto fondamentale delle interfacce � che servono per l'unione dei due bigrafi. Per esempio, prendendo il place graph, nell'operazione di composizione $A \circ C$,
i siti di A dovranno \emph{concordare} (vedremo una descrizione formale di questo concetto) con le radici di C. Per il link graph il concetto � lo stesso: gli innername di A dovranno
unirsi con gli outername di C.\\

Si consideri per esempio il bigrafo della figura \ref{fig:simpleBig}. Se vogliamo trovare due bigrafi A e C tali che $B = A \circ C$, allora dobbiamo rispettare le condizioni con cui
si pu� effettuare l'operazione di composizione. Le figure \ref{fig:bigC} e \ref{fig:bigA} rappresentano rispettivamente i due bigrafi C ed A. 
Si noti ancora una volta che le radici di C si \emph{uniscono} ai siti di A, e gli outername di C fanno lo stesso con gli innername di A. \newpage

%%%%%% BIGRAFO C
\begin{figure}[h]
\centering
\begin{tikzpicture}
%nodes
%\draw[help lines] (0,0) grid (8,5);
\draw[thick] (1.5,1.5) ellipse (0.5 and 0.5);%v1
\node[left] at (1.0,1.5) {$v_1$};
\draw[thick] (3.7,1.7) ellipse (0.3 and 0.3);%v3
\node[below right] at (3.5,1.5) {$v_3$};
\draw[thick] (6.5,3.0) ellipse (1.5 and 1.0);%v4
\node[above right] at (6.5,4.0) {$v_4$};
\draw[thick] (6.0,3.0) ellipse (0.5 and 0.5);%v5
\node[above right] at (6.5,3.0) {$v_5$};
%roots
\draw [rounded corners=5mm,dotted, thick] (0.0,0.0) rectangle (2.5,5.0);
\node [below right] at (0.0,5.0) {0};
\draw [rounded corners=5mm,dotted, thick] (2.7,0.0) rectangle (4.5,5.0);
\node [below right] at (2.7,5.0) {1};
\draw [rounded corners=5mm,dotted, thick] (4.7,0.0) rectangle (8.0,5.0);
\node [below right] at (4.7,5.0) {2};

	
%links
\draw [myGreen, thick] (1.5,1.0) to [out=320,in=230] (3.5,1.5);
\draw [fill] (1.5,1.0) circle [radius=0.05];
\draw [fill] (3.5,1.5) circle [radius=0.05];
\draw [myGreen, thick] (4.0,1.7) to [out=320,in=270] (5.6,2.7);
\draw [fill] (4.0,1.7) circle [radius=0.05];
\draw [fill] (5.6,2.7) circle [radius=0.05];
\draw [myGreen, thick] (5.0,3.0) to [out=270,in=180] (5.3,2.0);
\draw [fill] (5.0,3.0) circle [radius=0.05];

\draw [myGreen, thick] (1.5,2.0) to [out=90,in=230] (1.5,5.0);
\draw [fill] (1.5,5.0) circle [radius=0.05];
\draw [fill] (1.5,2.0) circle [radius=0.05];
\node [above] at (1.5,5.0) {x};
\draw [myGreen, thick] (6.0,4.0) to [out=90,in=230] (6.0,5.0);
\draw [fill] (6.0,4.0) circle [radius=0.05];
\draw [fill] (6.0,5.0) circle [radius=0.05];
\node [above] at (6.0,5.0) {y};


\end{tikzpicture}
\caption{Il bigrafo C.\label{fig:bigC}}
\end{figure}


\begin{figure}[h]
 \centering
 \subfigure[Place Graph di C]
 {
   	\begin{tikzpicture}
	%nodes
	\node at (1.0,4.0) {$0$};%0
	\draw [thick] (1.0,2.0) circle [radius=0.2];%v1
	\node[above right] at (1.0,2.0) {$v1$};
	
	\node at (2.0,4.0) {$1$};%1
	\draw [thick] (2.0,2.0) circle [radius=0.2];%v3
	\node[above right] at (2.0,2.0) {$v3$};
	
	\node at (3.0,4.0) {$2$};%2
	\draw [thick] (3.0,3.0) circle [radius=0.2];%v4
	\node[above right] at (3.0,3.0) {$v4$};
	\draw [thick] (3.0,2.0) circle [radius=0.2];%v5
	\node[above right] at (3.0,2.0) {$v5$};
	
	
	%links
	\draw (1.0,3.8) -- (1.0,2.2);% 0 - v1
	\draw (2.0,3.8) -- (2.0,2.2);% 1 - v3
	\draw (3.0,3.8) -- (3.0,3.2);% 2 - v4	
	\draw (3.0,2.8) -- (3.0,2.2);% v4 - v5

	\end{tikzpicture}
}
\hspace{5mm}
\subfigure[Link Graph di C]
{
  	\begin{tikzpicture}
	%nodes
	\draw[thick] (2.0,0.2) circle [radius=0.2];%v3
	\node[above left] at (2.0,0.2) {$v_3$};
	\draw[thick] (1.0,3.0) circle [radius=0.2];%v1
	\node[above left] at (1.0,3.0) {$v_1$};
	\draw[thick] (4.0,0.2) circle [radius=0.2];%v5
	\node[above right] at (4.0,0.2) {$v_5$};
	\draw[thick] (4.5,2.0) circle [radius=0.2];%v4
	\node[above right] at (4.5,2.0) {$v_4$};
	
	%links
	\draw [myGreen, thick] (1.0,2.8) to [out=270,in=90] (2.0,0.4);
	\draw [fill] (1.0,2.8) circle [radius=0.05];
	\draw [fill] (2.0,0.4) circle [radius=0.05];
	\draw [myGreen, thick] (1.2,3.0) to [out=0,in=180] (1.5,3.5);
	\draw [fill] (1.5,3.5) circle [radius=0.05];
	\node[above] at (1.5,3.5) {x};
	\draw [fill] (1.2,3.0) circle [radius=0.05];
	\draw [myGreen, thick] (2.2,0.2) to [out=0,in=180] (4.3,2.0);
	\draw [fill] (2.2,0.2) circle [radius=0.05];
	\draw [fill] (4.3,2.0) circle [radius=0.05];
	\draw [myGreen, thick] (4.0,0.4) to [out=90,in=0] (3.2,1.0);
	\draw [fill] (4.0,0.4) circle [radius=0.05];
	\draw [myGreen, thick] (4.5,2.2) to [out=180,in=0] (4.3,3.5);
	\draw [fill] (4.5,2.2) circle [radius=0.05];
	\draw [fill] (4.3,3.5) circle [radius=0.05];
	\node[above] at (4.3,3.5) {y};
	
	\end{tikzpicture}
}
 \caption{Bigrafo C.\label{fig:bigCdecap}}
 \end{figure}
 
 
 
 
%%%%%%%%%%%BIGRAFO A
\begin{figure}[!htbp]
\centering
\begin{tikzpicture}

%nodes
%\draw[help lines] (0,0) grid (8,5);
\draw[thick] (2.5,1.5) ellipse (2.5 and 1.5);%v0
\node[left] at (1.0,3.0) {$v_0$};
\draw[thick] (3.5,1.5) ellipse (1.0 and 1.0);%v2
\node[left] at (3.0,2.5) {$v_2$};
%roots
\draw[rounded corners=5mm,dotted, thick](0.0,0.0) rectangle (5.0,4.0);
\node [below right] at (0.0,4.0) {0};
\draw[rounded corners=5mm,dotted, thick] (5.3,0.0) rectangle (8.0,4.0);
\node [below right] at (5.3,4.0) {1};
%sites
\draw[rounded corners=1mm,dotted, thick, fill=myGrey](1.0,1.0) rectangle (2.0,2.0);
\node [below right] at (1.0,2.0) {0};
\draw[rounded corners=1mm,dotted, thick, fill=myGrey] (3.0,1.0) rectangle (4.0,2.0);
\node [below right] at (3.0,2.0) {1};
\draw[rounded corners=1mm,dotted, thick, fill=myGrey] (6.0,1.0) rectangle (7.0,2.0);
\node [below right] at (6.0,2.0) {2};
%links
\draw [myGreen, thick] (3.0,3.0) -- (3.0,3.35);
\draw [fill] (3.0,3.0) circle [radius=0.05];
\draw[myGreen, thick] (1.0,0.0) arc (190:-10:3.0);
\node [below] at (1.0,0.0) {x};
\node [below] at (7.0,0.0) {y};

\end{tikzpicture}
\caption{Il bigrafo A \label{fig:bigA}}
\end{figure}

\newpage



\begin{figure}[h]
 \centering
 \subfigure[Place Graph di A]
 {
   	\begin{tikzpicture}
	%\draw[help lines] (0,0) grid (5,5);
	%nodes
	\node at (2.0,4.0) {$0$};%0
	\draw [thick] (2.0,3.0) circle [radius=0.2];%v0
	\node[above right] at (2.0,3.0) {$v0$};
	\draw [thick] (3.0,2.0) circle [radius=0.2];%v2
	\node[above right] at (3.0,2.0) {$v2$};
	\node at (1.0,1.0) {$0$};%0
	\node at (3.0,1.0) {$1$};%1
	
	\node at (4.0,4.0) {$1$};%1
	\node at (4.0,1.0) {$2$};%2	
	
	%links
	\draw (2.0,3.8) -- (2.0,3.2);% 0 - v0
	\draw (2.0,2.8) -- (3.0,2.2);% v0 - v2
	\draw (3.0,1.8) -- (3.0,1.2);% v2 - 1	
	\draw (2.0,2.8) -- (1.0,1.2);% v0 - 0
	\draw (4.0,3.8) -- (4.0,1.2);% 1 - 2

	\end{tikzpicture}
}
\hspace{5mm}
\subfigure[Link Graph di A]
{
  	\begin{tikzpicture}
	%\draw[help lines] (0,0) grid (5,5);
	%nodes
	\draw[thick] (1.0,1.0) circle [radius=0.2];%v0
	\node[above left] at (1.0,1.0) {$v_0$};
	\draw[thick] (3.0,1.0) circle [radius=0.2];%v2
	\node[above right] at (3.0,1.0) {$v_2$};
	%links
	\draw[myGreen, thick] (0.0,0.0) arc (190:-10:2.0);
	\node [below] at (0.0,0.0) {x};
	\node [below] at (4.0,0.0) {y};
	\draw [fill] (1.0,1.2) circle [radius=0.05];
	\draw [myGreen, thick] (1.0,1.2) to [out=90,in=200] (1.5,2.3);
	
	
	\end{tikzpicture}
}
 \caption{Bigrafo A.\label{fig:bigAdecap}}
 \end{figure}









%%%%%%%%%%%%%%%%%%%%%%%%%%%%%%%%%%%%%%%%%%%%%%%%%%%%%%%%%%%%%%%%%%%%%%%%%%%%%%%%%%
\section{Definizione formale di Bigrafo}\label{sec:formalBigraphs}
Siamo ora pronti per definire separatamente i concetti di \emph{Place Graph} e \emph{Link Graph}. Quelli descritti, che noi chiameremo semplicemente bigrafi, in realt�
si chiamano \emph{concrete bigraphs}, per distinguerli da quelli astratti. In questa sede si tratter� solo di bigrafi concreti.\\
Si rimanda all'appendice \ref{ch:appA} per la notazione usata.

%PlaceGraph
\subsection{Place Graph}
Il place graph � una delle due strutture fondamentali di ogni bigrafo. E' una \emph{foresta} e rappresenta l'informazione di \emph{nesting}, ovvero quali nodi si trovano
all'interno di altri. Come gi� notato in figura \ref{fig:placeLink}.a, le radici ed i siti sono \emph{ordinati}, essendo essi rappresentati da un ordinale. In definitiva, quindi,
il place graph � una foresta ordinata, in cui solo le radici ed i siti sono ordinati.

\begin{define}[Place Graph]
Un place graph 
\begin{center}
$F=(V_F,ctrl_F,prnt_F) : m\to n$
\end{center}
� una tripla avente un' interfaccia interna $m$ ed un' interfaccia esterna $n$, entrambe ordinali finiti.Queste indicano rispettivamente i \emph{siti} e le \emph{radici} del
place graph. F ha un insieme finito $V_F$ di \emph{nodi}, una \emph{control map} $ctrl:V_F \to K$, e una \emph{parent map}
\begin{center}
$prnt:m \uplus V_F \to V_F \uplus n$
\end{center}
che � aciclica, cio� se $prnt_{F}^{i}(v)=v$ allora $i=0$.
\end{define}

Questa definizione formale ricalca ci� che � gi� stato notato nell' esempio della sottosezione \ref{sub:esempioInit}. Infatti, si noti che il place graph $F$ ha $m$ siti
ed $n$ radici, entrambi ordinati, che costituiscono rispettivamente la sua interfaccia interna ed esterna.\\
La funzione $ctrl_F$ associa ad ogni nodo un controllo, cio� un nome. Come gi� notato, l'equivalente grafico sta nel disegnare con simboli diversi nodi con controllo diverso.
Infine, la funzione $prnt_F$ associa ad ogni nodo interno o sito il suo genitore, che pu� essere a sua volta un altro nodo interno o una radice.\\
Questa funzione � di fondamentale importanza, in quanto � il cuore del place graph: rappresenta l'informazione di quali nodi si trovano all'interno di altri. E' \emph{aciclica},
nel senso che $n$ sue \emph{composizioni} non porteranno mai al nodo di partenza. In formule: 
\begin{center}
$prnt_F \circ prnt_F \circ \dots \circ prnt_F(v) \ne v$
\end{center}
Questo � equivalente a dire che la struttura dati rappresentata dal place graph � una \emph{foresta}.


%Link Graph
\subsection{Link Graph}\label{sub:linkGraph}
La seconda struttura dati fondamentale � il link graph. Esso rappresenta l'informazione di \emph{connessione} tra i nodi del bigrafo. E' un \emph{ipergrafo}, infatti
un arco pu� collegare due o pi� nodi.

\begin{define}[Link Graph]
Un link graph
\begin{center}
$F=(V_F,E_F,ctrl_F,link_F):X \to Y$
\end{center}
� una quadrupla avente un' interfaccia interna $X$ ed una interfaccia esterna $Y$, chiamate rispettivamente gli \emph{inner names} e \emph{outer names} del link graph.
F ha un insieme finito $V_F$ di nodi e $E_F$ di archi (\emph{edges}), una \emph{control map} $ctrl:V_F \to K$, e una \emph{link map}:
\begin{center}
$link_F: X \uplus P_F \to E_F \uplus Y$
\end{center}
dove $P_F=\left \{(v,i) \mid v \in V_F \land i \in ar(ctrl_F(v)) \right \}$ � l'insieme delle \emph{porte} di F. Quindi, $(v,i)$ � l' i-esima porta del nodo v. Chiamiamo $X \uplus P_F$ i
\emph{punti} di F, mentre $E_F \uplus Y$ i suoi \emph{link}.
\end{define}

Notiamo subito come alcune nozioni rimangono invariate dal place graph. Per esempio, $V_F$ e $ctrl_F$ non cambiano. Si aggiungono per� due concetti:
\begin{itemize}
	\item
	$E_F$: � l'insieme di archi del link graph F. Sono oggetti del tutto indipendenti dai nodi.
	\item
	$link_F$: � la funzione che consente di creare l'\emph{ipergrafo}. 
\end{itemize}

Dalla definizione \ref{def:sign}, sappiamo che un dato controllo ha un ben preciso numero di porte. La funzione $ar:K \to \mathbb{N}$ consente di calcolare
il numero di porte di un dato controllo: $ar(ctrl_F(v)$ restituisce un numero naturale, che � il  numero di porte del controllo del nodo v. 
Grazie alla funzione \emph{ar}, nel link graph ci sleghiamo totalmente dal concetto di \emph{nodo}, e siamo in grado di sostituirlo con il concetto di \emph{porta}.\\

Introduciamo ora due concetti:
\begin{itemize}
	\item
	l'insieme dei \emph{punti} di F � l'insieme che comprende tutte le sue porte ($P_F$) e tutti i suoi inner names (X).
	\item
	l'insieme dei \emph{link} di F � l'insieme che comprende tutti i suoi archi (edges) e tutti i suoi outer names (Y).
\end{itemize}

La funzione $link_F$ avr� come dominio l'insieme dei punti e come codominio l'insieme dei link. Possiamo quindi pensarla come � raffigurata in figura \ref{fig:linkDecomp}.b.
La funzione $link_F$ � perci� quella che crea il vero e proprio ipergrafo: associa ad ogni punto \emph{uno ed un solo} link. Sfruttando la \emph{non-iniettivit�} della
funzione possiamo creare l'ipergrafo.\\

\begin{figure}[h]
 \centering
 \subfigure[Notazione grafica]
{
	\begin{tikzpicture}
	%\draw[help lines] (0,0) grid (5,5);
	%nodes
	\draw[thick] (1.0,1.0) circle [radius=0.2];%v0
	\node[left] at (0.8,1.0) {$v_0$};
	\draw[thick] (3.0,1.0) circle [radius=0.2];%v2
	\node[above right] at (3.0,1.0) {$v_2$};
	%links
	\draw[myGreen, thick] (0.0,0.0) arc (190:-10:2.0);
	\node [below] at (0.0,0.0) {x};
	\node [below] at (4.0,0.0) {y};
	\draw [fill] (1.0,1.2) circle [radius=0.05];
	\node [above right] at (1.0,1.2) {$p_0$};
	\draw [myGreen, thick] (1.0,1.2) to [out=90,in=200] (1.5,2.3);
	\node at (2.0,2.0) {$e_1$};
	
	\end{tikzpicture}
}
\hspace{5mm}
\subfigure[Notazione funzionale]
{
  	\begin{tikzpicture}
	%Points
	\draw [fill] (1.0,4.0) circle [radius=0.05];
	\node [left] at (1.0,4.0) {$P_0$};
	\draw [fill] (1.0,2.0) circle [radius=0.05];
	\node [left] at (1.0,2.0) {x};
	\draw [fill] (1.0,0.0) circle [radius=0.05];
	\node [left] at (1.0,0.0) {y};
	%Links
	\draw [fill] (4.0,2.0) circle [radius=0.05];
	\node [right] at (4.0,2.0) {$e_1$};
	%Points-Links
	\draw [thick] (1.0,4.0) to [out=0,in=180] (4.0,2.0);
	\draw [thick] (1.0,2.0) to [out=0,in=180] (4.0,2.0);
	\draw [thick] (1.0,0.0) to [out=0,in=180] (4.0,2.0);
	\end{tikzpicture}
	
}
 \caption{Link Graph decomposto.\label{fig:linkDecomp}}
 \end{figure}

In figura \ref{fig:linkDecomp}.a si vede il link graph nella sua notazione usuale, cio� con anche i nodi disegnati. Invece, in \ref{fig:linkDecomp}.b si vede la notazione sotto
forma di funzione, dove si specifica dominio ($p_0$, $x$ e $y$) e codominio ($e_1$).




%Bigraph
\subsection{Bigrafo}
Un bigrafo � semplicemente l'unione del place graph e del link graph. E' importante notare che esse condividono lo \emph{stesso} insieme di nodi.
\begin{define}[Bigrafo]
Un bigrafo
\begin{center}
$F=(V_F, E_F, ctrl_F, prnt_F, link_F): \left \langle n , X \right \rangle \to \left \langle m , Y \right \rangle $
\end{center}
consiste in un place graph $F^P = (V_F, ctrl_f, prnt_F): n \to m$ e in un link graph $F^L=(V_F, E_F, ctrl_F, link_F):X \to Y$.
\end{define}

\begin{notaz}[Bigrafo]
Un bigrafo F viene spesso indicato tramite le sue interfacce:
\begin{center}
$F = \langle n, X \rangle \to \langle m, Y \rangle$
\end{center}
stando ad indicare che F ha $n$ siti, $X$ inner names, $m$ radici e $Y$ outer names.
\end{notaz}


\begin{figure}[h]
\centering
\begin{tikzpicture}
%nodes
%\draw[help lines] (0,0) grid (8,5);
\draw[thick] (1.5,1.5) ellipse (0.5 and 0.5);%v1
\node[left] at (1.0,1.5) {$v_1$};
\draw[thick] (3.7,1.7) ellipse (0.3 and 0.3);%v3
\node[below right] at (3.5,1.5) {$v_3$};
\draw[thick] (6.5,3.0) ellipse (1.5 and 1.0);%v4
\node[above right] at (6.5,4.0) {$v_4$};
\draw[thick] (6.0,3.0) ellipse (0.5 and 0.5);%v5
\node[above right] at (6.5,3.0) {$v_5$};
%roots
\draw [rounded corners=5mm,dotted, thick] (0.0,0.0) rectangle (2.5,5.0);
\node [below right] at (0.0,5.0) {0};
\draw [rounded corners=5mm,dotted, thick] (2.7,0.0) rectangle (4.5,5.0);
\node [below right] at (2.7,5.0) {1};
\draw [rounded corners=5mm,dotted, thick] (4.7,0.0) rectangle (8.0,5.0);
\node [below right] at (4.7,5.0) {2};

	
%links
\draw [myGreen, thick] (1.5,1.0) to [out=320,in=230] (3.5,1.5);
\draw [fill] (1.5,1.0) circle [radius=0.05];
\draw [fill] (3.5,1.5) circle [radius=0.05];
\draw [myGreen, thick] (4.0,1.7) to [out=320,in=270] (5.6,2.7);
\draw [fill] (4.0,1.7) circle [radius=0.05];
\draw [fill] (5.6,2.7) circle [radius=0.05];
\draw [myGreen, thick] (5.0,3.0) to [out=270,in=180] (5.3,2.0);
\draw [fill] (5.0,3.0) circle [radius=0.05];

\draw [myGreen, thick] (1.5,2.0) to [out=90,in=230] (1.5,5.0);
\draw [fill] (1.5,5.0) circle [radius=0.05];
\draw [fill] (1.5,2.0) circle [radius=0.05];
\node [above] at (1.5,5.0) {x};
\draw [myGreen, thick] (6.0,4.0) to [out=90,in=230] (6.0,5.0);
\draw [fill] (6.0,4.0) circle [radius=0.05];
\draw [fill] (6.0,5.0) circle [radius=0.05];
\node [above] at (6.0,5.0) {y};


\end{tikzpicture}
\caption{Esempio di bigrafo.\label{fig:innerOuterBig}}
\end{figure}

Il bigrafo della figura \ref{fig:innerOuterBig}, rispettando la notazione, � scritto in questo modo: $F = \langle 0, \emptyset  \rangle \to \langle 3, \{x,y\} \rangle$



%%%%%%%%%%%%%%%%%%%%%%%%%%%%%%%%%%%%%%%%%%%%%%%%%%%%%%%%%%%%%%%%%%%%%%%%%%%%
%Operazioni sui bigrafi
\section{Operazioni sui Bigrafi}\label{sec:operations}
Nelle sezioni precedenti, si � definita formalmente la nozione di bigrafo. Ora, si studiano le varie operazioni possibili, ovvero come ottenere un nuovo bigrafo da
due bigrafi di partenza. Queste operazioni saranno utili per la prossima sezione, che tratter� dell' algebra dei bigrafi.

\subsection{Traduzione di Supporto}
La prima operazione tratta di come si pu� ottenere un nuovo bigrafo avendo a disposizione un solo bigrafo base e una funzione biettiva. Si dimostra che applicando
tale biiezione ai nodi e archi del bigrafo di partenza si determina \emph{unicamente} il bigrafo risultante.

\begin{define}[Traduzione di Supporto]
Per ogni place graph, link graph e per ogni bigrafo � assegnato un insieme finito $|F|$, il suo \emph{supporto}. Per un place graph, definiamo $|F| = V_F$, mentre per
un link graph o un bigrafo definiamo $|F| = V_F \uplus E_F$.
Per due bigrafi F e G, una \emph{traduzione di supporto}  $p: |F| \to |G|$ da F a G consiste in due biiezioni $p_V: V_F \to V_G$ and $p_E: E_F \to E_G$ che rispettano la
struttura, nel seguente senso:
\begin{itemize}
	\item
	$p$ preserva i controlli, cio�: $ctrl_G \circ p_V = ctrl_F$. Ne segue che $p$ induce una biiezione sulle porte $p_P: P_F \to P_G$, definita da 
	$p_P((v, i)) = (p_V(v), i)$.
	\item
	$p$ modifica le mappe sulla struttura ($prnt$ e $link$) in questo modo:
	\begin{center}
	$prnt_G \circ (id_m \uplus p_V) = (id_n \uplus p_V) \circ prnt_F$ \\
	$link_G \circ (id_X \uplus p_P) = (id_Y \uplus p_E) \circ link_F$
	\end{center}
\end{itemize}
\end{define}

Come detto prima, data la biiezione $p$ e il bigrafo $F$, queste condizioni determinano unicamente G, che denotiamo con $p \cdot F$ e chiamiamo \emph{traduzione
di supporto di F tramite p}. Chiamiamo $F$ e $G$ \emph{support equivalent} ($F \bumpeq G$) se esiste una tale traduzione di supporto.\\

Dati due bigrafi, il problema di trovare una traduzione di supporto tra i due � equivalente al problema di stabilire quando essi sono uguali. Per cui, da qui in seguito 
si user� il termine \emph{traduzione di supporto} come sinonimo di \emph{isomorfismo}.\\

\begin{define}[Isomorfismo]\label{def:iso}
Due bigrafi F e G si dicono isomorfi se e solo se esiste una traduzione di supporto tra F e G, cio� se e solo se F e G sono \emph{support equivalent} ($F \bumpeq G$).
\end{define}





\subsection{Composizione}
L'operazione di \emph{composizione} � denotata dal simbolo $\circ$ e permette di scrivere equazioni del tipo $B = A \circ C$, come nell'esempio
\ref{sub:esempioInit}. La composizione necessita del concetto di interfaccia: come gi� notato, l'interfaccia interna di A deve concordare con l'interfaccia esterna di C.
Con una prima approssimazione, possiamo dire che C deve essere incluso \emph{dentro} A. 

\begin{define}[Composizione]
Si trattano separatamente i casi del place graph e del link graph:
\begin{itemize}
	\item
	\emph{Place Graph}: Se $F: k \to m$ e $G: m \to n$ sono due place graph con supporti disgiunti ( $|F|  \#  |G| $), la loro composizione 
		\begin{center}
		$G \circ F = (V, ctrl, prnt): k \to n$
		\end{center}
	ha i nodi $V = V_F \uplus V_G$ e la control map $ctrl = ctrl_F \uplus ctrl_G$. La sua parent map $prnt$ � definita come segue: se $w \in k \uplus V_F \uplus V_G$
	� un sito o un nodo di $G \circ F$, allora :
		\begin{center}
		$prnt(w) \stackrel{def}{=}
			\begin{cases} 
			prnt_F(w), & \mbox{se }w \in k \uplus V_F \ \land \ prnt_F(w) \in V_F \\ 
			prnt_G(j), & \mbox{se }w \in k \uplus V_F \ \land \ prnt_F(w) = j \in m \\ 
			prnt_G(w), & \mbox{se }w \in V_G 
			\end{cases}
		$
		\end{center}
		Il place graph \emph{identit�} su $m$ � $id_m \stackrel{def}{=} (\emptyset, \emptyset, id_m): m \to m$
		
	Si noti come l'unione delle due interfacce (quella esterna di A e quella interna di B) sia modellata dalla seconda riga. In particolare, $prnt_F(w) = j$ sta ad indicare che
	$j$ � una radice di $F$ e $w$ � uno dei suoi figli. Dato che $j$ � una radice, essa sar� un intero nell'insieme $\{0, \dots, m-1\}$, e dovr� quindi appartenere anche 
	all'interfaccia interna di $G$, dove $j$ sar� un sito. La parent map di $G \circ F$su $w$ sar� quella di G sul suo sito $j$.
	Quindi la seconda riga modella la seguente azione: mantengo la parent map per tutti i nodi interni dei due bigrafi, e
	al momento dell'unione delle due interfacce unisco la radice i-esima di F con il sito i-esimo di G, per ogni $i \in \{0, \dots, m-1\}$.
	\item
	\emph{Link Graph}: Se $F: X \to Y$ e $G: Y \to Z$ sono due link graph con supporti disgiunti ( $|F|  \#  |G| $), la loro composizione
		\begin{center}
		$G \circ F = (V, E, ctrl, link): X \to Z$
		\end{center}
	ha $V = V_F \uplus V_G$, $E = E_F \uplus E_G$, $ctrl = ctrl_F \uplus ctrl_G$ e la sua link map $link$ � definita come segue: se $q \in X \uplus P_F \uplus P_G$
	� un punto di $G \circ F$, allora
		\begin{center}
		$link(q) \stackrel{def}{=}
				\begin{cases} 
				link_F(q), & \mbox{se }q \in X \uplus P_F \ \land \ link_F(q) \in E_F \\ 
				link_G(y), & \mbox{se }q \in X \uplus P_F \ \land \ link_F(q) = y \in Y \\ 
				link_G(q), & \mbox{se }q \in P_G 
				\end{cases}
		$
		\end{center}
	Il link graph \emph{identit�} su $X$ � $id_X \stackrel{def}{=} (\emptyset, \emptyset, \emptyset, id_X): X \to X$
	Anche qui, si noti come la seconda riga modelli l'unione tra le interfacce. In particolare, se $link_F(q) = y$ e $y \in Y$, si ha che il punto $q$ del bigrafo F � collegato
	al suo outer name y. Dato che $Y$ � anche l'interfaccia interna di G, si ha che y � un inner name di G. La seconda riga quindi rappresenta l'unione tra gli outername
	di F e gli inner name di G che hanno lo stesso nome.
	\item
	\emph{Bigrafo}: Se $F: I \to J$ e $G: J \to K$ sono due bigrafi con supporti disgiunti ( $|F|  \#  |G| $), la loro composizione
		\begin{center}
		$G \circ F \stackrel{def}{=} \langle G_P \circ F_P, G_L \circ F_L \rangle : I \to K$
		\end{center}
	ed il bigrafo \emph{identit�} su $I=\langle m, X \rangle$ � $\langle id_m, id_x \rangle$. 
\end{itemize}
\end{define}


Si noti come l'operazione di composizione \emph{unisca} le interfacce dei due bigrafi, creando un unico bigrafo risultato, cio� un' unica funzione $prnt$ ed un' unica 
funzione $link$. \\

\subsection{Giustapposizione}
Si definisce ora un' altra operazione per creare un bigrafo da altri due base. Si chiama \emph{giustapposizione} e consiste nell'affiancare un bigrafo ad un altro. Questa 
operazione � possibile solo se i due bigrafi sono \emph{disgiunti}. Spesso viene anche chiamata \emph{prodotto}.

\begin{define}[Bigrafi disgiunti]
Due place graphs $F_i (i= 0, 1)$ sono disgiunti se $|F_0| \# |F_1|$. Due link graph $F_i: X_i \to Y_i$ sono \emph{disgiunti} se $X_0 \# X_1$, $Y_0 \# Y_1$ e $|F_0| \# |F_1|$.
Due bigrafi $F_i (i=0, 1)$ sono \emph{disgiunti} se $F_0^P \# F_1^P$ e $F_1^L \# F_1^L$.
\end{define}

L'operazione di giustapposizione � monoidale, cio� � associativa ed ha un unit�. Si definiranno quindi le propriet� dell'operazione e le sue unit�.

\begin{define}[Giustapposizione]
Definiamo separatamente i casi del place graph e del link graph:
\begin{itemize}
	\item
	\emph{Place Graph}: la giustapposizione di due interfacce $m_i (i = 0, 1)$ � $m_0 + m_1$ e l'unit� � 0. Se $F_i = (V_i, ctrl_i, prnt_i): m_i \to n_i$ sono place graphs
	disgiunti (i= 0, 1), la loro giustapposizione $F_0 \otimes F_1: m_0 + m_1 \to n_0 + n_1$ � data da: 
		\begin{center}
		$F_0 \otimes F_1 = (V_0 \uplus V_1, ctrl_0 \uplus ctrl_1, prnt_0 \uplus prnt_1^{'})$,
		\end{center}
	dove $prnt_1^{'}(m_0 + i) = n_0 + j$  ogni volta che  $prnt_1(i) = j$. \\
	Informalmente, affianco due place graph avendo come risultato una foresta. Le radici aumenteranno quindi di numero. Per questo motivo � necessaria la funzione
	$prnt_1^{'}$: supponiamo che $F_0$ abbia due radici e $F_1$ una sola.
	I nodi figli diretti di $F_1$ non possono pi� puntare alle loro vecchie radici, per esempio la radice 0, perch� ora essa � la radice di $F_0$. La parent map di $F_1$ dovr�
	quindi venire traslata in questo modo: tutti i nodi di $F_1$ che puntano alla sua radice 0, ora punteranno alla nuova radice $0+2 = 2$. Questo � quello che fa la funzione
	$prnt_1^{'}$.
	\item
	\emph{Link Graph}: la giustapposizione di due interfaccie disgiunte di due link graph � $X_0 \uplus X_1$ e l'unit� � $\emptyset$. Se 
	$F_i = (V_i, E_i, ctrl_i, link_i): X_i \to Y_i$ sono due link graph disgiunti ($i= 0, 1$), la loro giustapposizione $F_0 \otimes F_1: X_0 \uplus X_1 \to Y_0 \uplus Y_1$
	� data da
	\begin{center}
	$F_0 \otimes F_1 = (V_0 \uplus V_1, E_0 \uplus E_1, ctrl_0 \uplus ctrl_1, link_0 \uplus link_1)$
	\end{center}
	
	\item
	\emph{Bigrafi}: la giustapposizione di due interfacce disgiunte $I_i = \langle m_i, X_i \rangle (i= 0, 1)$ � $\langle m_0+m_1, X_0 \uplus X_1 \rangle$ e l'unit�
	� $\varepsilon = \langle 0, \emptyset \rangle$. Se $F_i: I_i \to J_i$ sono bigrafi disgiunti ($i= 0, 1$), la loro giustapposizione 
	$F_0 \otimes F_1: I_0 \otimes I_1 \to J_0 \uplus J_1$ � data da:
	\begin{center}
	$F_0 \otimes F_1 = \langle F_0^P \otimes F_1^P, F_0^L \otimes F_1^L \rangle$
	\end{center}
\end{itemize}
\end{define}

Tutte queste tre operazioni saranno utili per definire l'algebra dei bigrafi della prossima sezione.




%%%%%%%%%%%%%%%%%%%%%%%%%%%%%%%%%%%%%%%%%%%%%%%%%%%%%%%%%%%%%%%%%%%%%%%%%%%%%%%%%
%Algebra
\section{L' algebra dei bigrafi}\label{sec:algebra}
In questa sezione verr� illustrato come poter ottenere nuovi bigrafi da bigrafi base tramite le operazioni di composizione, identit� e prodotto (giustapposizione). Si definisce
quindi una vera e propria \emph{algebra} per la teoria dei bigrafi, in cui si dimostra che ogni bigrafo pu� essere derivato da alcuni bigrafi base. Si incomincia nelle prime
tre sezioni ad illustrare i bigrafi base su cui si appogger� l'algebra, ovvero placing, linking e ioni.Infine, si enuncia un importante risultato che � la decomponibilit� di
ogni bigrafo in una sua forma normale.\\
Si rimanda all' appendice \ref{ch:appA} per i dettagli sulla notazione usata.

\subsection{Placing elementari}
\begin{define}[Placing, Permutazioni, Merge]
Un bigrafo senza nodi e senza link viene detto \emph{placing} ($\varnothing$). Un placing che � biiettivo dai siti alle radici � detto \emph{permutazione} ($\pi$). Un placing
sono una radice e $n$ siti � denotata da $merge_n$.
\end{define}


\begin{figure}[h]
 \centering
 \subfigure[$\gamma_{1,1}: 2 \to 2$]
{
	\begin{tikzpicture}
	%\draw[help lines] (0,0) grid (3,3);
	%Roots
	\draw[rounded corners=2mm,dotted, thick] (0.0,0.0) rectangle (1.4,2.0);
	\node[below right] at (0.0,2.0) {0};
	\draw[rounded corners=2mm,dotted, thick] (1.6,0.0) rectangle (3.0,2.0);
	\node[below right] at (1.6,2.0) {1};
	%Sites
	\draw[rounded corners=1mm,dotted, thick, fill=myGrey] (0.2,0.5) rectangle (1.0,1.2);
	\node[below right] at (0.2,1.2) {1};
	\draw[rounded corners=1mm,dotted, thick, fill=myGrey] (2.0,0.5) rectangle (2.8,1.2);
	\node[below right] at (2.0,1.2) {0};
	\end{tikzpicture}
}
\hspace{5mm}
\subfigure[$1: 0 \to 1$]
{
  	\begin{tikzpicture}
	%\draw[help lines] (0,0) grid (3,3);
	%Roots
	\draw[rounded corners=2mm,dotted, thick] (0.5,0.0) rectangle (2.5,2.0);
	\node[below right] at (0.5,2.0) {0};
	\end{tikzpicture}
	
}
\hspace{5mm}
\subfigure[$join: 2 \to 1$]
{
  	\begin{tikzpicture}
	%\draw[help lines] (0,0) grid (3,3);
	%Roots
	\draw[rounded corners=2mm,dotted, thick] (0.0,0.0) rectangle (3.0,2.0);
	\node[below right] at (0.0,2.0) {0};
	%Sites
	\draw[rounded corners=1mm,dotted, thick, fill=myGrey] (0.5,0.5) rectangle (1.4,1.5);
	\node[below right] at (0.5,1.5) {0};
	\draw[rounded corners=1mm,dotted, thick, fill=myGrey] (1.6,0.5) rectangle (2.4,1.5);
	\node[below right] at (1.6,1.5) {1};
	\end{tikzpicture}
	
}
 \caption{Placing elementari.\label{fig:placingElementari}}
 \end{figure}

Un placing ha quindi solo siti e radici.\\
Il risultato importante � che si pu� ottenere qualsiasi \emph{placing} dai tre \emph{placing elementari} di figura \ref{fig:placingElementari}, tramite le operazioni di 
composizione, identit� e prodotto. In particolare:
\begin{itemize}
	\item
	ogni permutazione $\pi$ pu� essere ottenuta dalla simmetria elementare $\gamma_{1,1}$. Per esempio, se si vuole ottenere la permutazione di ordine 3 ($\pi_3$),
	allora si pu� scrivere la seguente equazione: $\pi_3 = (\gamma_{1,1} \circ \gamma_{1,1}) \otimes id_1$.
	\item
	ogni placing $\varnothing$ (e quindi anche ogni merge) pu� essere ottenuto dai tre placing elementari $\gamma_{1,1}, 1$ e $join$. Per esempio: $merge_0 = 1$ e 
	$merge_{n+1} = join \circ (id_1 \otimes merge_n)$.
\end{itemize}



 
\subsection{Linking elementari}
\begin{define}[Linking, Sostituzioni, Chiusure]
Un bigrafo senza nodi e senza places � detto \emph{linking} ($\delta$). Una \emph{sostituzione} ($\sigma$) � il prodotto tra sostituzioni elementari (\ref{}).
Una sostituzione biiettiva � detta \emph{rinomina} ($\alpha$).
Una \emph{chiusura} � il prodotto tra chiusure elementari(\ref{}).
\end{define}


\begin{figure}[h]
 \centering
 \subfigure[$^y / X: X \to y$]
{
	\begin{tikzpicture}
	%\draw[help lines] (0,0) grid (5,4);
	%Inners
	\draw [fill] (1.0,1.0) circle [radius=0.05];
	\node [below left] at (1.0,1.0) {$x_0$};
	\draw [fill] (2.0,1.0) circle [radius=0.05];
	\draw [fill] (3.0,1.0) circle [radius=0.05];
	\draw [fill] (4.0,1.0) circle [radius=0.05];
	\node [below right] at (4.0,1.0) {$x_n$};
	\node [below] at (2.5,1.0) {$\dots$};
	%Outers
	\draw [fill] (2.5,4.0) circle [radius=0.05];
	\node [above right] at (2.5,4.0) {$y$};
	\draw [myGreen, thick] (2.5,4.0) to [out=250,in=90] (1.0,1.0);% y - x1
	\draw [myGreen, thick] (2.5,4.0) to [out=260,in=90] (2.0,1.0);% y - x2
	\draw [myGreen, thick] (2.5,4.0) to [out=270,in=90] (3.0,1.0);% y - x3
	\draw [myGreen, thick] (2.5,4.0) to [out=280,in=90] (4.0,1.0);% y - xn
	\end{tikzpicture}
}
\hspace{5mm}
\subfigure[$/x: x \to \varepsilon$]
{
  	\begin{tikzpicture}
	%\draw[help lines] (0,0) grid (5,4);
	%Inners
	\draw [fill] (3.0,1.0) circle [radius=0.05];
	\node [below] at (3.0,1.0) {$x$};
	\draw [myGreen, thick] (3.0,1.0) to [out=90,in=270] (3.0,2.0);
	\draw [myGreen, thick] (2.8,2.0) to [out=0,in=180] (3.2,2.0);
	%Spaces
	\draw [white] (0.0,0.0) to [out=0,in=180] (0.0,0.1);
	\draw [white] (4.8,0.0) to [out=0,in=180] (5.0,0.0);
	\end{tikzpicture}
	
}
 \caption{Linking elementari.\label{fig:linkingElementari}}
 \end{figure}

Un linking ha solo inner names e outer names.\\
Il risultato importante � che ogni \emph{linking} pu� essere generato tramite composizione, prodotto e identit� a partire dai due \emph{linking elementari} di figura \ref{}:
sostituzioni elementari $^y / X$ e chiusure elementari $/x: x \to \varepsilon$.



\subsection{Ioni}
Si descrive un altro tipo di bigrafo base: questa volta esso contiene nodi, a differenza dei placing e dei linking. 

\begin{define}[Ione]
Per ogni controllo $K:n$, il bigrafo $K_{\vec x}: 1 \to \langle 1, \{ \vec x \} \rangle$ avente un singolo nodo di tipo $K$ le cui porte sono collegate biiettivamente con $n$
distinti nomi $\vec x$, � chiamato un \emph{ione discreto}.
\end{define}


\begin{figure}[h]
\centering
\begin{tikzpicture}
%\draw[help lines] (0,0) grid (5,4);
%Roots
\draw[rounded corners=2mm,dotted, thick] (0.0,0.0) rectangle (4.0,3.0);
%Nodes
\draw[thick] (2.0,1.5) ellipse ( 1.5 and 0.8);%K
\node[above right] at (3.0,2.0) {K};
%Sites
\draw[rounded corners=1mm,dotted, thick, fill=myGrey]  (1.5,1.0) rectangle (3.0,2.0);
%Outers
\draw [fill] (1.5,2.25) circle [radius=0.05];
\draw [fill] (2.0,2.3) circle [radius=0.05];
\draw [fill] (2.5,2.25) circle [radius=0.05];
\draw [myGreen, thick] (1.5,2.25) to [out=90,in=270] (1.0,3.5);
\node[above ] at (1.0,3.5) {$x_1$};
\draw [myGreen, thick] (2.0,2.3) to [out=90,in=270] (2.0,3.5);
\node[above ] at (2.0,3.5) {$\dots$};
\node[above right] at (2.0,2.3) {$\dots$};
\draw [myGreen, thick] (2.5,2.25) to [out=90,in=270] (3.0,3.5);
\node[above ] at (3.0,3.5) {$x_n$};

\end{tikzpicture}
\caption{$K_{\vec x}$.\label{fig:ione}}
\end{figure}


Riassumendo, i tre tipi di \emph{bigrafi elementari} sono i placing, i linking e gli ioni. Tramite questi possiamo esprimere algebricamente ogni bigrafo in termini di composizione,
prodotto e identit�. Per esempio: 
\begin{itemize}
	\item
	per formare un atomo (vedi definizione \ref{def:atomo}) usiamo la composizione tra un ione ed un placing (\ref{fig:atomoMolecola}.a).
	\item
	per formare una molecola (vedi definizione \ref{def:molecola}) usiamo anche qui la composizione tra un ione e un bigrafo discreto, ottenuto a sua volta ricorsivamente
	usando solo placing, linking e ioni (\ref{fig:atomoMolecola}.b).
\end{itemize}	

\begin{define}[Bigrafo Discreto]\label{def:bigrafoDiscreto}
Un bigrafo si dice discreto se non ha link chiusi e la sua link map � biiettiva.
\end{define}

\begin{define}[Atomo]\label{def:atomo}
Se il sito di un K-ione viene riempito dal placing $1: 1 \to 0$ (vedi \ref{fig:placingElementari}.b), il risultato � un \emph{atomo discreto} $K_{\vec x} \circ 1$.
\end{define}

\begin{define}[Molecola]\label{def:molecola}
Se il sito di un K-ione viene riempito da un bigrafo discreto $G: I \to \langle 1, Y \rangle$, il risultato � una \emph{molecola discreta} 
$(K_{\vec x} \otimes id_Y) \circ G$.
\end{define}


\begin{figure}[h]
 \centering
 \subfigure[$A_{pq} \circ 1$]
{
	\begin{tikzpicture}
	%\draw[help lines] (0,0) grid (4,4);
	%Roots
	\draw[rounded corners=2mm,dotted, thick] (0.0,0.0) rectangle (3.0,2.5);
	%Nodes
	\draw[thick] (1.5,1.5) ellipse ( 1.0 and 0.6);%A
	\node[above right] at (1.0,1.0) {A};
	%Outers
	\draw [myGreen, thick] (1.0,2.0) to [out=120,in=230] (1.0,3.5);
	\node[above ] at (1.0,3.5) {$p$};
	\draw [myGreen, thick] (2.0,2.0) to [out=60,in=310] (2.0,3.5);
	\node[above ] at (2.0,3.5) {$q$};
	\draw [fill] (1.0,2.0) circle [radius=0.05];
	\draw [fill] (2.0,2.0) circle [radius=0.05];
	
	\end{tikzpicture}
}
\hspace{5mm}
\subfigure[$(K_{xyz} \otimes id_{pq}) \circ L_{pq}$]
{
  	\begin{tikzpicture}
	%\draw[help lines] (0,0) grid (5,4);
	%Roots
	\draw[rounded corners=5mm,dotted, thick] (0.0,0.0) rectangle (5.0,3.0);
	%Nodes
	\draw[thick] (2.5,1.5) ellipse ( 2.0 and 1.3);%K
	\node[above left] at (1.0,2.2) {K};
	\draw[thick] (2.5,1.5) ellipse ( 1.0 and 0.6);%L
	\node[above right] at (1.0,1.0) {L};
	%Sites
	\draw[rounded corners=1mm,dotted, thick, fill=myGrey]  (2.0,1.2) rectangle (3.0,1.8);
	%Outers
	\draw [myGreen, thick] (1.0,2.35) to [out=120,in=230] (1.0,3.5);
	\draw [myGreen, thick] (2.0,2.75) to [out=100,in=250] (2.0,3.5);
	\draw [myGreen, thick] (4.0,2.35) to [out=60,in=310] (4.0,3.5);
	\draw [myGreen, thick] (2.0,2.0) to [out=120,in=240] (1.5,3.5);
	\draw [myGreen, thick] (3.0,2.0) to [out=60,in=300] (2.5,3.5);
	\draw [fill] (1.0,2.35) circle [radius=0.05];
	\node[above] at (1.0,3.5) {x};
	\draw [fill] (2.0,2.75) circle [radius=0.05];
	\node[above] at (2.0,3.5) {y};
	\draw [fill] (4.0,2.35) circle [radius=0.05];
	\node[above] at (4.0,3.5) {z};
	\draw [fill] (2.0,2.0) circle [radius=0.05];
	\node[above] at (1.5,3.5) {p};
	\draw [fill] (3.0,2.0) circle [radius=0.05];
	\node[above] at (2.5,3.5) {q};
	\end{tikzpicture}
}
 \caption{Atomo e molecola discreti.\label{fig:atomoMolecola}}
 \end{figure}

\begin{teor}
Ogni bigrafo pu� essere costruito a partire da placing elementari, linking elementari e ioni tramite le tre operazioni di composizione, identit� e giustapposizione.
\end{teor}


\subsection{Forma normale discreta}
Si � ora arrivati al risultato principale: la decomponibilit� di ogni bigrafo in pi� bigrafi base. Per descrivere questo enunciato, sono necessarie le definizioni di 
\emph{bigrafo discreto} (definizione \ref{def:bigrafoDiscreto}) e di \emph{bigrafo primo}.

\begin{define}[Bigrafo primo]
Un bigrafo si dice \emph{primo} se non ha inner names ed ha un' interfaccia esterna unaria. Esso prende la segue forma: $m \to \langle X \rangle$.
\end{define}

Un importante esempio di bigrafo primo � $merge_n: n \to 1$, dove $n \ge 0$. L'assenza di inner names nei bigrafi primi � fondamentale: essa assicura che ci sia una
ed una sola decomposizione di ogni bigrafo in linking e primi discreti, come segue:

\begin{prop}[Forma normale discreta]\label{prop:fnd}
Ogni bigrafo $G: \langle m, X \rangle \to \langle n, Z \rangle$ pu� essere espresso univocamente, con al pi� una rinomina su Y, come:
	\begin{center}
	$G = (id_n \otimes \lambda) \circ D$ 
	\end{center}
dove $\lambda: Y \to Z$ � un linking e $D:\langle m, X \rangle \to \langle n, Y \rangle$ � un bigrafo discreto. Inoltre, ogni bigrafo discreto D pu� essere fattorizzato 
univocamente, con al pi� una permutazione dei siti di ogni fattore, come
	\begin{center}
	$D = \alpha \otimes ((P_0 \otimes \dots \otimes P_{n-1}) \circ \pi)$
	\end{center}
dove $\alpha$ � una rinomina, ogni $P_i$ � primo e discreto, e $\pi$ � una permutazione di tutti i siti.
\end{prop}

Questa proposizione � cruciale per dimostrare la completezza della teoria algebrica dei bigrafi, che risulta quindi sia corretta che completa. 



%%%%%%%%%%%%%%%%%%%%%%%%%%%%%%%%%%%%%%%%%%%%%%%%%%%%%%%%%%%%%%%%%%%%%

%\section{Operazioni Derivate}
%Il fatto che ogni bigrafo si possa ottenere a partire da altri bigrafi base secondo l' equazione \ref{prop:fnd} spesso non aiuta il progettista a disegnare il sistema
%distribuito voluto. Il problema � che la scomposizione in parti non � il modo migliore per costruire un bigrafo. In questa sezione si introducono quindi alcune 
%operazioni derivate, con lo scopo di semplificare il lavoro del progettista nella costruzione del bigrafo.
%
%La prima operazione che si prende in esame � il \emph{prodotto parallelo} $\rVert$, che generalizza l' operazione di prodotto $\otimes$.
%\begin{define}[Prodotto parallelo]
%Il prodotto parallelo sulle interfacce � dato da:
%\begin{center}
%$\langle m, X \rangle \rVert \langle n, Y \rangle = \langle m + n, X \cup Y \rangle$
%\end{center}
%Siano ora $G_i: I_i \to J_i (i=0, 1)$ due bigrafi con supporti disgiunti. 
%\end{define}




%%%%%%%%%%%%%%%%%%%%%%%%%%%%%%%%%%%%%%%%%%%%%%%%%%%%%%%%%%%%%%%%%%%%%
\section{Bigraphical Reactive Systems}
Uno degli aspetti fondamentali di un bigrafo � la sua capacit� di evolversi sulla base di regole ben precise. Un BRS (Bigraphical Reactive System) consente di fare
evolvere lo stato di un sistema (bigrafo iniziale) e di dedurre nuove informazioni con specifiche regole di reazione. Un BRS costituisce quindi la \emph{dinamica} dei
bigrafi.\\

La definizione di un BRS richiede prima la definizione di altri concetti, come quello delle \emph{occorrenza} e di \emph{regola di reazione parametrica}.

\begin{define}[Occorrenza]\label{def:occorrenza}
Un bigrafo F occorre in un bigrafo G se l'equazione $G = C_1 \circ (F \otimes id_I) \circ C_0$ esiste per qualche interfaccia I e per qualche bigrafo $C_0$ e $C_1$.
\end{define}

L' identit� $id_I$ � importante: consente a $C_1$ di avere figli in $C_0$ e in $F$, e consente a $C_0$ e $C_1$ di condividere link che non riguardano $F$.

Il secondo concetto importante � quello di \emph{regola di reazione}. 

\begin{define}[Regola di reazione]
Una regola di reazione � della forma
	\begin{center}
	$R \to R'$
	\end{center}
dove $R$ si dice essere il \emph{redex} della regola, cio� il pattern da trovare e cambiare nel bigrafo a cui si applica la regola, e $R'$ � il \emph{reactum}, cio� il pattern
che andr� a sostituire il redex.
\end{define}

Redex e reactum sono entrambi bigrafi, e possono occorrere (secondo la definizione \ref{def:occorrenza}) nel bigrafo a cui si applica la regola di reazione. Le regole di 
reazione possono coinvolgere sia il link graph sia il place graph, come si pu� vedere da questo esempio.


\subsection{Esempio}\label{sub:esempioBuilding}
Nella figura \ref{fig:buildings} , � rappresentato un bigrafo che indica lo stato di un ambiente. Esso � formato da due costruzioni B (buildings), da quattro stanze R (rooms), da
cinque agenti A e da quattro computer C. Il bigrafo ha le seguente segnatura: $K = \{A:2, B:1, C:2, R:0\}$.

Il bigrafo rappresenta lo stato dell' ambiente, in cui i cinque agenti A stanno tenendo una video conferenza: gli agenti che stanno partecipando sono collegati dal link $e_0$. 
Ogni agente che vuole partecipare deve essere collegato ad un computer. A sua volta i vari computer di una stessa costruzione sono collegati insieme, formando una LAN.\\


\begin{figure}[h]
\centering
\begin{tikzpicture}
%\draw[help lines] (0,0) grid (12,6);
%Buildings
\draw[thick] (3.0,2.5) ellipse ( 2.5 and 2.0);%B1
\node[above left] at (5.3,4.0) {$B_1$};
\draw[thick] (9.0,2.5) ellipse ( 2.5 and 2.0);%B2
\node[above right] at (11.0,4.0) {$B_2$};
%Rooms
\draw[thick] (2.0,2.5) ellipse ( 0.8 and 1.5);%R1
\draw[thick] (4.5,2.5) ellipse ( 0.8 and 1.2);%R2
\draw[thick] (8.5,2.0) ellipse (  0.8 and 1.2);%R3
\draw[thick] (10.4,2.5) ellipse ( 0.8 and 1.5);%R4
%Agents
\draw[very thick] (1.7,3.0) -- (2.3,3.0) -- (2.0,3.8) -- (1.7,3.0);%A1
\draw[very thick] (3.0,3.0) -- (3.5,3.0) -- (3.25,4.0) -- (3.0,3.0);%A2 
\draw[very thick] (4.2,2.5) -- (4.8,2.5) -- (4.5,3.5) -- (4.2,2.5);%A3 
\draw[very thick] (7.0,2.5) -- (7.5,2.5) -- (7.25,3.5) -- (7.0,2.5);%A4 
\draw[very thick] (10.0,2.5) -- (10.5,2.5) -- (10.25,3.5) -- (10.0,2.5);%A5 
%Computers
\draw[very thick] (1.8,2.0) rectangle (2.2,2.2);%C1
\draw[very thick] (4.3,1.8) rectangle (4.8,2.0);%C2
\draw[very thick] (8.3,1.5) rectangle (8.8,1.8);%C3
\draw[very thick] (10.0,1.5) rectangle (10.5,1.8);%C4
%Edges
\node[below] at (4.0,5.0) {$e_0$};
\draw [myGreen, thick] (2.0,3.8) to [out=90,in=180] (3.0,5.0) -- (9.0,5.0) to [out=0,in=90] (10.25,3.5);
\draw [myGreen, thick] (3.25,4.0) to [out=90,in=180] (4.0,5.0);
\draw [myGreen, thick] (4.5,3.5) to [out=90,in=180] (5.5,5.0);
\draw [myGreen, thick] (7.25,3.5) to [out=90,in=0] (6.5,5.0);
%Edges Agent Computer
\draw [myGreen, thick] (2.0,3.0) -- (2.0,2.2);%A1-C1
\draw [myGreen, thick] (3.25,3.0) -- (3.25,2.7);%A2 -
\draw [myGreen, thick] (4.5,2.5) -- (4.5,2.0);%A3-C2
\draw [myGreen, thick] (7.25,2.5) -- (7.25,2.25);%A4-
\draw [myGreen, thick] (8.5,1.8) -- (8.5,2.1);%C3-
\draw [myGreen, thick] (10.25,2.5) -- (10.25,1.8);%A5-C4
%Edges Computer Building
\draw [myGreen, thick] (2.0,2.0) to [out=270,in=90] (3.0,0.5);%C1
\draw [myGreen, thick] (4.5,1.8) to [out=270,in=90] (3.0,0.5);%C2
\draw [myGreen, thick] (8.5,1.5) to [out=270,in=90] (9.0,0.5);%C3
\draw [myGreen, thick] (10.25,1.5) to [out=270,in=90] (9.0,0.5);%C4

\end{tikzpicture}
\caption{Esempio.\label{fig:buildings}}
\end{figure}


Alcune regole di reazione possono essere quelle delle figure \ref{fig:buildingsReactions1}, \ref{fig:buildingsReactions2}, \ref{fig:buildingsReactions3}. Esse agiscono sul bigrafo di partenza (figura \ref{fig:buildings}) e sono cos� definite:

\begin{itemize}
	\item
	La prima � la pi� semplice: un agente pu� lasciare la video chiamata. Il redex � la parte a sinistra della regola e pu� matchare qualsiasi agent. I due outername del redex
	indicano che le porte dell' agent devono essere collegate (prima dello scatto della regola) a zero o pi� porte. Se l'agent � collegato ad altri in una videochiamata, egli
	verr� disconnesso, mantenendo per� attivo il collegamento al computer: questa informazione � espressa dal reactum, la parte destra della regola.
	
	\begin{figure}[!htbp]
	\centering
	\begin{tikzpicture}
	%\draw[help lines] (0,0) grid (8,4);
	%Redex
	\draw[rounded corners=4mm,dotted, thick] (0.0,0.0) rectangle (3.0,4.0);
	\draw[very thick] (1.0,1.0) -- (2.0,1.0) -- (1.5,3.0) -- (1.0,1.0);
	\draw [myGreen, thick] (1.5,1.0) to [out=270, in=180] (2.0,0.5) to [out=0, in=270] (2.5,4.0);
	\draw [myGreen, thick] (1.5,3.0) to [out=90, in=270] (1.5,4.0);
		
	\draw[->, very thick, red] (3.5,2.0) -- (4.5,2.0);

	%Reactum
	\draw[rounded corners=4mm,dotted, thick] (5.0,0.0) rectangle (8.0,4.0);
	\draw[very thick] (6.0,1.0) -- (7.0,1.0) -- (6.5,3.0) -- (6.0,1.0);
	\draw [myGreen, thick] (6.5,1.0) to [out=270, in=180] (7.0,0.5) to [out=0, in=270] (7.5,4.0);
	\draw [myGreen, thick] (6.5,3.0) to [out=90, in=270] (6.7,3.5);
	\draw [myGreen, thick] (6.3,3.5) to [out=90, in=270] (6.5,4.0);
	\end{tikzpicture}
	\caption{Prima regola di reazione\label{fig:buildingsReactions1}}
	\end{figure}
	
	\item
	La seconda regola prevede di matchare solo gli agent che non sono collegati a nessun computer e che si trova nella loro stessa stanza. Il reactum prevede che 
	l'agent si ricolleghi a un tale computer. Se ci sono pi� computer nella stanza, la regola non si esprime su quale computer l' agent si debba ricollegare. Si noti un' 
	importante dettaglio: agent e computer si trovano sotto la stessa radice, e quindi la regola scatta solo se essi si trovano dentro la stessa stanza, o dentro entrambi 
	dentro lo stesso building.
	
	
	\begin{figure}[!htbp]
	\centering
	\begin{tikzpicture}
	%\draw[help lines] (0,0) grid (8,4);
	%Redex
	\draw[rounded corners=4mm,dotted, thick] (0.0,0.0) rectangle (3.0,4.0);
	%Redex Agent
	\draw[very thick] (1.2,2.0) -- (1.8,2.0) -- (1.5,3.0) -- (1.2,2.0);
	%Redex Computer
	\draw[very thick] (1.2,0.8) rectangle (1.8,1.0);
	%Redex Edges
	\draw [myGreen, thick] (1.8,0.9) to [out=0, in=270] (2.5,4.0);
	\draw [myGreen, thick] (1.5,3.0) to [out=90, in=270] (1.5,4.0);
	\draw [myGreen, thick] (1.3,1.5) to [out=90, in=270] (1.5,2.0);
	\draw [myGreen, thick] (1.5,1.0) to [out=90, in=270] (1.7,1.5);
	
	\draw[->, very thick, red] (3.5,2.0) -- (4.5,2.0);

	%Reactum
	\draw[rounded corners=4mm,dotted, thick] (5.0,0.0) rectangle (8.0,4.0);
	%Reactum Agent
	\draw[very thick] (6.2,2.0) -- (6.8,2.0) -- (6.5,3.0) -- (6.2,2.0);
	%Reactum Computer
	\draw[very thick] (6.2,0.8) rectangle (6.8,1.0);
	%Reactum Edges
	\draw [myGreen, thick] (6.8,0.9) to [out=0, in=270] (7.5,4.0);
	\draw [myGreen, thick] (6.5,3.0) to [out=90, in=270] (6.5,4.0);
	\draw [myGreen, thick] (6.5,1.0) to [out=90, in=270] (6.5,2.0);
	
	\end{tikzpicture}
	\caption{Seconda regola di reazione\label{fig:buildingsReactions2}}
	\end{figure}
		
	\item
	Mentre le prime due regole modificavano solamente il link graph, questa terza regola agisce anche sul place graph. Essa prevede lo spostamento di un agent in
	una stanza. Si noti ancora una volta che l' agent e la stanza si trovano sotto la stessa radice. Questo implica che se assumiamo che ogni stanza sia dentro una
	costruzione (building), allora questa regola scatta se e solo se l'agent si trova nella stesso building della stanza. In altre parole, non � possibile che un agent fuori
	da un building entri direttamente in una stanza. Infine, si noti il sito presente sia nel redex che nel reactum: esso rappresenta i parametri della regola. Dentro questo
	sito ci possono essere altri computer o altri agent. Se si togliesse tale sito dal redex, allora la regola scatterebbe solo se in tale stanza non ci fossero n� computer
	n� agenti, cio� solo se fosse vuota.
	
	
	\begin{figure}[h]
	\centering
	\begin{tikzpicture}
	%\draw[help lines] (0,0) grid (8,4);
	%Redex
	\draw[rounded corners=4mm,dotted, thick] (0.0,0.0) rectangle (3.0,4.0);
	\draw[thick] (2.0,1.5) ellipse (1.0 and 1.5);
	\draw[very thick] (0.0,2.0) -- (0.5,2.0) -- (0.25,3.0) -- (0.0,2.0);
	\draw [myGreen, thick] (0.25,2.0) to [out=270, in=180] (0.5,1.8) to [out=0, in=270] (0.5,4.0);
	\draw [myGreen, thick] (0.25,3.0) to [out=90, in=270] (0.25,4.0);
	\draw[rounded corners=1mm,dotted, thick, fill=myGrey](1.5,0.7) rectangle (2.5, 1.2);
	
	\draw[->, very thick, red] (3.5,2.0) -- (4.5,2.0);
	
	%Reactum
	\draw[rounded corners=4mm,dotted, thick] (5.0,0.0) rectangle (8.0,4.0);
	\draw[thick] (7.0,1.5) ellipse (1.0 and 1.5);
	\draw[very thick] (7.0,1.5) -- (7.5,1.5) -- (7.25,2.5) -- (7.0,1.5);
	\draw [myGreen, thick] (7.25,1.5) to [out=270, in=180] (7.5,1.3) to [out=0, in=270] (7.5,4.0);
	\draw [myGreen, thick] (7.25,2.5) to [out=90, in=270] (7.25,4.0);
	\draw[rounded corners=1mm,dotted, thick, fill=myGrey](6.5,0.7) rectangle (7.5, 1.2);%Site
	
	\end{tikzpicture}
	\caption{Terza regola di reazione\label{fig:buildingsReactions3}}
	\end{figure}	
\end{itemize}



\subsection{Regole di reazione parametriche}

\begin{define}[Matching]
Il problema di trovare una o pi� occorrenze di un \emph{redex R} all'interno di un bigrafo B si chiama \emph{matching}. Nel caso in cui un \emph{match} venga trovato,
esso viene denotato tramite la seguente equazione:
\begin{center}
$B = C \circ (R \otimes id_I) \circ D$
\end{center}
dove C � detto il \emph{contesto} e D sono i \emph{parametri} del match.
\end{define}

I bigrafi e la loro teoria ereditano ed estendono molte caratteristiche del CCS (\emph{Calculus of Communicating Systems}). Una di queste � la possibilit� di isolare
le zone che possono evolversi, cio� determinare quali nodi del bigrafo possono essere soggetti al processo di matching. Introduciamo quindi la nozione di 
\emph{segnatura dinamica}.

\begin{define}[Segnatura dinamica]
Una segnatura � \emph{dinamica} se assegna ad ogni controllo $K$ uno stato nell'insieme $\{attivo, passivo \}$. Diciamo che un K-nodo � \emph{attivo}
se il suo controllo � assegnato allo stato attivo; lo stesso vale per lo stato passivo.

Un bigrafo $G: \left <m, X \right > \to \left <n, Y \right >$ � attivo su $i \in m$ se ogni nodo antenato del sito $i$ � attivo. Un bigrafo G si dice attivo, se � attivo su ogni
sito.
\end{define}

Prima di scattare, una regola di reazione necessita del processo di matching, che trova l'occorrenza del redex della regola nel bigrafo. L'occorrenza sar� della forma 
$B = C \circ (R \otimes id_I) \circ D$. Se il contesto C � attivo, allora la regola scatta, sostituendo l'occorrenza del redex con il reactum. Altrimenti, la regola non scatta.
In questo senso si riesce ad isolare parti del sistema che non si vogliono fare evolvere.\\

Si fornisce ora una definizione precisa di \emph{regola di reazione}. Si tenga a mente l'esempio \ref{sub:esempioBuilding}, in cui si presta particolare attenzione alla
sostituzione del redex con il reactum.

\begin{define}[Regola di reazione]
Una regola di reazione $R \to R'$, dove il redex $R$ e il reactum $R'$ hanno la stessa interfaccia, � una trasformazione che, se applicata al bigrafo $B$, produce un nuovo
bigrafo $B'$ secondo questa relazione:
\begin{center}
$B: C \circ (R \otimes id_I) \circ D \qquad \to \qquad B': C \circ (R' \otimes id_I) \circ D $
\end{center}
dove il contesto C � attivo.
\end{define}

Si noti che questa definizione richiede che $R$ e $R'$ abbiano la stessa interfaccia. Questo � in genere un vincolo troppo forte, che obbliga il progettista a scrivere regole
di reazione limitative. Per esempio, non potrei copiare il contenuto, perch� i siti del reactum sarebbero uno in pi� di quelli del redex, rendendo la loro interfaccia interna
diversa. Per indebolire questo vincolo, introduciamo il concetto di \emph{mappa di istanziazione} che porter� al concetto di \emph{regole di reazione parametriche}.

\begin{define}[Mappa di istanziazione]
Siano $\left <m, X \right >$ e $\left <m', X \right >$ le interfacce interne rispettivamente di $R$ e $R'$. Una \emph{mappa di istanzazione} � una funzione $\eta: m \to m'$
che mappa siti di $R'$ in siti di $R$.
\end{define}

\begin{define}[Regola di riscrittura parametrica]
Una regola di reazione parametrica � una tripla della forma:
\begin{center}
$(R: m \to I , R': m' \to I , \eta)$
\end{center}
dove $R$ e $R'$ sono rispettivamente il redex e il reactum della regola. Questa volta devono concordare solo sull'interfaccia esterna e sugli inner names.

La regola di reazione parametrica, se applicata al bigrafo B, produce la reazione:
\begin{center}
$B = C \circ (R \otimes id_I) \circ D \qquad \to \qquad B' = C \circ (R' \otimes id_I) \circ \overline{\eta}(D)$
\end{center}
in cui la funzione $\overline{\eta}$ � definita come segue: 

sia $g: \left <m, X \right >$ un bigrafo la cui FND � $g = \lambda \circ (d_0 \otimes \dots \otimes d_{m-1})$, allora
\begin{center}
$\overline{\eta}(g) = \lambda \circ (d'_0 || \dots || d'_{m-1})$
\end{center}
dove $d'_i \bumpeq d_{\eta(i)}$
\end{define}


\subsection{BRS}
Si sono date tutte le nozioni necessarie per definire formalmente la dinamica di un bigrafo. Essa � costituita da un BRS (\emph{bigraphical reactive system}), che fa evolvere
il bigrafo di partenza sulla base di regole ben precise, contenute al suo interno. Ogni BRS � costituito da una segnatura e da un insieme di regole: risulta quindi essere
un sistema deduttivo, perch� composto rispettivamente da sintassi e semantica. 

\begin{define}[BRS]
Un sistema reattivo bigrafico (BRS) � definito attraverso una coppia $(K, R)$, dove $K$ � una segnatura e $R$ � un insieme di regole di reazione, e si indica con
$BG(K, R)$. \\
L'insieme delle regole $R$ � chiuso rispetto all'\emph{equivalenza sul supporto}: se $R \bumpeq S$ e $R' \bumpeq S'$ e $(R, R', \eta) \in R$ per un certo $\eta$, allora
$(S, S', \eta) \in R$.
\end{define}

Si indichi ora con $B \to* B'$ il fatto che il bigrafo B' sia raggiungibile da B con zero o pi� regole di reazione. Si dimostra che, dato un bigrafo B la cui segnatura � K, il BRS
denotato da $BG(K, R)$ computa ogni bigrafo B' tale che $B \to* B'$. Se si considerano quindi tutti i possibili B', si ha l'insieme di tutti i possibili stati in cui il sistema B pu�
evolversi secondo le regole R.



\subsection{Esempio}\label{sub:exeIntro}
Vediamo ora un ultimo esempio, in cui si useranno le regole di reazione parametriche. Si vuole rappresentare l'operazione di moltiplicazione tramite i bigrafi. Incominciamo con la segnatura: ci sono tre controlli
\begin{itemize}
	\item
	Mul, che rappresenta l'operazione di moltiplicazione. Ha ariet� 0. Attivo.
	\item
	Num, che rappresenta il numero, anch'esso 0 porte. Passivo.
	\item
	One, rappresenta l'unit�, con ariet� 0. Passivo.
\end{itemize}

L'idea principale � che il nodo \emph{mul} contiene due numeri, cio� due nodi \emph{num}, che saranno quelli che dovranno essere moltiplicati. Si pu� anche optare per una versione ricorsiva, permettendo al nodo \emph{mul} di contenere anche altri nodi \emph{mul}. Infine, ogni numero � rappresentato come unione di varie unit�. Per esempio, il numero 4 verr� rappresentano tramite il nodo \emph{num} contenente quattro nodi \emph{one}.

Se vogliamo operare la moltiplicazione fra 4 e 2, allora il bigrafo equivalente sar� quello di figura \ref{fig:bigMul}.

\begin{figure}[!h]
\centering
\begin{tikzpicture}
%\draw[help lines] (0,0) grid (8,5);
%Root
\draw[rounded corners=4mm,dotted, thick] (1.0,0.0) rectangle (7.0,5.0);
\node[below right] at (1.0,5.0) {0};
%Nodes
\draw[rounded corners=4mm, thick] (1.5,0.5) rectangle (6.5,4.5);%mul
\node[below right] at (1.5,4.5) {mul};
\draw[thick] (2.7,3.0) circle [radius=1.0] ;%num1
\node[below] at (2.6,3.9) {num};
\draw[thick] (2.0,3.0) rectangle (2.4,3.4);%one
\node[above right] at (2.0,2.95) {1};
\draw[thick] (2.8,3.0) rectangle (3.2,3.4);%one
\node[above right] at (2.8,2.95) {1};
\draw[thick] (2.0,2.4) rectangle (2.4,2.8);%one
\node[above right] at (2.0,2.35) {1};
\draw[thick] (2.8,2.4) rectangle (3.2,2.8);%one
\node[above right] at (2.8,2.35) {1};
\draw[thick] (5.3,3.0) circle [radius=1.0] ;%num2
\node[below] at (5.2,3.9) {num};
\draw[thick] (4.6,3.0) rectangle (5.0,3.4);%one
\node[above right] at (4.6,2.95) {1};
\draw[thick] (5.4,3.0) rectangle (5.8,3.4);%one
\node[above right] at (5.4,2.95) {1};

\end{tikzpicture}
\caption{Moltiplicazione tramite bigrafi \label{fig:bigMul}}
\end{figure}

Per eseguire la moltiplicazione si usano solamente due regole di reazione, ma in maniera \emph{ricorsiva}. Esse ricalcano nel formalismo dei bigrafi questa semplice operazione: $a*b = (a-1)*b+b$. Distinguiamo quindi fra caso ricorsivo e caso base.

\begin{figure}[!h]
\centering
\begin{tikzpicture}
%\draw[help lines] (0,0) grid (13,5);
%%% REDEX %%%
%Root
\draw[rounded corners=4mm,dotted, thick] (0.0,0.0) rectangle (6.0,5.0);
\node[below right] at (0.0,5.0) {0};
%Nodes
\draw[rounded corners=4mm, thick] (0.5,0.5) rectangle (5.5,4.5);%mul
\node[below right] at (0.5,4.5) {mul};
\draw[thick] (1.7,3.0) circle [radius=1.0] ;%num1
\node[below] at (1.6,3.9) {num};
\draw[thick] (1.0,3.0) rectangle (1.4,3.4);%one
\node[above right] at (1.0,2.95) {1};
\draw[rounded corners=1mm,dotted, thick, fill=myGrey] (1.0,2.5) rectangle (2.1,2.9) ;%site 0
\node[above] at (1.5,2.45) {0};
\draw[thick] (4.3,3.0) circle [radius=1.0] ;%num2
\node[below] at (4.2,3.9) {num};
\draw[rounded corners=1mm,dotted, thick, fill=myGrey] (3.5,3.0) rectangle (5.0,3.5) ;%site 1
\node[above] at (4.3,3.0) {1};
\draw[rounded corners=1mm,dotted, thick, fill=myGrey] (2.0,1.0) rectangle (4.0,1.5) ;%site 2
\node[above] at (3.0,1.0) {2};

\draw[->, very thick, red] (6.1,2.5) -- (6.9,2.5);

%%% REACTUM %%%
%Root
\draw[rounded corners=4mm,dotted, thick] (7.0,0.0) rectangle (13.0,5.0);
\node[below right] at (7.0,5.0) {0};
%Nodes
\draw[rounded corners=4mm, thick] (7.5,0.5) rectangle (12.5,4.5);%mul
\node[below right] at (7.5,4.5) {mul};
\draw[thick] (8.7,3.0) circle [radius=1.0] ;%num1
\node[below] at (8.6,3.9) {num};
\draw[rounded corners=1mm,dotted, thick, fill=myGrey] (8.0,2.5) rectangle (9.1,2.9) ;%site 0
\node[above] at (8.5,2.45) {0};
\draw[thick] (11.3,3.0) circle [radius=1.0] ;%num2
\node[below] at (11.2,3.9) {num};
\draw[rounded corners=1mm,dotted, thick, fill=myGrey] (10.5,3.0) rectangle (12.0,3.5) ;%site 1
\node[above] at (11.3,3.0) {1};
\draw[rounded corners=1mm,dotted, thick, fill=myGrey] (8.0,1.0) rectangle (10.0,1.5) ;%site 2
\node[above] at (9.0,1.0) {2};
\draw[rounded corners=1mm,dotted, thick, fill=myGrey] (11.0,1.0) rectangle (12.0,1.5) ;%site 3
\node[above] at (11.5,1.0) {3};

%%% ARCHI PARAMETRICI %%%
\draw[dashed, blue, very thick] (8.5,1.2) to [out=270,in=0] (3.7,1.2); % 2-2
\draw[dashed, blue, very thick] (8.2,2.7) to [out=90,in=90] (2.0,2.7); % 0-0
\draw[dashed, blue, very thick] (11.2,1.3) to [out=120,in=270] (4.8,3.3); % 3-1
\draw[dashed, blue, very thick] (10.8,3.3) to [out=90,in=90] (4.0,3.3); % 1-1

\end{tikzpicture}
\caption{Caso ricorsivo \label{fig:bigMulRicorsive}}
\end{figure}

La regola del caso ricorsivo mette bene in evidenza l'importanza delle regole parametriche. Un sito del redex (1) viene copiato in una regione diversa, cio� dentro \emph{mul}. Questo ci consente di \emph{copiare} tutto il contenuto del secondo nodo \emph{num} dentro il nodo \emph{mul}. Infatti, il contenuto di \emph{mul} al di fuori dei nodi \emph{num} rappresenta la somma ($+b$) dell'uguaglianza $a*b = (a-1)*b + b$.\\
Si noti che questa regola va applicata ricorsivamente fino a quando il primo nodo \emph{num} ha contenuto vuoto, cio� fino a quando il caso base � applicabile.



\begin{figure}[!h]
\centering
\begin{tikzpicture}
%\draw[help lines] (0,0) grid (13,5);
%%% REDEX %%%
%Root
\draw[rounded corners=4mm,dotted, thick] (0.0,0.0) rectangle (6.0,5.0);
\node[below right] at (0.0,5.0) {0};
%Nodes
\draw[rounded corners=4mm, thick] (0.5,0.5) rectangle (5.5,4.5);%mul
\node[below right] at (0.5,4.5) {mul};
\draw[thick] (1.7,3.0) circle [radius=1.0] ;%num1
\node[below] at (1.6,3.9) {num};
\draw[thick] (4.3,3.0) circle [radius=1.0] ;%num2
\node[below] at (4.2,3.9) {num};
\draw[rounded corners=1mm,dotted, thick, fill=myGrey] (3.5,3.0) rectangle (5.0,3.5) ;%site 0
\node[above] at (4.3,3.0) {0};
\draw[rounded corners=1mm,dotted, thick, fill=myGrey] (2.0,1.0) rectangle (4.0,1.5) ;%site 1
\node[above] at (3.0,1.0) {1};

\draw[->, very thick, red] (6.1,2.5) -- (6.9,2.5);

%%% REACTUM %%%
%Root
\draw[rounded corners=4mm,dotted, thick] (7.0,0.0) rectangle (10.0,5.0);
\node[below right] at (7.0,5.0) {0};
%Nodes
\draw[thick] (8.7,3.0) circle [radius=1.0] ;%num1
\node[below] at (8.6,3.9) {num};
\draw[rounded corners=1mm,dotted, thick, fill=myGrey] (8.0,2.5) rectangle (9.1,2.9) ;%site 0
\node[above] at (8.5,2.45) {0};

%%% ARCHI PARAMETRICI %%%
\draw[dashed, blue, very thick] (8.6,2.5) to [out=270,in=0] (3.7,1.2); % 0-1

\end{tikzpicture}
\caption{Caso Base \label{fig:bigMulBase}}
\end{figure}


Il caso base copia il contenuto di \emph{mul} dentro un nuovo nodo \emph{num}, che rappresenta il risultato della moltiplicazione. Nell'aritmetica classica, per risolvere ricorsivamente la moltiplicazione $2*4$ si operano questi passi:
\begin{center}
$result = 4*2 = (4-1)*2+2 = 3*2+2 \qquad$ (caso ricorsivo)
$3*2 = (3-1)*2+2 = 2*2 +2  \qquad$ (caso ricorsivo)
$2*2 = (2-1)*2+2 = 1*2 + 2 \qquad$ (caso ricorsivo)
$1*2 = (1-1)*2+2 = 0*2+2 = 2 \qquad$ (caso base)
\end{center}
Seguendo la catena di uguaglianze, si trova che il risultato finale della moltiplicazione �: 
\begin{center}
$result = 4*2 = (((0*2+2)+2)+2)+2 = 2 + 2 + 2 + 2 = 8$.
\end{center}



Vediamo ora l'applicazione della prima regola ricorsiva sul bigrafo di partenza (figura \ref{fig:bigMul}). La prima operazione da eseguire � trovare l'occorrenza del redex nel
bigrafo, che formalmente significa rispettare l'equazione $B = C \circ (R \otimes id_I) \circ D$, dove C � il contesto, R � il redex e D sono i parametri. Dato che tutti e tre sono
bigrafi, possiamo disegnarli, come � stato fatto in figura \ref{fig:bigMulMatch}.

\newpage



\begin{figure}[!htbp]
\centering
\begin{tikzpicture}
%\draw[help lines] (0,0) grid (14,6);
%%%   Contesto   %%%
\node[above] at (1.5,5.5) {Contesto};
%Root
\draw[rounded corners=4mm,dotted, thick] (0.0,0.0) rectangle (3.0,5.0);
\node[below right] at (0.0,5.0) {0};
%Site
\draw[rounded corners=1mm,dotted, thick, fill=myGrey] (1.0,1.0) rectangle (2.0,4.0) ;%site 0
\node at (1.5,2.0) {0};

%%%   Redex   %%%
\node[above] at (7.0,5.5) {Redex};
%Root
\draw[rounded corners=4mm,dotted, thick] (4.0,0.0) rectangle (10.0,5.0);
\node[below right] at (4.0,5.0) {0};
%Nodes
\draw[rounded corners=4mm, thick] (4.5,0.5) rectangle (9.5,4.5);%mul
\node[below right] at (4.5,4.5) {mul};
\draw[thick] (5.7,3.0) circle [radius=1.0] ;%num1
\node[below] at (5.6,3.9) {num};
\draw[thick] (5.0,3.0) rectangle (5.4,3.4);%one
\node[above right] at (5.0,2.95) {1};
\draw[rounded corners=1mm,dotted, thick, fill=myGrey] (5.0,2.5) rectangle (6.1,2.9) ;%site 0
\node[above] at (5.5,2.45) {0};
\draw[thick] (8.3,3.0) circle [radius=1.0] ;%num2
\node[below] at (8.2,3.9) {num};
\draw[rounded corners=1mm,dotted, thick, fill=myGrey] (7.5,3.0) rectangle (9.0,3.5) ;%site 1
\node[above] at (8.3,3.0) {1};
\draw[rounded corners=1mm,dotted, thick, fill=myGrey] (6.0,1.0) rectangle (8.0,1.5) ;%site 2
\node[above] at (7.0,1.0) {2};

%%%   Parametri   %%%
\node[above] at (12.5,5.5) {Parametri};
\draw[rounded corners=4mm,dotted, thick] (10.8,0.0) rectangle (13.9,1.4);
\node[below right] at (10.9,1.4) {0};
\draw[thick] (11.0,0.5) rectangle (11.4,0.9);%one
\node[above right] at (11.0,0.45) {1};
\draw[thick] (12.0,0.5) rectangle (12.4,0.9);%one
\node[above right] at (12.0,0.45) {1};
\draw[thick] (13.0,0.5) rectangle (13.4,0.9);%one
\node[above right] at (13.0,0.45) {1};

\draw[rounded corners=4mm,dotted, thick] (10.8,1.6) rectangle (13.9,3.0);
\node[below right] at (10.9,3.0) {1};
\draw[thick] (12.0,2.0) rectangle (12.4,2.4);%one
\node[above right] at (12.0,1.95) {1};
\draw[thick] (13.0,2.0) rectangle (13.4,2.4);%one
\node[above right] at (13.0,1.95) {1};
\draw[rounded corners=4mm,dotted, thick] (10.8,3.2) rectangle (13.9,4.6);
\node[below right] at (10.9,4.6) {2};


\end{tikzpicture}
\caption{Match decomposto \label{fig:bigMulMatch}}
\end{figure}


Il bigrafo finale, dopo 3 applicazioni della regola ricorsiva ed una della regola base, sar� quello di figura \ref{fig:bigMulFinal}.


\begin{figure}[!htbp]
\centering
\begin{tikzpicture}
%\draw[help lines] (0,0) grid (6,5);
%Root
\draw[rounded corners=4mm,dotted, thick] (0.0,0.0) rectangle (5.0,5.0);
\node[below right] at (0.0,5.0) {0};
%Nodes
\draw[thick] (2.5,2.5) circle [radius=2.0];
\node[above] at (2.5,4.0) {num};
\draw[thick] (1.0,3.0) rectangle (1.4,3.4);%one
\node[above right] at (1.0,2.95) {1};
\draw[thick] (2.0,3.0) rectangle (2.4,3.4);%one
\node[above right] at (2.0,2.95) {1};
\draw[thick] (3.0,3.0) rectangle (3.4,3.4);%one
\node[above right] at (3.0,2.95) {1};
\draw[thick] (1.0,2.0) rectangle (1.4,2.4);%one
\node[above right] at (1.0,1.95) {1};
\draw[thick] (2.0,2.0) rectangle (2.4,2.4);%one
\node[above right] at (2.0,1.95) {1};
\draw[thick] (3.0,2.0) rectangle (3.4,2.4);%one
\node[above right] at (3.0,1.95) {1};
\draw[thick] (2.0,1.0) rectangle (2.4,1.4);%one
\node[above right] at (2.0,0.95) {1};
\draw[thick] (3.0,1.0) rectangle (3.4,1.4);%one
\node[above right] at (3.0,0.95) {1};

\end{tikzpicture}
\caption{Bigrafo finale \label{fig:bigMulFinal}}
\end{figure}

L'implementazione di questo esempio si pu� trovare in \ref{}.

Questo esempio mostra bene come le regole di reazione che formano il BRS rappresentano la semantica del sistema. Per evitare di sprecare memoria sul calcolatore, � necessario progettare con cura la segnature e l'insieme di regole del BRS. Nell'esempio di cui sopra, le regole erano soltanto due e potevano essere applicate ricorsivamente. Questo � preferibile ad avere $k$ regole e applicarle una dopo l'altra, appunto a causa dello spreco di memoria che si otterrebbe.








\chapter{Isomorfismo tra bigrafi}
Si � visto nel precedente capitolo come un bigrafo sia capace di evolversi all'interno di un BRS. In alcune situazioni, si vuole evitare evoluzioni infinite di un bigrafo, perch�
per esempio porterebbero sempre a stati uguali fra di loro.\\
Per tenere traccia all'istante $t_k$ di tutti gli stati precedentemente assunti da un BRS ($t_0 \dots t_{k-1}$), si � costruita una struttura dati a grafo, dove ogni nodo
� uno stato, cio� un bigrafo: si chiamer� questo grafo  ``\emph{grafo degli stati}". Se un BRS parte dallo stato $S_0$, � possibile che dopo $K$ regole di reazione lo stato $S_k$ sia uguale allo stato iniziale $S_0$. 
Questo significa che $S_0$ e $S_k$ hanno la stessa \emph{semantica}, cio� rappresentano lo stesso stato del sistema. Per cui, nel grafo degli stati, essi dovranno essere
lo stesso medesimo nodo. \\
Questo � equivalente a scrivere: ``se lo stato $S_{k}$ � stato ottenuto da $S_{k-1}$ tramite la regola $R$ e $S_k$ e $S_0$ sono isomorfi, allora da $S_{k-1}$ si ottiene
$S_0$"
\begin{center}
se $S_{k-1}\stackrel{R}{\longrightarrow}S_{k}$ e $ S_k \bumpeq S_0 \qquad$ allora $S_{k-1}\stackrel{R}{\longrightarrow}S_{0}$
\end{center}
Capire quando due bigrafi sono isomorfi e quindi \emph{semanticamente equivalenti} � di fondamentale importanza se si vogliono evitare i cos� detti \emph{loop} fra regole: per evitare che fra $S_{i}$ e $S_{i+1}$ si continuino ad applicare sempre le due stesse regole $R_1$ e $R_2$ all'infinito, devo capire che:
\begin{itemize} 
	\item
	da $S_{i}$, tramite la regola $R_1$, ottengo un bigrafo isomorfo a $S_{i+1}$
	\item
	da $S_{i+1}$, tramite la regola $R_2$, ottengo un bigrafo isomorfo a $S_{i}$
\end{itemize}
Questo concetto verr� chiarito con gli esempi sottostanti.

\section{Esempio:}\label{sub:networkExe}
Si prenda l'esempio di una rete modellata tramite un bigrafo: si vuole fare in modo che, dato un pacchetto iniziale che ha
come mittente l'host A, esso arrivi al destinatario B. Ci saranno quindi delle regole di reazione per i router, che permetteranno di inoltrare i pacchetti verso le sue interfacce
di uscita. Poich�, utilizzando i soli bigrafi, servirebbero troppe regole di reazione per modellare il fatto che se il destinario � X allora l'interfaccia di uscita del pacchetto � la numero N,  si pu� pensare di inoltrare il pacchetto verso tutte le uscite in modo non deterministico. Cos� facendo, il pacchetto arriver� sicuramente al destinatario.
I problemi sono ora due:
\begin{itemize}
	\item
	il pacchetto arriver� a destinatari non corretti. Si pu� introdurre una regola di reazione che elimini dall'host ogni pacchetto che non ha come destinatario l'host stesso.
	\item
	se prendiamo il $k-esimo$ router $R_k$, allora il pacchetto ritorner� al $(k-1)-esimo$ router $R_{k-1}$ , che a sua volta lo inoltrer� verso tutte le sue interfacce, e quindi 
	anche nuovamente verso $R_{k}$. Si ha cos� un ciclo infinito di pacchetti tra $R_k$ e $R_{k-1}$.
\end{itemize}

In questa sede si tratter� il secondo di questi problemi, che ha un'apparentemente semplice soluzione: capire quando due bigrafi sono uguali. In questo esempio, � 
facile capire che per risolvere il problema basta verificare se gli stati $S_{k+i}$ e $S_{k-1}$ sono uguali: se $S_{k}$ � il bigrafo in cui il pacchetto � nel router $R_{k}$, allora
si applica la regola di inoltro e si generano tanti stati ($S_{k+1} \dots S_{k+j}$) quanti sono i router vicini a $R_k$; tra di questi ci sara anche $R_{k-1}$. Quindi, se supponiamo
che nello stato $S_{k+i}$ il pacchetto torni indietro al router $R_{k-1}$, allora il problema � capire che $S_{k+i}$ � uguale $S_{k-1}$, cio� verificare che i due bigrafi siano
isomorfi. In questo modo si sa che dallo stato $S_{k-1}$ non si dovr� pi� applicare la regola di inoltro verso $S_{k}$, perch� questo causerebbe un ciclo infinito.

\section{Formulazione del problema}
Si � visto che il termine isomorfismo ci aiuta a capire quando due bigrafi sono uguali. Riportiamo di nuovo la sua definizione formale (\ref{def:iso}):
\begin{define}[Isomorfismo]
Due bigrafi F e G si dicono isomorfi se e solo se esiste una traduzione di supporto tra F e G, cio� se e solo se F e G sono \emph{support equivalent} ($F \bumpeq G$).
\end{define}

Il problema consiste quindi nel trovare un funzione biettiva che ha come dominio i nodi e gli archi del primo bigrafo, come codominio quelli del secondo bigrafo e che ne rispetti
la struttura del primo. Come gi� visto, tale funzione si chiama \emph{traduzione di supporto}.\\
Si noti come la traduzione di supporto consenta di ritenere isomorfi due bigrafi che sono uguali \emph{modulo permutazione}: tale funzione pu� infatti operare una permutazione sui nomi dei nodi e degli archi, cos� come sugli inner names e outer names. In altre parole, non c'� nessun vincolo sui nomi, ma solo sulla struttura del bigrafo. Si capir� meglio questo concetto negli esempi che seguiranno.

\section{Strategia di soluzione}
Dal punto di vista teorico, si � trattato questo problema come una \emph{rete di flusso}: esse consentono di specificare in modo molto preciso delle condizioni sulla \emph{struttura} dei grafi. Qui l'obbiettivo � trovare una traduzione di supporto, che conservi la struttura e sia biettiva: quest'ultima caratteristica per esempio pu� essere espressa come
il fatto che ogni nodo del primo bigrafo deve essere associato ad uno ed un solo nodo del secondo bigrafo. Tramite la rete di flusso, si pu� specificare questa condizione 
dicendo che il flusso in uscita da ogni nodo deve essere esattamente 1. Questo � un primo esempio del motivo per cui si � scelta una rete di flusso per la modellizzazione.

Si � scelto di trattare questo problema tramite la programmazione a vincoli. Questo paradigma permette di rappresentare al calcolatore un sistema di equazioni, che esso
risolver�. Le equazioni saranno i vincoli che la rete di flusso dovr� rispettare. Nel nostro caso, ci baster� capire se esiste una soluzione al sistema, e non ci interessa sapere quali sono le sue soluzioni. Tutte le equazioni scritte sono lineari,
con la conseguenza che si avr� un sistema lineare: questo permette di abbassare la complessit� dell'algoritmo.\\
Si � usata la libreria Java \emph{Choco vs 3.3.1} per esprimere tutti i vincoli.\\

Si � precedentemente visto che ogni bigrafo � formato da due strutture ortogonali e totalmente indipendenti, il place graph ed il link graph. Il problema dell'isomorfismo verr� quindi trattato
separatamente per le due strutture: ci saranno dei vincoli solamente per il place graph, ed altri solamente per il link graph. Infine, gli ultimi vincoli serviranno per conciliare le due soluzioni: si vedr� che senza di questi il calcolatore riconoscer� come uguali due bigrafi che hanno lo stesso link graph e lo stesso place graph modulo permutazione ma che non sono isomorfi (figura \ref{fig:falseIso}).\\

La definizione di isomorfismo vista precedentemente pu� far credere che esso sia un termine puramente sintattico, cio� che riguardi solamente la struttura dei due bigrafi. L'isomorfismo � invece un'operazione che riguarda anche la semantica: due bigrafi strutturalmente uguali sono due bigrafi semanticamente equivalenti (figura \ref{fig:bigMulIso}).

\subsection{Esempi}\label{sub:isoExamples}
Non � sempre banale capire quando due bigrafi sono isomorfi. Si forniscono quindi alcuni esempi per acquisire familiarit� con questo concetto. 

Il primo riprende il bigrafo di figura \ref{fig:bigMul} per la moltiplicazione tra numeri naturali. Si vuole mostrare che, semanticamente, l'operazione $2*4$ � equivalente
all'operazione $4*2$. In altre parole, non importa l'ordine dei nodi.

Incominciamo notando che i due link graph sono isomorfi, perch� non hanno archi e presentano lo stesso numero di nodi. Prendiamo ora in esame il place graph: se si ritorna alla sua definizione, si nota subito come i due place graph siano uguali perch� i nodi interni non sono ordinati.

Il fatto di dimostrare che i due bigrafi sono isomorfi e semanticamente equivalenti, dimostra che la propriet� commutativa della moltiplicazione vale anche nella sua versione bigrafica.\\

\begin{figure}[!h]
\centering
\begin{tikzpicture}
%\draw[help lines] (0,0) grid (14,5);
%Root
\draw[rounded corners=4mm,dotted, thick] (1.0,0.0) rectangle (7.0,5.0);
\node[below right] at (1.0,5.0) {0};
%Nodes
\draw[rounded corners=4mm, thick] (1.5,0.5) rectangle (6.5,4.5);%mul
\node[below right] at (1.5,4.5) {mul};
\draw[thick] (2.7,3.0) circle [radius=1.0] ;%num1
\node[below] at (2.6,3.9) {num};
\draw[thick] (2.0,3.0) rectangle (2.4,3.4);%one
\node[above right] at (2.0,2.95) {1};
\draw[thick] (2.8,3.0) rectangle (3.2,3.4);%one
\node[above right] at (2.8,2.95) {1};
\draw[thick] (2.0,2.4) rectangle (2.4,2.8);%one
\node[above right] at (2.0,2.35) {1};
\draw[thick] (2.8,2.4) rectangle (3.2,2.8);%one
\node[above right] at (2.8,2.35) {1};
\draw[thick] (5.3,3.0) circle [radius=1.0] ;%num2
\node[below] at (5.2,3.9) {num};
\draw[thick] (4.6,3.0) rectangle (5.0,3.4);%one
\node[above right] at (4.6,2.95) {1};
\draw[thick] (5.4,3.0) rectangle (5.8,3.4);%one
\node[above right] at (5.4,2.95) {1};

%Root
\draw[rounded corners=4mm,dotted, thick] (8.0,0.0) rectangle (14.0,5.0);
\node[below right] at (8.0,5.0) {0};
%Nodes
\draw[rounded corners=4mm, thick] (8.5,0.5) rectangle (13.5,4.5);%mul
\node[below right] at (8.5,4.5) {mul};
\draw[thick] (9.7,3.0) circle [radius=1.0] ;%num1
\node[below] at (9.6,3.9) {num};
\draw[thick] (9.0,3.0) rectangle (9.4,3.4);%one
\node[above right] at (9.0,2.95) {1};
\draw[thick] (9.8,3.0) rectangle (10.2,3.4);%one
\node[above right] at (9.8,2.95) {1};
\draw[thick] (12.3,3.0) circle [radius=1.0] ;%num2
\node[below] at (12.2,3.9) {num};
\draw[thick] (11.6,3.0) rectangle (12.0,3.4);%one
\node[above right] at (11.6,2.95) {1};
\draw[thick] (12.4,3.0) rectangle (12.8,3.4);%one
\node[above right] at (12.4,2.95) {1};
\draw[thick] (11.6,2.4) rectangle (12.0,2.8);%one
\node[above right] at (11.6,2.35) {1};
\draw[thick] (12.4,2.4) rectangle (12.8,2.8);%one
\node[above right] at (12.4,2.35) {1};

\end{tikzpicture}
\caption{Bigrafi Isomorfi: $(4*2) = (2*4)$ \label{fig:bigMulIso}}
\end{figure}


Presentiamo un altro esempio, questa volta su un caso negativo. Si � specificato che per risolvere il problema si trattano separatamente i casi del place graph e del link graph: questo approccio per� necessita di avere dei vincoli di ``coerenza" che uniscano le due soluzioni. Senza questi vincoli, si pu� incorrere nel seguente problema.

Si considerino i due bigrafi di figura \ref{fig:falseIso} e i loro relativi place graph e link graph (figura \ref{fig:falsoIsoDecap}). Si nota subito che i due bigrafi \emph{non} sono isomorfi. Prendendo per� i due place graph ci si accerta che essi lo sono. Lo stesso vale per i due link graph: sono entrambi formati da tre nodi, in cui c'� solamente un edge che collega il cerchio ad un quadrato.

I casi separati del link graph e del place graph sembrano quindi suggerire che i due bigrafi siano isomorfi, mentre si vede subito che non � cos�. Quello che non abbiamo considerato sono appunto i \emph{vincoli di coerenza}, che informalmente dicono che il quadrato collegato al cerchio (nel primo link graph) deve essere quello che contiene un altro quadrato (nel primo place graph). Aggiunta questa condizione, si nota che le due soluzioni sono \emph{incompatibili}, avendo quindi che i due bigrafi \emph{non} sono isomorfi.




\begin{figure}[!h]
\centering
\begin{tikzpicture}
%\draw[help lines] (0,0) grid (14,5);
%%%   First Bigraph   %%%
%Root
\draw[rounded corners=4mm,dotted, thick] (0.0,0.0) rectangle (6.0,4.0);
\node[below right] at (0.0,4.0) {0};
%Nodes
\draw[rounded corners=3mm, thick] (0.5,0.5) rectangle (3.0,3.0);
\node[above right] at (0.5,3.0) {$Q_1$};
\draw[thick] (4.5,1.5) circle [radius=1.0];
\node[above] at (4.5,2.5) {$C_1$};
\draw[rounded corners=2mm, thick] (1.0,1.0) rectangle (2.0,2.0);
\node[above right] at (1.0,2.0) {$Q_2$};
%Edges
\draw[myGreen, thick] (2.0,3.0) to [out=90,in=90] (4.0,2.35);
\draw[fill=black] (2.0,3.0) circle [radius=0.05];
\draw[fill=black] (4.0,2.35) circle [radius=0.05];


%%%   First Bigraph   %%%
%Root
\draw[rounded corners=4mm,dotted, thick] (7.0,0.0) rectangle (13.0,4.0);
\node[below right] at (7.0,4.0) {0};
%Nodes
\draw[rounded corners=3mm, thick] (7.5,0.5) rectangle (10.0,3.0);
\node[above right] at (7.5,3.0) {$Q_1$};
\draw[thick] (11.5,1.5) circle [radius=1.0];
\node[above] at (11.5,2.5) {$C_1$};
\draw[rounded corners=2mm, thick] (8.0,1.0) rectangle (9.0,2.0);
\node[above right] at (8.0,2.0) {$Q_2$};
%Edges
\draw[myGreen, thick] (8.7,2.0) to [out=90,in=90] (11.0,2.35);
\draw[fill=black] (8.7,2.0) circle [radius=0.05];
\draw[fill=black] (11.0,2.35) circle [radius=0.05];


\end{tikzpicture}
\caption{Bigrafi non Isomorfi\label{fig:falseIso}}
\end{figure}




\begin{figure}[!h]
\centering
\subfigure[Place Graph isomorfi]{
\begin{tikzpicture}
%\draw[help lines] (0,0) grid (14,5);
%%%   First Place Graph   %%%
%Root
\node at (2.5,4.5){0};
%Nodes
\node at (1.5,3.5){$Q_1$};
\node at (3.5,3.5){$C_1$};
\node at (1.5,1.5){$Q_2$};
%Edges
\draw (2.5,4.2) -- (1.5,3.7);
\draw (2.5,4.2) -- (3.5,3.7);
\draw (1.5,3.2) -- (1.5,1.7);

%%%   Second Place Graph   %%%
%Root
\node at (11.5,4.5){0};
%Nodes
\node at (10.5,3.5){$Q_1$};
\node at (12.5,3.5){$C_1$};
\node at (10.5,1.5){$Q_2$};
%Edges
\draw (11.5,4.2) -- (10.5,3.7);
\draw (11.5,4.2) -- (12.5,3.7);
\draw (10.5,3.2) -- (10.5,1.7);

\end{tikzpicture}
}
\hspace{5mm}
\subfigure[Link Graph isomorfi]{
\begin{tikzpicture}
%\draw[help lines] (0,0) grid (14,5);
%%%   First Link Graph   %%%
\draw[thick] (1.0,2.0) circle [radius=0.2];
\node[above left] at (1.0,2.0) {$Q_1$};
\draw[thick] (2.0,4.0) circle [radius=0.2];
\node[above left] at (2.0,4.0) {$C_1$};
\draw[thick] (4.0,2.0) circle [radius=0.2];
\node[above right] at (4.0,2.0) {$Q_2$};
\draw[myGreen,thick] (1.0,2.2) to [out=90,in=270] (2.0,3.8);

%%%   Second Link Graph   %%%
\draw[thick] (10.0,2.0) circle [radius=0.2];
\node[above left] at (10.0,2.0) {$Q_2$};
\draw[thick] (11.0,4.0) circle [radius=0.2];
\node[above left] at (11.0,4.0) {$C_1$};
\draw[thick] (13.0,2.0) circle [radius=0.2];
\node[above right] at (13.0,2.0) {$Q_1$};
\draw[myGreen,thick] (10.0,2.2) to [out=90,in=270] (11.0,3.8);

\end{tikzpicture}
}
\caption{Place Graph e Link Graph del bigrafo \ref{fig:falseIso}\label{fig:falsoIsoDecap}}
\end{figure}






%%%%%%%%%			VINCOLI 				%%%%%%%%%%%
\section{Vincoli}
In questa sezione si presentano le equazioni necessarie per risolvere il problema dell'isomorfismo. Come gi� anticipato, si lavorer� sulle \emph{reti di flusso}, che sono grafi orientati pesati. Distinguiamo le equazioni per il place graph, per il link graph e per la cos� detta coerenza.

\subsection{Vincoli per il place graph}
L'isomorfismo tra place graphs � un isomorfismo tra foreste. Si vuole infatti vedere quando la prima foresta � isomorfa alla seconda, a meno di permutazioni delle radici e dei siti. Una rete di flusso per questo problema � quella in figura \ref{fig:placeFlow}.

\begin{figure*}[th]
\centering
\begin{tikzpicture}
	%root
	\draw [fill] (1.0,2.0) circle [radius=0.1];
	%level 1
	\draw [fill] (0.0,1.0) circle [radius=0.1];
	\draw [fill] (2.0,1.0) circle [radius=0.1];
	%level 2
	\draw [fill] (2.0,0.0) circle [radius=0.1];
	%tree-edges G
	\draw [myGreen, thick] (1.0,1.9) -- (0.0,1.1);
	\draw [myGreen, thick] (1.0,1.9) -- (2.0,1.1);
	\draw [myGreen, thick] (2.0,0.9) -- (2.0,0.1);
	
	
	%root
	\draw [fill] (5.0,2.0) circle [radius=0.1];
	%level 1
	\draw [fill] (4.0,1.0) circle [radius=0.1];
	\draw [fill] (6.0,1.0) circle [radius=0.1];
	%level 2
	\draw [fill] (6.0,0.0) circle [radius=0.1];
	%tree-edges G
	\draw [myGreen, thick] (5.0,1.9) -- (4.0,1.1);
	\draw [myGreen, thick] (5.0,1.9) -- (6.0,1.1);
	\draw [myGreen, thick] (6.0,0.9) -- (6.0,0.1);
%level 0
\draw [red] (1.1,2.0) to [out=60 , in=180 ] (4.9,2.0);
%level 1
\draw [red] (0.1,1.0) to [out=30 , in=150 ] (3.9,1.0);
\draw [red] (0.1,1.0) to [out=-30 , in=210] (5.9,1.0);
\draw [red] (2.1,1.0) to [out=-30 , in=180 ] (3.9,1.0);
\draw [red] (2.1,1.0) to [out=30 , in=180] (5.9,1.0);
%level 2
\draw [red] (2.1,0.0) to [out=330 , in=180] (5.9,0.0);


\end{tikzpicture}
\caption{Rete di flusso per l'isomorfismo tra place graphs.	\label{fig:placeFlow}}
\end{figure*}



Alle due foreste (cio� ai due place graph, costituiti da archi di colore verde), che chiameremo $P_F$ e $P_G$, si sono aggiunti altri archi (quelli rossi) \emph{solamente} tra nodi della stessa altezza, creando cos� un grafo. Questa � la vera e propria rete di flusso per il problema dell'isomorfismo tra place graph. Ogni arco rosso � \emph{orientato}, perch� va dai nodi della prima foresta ai nodi della seconda, e \emph{pesato}, perch� gli � assegnato un numero naturale $p \in \{ 0, 1\}$.

Nell'implementazione, trattata tramite la programmazione a vincoli, ogni arco rosso � una variabile. L'insieme delle variabili per il place graph sar� quindi:
\begin{center}
$M_{d,m,n} \in \{ 0, 1\} \qquad 0 \le d \le depth \qquad$\\ $\qquad \qquad m \in P_F^d$ \\ $\qquad \qquad n \in P_G^d$
\end{center}
dove \emph{depth} � l'altezza massima della prima foresta, e $P_F^d$ e $P_G^d$ indicano l'insieme di nodi/radici/siti che si trovano all'altezza $d$ rispettivamente nel place graph $P_F$ e $P_G$.\\
In altre parole, creo una variabile che pu� assumere valore 0 o 1 per ogni coppia di nodi $(a,b)$ che si trovano sulla stessa altezza, dove $a$ appartiene al primo place graph mentre $b$ al secondo.

I vincoli dovranno essere tali che, dopo l'esecuzione del sistema sul calcolatore, le variabili che assumeranno il valore 1 saranno quelle che \emph{formeranno} la vera e propria traduzione di supporto. Ovvero: 
\begin{prop}\label{prop:traduzioneSupporto}
La variabile $M_{d,m,n}$ assumer� il valore 1 \emph{se e solo se} esiste una traduzione di supporto $\rho$ tale che $\rho(m)=n$.
\end{prop}
Tutte le altre variabili dovranno assumere il valore 0. Un altro modo di vedere la soluzione � questa: la funzione di traduzione di supporto sar� definita da tutte e sole le variabili con valore 1. Infine: 
\begin{term}\label{term:matchVariabili}
Quando la variabile $M_{d,m,n}=1$ diremo che m e n costituiscono un match.
\end{term}

Nell'esempio di figura \ref{fig:placeFlow}, la soluzione esiste ed � definita come in figura \ref{fig:placeFlowRis}, dove i numeri sopra le variabili (archi rossi) indicano i valori che esse hanno assunto dopo la risoluzione del sistema di equazioni.



\begin{figure*}[th]
\centering
\begin{tikzpicture}
%\draw[help lines] (0,0) grid (10,5);
	%root
	\draw [fill] (1.0,2.0) circle [radius=0.1];
	%level 1
	\draw [fill] (0.0,1.0) circle [radius=0.1];
	\draw [fill] (2.0,1.0) circle [radius=0.1];
	%level 2
	\draw [fill] (2.0,0.0) circle [radius=0.1];
	%tree-edges G
	\draw [myGreen, thick] (1.0,1.9) -- (0.0,1.1);
	\draw [myGreen, thick] (1.0,1.9) -- (2.0,1.1);
	\draw [myGreen, thick] (2.0,0.9) -- (2.0,0.1);
	
	
	%root
	\draw [fill] (5.0,2.0) circle [radius=0.1];
	%level 1
	\draw [fill] (4.0,1.0) circle [radius=0.1];
	\draw [fill] (6.0,1.0) circle [radius=0.1];
	%level 2
	\draw [fill] (6.0,0.0) circle [radius=0.1];
	%tree-edges G
	\draw [myGreen, thick] (5.0,1.9) -- (4.0,1.1);
	\draw [myGreen, thick] (5.0,1.9) -- (6.0,1.1);
	\draw [myGreen, thick] (6.0,0.9) -- (6.0,0.1);
%level 0
\draw [red] (1.1,2.0) to [out=60 , in=180 ] (4.9,2.0);
\node[above] at (2.0,2.5) {1};
%level 1
\draw [red] (0.1,1.0) to [out=30 , in=150 ] (3.9,1.0);
\node[above] at (2.0,1.5) {1};
\draw [red] (0.1,1.0) to [out=-30 , in=210] (5.9,1.0);
\node[above] at (1.0,0.5) {0};
\draw [red] (2.1,1.0) to [out=-30 , in=180 ] (3.9,1.0);
\node[above] at (3.0,0.5) {0};
\draw [red] (2.1,1.0) to [out=30 , in=180] (5.9,1.0);
\node[above] at (5.5,1.0) {1};
%level 2
\draw [red] (2.1,0.0) to [out=330 , in=180] (5.9,0.0);
\node[above left] at (6.0,0.0) {1};


\end{tikzpicture}
\caption{Soluzione della rete di flusso \label{fig:placeFlowRis}}
\end{figure*}

Riassumendo, dobbiamo tradurre in vincoli il fatto che la funzione di traduzione di supporto sia biiettiva e che conservi la struttura del place graph $P_F$. Distingueremo quindi in vincoli di flusso, che serviranno per il primo problema, e in vincoli strutturali, per il secondo.

Incominciamo con i \textbf{vincoli strutturali}: si � adottata una versione ricorsiva per i vincoli. Distinguiamo quindi in caso base e passo ricorsivo.
\begin{itemize}
	\item
	\emph{Caso Base}: due nodi della stessa altezza che hanno un numero diverso di figli \emph{non} possono costituire un match. In formule:
	\begin{center}
	$M_{d,m,n} = 0 \qquad$ se $|prnt_F^{-1}(m)| \ne |prnt_G^{-1}(n)|$ \\ $\qquad \forall d\leq depth - 1$ \\ $ \qquad \qquad \forall m \in P_F^d$ e $n \in P_G^d$
	\end{center}
	Nella figura sottostante, si vede subito che la variabile $M_{0,r_0,r_0}=0$ perch� $prnt_F^{-1}(r_0)=1 \ne 2=prnt_G^{-1}(r_0)$.
	
	\begin{figure*}[th]
	\centering
	\begin{tikzpicture}
	%\draw[help lines] (0,0) grid (10,5);
	%root 1
	\draw [fill] (1.0,2.0) circle [radius=0.1];
	%level 1.1
	\draw [fill] (0.0,1.0) circle [radius=0.1];
	\draw [fill] (2.0,1.0) circle [radius=0.1];
	%level 1.2
	\draw [fill] (2.0,0.0) circle [radius=0.1];
	%tree-edges G
	\draw [myGreen, thick] (1.0,1.9) -- (0.0,1.1);
	\draw [myGreen, thick] (1.0,1.9) -- (2.0,1.1);
	\draw [myGreen, thick] (2.0,0.9) -- (2.0,0.1);
	
	%root 2
	\draw [fill] (3.0,2.0) circle [radius=0.1];
	\node[above left] at (3.0,2.0){$r_0$};
	%level 2.1
	\draw [fill] (3.0,1.0) circle [radius=0.1];
	%level 2.2
	\draw [fill] (3.0,0.0) circle [radius=0.1];
	%tree-edges G
	\draw [myGreen, thick] (3.0,1.9) -- (3.0,1.1);
	\draw [myGreen, thick] (3.0,0.9) -- (3.0,0.1);
	
	%%%%%%%%%%%%%%%%%%%%%
	
	%root 1
	\draw [fill] (8.0,2.0) circle [radius=0.1];
	\node[above right] at (8.0,2.0){$r_0$};
	%level 1.1
	\draw [fill] (7.0,1.0) circle [radius=0.1];
	\draw [fill] (9.0,1.0) circle [radius=0.1];
	%level 1.2
	\draw [fill] (9.0,0.0) circle [radius=0.1];
	%tree-edges G
	\draw [myGreen, thick] (8.0,1.9) -- (7.0,1.1);
	\draw [myGreen, thick] (8.0,1.9) -- (9.0,1.1);
	\draw [myGreen, thick] (9.0,0.9) -- (9.0,0.1);
	
	%root 2
	\draw [fill] (6.0,2.0) circle [radius=0.1];
	%level 2.1
	\draw [fill] (6.0,1.0) circle [radius=0.1];
	%level 2.2
	\draw [fill] (6.0,0.0) circle [radius=0.1];
	%tree-edges G
	\draw [myGreen, thick] (6.0,1.9) -- (6.0,1.1);
	\draw [myGreen, thick] (6.0,0.9) -- (6.0,0.1);
	
	%%%%%%%%%%%%%%%%%%%%%%
	
	\draw[red] (3.0,2.1) to [out=20,in=160] (8.0,2.1);
	\node[above] at (5.0,2.5){0}; 
	
	\end{tikzpicture}
	\caption{Esempio per il primo vincolo \label{fig:firstConstraintExe}}
	\end{figure*}
	Si noti che questo vincolo viene applicato a tutti i nodi tranne ai siti, come specificato dalla condizione $\forall d\leq depth - 1$.
	
	\item
	\emph{Caso Ricorsivo}:  se due nodi alla stessa altezza \emph{non} costituiscono un match, allora neanche i loro figli lo fanno. In formule:
	\begin{center}
	$M_{d,m,n} \le M_{d-1,prnt(m), prnt(n)} \qquad \forall d \ge 1\qquad \qquad$ \\ $\qquad \qquad \qquad \qquad \qquad \qquad \forall m \in P_F^d$ \\ $\qquad \qquad \qquad \qquad \qquad \qquad 	\forall n \in P_G^d$
	\end{center}
	Questo vincolo modella la seguente implicazione: 
	\begin{center}
	$M_{d-1,prnt(m), prnt(n)}=0 \qquad \Rightarrow \qquad M_{d,m,n}=0$
	\end{center}
	Esso costituisce il caso ricorsivo perch� vale per tutti i nodi tranne le radici, e le radici sono ricoperte dal caso base. Si veda la figura \ref{fig:secondConstraintExe}: la variabile 
	$M_{1,n_1,n_2}=0$ perch� dalla figura \ref{fig:firstConstraintExe} sappiamo che $M_{0,r_0,r_0}=0$ e perch� $prnt_F(n_1)=r_0$ e $prnt_G(n_2)=r_0$. Lo stesso ragionamento vale per la variabile $M_{1,n_1,n_3}$. \\ \\ \\ \\ \\ \\
	
	
		
	\begin{figure*}[th]
	\centering
	\begin{tikzpicture}
	%\draw[help lines] (0,0) grid (10,5);
	%root 1
	\draw [fill] (1.0,2.0) circle [radius=0.1];
	%level 1.1
	\draw [fill] (0.0,1.0) circle [radius=0.1];
	\draw [fill] (2.0,1.0) circle [radius=0.1];
	%level 1.2
	\draw [fill] (2.0,0.0) circle [radius=0.1];
	%tree-edges G
	\draw [myGreen, thick] (1.0,1.9) -- (0.0,1.1);
	\draw [myGreen, thick] (1.0,1.9) -- (2.0,1.1);
	\draw [myGreen, thick] (2.0,0.9) -- (2.0,0.1);
	
	%root 2
	\draw [fill] (3.0,2.0) circle [radius=0.1];
	\node[above left] at (3.0,2.0){$r_0$};
	%level 2.1
	\draw [fill] (3.0,1.0) circle [radius=0.1];
	\node[above left] at (3.0,1.0) {$n_1$};
	%level 2.2
	\draw [fill] (3.0,0.0) circle [radius=0.1];
	%tree-edges G
	\draw [myGreen, thick] (3.0,1.9) -- (3.0,1.1);
	\draw [myGreen, thick] (3.0,0.9) -- (3.0,0.1);
	
	%%%%%%%%%%%%%%%%%%%%%
	
	%root 1
	\draw [fill] (8.0,2.0) circle [radius=0.1];
	\node[above right] at (8.0,2.0){$r_0$};
	%level 1.1
	\draw [fill] (7.0,1.0) circle [radius=0.1];
	\node[above left] at (7.0,1.0){$n_2$};
	\draw [fill] (9.0,1.0) circle [radius=0.1];
	\node[above right] at (9.0,1.0){$n_3$};
	%level 1.2
	\draw [fill] (9.0,0.0) circle [radius=0.1];
	%tree-edges G
	\draw [myGreen, thick] (8.0,1.9) -- (7.0,1.1);
	\draw [myGreen, thick] (8.0,1.9) -- (9.0,1.1);
	\draw [myGreen, thick] (9.0,0.9) -- (9.0,0.1);
	
	%root 2
	\draw [fill] (6.0,2.0) circle [radius=0.1];
	%level 2.1
	\draw [fill] (6.0,1.0) circle [radius=0.1];
	%level 2.2
	\draw [fill] (6.0,0.0) circle [radius=0.1];
	%tree-edges G
	\draw [myGreen, thick] (6.0,1.9) -- (6.0,1.1);
	\draw [myGreen, thick] (6.0,0.9) -- (6.0,0.1);
	
	%%%%%%%%%%%%%%%%%%%%%%
	
	\draw[red] (3.0,2.1) to [out=20,in=160] (8.0,2.1);
	\node[above] at (5.0,2.5){0}; 
	\draw[red] (3.1,1.0) to [out=20,in=170] (6.9,1.0);
	\node[above] at (4.0,1.2){0};
	\draw[red] (3.1,1.0) to [out=-20,in=190] (8.9,1.0);
	\node[above] at (5.0,0.5){0};
	
	\end{tikzpicture}
	\caption{Esempio per il secondo vincolo \label{fig:secondConstraintExe}}
	\end{figure*}
\end{itemize}


Questi vincoli non sono per� sufficienti per determinare un isomorfismo tra i due place graph $P_F$ e $P_G$. In particolare, la traduzione di supporto deve essere biiettiva, cio� associare un nodo/radice/sito di $P_F$ ad uno ed un solo nodo/radice/sito di $P_G$. Si sono quindi aggiunti i \textbf{vincoli di flusso}.

\begin{itemize}
	\item
	\emph{Flusso in uscita}: il flusso totale in uscita da ogni nodo deve essere esattamente pari a 1. 
	\begin{notaz}
	Questo vincolo lo indicheremo con la notazione: $\delta^{+}(m)=1$, dove $m \in P_F^d$.
	\end{notaz}
	Esso si traduce nel fatto che la somma di tutte le variabili in uscita da ogni nodo/radice/sito di $P_F$ deve essere 1, ovvero: ogni nodo di F pu� costituire un match solamente con uno ed un solo altro nodo di G. In formule:
	\begin{center}
	$\sum\limits_{n}M_{d,m,n} = 1 \qquad 0 \le d \le depth \qquad  $ \\ $ \qquad \qquad m \in P_F^d$ \\ $ \qquad \qquad n \in P_G^d$
	\end{center}
	
	\begin{figure*}[th]
	\centering
	\begin{tikzpicture}
	%\draw[help lines] (0,0) grid (10,5);
	%root 1
	\draw [fill] (1.0,2.0) circle [radius=0.1];
	%level 1.1
	\draw [fill] (0.0,1.0) circle [radius=0.1];
	\draw [fill] (2.0,1.0) circle [radius=0.1];
	%level 1.2
	\draw [fill] (2.0,0.0) circle [radius=0.1];
	%tree-edges G
	\draw [myGreen, thick] (1.0,1.9) -- (0.0,1.1);
	\draw [myGreen, thick] (1.0,1.9) -- (2.0,1.1);
	\draw [myGreen, thick] (2.0,0.9) -- (2.0,0.1);
	
	%root 2
	\draw [fill] (3.0,2.0) circle [radius=0.1];
	%level 2.1
	\draw [fill] (3.0,1.0) circle [radius=0.1];
	\node[above left] at (3.0,1.0) {$n_1$};
	%level 2.2
	\draw [fill] (3.0,0.0) circle [radius=0.1];
	%tree-edges G
	\draw [myGreen, thick] (3.0,1.9) -- (3.0,1.1);
	\draw [myGreen, thick] (3.0,0.9) -- (3.0,0.1);
	
	%%%%%%%%%%%%%%%%%%%%%
	
	%root 1
	\draw [fill] (8.0,2.0) circle [radius=0.1];
	%level 1.1
	\draw [fill] (7.0,1.0) circle [radius=0.1];
	\node[right] at (7.0,1.0){$n_2$};
	\draw [fill] (9.0,1.0) circle [radius=0.1];
	\node[above right] at (9.0,1.0){$n_3$};
	%level 1.2
	\draw [fill] (9.0,0.0) circle [radius=0.1];
	%tree-edges G
	\draw [myGreen, thick] (8.0,1.9) -- (7.0,1.1);
	\draw [myGreen, thick] (8.0,1.9) -- (9.0,1.1);
	\draw [myGreen, thick] (9.0,0.9) -- (9.0,0.1);
	
	%root 2
	\draw [fill] (6.0,2.0) circle [radius=0.1];
	%level 2.1
	\draw [fill] (6.0,1.0) circle [radius=0.1];
	\node[above right] at (6.0,1.0){$n_1$};
	%level 2.2
	\draw [fill] (6.0,0.0) circle [radius=0.1];
	%tree-edges G
	\draw [myGreen, thick] (6.0,1.9) -- (6.0,1.1);
	\draw [myGreen, thick] (6.0,0.9) -- (6.0,0.1);
	
	%%%%%%%%%%%%%%%%%%%%%%
	
	\draw[red] (3.1,1.0) to [out=50,in=130] (6.9,1.0);%n1-n2
	\node at (4.0,2.0){0};
	\draw[red] (3.1,1.0) to [out=-30,in=210] (8.9,1.0);%n1-n3
	\node[above] at (4.0,0.0){0};
	\draw[red] (3.1,1.0) to [out=20,in=170] (5.9,1.0);%n1-n1
	\node[above] at (5.0,1.2){1};
	
	
	\end{tikzpicture}
	\caption{Esempio per il vincolo sul flusso in uscita \label{fig:thirdConstraintExe}}
	\end{figure*}
	In figura \ref{fig:thirdConstraintExe} si pu� vedere come la somma degli archi in uscita dal nodo $n_1$ sia 1, ovvero: 
	$M_{1,n_1,n_1}+M_{1,n_1,n_2}+M_{1,n_1,n_3}=1$. \\
	
	\item
	\emph{Flusso in entrata}: il flusso totale in entrata da ogni nodo deve essere esattamente pari a 1.
	\begin{notaz}
	Questo vincolo lo indicheremo con la notazione: $\delta^{-}(n)=1$, dove $n \in P_G^d$.
	\end{notaz}
	Esso � equivalente a dire che la somma di tutte le variabili in entrata da ogni nodo/radice/sito di $P_G$ deve essere 1. In altre parole: ogni nodo di G pu� costituire un match solamente con uno ed un solo altro nodo di F. In formule:
	\begin{center}
	$\sum\limits_{m}M_{d,m,n} = 1 \qquad 0 \le d \le depth \qquad  $ \\ $ \qquad \qquad m \in P_F^d$ \\ $ \qquad \qquad n \in P_G^d$
	\end{center}
	
	
	
	\begin{figure*}[th]
	\centering
	\begin{tikzpicture}
	%\draw[help lines] (0,0) grid (10,5);
	%root 1
	\draw [fill] (1.0,2.0) circle [radius=0.1];
	%level 1.1
	\draw [fill] (0.0,1.0) circle [radius=0.1];
	\node[above left] at (0.0,1.0){$n_2$};
	\draw [fill] (2.0,1.0) circle [radius=0.1];
	\node[below left] at (2.0,1.0){$n_3$};
	%level 1.2
	\draw [fill] (2.0,0.0) circle [radius=0.1];
	%tree-edges G
	\draw [myGreen, thick] (1.0,1.9) -- (0.0,1.1);
	\draw [myGreen, thick] (1.0,1.9) -- (2.0,1.1);
	\draw [myGreen, thick] (2.0,0.9) -- (2.0,0.1);
	
	%root 2
	\draw [fill] (3.0,2.0) circle [radius=0.1];
	%level 2.1
	\draw [fill] (3.0,1.0) circle [radius=0.1];
	\node[above left] at (3.0,1.0) {$n_1$};
	%level 2.2
	\draw [fill] (3.0,0.0) circle [radius=0.1];
	%tree-edges G
	\draw [myGreen, thick] (3.0,1.9) -- (3.0,1.1);
	\draw [myGreen, thick] (3.0,0.9) -- (3.0,0.1);
	
	%%%%%%%%%%%%%%%%%%%%%
	
	%root 1
	\draw [fill] (8.0,2.0) circle [radius=0.1];
	%level 1.1
	\draw [fill] (7.0,1.0) circle [radius=0.1];
	\draw [fill] (9.0,1.0) circle [radius=0.1];
	%level 1.2
	\draw [fill] (9.0,0.0) circle [radius=0.1];
	%tree-edges G
	\draw [myGreen, thick] (8.0,1.9) -- (7.0,1.1);
	\draw [myGreen, thick] (8.0,1.9) -- (9.0,1.1);
	\draw [myGreen, thick] (9.0,0.9) -- (9.0,0.1);
	
	%root 2
	\draw [fill] (6.0,2.0) circle [radius=0.1];
	%level 2.1
	\draw [fill] (6.0,1.0) circle [radius=0.1];
	\node[above right] at (6.0,1.0){$n_1$};
	%level 2.2
	\draw [fill] (6.0,0.0) circle [radius=0.1];
	%tree-edges G
	\draw [myGreen, thick] (6.0,1.9) -- (6.0,1.1);
	\draw [myGreen, thick] (6.0,0.9) -- (6.0,0.1);
	
	%%%%%%%%%%%%%%%%%%%%%%
	
	\draw[red] (0.1,1.0) to [out=30,in=150] (5.9,1.0);%n2-n1
	\node at (4.0,2.0){0};
	\draw[red] (2.1,1.0) to [out=-30,in=210] (5.9,1.0);%n3-n1
	\node[above] at (4.0,0.0){0};
	\draw[red] (3.1,1.0) to [out=20,in=170] (5.9,1.0);%n1-n1
	\node[above] at (4.0,1.15){1};
	
	
	\end{tikzpicture}
	\caption{Esempio per il vincolo sul flusso in entrata \label{fig:fourthConstraintExe}}
	\end{figure*}
	In figura \ref{fig:fourthConstraintExe}, � stato espresso il vincolo che il flusso in entrata verso il nodo $n_1$ deve essere 1, cio�: 
	$M_{1,n_1,n_1}+M_{1,n_2,n_1}+M_{1,n_3,n_1}=1$.
	
\end{itemize}

Si capisce bene come questi due ultimi vincoli assicurino che la traduzione di supporto sia rispettivamente iniettiva e suriettiva, rendendola quindi \emph{biiettiva} come si voleva.



%%%   LINK GRAPH   %%%
\subsection{Vincoli per il link graph}
L'isomorfismo tra link graph � un isomorfismo tra ipergrafi. In questo problema, sfruttiamo soprattutto la definizione di link graph: la sua struttura � definita dalla funzione \emph{link}, che collega Punti a Handles. I primi sono l'insieme delle porte e degli inner names, mentre i secondi sono l'insieme degli archi e degli outer names (sottosezione \ref{sub:linkGraph}). Si pu� quindi vedere ogni link graph come una funzione che ha come dominio i Punti e come codominio gli Handles. 

Sulla base di queste osservazioni, possiamo costruire la \emph{rete di flusso} per questo problema nel seguente modo: innanzitutto chiamiamo i due link graph rispettivamente $L_F$ e $L_G$, e le loro funzioni come $link_F$ e $link_G$ (definite dagli archi verdi di figura \ref{fig:linkFlow}). Possiamo collegare tutti gli elementi del dominio di $link_F$ a tutti gli elementi del dominio di $link_G$, creando cos� archi \emph{orientati}, perch� vanno da punti di $L_F$ a punti di $L_G$, e \emph{pesati}, perch� possono assumere un valore $p \in \{0, 1\}$.
Infine, facciamo lo stesso con i loro Handle: colleghiamo tutti gli elementi del codominio di $link_F$ a tutti gli elementi del codominio di $link_G$. \\
Si � cos� creata la rete di flusso in figura \ref{fig:linkFlow}, dove $D_X$ (con $X \in \{ F, G\}$) indica il dominio di $link_X$ e $C_X$ il suo codominio. \\
Per chiarezza visiva, si sono omessi alcuni archi rossi, ma si deve immaginare che ogni elemento di $D_F$ abbiamo tre archi verso ognuno degli elementi di $D_G$. Lo stesso vale per $C_F$.


\begin{figure*}[th]
	\centering
	\begin{tikzpicture}
	%\draw[help lines] (0,0) grid (8,5);
	%%%   First Link Graph    %%%
	%%%   Domain   %%%
	\draw[rounded corners = 2mm, dashed] (0.8,2.8) rectangle (2.0,4.2);
	\node[above] at (1.5,4.2) {$D_F$};
	%%%   Codomain   %%%
	\draw[rounded corners = 2mm, dashed] (0.8,0.8) rectangle (2.0,2.2);
	\node[above] at (1.5,0.1) {$D_G$};
	%points G
	\draw [fill] (1.5,3.0) circle [radius=0.1];%p3
	\draw [fill] (1.0,3.5) circle [radius=0.1];%p2
	\draw [fill] (1.5,4.0) circle [radius=0.1];%p1
	%handles G
	\draw [fill] (4.5,3.0) circle [radius=0.1];%h3
	\draw [fill] (5.0,3.5) circle [radius=0.1];%h2
	\draw [fill] (4.5,4.0) circle [radius=0.1];%h1
	%hyper-edges G
	\draw [myGreen, thick] (1.6,4.0) -- (4.4,4.0);
	\draw [myGreen, thick] (1.1,3.5) -- (4.9,3.5);
	\draw [myGreen, thick] (1.6,3.0) to [out=0,in=180] (3.5,3.5);
	
	%%%   Second Link Graph    %%%
	%%%   Domain   %%%
	\draw[rounded corners = 2mm, dashed] (4.0,2.8) rectangle (5.2,4.2);
	\node[above] at (4.7,4.2) {$C_F$};
	%%%   Codomain   %%%
	\draw[rounded corners = 2mm, dashed] (4.0,0.8) rectangle (5.2,2.2);
	\node[above] at (4.7,0.1) {$C_G$};
	%points H
	\draw [fill] (1.5,1.0) circle [radius=0.1];%p3
	\draw [fill] (1.0,1.5) circle [radius=0.1];%p2
	\draw [fill] (1.5,2.0) circle [radius=0.1];%p1
	%handles H
	\draw [fill] (4.5,1.0) circle [radius=0.1];%h3
	\draw [fill] (5.0,1.5) circle [radius=0.1];%h2
	\draw [fill] (4.5,2.0) circle [radius=0.1];%h1
	%hyper-edges H
	\draw [myGreen, thick] (1.6,2.0) -- (4.4,2.0);
	\draw [myGreen, thick] (1.1,1.5) -- (4.9,1.5);
	\draw [myGreen, thick] (1.6,1.0) to [out=0,in=180] (3.5,1.5);
	
	%%%   Red Edges   %%%
	\draw [red, ->] (1.4,3.0) to [out=180,in=180] (1.4,2.0);%p3-p1
	\draw [red, ->] (1.4,3.0) to [out=180,in=150] (0.9,1.5);%p3-p2
	\draw [red, ->] (1.4,3.0) to [out=180,in=120] (1.4,1.0);%p3-p3
	
	\draw [red, ->] (4.6,3.0) to [out=0,in=0] (4.6,2.0);%h3-h1
	\draw [red, ->] (4.6,3.0) to [out=0,in=30] (5.1,1.5);%h3-h2
	\draw [red, ->] (4.6,3.0) to [out=0,in=60] (4.6,1.0);%h3-h3
	
	
	\end{tikzpicture}
	\caption{Esempio di rete di flusso per il problema di isomorfismo tra link graphs. \label{fig:linkFlow}}
\end{figure*}


Incominciamo quindi con definire le variabili del sistema di equazioni:
\begin{center}
$x_{h,h'} \in \{0, 1\} \qquad h \in C_F = Y_F \uplus E_F \qquad$ \\ $\qquad \qquad \qquad \qquad h' \in C_G = Y_G \uplus E_G \qquad$
\end{center}
\begin{center}
$y_{p,p'} \in \{0, 1\} \qquad p \in D_F = X_F \uplus P_F \qquad$ \\ $\qquad \qquad \qquad \qquad p' \in D_G = X_G \uplus P_G \qquad$
\end{center}

Si sono distinte le variabili che vanno dal dominio $D_F$ a $D_G$, che vengono chiamate $y$, da quelle che vanno dal codominio $C_F$ a $C_G$, chiamate $x$.
Le soluzioni del sistema hanno lo stesso significato che avevano nel place graph: dopo l'esecuzione, le variabili che assumeranno il valore 1 saranno solamente quelle che costituiranno la vera e propria traduzione di supporto. Quindi, anche per questo problema, valgono la proposizione \ref{prop:traduzioneSupporto} e la notazione \ref{term:matchVariabili}.

I vincoli che si devono scegliere hanno il compito di ``costituire" la funzione di traduzione di supporto, e devono quindi assicurare che essa sia biiettiva e che mantenga la struttura del primo link graph. Perci�, anche in questo caso, distinguiamo in \emph{vincoli di flusso}, per il primo problema, e in \emph{vincoli strutturali}, per il secondo.

Incominciamo con il definire i \textbf{vincoli strutturali}. Essi hanno il compito di definire una funzione che va dal primo link graph al secondo, che sia in grado di mantenere la struttura del primo. In altre parole, devono controllare che le due strutture siano \emph{compatibili}. I vincoli strutturali fanno riferimento al caso negativo, cio� descrivono nel sistema quando due punti o due handle \emph{non} possono costituire un match. Da qui i due vincoli:

\begin{itemize}
	\item
	\emph{Primo vincolo strutturale}: due handles (il primo di $L_F$ e il secondo di $L_G$) che hanno un numero diverso di pre-immagini \emph{non} possono costituire un match. In altre parole: se l'handle $h$ � immagine di due punti ma l'handle $h'$ lo � di uno solo, allora $h$ e $h'$ non possono essere associati. In formule:
	\begin{center}
	$x_{h,h'} = 0 \qquad \qquad \qquad |link^{-1}_F(h)| \ne |link^{-1}_G(h')| \qquad $ \\ 
	$\qquad \qquad h \in C_F = Y_F \uplus E_F$ \\ 
	$\qquad \qquad h' \in C_G = Y_G \uplus E_G$
	\end{center}
	In figura \ref{fig:firstLinkConstraint}, si pu� vedere come la variabile $x_{h_3,h_1}$ sia vincolata ad assumere il valore 0, infatti: $|link_F^{-1}(h_3)| = 0$ ma 
	$|link_G^{-1}(h_1)| = 1$, quindi $x_{h_3,h_1} = 0$.
	
	\begin{figure*}[th]
	\centering
	\begin{tikzpicture}
	%\draw[help lines] (0,0) grid (8,5);
	%%%   First Link Graph    %%%
	%points G
	\draw [fill] (1.5,3.0) circle [radius=0.1];%p3
	\draw [fill] (1.0,3.5) circle [radius=0.1];%p2
	\draw [fill] (1.5,4.0) circle [radius=0.1];%p1
	%handles G
	\draw [fill] (4.5,3.0) circle [radius=0.1];%h3
	\draw [fill] (5.0,3.5) circle [radius=0.1];%h2
	\draw [fill] (4.5,4.0) circle [radius=0.1];%h1
	%hyper-edges G
	\draw [myGreen, thick] (1.6,4.0) -- (4.4,4.0);
	\draw [myGreen, thick] (1.1,3.5) -- (4.9,3.5);
	\draw [myGreen, thick] (1.6,3.0) to [out=0,in=180] (3.5,3.5);
	
	%%%   Second Link Graph    %%%
	%points H
	\draw [fill] (1.5,1.0) circle [radius=0.1];%p3
	\draw [fill] (1.0,1.5) circle [radius=0.1];%p2
	\draw [fill] (1.5,2.0) circle [radius=0.1];%p1
	%handles H
	\draw [fill] (4.5,1.0) circle [radius=0.1];%h3
	\draw [fill] (5.0,1.5) circle [radius=0.1];%h2
	\draw [fill] (4.5,2.0) circle [radius=0.1];%h1
	%hyper-edges H
	\draw [myGreen, thick] (1.6,2.0) -- (4.4,2.0);
	\draw [myGreen, thick] (1.1,1.5) -- (4.9,1.5);
	\draw [myGreen, thick] (1.6,1.0) to [out=0,in=180] (3.5,1.5);
	
	%%%   Red Edges   %%%
	\draw [red, ->] (4.6,3.0) to [out=0,in=0] (4.6,2.0);%h3-h1
	\node[right] at (5.0,3.0){$h_3$};
	\node[right] at (5.0,2.0){$h_1$};
	
	\end{tikzpicture}
	\caption{Esempio per il primo vincolo. \label{fig:firstLinkConstraint}}
	\end{figure*}

	
	\item
	\emph{Secondo vincolo strutturale}: se due handles ($h \in C_F$ e $h' \in C_G$) non costituiscono un match, allora neanche i punti che hanno $h$ e $h'$ come immagine lo fanno. Questo vincolo modella la seguente implicazione:
	\begin{center}
	$x_{h,h'} = 0 \qquad \Rightarrow \qquad y_{p,p'} = 0$
	\end{center}
	dove $p \in link^{-1}_F(h)$ e $p' \in link^{-1}_G(h')$. Esso pu� essere tradotto tramite la seguente equazione.
	\begin{center}
	$y_{p,p'} \le x_{link_F(p),link_G(p')} \qquad \qquad \qquad p \in D_F = X_F \uplus P_F \qquad \qquad \qquad \qquad$ \\ 
	$\qquad \qquad \qquad \qquad p' \in D_G = X_G \uplus P_G$
	\end{center}
	Si noti che questa equazione equivale all'implicazione precedente: se $x=0$, allora deve per forza essere che anche $y=0$.
	
	
	\begin{figure*}[th]
	\centering
	\begin{tikzpicture}
	%\draw[help lines] (0,0) grid (8,5);
	%%%   First Link Graph    %%%
	%points G
	\draw [fill] (1.5,3.0) circle [radius=0.1];%p3
	\node [above] at (1.5,3.0) {$p_3$};
	\draw [fill] (1.0,3.5) circle [radius=0.1];%p2
	\node [above] at (1.0,3.5) {$p_2$};
	\draw [fill] (1.5,4.0) circle [radius=0.1];%p1
	%handles G
	\draw [fill] (4.5,3.0) circle [radius=0.1];%h3
	\draw [fill] (5.0,3.5) circle [radius=0.1];%h2 	
	\node[right] at (5.0,3.5){$h_2$};
	\draw [fill] (4.5,4.0) circle [radius=0.1];%h1
	%hyper-edges G
	\draw [myGreen, thick] (1.6,4.0) -- (4.4,4.0);
	\draw [myGreen, thick] (1.1,3.5) -- (4.9,3.5);
	\draw [myGreen, thick] (1.6,3.0) to [out=0,in=180] (3.5,3.5);
	
	%%%   Second Link Graph    %%%
	%points H
	\draw [fill] (1.5,1.0) circle [radius=0.1];%p3
	\draw [fill] (1.0,1.5) circle [radius=0.1];%p2
	\draw [fill] (1.5,2.0) circle [radius=0.1];%p1
	\node [below] at (1.5,2.0) {$p_1$};
	%handles H
	\draw [fill] (4.5,1.0) circle [radius=0.1];%h3
	\draw [fill] (5.0,1.5) circle [radius=0.1];%h2
	\draw [fill] (4.5,2.0) circle [radius=0.1];%h1
	\node[right] at (5.0,2.0){$h_1$};
	%hyper-edges H
	\draw [myGreen, thick] (1.6,2.0) -- (4.4,2.0);
	\draw [myGreen, thick] (1.1,1.5) -- (4.9,1.5);
	\draw [myGreen, thick] (1.6,1.0) to [out=0,in=180] (3.5,1.5);
	
	%%%   Red Edges   %%%
	\draw [red, ->] (5.0,3.4) to [out=0,in=0] (4.6,2.0);%h2-h1
	\node at (5.5,2.5) {0};
	\draw [red, ->] (0.9,3.5) to [out=210,in=180] (1.4,2.0);%p2-p1
	\node at (0.5,3.0) {0};
	\draw [red, ->] (1.4,3.0) to [out=210,in=150] (1.4,2.0);%p3-p1
	\node at (1.5,2.5) {0};
	
	\end{tikzpicture}
	\caption{Esempio per il secondo vincolo. \label{fig:secondLinkConstraint}}
	\end{figure*}
	
	In figura \ref{fig:secondLinkConstraint}, si pu� vedere un esempio per questo vincolo. Dal primo vincolo sappiamo che $x_{h_2,h_1} = 0$ perch� 
	$|link_F^{-1}(h_2)| = 2$ mentre $|link_G^{-1}(h_1)| = 1$. Aggiungendo il vincolo appena descritto, ricaviamo quindi che $y_{p_2,p_1} = 0$ e $y_{p_3,p_1} = 0$. Infatti, si vede subito che i punti $p_2$ e $p_3$ non possono costituire un match con $p_1$.
\end{itemize}

Descriviamo adesso i \textbf{vincoli di flusso}, che consentono di avere una funzione biiettiva. Il caso � analogo a quello per il place graph e la notazione rimane la stessa.

\begin{itemize}
	\item
	\emph{Flusso in uscita}:
		\begin{itemize}
			\item
			\emph{$\delta^+(p)=1$}:
			Dobbiamo assicurare che ogni punto di $D_F$ sia associato ad uno e un solo punto di $D_G$, che si traduce in questa equazione:
				\begin{center}
				$\sum\limits_{p'}y_{p,p'} = 1 \qquad \qquad \qquad p \in X_F \uplus P_F \qquad p' \in X_G \uplus P_G$
				\end{center}
			
									
				
	\begin{figure*}[th]
	\centering
	\begin{tikzpicture}
	%\draw[help lines] (0,0) grid (8,5);
	%%%   First Link Graph    %%%
	%points G
	\draw [fill] (1.5,3.0) circle [radius=0.1];%p3
	\node [above] at (1.5,3.0) {$p_3$};
	\draw [fill] (1.0,3.5) circle [radius=0.1];%p2
	\draw [fill] (1.5,4.0) circle [radius=0.1];%p1
	%handles G
	\draw [fill] (4.5,3.0) circle [radius=0.1];%h3
	\node[above] at (4.5,3.0){$h_3$};
	\draw [fill] (5.0,3.5) circle [radius=0.1];%h2
	\draw [fill] (4.5,4.0) circle [radius=0.1];%h1
	%hyper-edges G
	\draw [myGreen, thick] (1.6,4.0) -- (4.4,4.0);
	\draw [myGreen, thick] (1.1,3.5) -- (4.9,3.5);
	\draw [myGreen, thick] (1.6,3.0) to [out=0,in=180] (3.5,3.5);
	
	%%%   Second Link Graph    %%%
	%points H
	\draw [fill] (1.5,1.0) circle [radius=0.1];%p3
	\draw [fill] (1.0,1.5) circle [radius=0.1];%p2
	\draw [fill] (1.5,2.0) circle [radius=0.1];%p1
	%handles H
	\draw [fill] (4.5,1.0) circle [radius=0.1];%h3
	\draw [fill] (5.0,1.5) circle [radius=0.1];%h2
	\draw [fill] (4.5,2.0) circle [radius=0.1];%h1
	%hyper-edges H
	\draw [myGreen, thick] (1.6,2.0) -- (4.4,2.0);
	\draw [myGreen, thick] (1.1,1.5) -- (4.9,1.5);
	\draw [myGreen, thick] (1.6,1.0) to [out=0,in=180] (3.5,1.5);
	
	%%%   Red Edges   %%%
	\draw [red, ->] (1.4,3.0) to [out=230,in=130] (1.4,2.0);%p3-p1
	\node at (0.5,2.0) {0};
	\draw [red, ->] (1.4,3.0) to [out=180,in=180] (0.9,1.5);%p3-p2
	\node at (1.0,1.0) {1};
	\draw [red, ->] (1.4,3.0) to [out=180,in=180] (1.4,1.0);%p3-p3
	\node at (1.5,2.5) {0};
	
	\draw [red, ->] (4.6,3.0) to [out=0,in=0] (4.6,2.0);%h3-h1
	\node at (4.5,2.5) {0};
	\draw [red, ->] (4.6,3.0) to [out=0,in=0] (5.1,1.5);%h3-h2
	\node at (5.5,2.0) {0};
	\draw [red, ->] (4.6,3.0) to [out=0,in=0] (4.6,1.0);%h3-h3
	\node at (5.0,0.8) {1};
	
	
	\end{tikzpicture}
	\caption{Esempio per i due vincoli sul flusso in uscita. \label{fig:outFlowLinkConstraint}}
	\end{figure*}
			
			\item
			\emph{$\delta^+(h)=1$}:
			E' l'analogo del caso precedente. Ogni handle di $C_F$ deve essere associato ad uno e un solo handle di $C_G$. La rispettiva equazione �:
				\begin{center}
				$\sum\limits_{h'}y_{h,h'} = 1 \qquad \qquad \qquad h \in Y_F \uplus E_F \qquad p' \in Y_F \uplus E_F$
				\end{center}

			
			La figura \ref{fig:outFlowLinkConstraint} mostra un esempio di questi due vincoli.

		\end{itemize}
		
		
	\item
	\emph{Flusso in entrata}:
		\begin{itemize}
			\item
			\emph{$\delta^-(p')=1$}: ogni punto di $D_G$ pu� essere associato ad uno e un solo punto di $D_F$. In formule:
				\begin{center}
				$\sum\limits_{p}y_{p,p'} = 1 \qquad \qquad \qquad p \in X_F \uplus P_F \qquad p' \in X_G \uplus P_G$
				\end{center}
				
			
			\item
			\emph{$\delta^-(h')=1$}: ogni handle di $C_G$ pu� essere associato ad uno e un solo handle di $C_F$. In formule:
				\begin{center}
				$\sum\limits_{h}y_{h,h'} = 1 \qquad \qquad \qquad h \in Y_F \uplus E_F \qquad p' \in Y_G \uplus E_G$
				\end{center}
			La figura \ref{fig:inFlowLinkConstraint} mostra un esempio per questi due vincoli.
			
			
	\begin{figure*}[th]
	\centering
	\begin{tikzpicture}
	%\draw[help lines] (0,0) grid (8,5);
	%%%   First Link Graph    %%%
	%points G
	\draw [fill] (1.5,3.0) circle [radius=0.1];%p3
	\draw [fill] (1.0,3.5) circle [radius=0.1];%p2
	\draw [fill] (1.5,4.0) circle [radius=0.1];%p1
	%handles G
	\draw [fill] (4.5,3.0) circle [radius=0.1];%h3
	\draw [fill] (5.0,3.5) circle [radius=0.1];%h2
	\draw [fill] (4.5,4.0) circle [radius=0.1];%h1
	%hyper-edges G
	\draw [myGreen, thick] (1.6,4.0) -- (4.4,4.0);
	\draw [myGreen, thick] (1.1,3.5) -- (4.9,3.5);
	\draw [myGreen, thick] (1.6,3.0) to [out=0,in=180] (3.5,3.5);
	
	%%%   Second Link Graph    %%%
	%points H
	\draw [fill] (1.5,1.0) circle [radius=0.1];%p3
	\draw [fill] (1.0,1.5) circle [radius=0.1];%p2
	\draw [fill] (1.5,2.0) circle [radius=0.1];%p1
	\node[below] at (1.5,2.0){$p_1$};
	%handles H
	\draw [fill] (4.5,1.0) circle [radius=0.1];%h3
	\draw [fill] (5.0,1.5) circle [radius=0.1];%h2
	\draw [fill] (4.5,2.0) circle [radius=0.1];%h1
	\node[below] at (4.5,2.0){$h_1$};
	%hyper-edges H
	\draw [myGreen, thick] (1.6,2.0) -- (4.4,2.0);
	\draw [myGreen, thick] (1.1,1.5) -- (4.9,1.5);
	\draw [myGreen, thick] (1.6,1.0) to [out=0,in=180] (3.5,1.5);
	
	%%%   Red Edges   %%%
	\draw [red, ->] (1.4,4.0) to [out=180,in=180] (1.4,2.0);%p1-p1
	\node at (1.0,4.0) {1};
	\draw [red, ->] (0.9,3.5) to [out=180,in=180] (1.4,2.0);%p2-p1
	\node at (0.5,3.5) {0};
	\draw [red, ->] (1.4,3.0) to [out=230,in=130] (1.4,2.0);%p3-p1
	\node at (1.5,2.5) {0};
	
	\draw [red, ->] (4.6,4.0) to [out=0,in=0] (4.6,2.0);%h1-h1
	\node at (5.0,4.0) {1};
	\draw [red, ->] (5.1,3.5) to [out=0,in=0] (4.6,2.0);%h2-h1
	\node at (5.5,3.0) {0};
	\draw [red, ->] (4.6,3.0) to [out=0,in=0] (4.6,2.0);%h3-h1
	\node at (4.5,2.5) {0};
	
	
	\end{tikzpicture}
	\caption{Esempio per i due vincoli sul flusso in entrata. \label{fig:inFlowLinkConstraint}}
	\end{figure*}
			
			\end{itemize}
\end{itemize}

Si noti che i quattro vincoli di flusso assicurano la iniettivit� e la suriettivit� della funzione di traduzione di supporto, rendendola quindi \emph{biiettiva}.


\subsection{Vincoli di coerenza}
Nella sottosezione \ref{sub:isoExamples}, dedicata ad alcuni esempi, la figura \ref{fig:falseIso} mostrava due bigrafi che, pur avendo i place graphs ed i link graphs isomorfi, \emph{non} erano tali. Questo perch� il sistema di equazioni non era completo: non bastano cio� i vincoli visti fino ad ora, e bisogna integrarli con dei \textbf{vincoli di coerenza} che sono dedicati all'unione delle due soluzioni.

Nell'esempio citato, si era gi� notato informalmente che il quadrato collegato al cerchio nel primo link graph deve essere quello pi� esterno nel primo place graph. Diamo ora le definizioni formali di questi vincoli:

\begin{itemize}
	\item
	\emph{Primo vincolo di coerenza}: due porte costituiscono un match nei due link graphs \emph{se e solo se} i rispettivi nodi lo fanno nei due place graphs e le due porte hanno lo stesso indice. In formule:
		\begin{center}
		$y_{p,p'} = M_{d,m,m'} \qquad \qquad \qquad p=(m,i) \qquad p'=(m',i) \qquad \qquad$ \\ 
		$\qquad \qquad \qquad p \in P_F \qquad p' \in P_G$ \\ 
		$\qquad \qquad \qquad m \in V_F \qquad m' \in V_G$ \\ 
		$i \in \mathbb{N}$ 
		\end{center}
	In altre parole, se due nodi $m$ e $m'$ non costituiscono un match nei due place graphs, allora neanche le loro porte lo fanno. D'altra parte, se $m$ e $m'$ costituiscono un match, cio� $M_{d,m,m'}=1$, allora le porte con lo stesso indice devono costituire un match: $y_{p,p'} = 1$.
	
	\item
	\emph{Secondo vincolo di coerenza}: due porte con indici diversi non possono costituire un match. In formule:
		\begin{center}
		$y_{p,p'} = 0 \qquad \qquad \qquad p=(m,i) \qquad p'=(m',i) \qquad $ \\ 
		$\qquad \qquad \qquad p \in P_F \qquad p' \in P_G$ \\ 
		$\qquad \qquad \qquad m \in V_F \qquad m' \in V_G$ \\ 
		$\qquad \qquad \qquad i, i' \in \mathbb{N} \qquad i \ne i'$ 
		\end{center}
	
	\item
	\emph{Terzo vincolo di coerenza}: se due punti sono di tipo diverso, cio� il primo � una porta e il secondo un inner name, allora essi non possono costituire un match.
		\begin{center}
		$y_{p,p'} = 0 \qquad \qquad \qquad p \in P_F \qquad p' \in X_G \qquad $ \\ 
		\end{center}
		Banalmente, vale anche il caso speculare, dove il primo punto � un inner name mentre il secondo � una porta:
		\begin{center}
		$y_{p,p'} = 0 \qquad \qquad \qquad p \in X_F \qquad p' \in P_G \qquad $ \\ 
		\end{center}
		
	\item
	\emph{Quarto vincolo di coerenza}: se due handles sono di tipo diverso, cio� il primo � un arco e il secondo un outername, allora essi non possono costituire un match.
		\begin{center}
		$y_{h,h'} = 0 \qquad \qquad \qquad h \in E_F \qquad h' \in Y_G \qquad $ \\ 
		\end{center}
		Il caso speculare, dove il primo handle � un outer name name mentre il secondo � un arco, � dato dalla formula:
		\begin{center}
		$y_{h,h'} = 0 \qquad \qquad \qquad h \in Y_F \qquad h' \in E_G \qquad $ \\ 
		\end{center}
	
	\item
	\emph{Quinto vincolo di coerenza}: un nodo pu� costituire un match solo con un altro nodo. Lo stesso vale per le radici e per i siti. Perci�, � impossibile per esempio che esista una variabile con valore 1 da una radice verso un nodo. I vincoli che coprono tutti i possibili casi sono:
		\begin{center}
		$M_{d,a,b} = 0 \qquad \qquad$ se $ \qquad a \in n \qquad b \in V_G $\\
		oppure se $ \qquad \qquad \qquad a \in n \qquad b \in m $\\
		oppure se $ \qquad \qquad \qquad a \in m \qquad b \in V_G $\\
		oppure se $ \qquad \qquad \qquad a \in m \qquad b \in n $\\
		oppure se $ \qquad \qquad \qquad a \in V_F \qquad b \in m $\\
		oppure se $ \qquad \qquad \qquad a \in V_F \qquad b \in n $\\
		\end{center}
	\item
	\emph{Sesto vincolo di coerenza}: due nodi con controlli diversi non possono costituire un match. Per esempio, in figura \ref{fig:falseIso} il quadrato pi� grande del primo bigrafo non pu� essere associato con il cerchio del secondo.
	\begin{center}
	$M_{d,a,b} = 0 \qquad \qquad \qquad ctrl_F(a) \ne ctrl_G(b)$
	\end{center}
	
\end{itemize}


%%%%%%%%%%%%%%%%%%%%%%%%%%%%%%
\section{Grafo degli stati}
Sono stati elencati tutti i vincoli necessari per determinare l'isomorfismo tra bigrafi. Si � gi� visto che l'utilit� di sapere quando due bigrafi sono uguali risiede nel fatto che � possibile fermare l'esecuzione del BRS, evitando cos� cicli infiniti tra due bigrafi uguali. Per memorizzare tutti gli stati assunti da un BRS durante la sua evoluzione, si � creato il \emph{grafo degli stati}: � una struttura dati a grafo dove ogni nodo � a sua volta un bigrafo. 


	\begin{figure*}[th]
	\centering
	\begin{tikzpicture}
	%\draw[help lines] (0,0) grid (8,5);
	%States
	\draw[thick] (0.5,0.5) circle [radius=0.5];
	\node at (0.5,0.5) {$S_0$};
	\draw[thick] (2.5,2.5) circle [radius=0.5];
	\node at (2.5,2.5) {$S_1$};
	\draw[thick] (4.5,1.5) circle [radius=0.5];
	\node at (4.5,1.5) {$S_2$};
	\draw[thick] (4.5,3.5) circle [radius=0.5];
	\node at (4.5,3.5) {$S_3$};
	\draw[thick] (6.5,2.5) circle [radius=0.5];
	\node at (6.5,2.5) {$S_4$};
	%Edges
	\draw[->, thick] (0.85,0.85) -- (2.15,2.15);%s0-s1
	\node[above left] at (1.5,1.5) {$R_1$};
	\draw[->, thick] (3.0,2.5) -- (4.0,3.5);%s1-s3
	\node at (3.5,1.5) {$R_1$};
	\draw[->, thick] (3.0,2.5) -- (4.0,1.5);%s1-s2
	\node at (3.5,3.5) {$R_1$};
	\draw[->, thick] (5.0,3.5) -- (6.0,2.5);%s3-s4
	\node at (5.5,1.5) {$R_0$};
	\draw[->, thick] (5.0,1.5) -- (6.0,2.5);%s2-s4
	\node at (5.5,3.5) {$R_0$};
	\draw[->, thick] (6.5,2.0) to [out=270,in=0] (5.5,0.0) -- (3.5,0.0) to [out=180,in=270] (2.5,2.0) ;%s4-s1
	\node[above] at (4.5,0.0) {$R_2$};
	
	\end{tikzpicture}
	\caption{Esempio di grafo degli stati. \label{fig:stateGraphExe}}
	\end{figure*}


Come si pu� vedere dalla figura \ref{fig:stateGraphExe}, ogni nodo � etichettato con la stringa $S_i$, perch� si tratta di uno stato, cio� di un bigrafo. All'interno di un BRS, abbiamo visto che un bigrafo pu� evolversi tramite le regole di reazione, motivo per cui ogni arco orientato del grafo degli stati � etichettato con il nome della regola che ha portato dal primo stato al secondo. Nell'esempio di cui sopra, dal bigrafo iniziale $S_0$ si passa a $S_1$ tramite la regola $R_1$.

Si noti un aspetto molto importante: da $S_0$ a $S_1$ la regola $R_1$ ha generato un solo nuovo stato. Questo vuol dire che il redex di $R_1$ ha una sola occorrenza in $S_0$, ed essa � stata sostituita dal reactum di $R_1$, creando cos� lo stato $S_1$. Per�, se prendiamo in considerazione quest'ultimo stato, si vede che ora la regola $R_1$ porta a due nuovi stati. Questo perch� in $S_1$ ci sono due occorrenze del redex di $R_1$, che vengono sostituite dal suo reactum, creando rispettivamente i due nuovi stati $S_2$ e $S_3$. In altre parole: una regola di reazione pu� generare un diverso numero di stati a seconda del bigrafo a cui � applicata.

Infine, si presti attenzione all'arco tra $S_4$ e $S_1$. Il significato � il seguente: applico la regola $R_2$ al bigrafo $S_4$, generando un nuovo stato $S_5$. Esso � per� isomorfo a $S_1$, cio�: $S_1$ e $S_5$ hanno la stessa \emph{semantica}, per cui collego $S_4$ a $S_1$. Se non ci fossimo accorti di questa propiet�, allora avremmo continuato ad applicare le regole $R_0$, $R_1$ e $R_2$ all'\emph{infinito}, generando sempre nuovi stati, come in figura \ref{fig:infiniteStateGraphExe}. Ora invece, dato che ci siamo accorti che da $S_4$ siamo ritornati ad $S_1$, non applichiamo pi� nessuna regola, avendo quindi un grafo degli stati \emph{finito}, cio� quello di figura \ref{fig:stateGraphExe}.\\ \\




	\begin{figure*}[th]
	\centering
	\begin{tikzpicture}
	%\draw[help lines] (0,0) grid (14,5);
	%States
	\draw[thick] (0.5,0.5) circle [radius=0.5];
	\node at (0.5,0.5) {$S_0$};
	\draw[thick] (2.5,2.5) circle [radius=0.5];
	\node at (2.5,2.5) {$S_1$};
	\draw[thick] (4.5,1.5) circle [radius=0.5];
	\node at (4.5,1.5) {$S_2$};
	\draw[thick] (4.5,3.5) circle [radius=0.5];
	\node at (4.5,3.5) {$S_3$};
	\draw[thick] (6.5,2.5) circle [radius=0.5];
	\node at (6.5,2.5) {$S_4$};
	\draw[thick] (8.5,2.5) circle [radius=0.5];
	\node at (8.5,2.5) {$S_5$};
	\draw[thick] (10.5,1.5) circle [radius=0.5];
	\node at (10.5,1.5) {$S_6$};
	\draw[thick] (10.5,3.5) circle [radius=0.5];
	\node at (10.5,3.5) {$S_7$};
	\draw[thick] (12.5,2.5) circle [radius=0.5];
	\node at (12.5,2.5) {$S_8$};
	
	%Edges
	\draw[->, thick] (0.85,0.85) -- (2.15,2.15);%s0-s1
	\node[above left] at (1.5,1.5) {$R_1$};
	\draw[->, thick] (3.0,2.5) -- (4.0,3.5);%s1-s3
	\node at (3.5,1.5) {$R_1$};
	\draw[->, thick] (3.0,2.5) -- (4.0,1.5);%s1-s2
	\node at (3.5,3.5) {$R_1$};
	\draw[->, thick] (5.0,3.5) -- (6.0,2.5);%s2-s4
	\node at (5.5,1.5) {$R_0$};
	\draw[->, thick] (5.0,1.5) -- (6.0,2.5);%s3-s4
	\node at (5.5,3.5) {$R_0$};
	\draw[->, thick] (7.0,2.5) -- (8.0,2.5);%s4-s5
	\node[above] at (7.5,2.5) {$R_2$};
	\draw[->, thick] (9.0,2.5) -- (10.0,1.5);%s5-s6
	\node at (9.5,1.5) {$R_1$};
	\draw[->, thick] (9.0,2.5) -- (10.0,3.5);%s5-s7
	\node at (9.5,3.5) {$R_1$};
	\draw[->, thick] (11.0,1.5) -- (12.0,2.5);%s6-s8
	\node at (11.5,3.5) {$R_0$};
	\draw[->, thick] (11.0,3.5) -- (12.0,2.5);%s7-s8
	\node at (11.5,1.5) {$R_0$};
	\draw[->, thick] (13.0,2.5) -- (13.9,2.5);%s8-...
	\node[above] at (13.5,2.5) {$R_2$};
	\node[below] at (13.5,2.5) {$\dots$};
	
	\end{tikzpicture}
	\caption{Grafo degli stati infinito. \label{fig:infiniteStateGraphExe}}
	\end{figure*}



\subsection{Esempio}\label{sub:networkExe}
In questa sottosezione, viene fornito un esempio concreto di grafo degli stati, con particolare attenzione al problema dell'evoluzione infinita. Si riprende l'esempio della rete della sottosezione \ref{sub:networkExe}. L'implementazione la si pu� trovare in [...]. 

Si introduce solo la segnatura del pacchetto, del router e del dominio: $K=\{pacchetto:2, \quad router:2, \quad dominio:0\}$. Chiameremo rispettivamente $Encap$ e $Decap$ le regole per l'incapsulamento e il decapsulamento dei pacchetto nei vari strati, per esempio da Http a Tcp, o da Ip a Ethernet. Ci concentriamo sulla regola di inoltro tra router, che � quella in figura \ref{fig:forwardRule}.



	\begin{figure*}[th]
	\centering
	\begin{tikzpicture}
	%\draw[help lines] (0,0) grid (14,5);
	%%%   Redex   %%%
	% Sx Big
	\draw[rounded corners = 3mm, dashed] (0.0,0.0) rectangle (3.0,4.0);%Root 1
	\draw[rounded corners = 3mm, thick] (0.3,0.3) rectangle (2.7,3.7);%Domain 1
	\draw[thick] (2.0,3.0) circle [radius=0.5];%Router 1
	\draw[rounded corners = 1mm, dashed, fill=myGrey] (2.0,0.5) rectangle (2.5,1.0);%Site 0
	\node at (2.3,0.75) {0};
	%Packet
	\draw[rounded corners=1mm, thick] (0.5,1.0) rectangle (1.7,1.7);
	\draw[rounded corners = 1mm, dashed, fill=myGrey] (1.0,1.1) rectangle (1.4,1.5);%Site 2
	\node at (1.2,1.3) {2};
	%idS
	\draw[myGreen, thick] (0.6,1.7) to [out=100,in=270] (0.5,4.2);
	\draw[fill] (0.6,1.7) circle [radius=0.05];
	\node[above] at (0.4,4.2) {$id_S$};
	%idR
	\draw[rounded corners = 5mm, myGreen, thick] (1.6,1.7) -- (2.0,1.9) -- (3.2,1.8) -- (5.0,1.5) -- (6.0,1.5) -- (6.0,4.2);
	\draw[fill] (1.6,1.7) circle [radius=0.05];
	\node[above] at (6.3,4.2) {$id_R$};
	
	%Dx Big
	\draw[rounded corners = 3mm, dashed] (3.5,0.0) rectangle (6.5,4.0);%Root 2
	\draw[rounded corners = 3mm, thick] (3.8,0.3) rectangle (6.2,3.7);%Domain 2
	\draw[thick] (4.5,3.0) circle [radius=0.5];%Router 2
	\draw[rounded corners = 1mm, dashed, fill=myGrey] (4.0,0.5) rectangle (4.5,1.0);%Site 1
	\node at (4.3,0.75) {1};
	
	%LinkR
	\draw[myGreen, thick] (2.0,3.5) to [out=90,in=270] (3.15,4.5);%r1-link
	\draw[fill] (2.0,3.5) circle [radius=0.05];
	\draw[myGreen, thick] (4.5,3.5) to [out=90,in=270] (3.15,4.5);%r2-link
	\draw[fill] (4.5,3.5) circle [radius=0.05];
	\node[above] at (3.15,4.5) {link};
	%LocalS
	\draw[rounded corners = 3mm, myGreen, thick] (2.0,2.5) -- (2.0,2.1) -- (1.2,2.1) -- (1.2,4.5);
	\draw[fill] (2.0,2.5) circle [radius=0.05];
	\node[above] at (1.2,4.5) {$local_S$};
	%LocalR
	\draw[rounded corners = 3mm, myGreen, thick] (4.5,2.5) -- (4.5,2.1) -- (5.3,2.1) -- (5.3,4.5);
	\draw[fill] (4.5,2.5) circle [radius=0.05];
	\node[above] at (5.3,4.5) {$local_R$};
	
	
	\draw[->, red, very thick] (6.8,2.0) -- (7.8,2.0);
	
	%%%   Reactum   %%%
	% Sx Big
	\draw[rounded corners = 3mm, dashed] (8.0,0.0) rectangle (11.0,4.0);%Root 1
	\draw[rounded corners = 3mm, thick] (8.3,0.3) rectangle (10.7,3.7);%Domain 1
	\draw[thick] (10.0,3.0) circle [radius=0.5];%Router 1
	\draw[rounded corners = 1mm, dashed, fill=myGrey] (10.0,0.5) rectangle (10.5,1.0);%Site 0
	\node [below] at (10.3,1.0) {0};
	
	%Dx Big
	\draw[rounded corners = 3mm, dashed] (11.3, 0.0) rectangle (13.99,4.0);%Root 2
	\draw[rounded corners = 3mm, thick] (11.6,0.3) rectangle (13.6,3.7);%Domain 2
	\draw[thick] (12.5,3.0) circle [radius=0.5];%Router 2
	\draw[rounded corners = 1mm, dashed, fill=myGrey] (12.0,0.4) rectangle (12.5,0.9);%Site 1
	\node [below] at (12.3,0.9) {1};
	%Packet
	\draw[rounded corners=1mm, thick] (12.0,1.0) rectangle (13.2,1.7);
	\draw[rounded corners = 1mm, dashed, fill=myGrey] (12.5,1.1) rectangle (13.0,1.5);%Site 2
	\node at (12.7,1.3) {2};
	%idS
	\draw[rounded corners = 5mm, myGreen, thick] (12.0,1.7) -- (11.0,1.9) -- (8.5,1.5) -- (8.5,4.2);
	\draw[fill] (12.0,1.7) circle [radius=0.05];
	\node[above] at (8.4,4.2) {$id_S$};
	%idR
	\draw[rounded corners = 5mm, myGreen, thick] (13.2,1.7) to [out=60,in=270] (13.9,4.2);
	\draw[fill] (13.2,1.7) circle [radius=0.05];
	\node[above] at (13.59,4.0) {$id_R$};
	
	%LinkR
	\draw[myGreen, thick] (10.0,3.5) to [out=90,in=270] (11.15,4.5);%r1-link
	\draw[fill] (10.0,3.5) circle [radius=0.05];
	\draw[myGreen, thick] (12.5,3.5) to [out=90,in=270] (11.15,4.5);%r2-link
	\draw[fill] (12.5,3.5) circle [radius=0.05];
	\node[above] at (11.15,4.5) {link};
	%LocalS
	\draw[rounded corners = 3mm, myGreen, thick] (10.0,2.5) -- (10.0,2.1) -- (9.2,2.1) -- (9.2,4.5);
	\draw[fill] (10.0,2.5) circle [radius=0.05];
	\node[above] at (9.2,4.5) {$local_S$};
	%LocalR
	\draw[rounded corners = 3mm, myGreen, thick] (12.5,2.5) -- (12.5,2.1) -- (13.2,2.1) -- (13.2,4.5);
	\draw[fill] (12.5,2.5) circle [radius=0.05];
	\node[above] at (13.0,4.5) {$local_R$};
	
	\end{tikzpicture}
	\caption{Regola di inoltro tra router. \label{fig:forwardRule}}
	\end{figure*}



La regola � molto semplice. Innanzitutto, incominciamo con i vari controlli. Figli diretti delle due radici sono i due rettangoli: essi sono i domini di cui fanno parte i due router. Questi ultimi, a loro volta, sono disegnati come circonferenze con due porte: la prima collegata a ``$link$" e la seconda a ``$Local$". Infine, il pacchetto � rappresentato da un rettangolo, anch'esso con due porte: la prima, che � quella pi� a sinistra, � collegata al mittente, mentre la seconda al destinatario. 

Passiamo ora a definire il ruolo degli outername. ``$Link$" � l'outername dedicato alla connessione tra router: se un terzo router volesse collegarsi a questi due, allora dovrebbe collegare (nel vero senso della parola) la sua porta superiore con l'outername ``$link$". Si ricordi quanto detto nella sottosezione \ref{sub:networkExe}: i router inoltrano non deterministicamente ogni pacchetto verso tutte le sue interfacce. Se tre router sono collegati all'outername ``$link$", allora uno di questi inoltra il pacchetto verso tutti gli altri due. Per cui ``$link$" rappresenta l'insieme di tutte le interfacce di un router.

Gli outername ``$Local_S$" e ``$Local_R$" sono a disposizione solamente per gli host del dominio corrente. Per esempio, a ``$Local_S$" (che sta per \emph{Local Sender}) si possono collegare solo gli host del dominio in cui si trova il primo router. Nella regola di figura \ref{fig:forwardRule}, questi host vengono inglobati dal sito numero 0. Il medesimo discorso vale per ``$Local_R$".

Il pacchetto � collegato a due outername. Il primo, quello pi� a sinistra, � ``$id_S$" (\emph{id Sender}), mentre il secondo �  ``$id_R$" (\emph{id Receiver}). Nella realt�, questi due outername saranno gli identificativi del mittente e del destinatario. Se per esempio il pacchetto in questione � IP, allora $id_S$ sar� l'indirizzo IP del mittente, mentre $id_R$ quello del destinatario.

Infine, si noti anche il sito numero 2: si trova dentro il pacchetto e permette di astrarre al tipo di pacchetto. Per esempio: se � un pacchetto IP, allora � probabile che abbia incapsulato al suo interno un pacchetto TCP, che a sua volta contiene un pacchetto HTTP. Onde evitare di scrivere regole ad hoc per ogni tipo di pacchetto, si introduce il sito numero 2, cos� che la regola trovi un occorrenza (e quindi scatti) per qualsiasi tipo di pacchetto.

Vediamo ora un esempio reale, proposto in figura \ref{fig:networkRuleExe}. Siano $D_S$ e $D_R$ i domini rispettivamente del mittente (sender) e del destinatario (receiver). Indicheremo con il triangolo il controllo per un host. Quindi $h_1$ e $h_2$ saranno rispettivamente all'interno di $D_S$ e $D_R$. Lo stesso vale per i router, cio� $R_S$ e $R_R$. Supponiamo che il pacchetto sia di tipo IP, e che $h_1$ sia collegato per esempio a $158.110.3.46$, e $h_2$ a $158.110.144.31$. Infine, per quanto detto prima a proposito di ``$Local_S$" e ``$Local_R$", alla porta inferiore di $R_S$ si collegher� $h_1$, mentre a quella di $R_R$ si collegher� $id_2$.


	\begin{figure*}[th]
	\centering
	\begin{tikzpicture}
	%\draw[help lines] (0,0) grid (10,5);
	%%%   Redex   %%%
	% Sx Big
	\draw[rounded corners = 3mm, dashed] (0.0,0.0) rectangle (3.0,4.0);%Root 1
	\draw[rounded corners = 3mm, thick] (0.3,0.3) rectangle (2.7,3.7);%Domain 1
	\node at (2.0,4.2){$D_S$};
	\draw[thick] (2.0,3.0) circle [radius=0.5];%Router 1
	\node at (2.0,3.0){$R_S$};
	\draw[rounded corners = 1mm, thick] (0.7,0.5) -- (1.5,0.5) -- (1.1,1.2) -- (0.7,0.5);%h1
	\node at (1.15,0.75) {$h_1$};
	%Packet
	\draw[rounded corners=1mm, thick] (1.4,1.0) rectangle (2.6,1.7);
	\draw[rounded corners = 1mm, thick, fill=myGrey] (1.6,1.1) rectangle (2.4,1.5);%tcp
	\node at (2.0,1.3) {tcp};
	%packet - idS
	\draw[myGreen, thick] (1.5,1.7) to [out=90,in=300] (0.83,2.5);
	\draw[fill] (1.5,1.7) circle [radius=0.05];
	%packet - idR
	\draw[myGreen, thick] (2.5,1.7) to [out=60,in=230] (5.72,2.5);
	\draw[fill] (2.5,1.7) circle [radius=0.05];
	
	
	%Dx Big
	\draw[rounded corners = 3mm, dashed] (3.5,0.0) rectangle (6.5,4.0);%Root 2
	\draw[rounded corners = 3mm, thick] (3.8,0.3) rectangle (6.2,3.7);%Domain 2
	\node at (4.5,4.2){$D_R$};
	\draw[thick] (4.5,3.0) circle [radius=0.5];%Router 2
	\node at (4.5,3.0){$R_R$};
	\draw[rounded corners = 1mm, thick] (5.0,0.5) -- (5.8,0.5) -- (5.4,1.3) -- (5.0,0.5);%h2
	\node at (5.4,0.75) {$h_2$};
	
	%LinkR
	\draw[myGreen, thick] (2.0,3.5) to [out=90,in=270] (3.15,4.5);%r1-link
	\draw[fill] (2.0,3.5) circle [radius=0.05];
	\draw[myGreen, thick] (4.5,3.5) to [out=90,in=270] (3.15,4.5);%r2-link
	\draw[fill] (4.5,3.5) circle [radius=0.05];
	\node[above] at (3.15,4.5) {link};
	%idS
	\node[above] at (0.4,4.2) {$158.110.3.46$};
	\draw[rounded corners = 3mm, myGreen, thick] (1.1,1.2) to [out=90,in=270] (0.4,4.2);
	\draw[fill] (1.1,1.2) circle [radius=0.05];
	%idR
	\node[above] at (6.3,4.2) {$158.110.144.31$};
	\draw[rounded corners = 3mm, myGreen, thick] (5.4,1.3) to [out=90,in=270] (6.3,4.2);
	\draw[fill] (5.4,1.3) circle [radius=0.05];
	%LocalS
	\draw[rounded corners = 3mm, myGreen, thick] (1.1,0.5) -- (1.1,0.2) --  (2.0,0.2) -- (2.0,2.5);
	\draw[fill] (2.0,2.5) circle [radius=0.05];
	\draw[fill] (1.1,0.5) circle [radius=0.05];
	%LocalR
	\draw[rounded corners = 3mm, myGreen, thick] (5.4,0.5) -- (5.4,0.2) -- (4.5,0.2) -- (4.5,2.5);
	\draw[fill] (4.5,2.5) circle [radius=0.05];
	\draw[fill] (5.4,0.5) circle [radius=0.05];
	
	
	\end{tikzpicture}
	\caption{Bigrafo di partenza \label{fig:networkRuleExe}}
	\end{figure*}


Chiamiamo il bigrafo della figura \ref{fig:networkRuleExe} come $S_0$, cio� lo stato iniziale del grafo degli stati. Il nostro BRS � formato da una sola regola, cio� quella di figura \ref{fig:forwardRule}, che chiameremo $R_0$. Si noti come il redex di $R_0$ trovi una ed una sola occorrenza in $S_0$, e quindi generer� un solo stato, ovvero $S_1$, dove il pacchetto si � spostano in $D_R$, arrivando a destinazione. Si pu� vedere questo risultato nel bigrafo di figura \ref{fig:forwardRuleBigPost}, che sar� quindi $S_1$.



	\begin{figure*}[th]
	\centering
	\begin{tikzpicture}
	%\draw[help lines] (0,0) grid (10,5);
	%%%   Redex   %%%
	% Sx Big
	\draw[rounded corners = 3mm, dashed] (0.0,0.0) rectangle (3.0,4.0);%Root 1
	\draw[rounded corners = 3mm, thick] (0.3,0.3) rectangle (2.7,3.7);%Domain 1
	\node at (2.0,4.2){$D_S$};
	\draw[thick] (2.0,3.0) circle [radius=0.5];%Router 1
	\node at (2.0,3.0){$R_S$};
	\draw[rounded corners = 1mm, thick] (0.7,0.5) -- (1.5,0.5) -- (1.1,1.2) -- (0.7,0.5);%h1
	\node at (1.15,0.75) {$h_1$};
	%Packet
	\draw[rounded corners=1mm, thick] (3.8,1.0) rectangle (5.0,1.7);
	\draw[rounded corners = 1mm, thick, fill=myGrey] (4.0,1.1) rectangle (4.8,1.5);%Tcp
	\node at (4.4,1.3) {tcp};
	%packet - idS
	\draw[myGreen, thick] (3.9,1.7) to [out=90,in=300] (0.83,2.5);
	\draw[fill] (3.9,1.7) circle [radius=0.05];
	%packet - idR
	\draw[myGreen, thick] (4.9,1.7) to [out=60,in=230] (5.72,2.5);
	\draw[fill] (4.9,1.7) circle [radius=0.05];
	
	
	%Dx Big
	\draw[rounded corners = 3mm, dashed] (3.5,0.0) rectangle (6.5,4.0);%Root 2
	\draw[rounded corners = 3mm, thick] (3.8,0.3) rectangle (6.2,3.7);%Domain 2
	\node at (4.5,4.2){$D_R$};
	\draw[thick] (4.5,3.0) circle [radius=0.5];%Router 2
	\node at (4.5,3.0){$R_R$};
	\draw[rounded corners = 1mm, thick] (5.0,0.5) -- (5.8,0.5) -- (5.4,1.3) -- (5.0,0.5);%h2
	\node at (5.4,0.75) {$h_2$};
	
	%LinkR
	\draw[myGreen, thick] (2.0,3.5) to [out=90,in=270] (3.15,4.5);%r1-link
	\draw[fill] (2.0,3.5) circle [radius=0.05];
	\draw[myGreen, thick] (4.5,3.5) to [out=90,in=270] (3.15,4.5);%r2-link
	\draw[fill] (4.5,3.5) circle [radius=0.05];
	\node[above] at (3.15,4.5) {link};
	%idS
	\node[above] at (0.4,4.2) {$158.110.3.46$};
	\draw[rounded corners = 3mm, myGreen, thick] (1.1,1.2) to [out=90,in=270] (0.4,4.2);
	\draw[fill] (1.1,1.2) circle [radius=0.05];
	%idR
	\node[above] at (6.3,4.2) {$158.110.144.31$};
	\draw[rounded corners = 3mm, myGreen, thick] (5.4,1.3) to [out=90,in=270] (6.3,4.2);
	\draw[fill] (5.4,1.3) circle [radius=0.05];
	%LocalS
	\draw[rounded corners = 3mm, myGreen, thick] (1.1,0.5) -- (1.1,0.2) --  (2.0,0.2) -- (2.0,2.5);
	\draw[fill] (2.0,2.5) circle [radius=0.05];
	\draw[fill] (1.1,0.5) circle [radius=0.05];
	%LocalR
	\draw[rounded corners = 3mm, myGreen, thick] (5.4,0.5) -- (5.4,0.2) -- (4.5,0.2) -- (4.5,2.5);
	\draw[fill] (4.5,2.5) circle [radius=0.05];
	\draw[fill] (5.4,0.5) circle [radius=0.05];
	
	
	\end{tikzpicture}
	\caption{Bigrafo dopo l'applicazione della regola $R_0$ \label{fig:forwardRuleBigPost}}
	\end{figure*}


Di per s�, il BRS si dimentica degli stati precedenti, in questo caso $S_0$. Per cui, da $S_1$ scatterebbe di nuovo la regola $R_0$, creando lo stato $S_2$. A sua volta, da $S_2$, eseguirebbe di nuovo $R_0$ e causerebbe cos� un'esecuzione infinita, dando luogo al grafo degli stati in figura \ref{fig:infiniteStateGraphNetwork}.

Grazie all'isomorfismo tra bigrafi, riusciamo a collegare lo stato $S_1$ a $S_0$. Poich� da $S_1$ tramite la regola $R_0$ viene creato lo stato $S_2$, siamo ora in grado non doverlo pi� memorizzare, in quanto � uguale (o meglio, � semanticamente equivalente) allo stato $S_0$. Per cui, ritrovandoci di nuovo in $S_0$, sappiamo che non dobbiamo applicare pi� nessuna regola, pena un'esecuzione infinita del BRS. Tramite l'isomorfismo, si � potuto generare il \emph{grafo degli stati} della figura \ref{fig:finiteStateGraphNetwork}.



\begin{figure*}[th]
	\centering
	\begin{tikzpicture}
	%\draw[help lines] (0,0) grid (14,5);
	%States
	\draw[thick] (0.5,0.5) circle [radius=0.5];
	\node at (0.5,0.5) {$S_0$};
	\draw[thick] (2.5,0.5) circle [radius=0.5];
	\node at (2.5,0.5) {$S_1$};
	\draw[thick] (4.5,0.5) circle [radius=0.5];
	\node at (4.5,0.5) {$S_2$};
	\draw[thick] (6.5,0.5) circle [radius=0.5];
	\node at (6.5,0.5) {$S_3$};
	
	%Edges
	\draw[->, thick] (1.0,0.5) -- (2.0,0.5);%s0-s1
	\node[above] at (1.5,0.5) {$R_0$};
	\draw[->, thick] (3.0,0.5) -- (4.0,0.5);%s1-s2
	\node[above] at (3.5,0.5) {$R_0$};
	\draw[->, thick] (5.0,0.5) -- (6.0,0.5);%s2-s3
	\node[above] at (5.5,0.5) {$R_0$};
	\draw[->, thick] (7.0,0.5) -- (8.0,0.5);%s3-...
	\node[above] at (7.5,0.5) {$\dots$};
	
	\end{tikzpicture}
	\caption{Grafo degli stati infinito. \label{fig:infiniteStateGraphNetwork}}
\end{figure*}




\begin{figure*}[th]
	\centering
	\begin{tikzpicture}
	%\draw[help lines] (0,0) grid (14,5);
	%States
	\draw[thick] (1.5,1.5) circle [radius=0.5];
	\node at (1.5,1.5) {$S_0$};
	\draw[thick] (3.5,1.5) circle [radius=0.5];
	\node at (3.5,1.5) {$S_1$};
	
	%Edges
	\draw[->, thick] (1.85,1.85) to [out=45,in=135] (3.15,1.85);%s0-s1
	\node at (2.5,2.5) {$R_0$};
	\draw[<-, thick] (1.85,1.15) to [out=-45,in=-135] (3.15,1.15);%s1-s0
	\node at (2.5,0.5) {$R_0$};
		
	\end{tikzpicture}
	\caption{Grafo degli stati finito. \label{fig:finiteStateGraphNetwork}}
\end{figure*}




\section{Complessit�}\label{sub:isoComplexity}
Il problema dell'isomorfismo tra bigrafi, come si � visto, � stato affrontato tramite il \emph{constraint programming}. Si � scelta questa tecnica per i seguenti motivi:
\begin{itemize}
	\item
	la programmazione a vincoli consente di ridurre notevolmente il tempo di sviluppo dell'algoritmo e il numero di errori.
	\item
	il motore che risolve il sistema di equazioni (in questo caso \emph{Choco vs 3.3.1}) � di solito soggetto a molti test, ed � quindi affidabile. C'� anche la possibilit� di imporre delle euristiche che migliorano notevolmente le prestazioni, rendendolo quindi anche efficiente.
	\item
	il motore pu� venire considerato come una \emph{black box}, astraendosi quindi alla sua implementazioni interna, che pu� per esempio venire cambiata o migliorata nel tempo, evitando di affliggere il software che lo usa.
\end{itemize}

Stabilire la complessit� del problema dell'isomorfismo tra bigrafi � molto importante, perch� relazionandolo con altri problemi � possibile trovare un algoritmo efficiente per risolverlo. Facciamo un esempio: ipotizziamo che il problema dell'isomorfismo tra bigrafi (che chiameremo $P_{big}$) si trovi nella classe C, e che il problema $P_c$ sia \emph{completo} rispetto a questa classe. Per definizione di \emph{completezza}, ogni istanza di ogni problema appartenente alla classe C, quindi anche ogni istanza di $P_{big}$, pu� essere tradotta (cio� \emph{ridotta}) con una certa complessit� al problema $P_c$.

L'importanza di questa considerazione risiede in questo fatto: se si riesce a scoprire un algoritmo che operi in tempo polinomiale per risolvere il problema $P_c$ allora tutti i problemi della classe $C$, incluso $P_{big}$, sarebbero risolvibili in tempo polinomiale.

Si dimostrer� il seguente risultato:

\begin{teor}[Complessit� di $P_{big}$]
Il problema dell'isomorfismo tra bigrafi � GI-completo.
\end{teor}

\begin{corol}
Il problema dell'isomorfismo tra bigrafi � \textbf{equivalente} al problema dell'isomorfismo tra grafi.
\end{corol}

\begin{dimos}
Per dimostrare l'equivalenza, procederemo in due direzioni: nella prima dimostreremo che esiste una riduzione in tempo polinomiale da $P_{graph}$ a $P_{big}$, cio� che ogni istanza del problema $P_{graph}$ pu� venire tradotta in un'istanza del problema $P_{big}$. Invece, nella seconda direzione, faremo l'inverso. Si dimostreranno quindi che esistono le seguenti riduzioni:\\

$
\begin{cases}   
P_{graph} \le P_{big}\\
P_{big} \le P_{graph}
\end{cases}
$
\\

\begin{itemize}
	\item
	$P_{graph} \le P_{big}$:\\
	Dando una particolare segnatura al bigrafo, ogni termine del $\pi$-Calcolo pu� essere tradotto in un bigrafo, quindi si ha che ogni istanza del problema $\pi$-SC ($\pi$  Structural Conguence), che controlla quando due termini sono equivalenti, pu� essere tradotta in tempo polinomiale in un'istanza del problema $P_{big}$, che controlla quando due bigrafi sono isomorfi. Questa � una \emph{Karp-riduzione} dal problema $P_{\pi}$ a $P_{big}$. In [...] si dimostra che il problema di congruenza tra termini del $\pi$-Calcolo, cio� $\pi$-SC, � equivalente al problema dell'isomorfismo tra \emph{grafi}, $P_{graph}$. Quindi, abbiamo trovato una riduzione che traduce ogni istanza del problema di isomorfismo tra grafi in un'istanza del problema di isomorfismo tra bigrafi.
	
	\item
	$P_{big} \le P_{graph}$:\\
	Procediamo ora con la seconda parte della dimostrazione: vogliamo dimostrare che ogni istanza del problema $P_{graph}$ pu� venire tradotta in tempo polinomiale in un'istanza del problema $P_{big}$. Si � visto nell'introduzione che ogni bigrafo � decomponibile in due strutture tra loro ortogonali (place graph e link graph), e che quindi l'isomorfismo pu� venire trattato separatamente per i due casi. Trattandosi di grafi, ogni istanza di $P_{big}$ pu� venire ridotta a due istanza di $P_{graph}$. Abbiamo quindi trovato una riduzione da $P_{big}$ a $P_{graph}$. 	
\end{itemize}

Si � quindi dimostrata l'equivalenza fra il problema dell'isomorfismo tra bigrafi e quella tra grafi, concludendo quindi che il problema $P_{big}$ � equivalente al problema $P_{graph}$.
\end{dimos}

E' noto che $P_{graph}$ appartiene alla classe NP, ma non � ancora stato dimostrato che sia completo. Per qui si � creata una nuova classe di complessit�, nota con il nome di GI, che raccoglie tutti i problemi che possono essere ridotti in tempo polinomiale al problema $P_{graph}$. Quindi si ha che $P_{big} \in GI$.

Come gi� accennato, l'importanza di trovare la classe di complessit� a cui appartiene un dato problema � enorme: se si trovasse un algoritmo che risolva in tempo polinomiale un problema NP-completo (cio� se la famosa domanda \emph{P is NP?} avesse risposta affermativa), allora tutti i problemi appartenenti alla classe NP, incluso $P_{big}$, potrebbero essere ridotti in tempo polinomiale al problema NP-completo di cui si ha l'algoritmo polinomiale. In questo modo, anche l'isomorfismo tra bigrafi sarebbe risolvibile in tempo polinomiale.







\chapter{Model Checker per bigrafi}
Nel capitolo precedente si � visto come il fatto di poter riconoscere quando due bigrafi sono uguali permetta di poter arrestare l'esecuzione del BRS, evitando sue evoluzioni infinite. In questo capitolo invece si vedr� l'altro principale problema: come poter verificare date propriet� sul BRS. Riprendendo l'esempio della rete, una propriet� che potremmo verificare � l'arrivo a destinazione di un pacchetto, oppure assicurarci che nessun pacchetto non autorizzato passi attraverso un firewall.

Come per l'isomorfismo, anche questo problema necessita di una soluzione generale, che prescinde dal dominio che i bigrafi rappresentano. Si � creata quindi una semplice logica a predicati, con cui � possibile esprimere le propriet� che si vuole verificare. Essa andr� a formare la \emph{politica} per il Model Checker, che servir� a verificare le propriet� sul grafo degli stati visto nel capitolo precedente.


\section{Model Checker}
Un Model Checker (MC) � un metodo per verificare delle propriet� in un sistema formale. Nel nostro caso, si � costruito un MC basato sul grafo degli stati (come quello di figura \ref{fig:bsgExe}): il problema sar� capire se un nodo rispetti le propriet� specificate.

Si � visto come nel \emph{grafo degli stati} ogni nodo sia a sua volta un bigrafo. In figura \ref{fig:bsgExe} c'� il grafo degli stati dell'esempio \ref{sub:networkExe} sullo scambio di pacchetti tra due router. Ci possiamo chiedere se in uno dei due nodi il pacchetto sia arrivato a destinazione, cio� se uno dei due stati $S_i$ ($i \in \{0, 1\}$) il pacchetto sia nello stesso dominio dell' host destinazione.

\begin{figure*}[th]
	\centering
	\begin{tikzpicture}
	%\draw[help lines] (0,0) grid (14,5);
	%States
	\draw[thick] (1.5,1.5) circle [radius=0.5];
	\node at (1.5,1.5) {$S_0$};
	\draw[thick] (3.5,1.5) circle [radius=0.5];
	\node at (3.5,1.5) {$S_1$};
	
	%Edges
	\draw[->, thick] (1.85,1.85) to [out=45,in=135] (3.15,1.85);%s0-s1
	\node at (2.5,2.5) {$R_0$};
	\draw[<-, thick] (1.85,1.15) to [out=-45,in=-135] (3.15,1.15);%s1-s0
	\node at (2.5,0.5) {$R_0$};
		
	\end{tikzpicture}
	\caption{Grafo degli stati. \label{fig:bsgExe}}
\end{figure*}

Nei MC queste propriet� sono di solito esprimibili attraverso una qualche logica, per cui possiamo esprimere formalmente cosa significa che un MC verifichi una certa propriet�.

\begin{prop}
Il problema della verifica di una propriet� da parte di un MC � esprimibile come:
	\begin{center}
	$MC,S_0 \models p$
	\end{center}
dove MC � un model checker, $S_0$ � lo stato iniziale e $p$ � una propriet� espressa in una qualche logica.
\end{prop}

Ovviamente, dallo stato $S_0$ il MC evolver� secondo precise regole per formare tutti i possibili stati $S_0 \dots S_n$: nel nostro caso, ogni arco tra due nodi del \emph{grafo degli stati} � una regola di reazione. Per cui il model checker controller� l'intero grafo: appena trova uno stato $S_i$ che soddisfa $p$ ($MC,S_i \models p$) ritorna True, altrimenti, cio� nel caso in cui \emph{tutti} gli stati del grafo non rispettino la propriet�, ritorna False.

Queste considerazioni ci portano a definire il comportamento del MC:

\begin{prop}
Il comportamento di un model checker MC � definito dalla seguente relazione:
	\begin{center}
	$\begin{cases}
	return \ \ True \qquad if \ \exists S_i\in MC \ : \ MC,S_i\models p \\
	return \ \ False \qquad otherwise \ (\forall S_i\in MC \ (MC,S_i \not\models p))
	\end{cases}$
	\end{center}
\end{prop}

Nell'esempio \ref{fig:bsgExe}, il problema � quindi banale: il model checker ritorna vero se e solo se � vera la formula $MC,S_0 \models p \lor MC,S_1 \models p$.

\subsection{Generazione degli stati}
Si � appena visto che in un model checker si possono esprimere delle propriet�: ogni MC ha per� anche un altro grado di libert�, che riguarda la generazione degli stati. Come creare il grafo degli stati? E con che ordine?

Nel MC costruito per questa tesi, che chiameremo $MC_{big}$, ci sono varie strategie ed ognuna � adatta per certi scopi. Vediamone alcune:
\begin{itemize}
	\item
	\emph{Strategia Breadth First}: per ogni stato $S_i$ vengono generati tutti gli stati possibili adiacenti ad $S_i$. Per esempio, in figura \ref{fig:BSGen}, si mostrano i primi tre passi della strategia Breadth First. Il suo vantaggio � che non si tralascia nessuno stato, ottenendo un grafo degli stati \emph{completo}. Inoltre, se un nodo porta ad un vicolo cieco (cio� se non genera nessun bigrafo tramite nessuna regola) allora questo viene semplicemente tolto dalla coda. Lo svantaggio � che pu� essere molto lenta: se da ogni stato si generano k stati (con k molto alto), allora prima di verificare la propriet� potrebbero volerci molte iterazioni.
	
	\begin{figure*}[th]
	\centering
	\subfigure[Primo passo]{
	\begin{tikzpicture}
	%\draw[help lines] (0,0) grid (8,5);
	%States
	\draw[thick] (0.5,0.5) circle [radius=0.5];
	\node at (0.5,0.5) {$S_0$};
	\draw [draw=white, fill=white] (3.0,0.0) circle [radius=0.1];
	\end{tikzpicture}
	}
	
	\hspace{5mm}
	
	\subfigure[Secondo passo]{
	\begin{tikzpicture}
	%\draw[help lines] (0,0) grid (8,5);
	%States
	\draw[thick] (0.5,0.5) circle [radius=0.5];
	\node at (0.5,0.5) {$S_0$};
	\draw[thick] (2.5,2.5) circle [radius=0.5];
	\node at (2.5,2.5) {$S_1$};
	%Edges
	\draw[->, thick] (0.85,0.85) -- (2.15,2.15);%s0-s1
	\node[above left] at (1.5,1.5) {$R_1$};
	\end{tikzpicture}
	}
	
	\hspace{5mm}
	
	\subfigure[Terzo passo]{
	\begin{tikzpicture}
	%\draw[help lines] (0,0) grid (8,5);
	%States
	\draw[thick] (0.5,0.5) circle [radius=0.5];
	\node at (0.5,0.5) {$S_0$};
	\draw[thick] (2.5,2.5) circle [radius=0.5];
	\node at (2.5,2.5) {$S_1$};
	\draw[thick] (4.5,1.5) circle [radius=0.5];
	\node at (4.5,1.5) {$S_2$};
	\draw[thick] (4.5,3.5) circle [radius=0.5];
	\node at (4.5,3.5) {$S_3$};
	
	%Edges
	\draw[->, thick] (0.85,0.85) -- (2.15,2.15);%s0-s1
	\node[above left] at (1.5,1.5) {$R_1$};
	\draw[->, thick] (3.0,2.5) -- (4.0,3.5);%s1-s3
	\node at (3.5,1.5) {$R_1$};
	\draw[->, thick] (3.0,2.5) -- (4.0,1.5);%s1-s2
	\node at (3.5,3.5) {$R_1$};	
	\end{tikzpicture}
	}
	\caption{Esempio di generazione Breadth First. \label{fig:BSGen}}
	\end{figure*}

	 
	\item
	\emph{Strategia Random}: se dallo stato $S_i$ si possono applicare k regole allora viene scelto in maniera random un numero naturale $m \in \{1 \dots k\}$ e si genera solamente lo stato $S_m$. Questo consente di non memorizzare l'intero grafo degli stati (che in certi casi pu� essere molto grande) e verificare al momento della generazione di $S_k$ se $MC,S_k \models p$. Lo svantaggio � quello che l'esecuzione potrebbe andare avanti all'infinito, infatti il grafo \emph{non � completo}. Per cui potenzialmente potrebbero occorrere infinite evoluzioni prima di verificare una propriet�. Un esempio di generazione con la strategia random � quello in figura \ref{fig:BSGenRandom}.
	
	\begin{figure*}[!h]
	\centering
	\subfigure[Primo passo]{
	\begin{tikzpicture}
	%\draw[help lines] (0,0) grid (8,5);
	%States
	\draw[thick] (0.5,0.5) circle [radius=0.5];
	\node at (0.5,0.5) {$S_0$};
	\draw [draw=white, fill=white] (3.0,0.0) circle [radius=0.1];
	\end{tikzpicture}
	}
	
	\hspace{5mm}
	
	\subfigure[Secondo passo]{
	\begin{tikzpicture}
	%\draw[help lines] (0,0) grid (8,5);
	%States
	\draw[thick] (0.5,0.5) circle [radius=0.5];
	\node at (0.5,0.5) {$S_0$};
	\draw[thick] (2.5,2.5) circle [radius=0.5];
	\node at (2.5,2.5) {$S_1$};
	%Edges
	\draw[->, thick] (0.85,0.85) -- (2.15,2.15);%s0-s1
	\node[above left] at (1.5,1.5) {$R_1$};
	\end{tikzpicture}
	}
	
	\hspace{5mm}
	
	\subfigure[Terzo passo]{
	\begin{tikzpicture}
	%\draw[help lines] (0,0) grid (8,5);
	%States
	\draw[thick] (0.5,0.5) circle [radius=0.5];
	\node at (0.5,0.5) {$S_0$};
	\draw[thick] (2.5,2.5) circle [radius=0.5];
	\node at (2.5,2.5) {$S_1$};
	\draw[thick] (4.5,1.5) circle [radius=0.5];
	\node at (4.5,1.5) {$S_2$};
	
	%Edges
	\draw[->, thick] (0.85,0.85) -- (2.15,2.15);%s0-s1
	\node[above left] at (1.5,1.5) {$R_1$};
	\draw[->, thick] (3.0,2.5) -- (4.0,1.5);%s1-s2
	\node at (3.5,2.5) {$R_1$};	
	\end{tikzpicture}
	}
	
	\hspace{5mm}
	
	\subfigure[Quarto passo]{
	\begin{tikzpicture}
	%\draw[help lines] (0,0) grid (8,5);
	%States
	\draw[thick] (0.5,0.5) circle [radius=0.5];
	\node at (0.5,0.5) {$S_0$};
	\draw[thick] (2.5,2.5) circle [radius=0.5];
	\node at (2.5,2.5) {$S_1$};
	\draw[thick] (4.5,1.5) circle [radius=0.5];
	\node at (4.5,1.5) {$S_2$};
	\draw[thick] (6.5,2.5) circle [radius=0.5];
	\node at (6.5,2.5) {$S_4$};
	%Edges
	\draw[->, thick] (0.85,0.85) -- (2.15,2.15);%s0-s1
	\node[above left] at (1.5,1.5) {$R_1$};
	\draw[->, thick] (3.0,2.5) -- (4.0,1.5);%s1-s2
	\node at (3.5,2.5) {$R_1$};
	\draw[->, thick] (5.0,1.5) -- (6.0,2.5);%s2-s4
	\node at (5.5,2.5) {$R_0$};
	\end{tikzpicture}
	}
	
	\hspace{5mm}
	
	\subfigure[Quinto passo]{
	\begin{tikzpicture}
	%\draw[help lines] (0,0) grid (8,5);
	%States
	\draw[thick] (0.5,0.5) circle [radius=0.5];
	\node at (0.5,0.5) {$S_0$};
	\draw[thick] (2.5,2.5) circle [radius=0.5];
	\node at (2.5,2.5) {$S_1$};
	\draw[thick] (4.5,1.5) circle [radius=0.5];
	\node at (4.5,1.5) {$S_2$};
	\draw[thick] (6.5,2.5) circle [radius=0.5];
	\node at (6.5,2.5) {$S_4$};
	%Edges
	\draw[->, thick] (0.85,0.85) -- (2.15,2.15);%s0-s1
	\node[above left] at (1.5,1.5) {$R_1$};
	\draw[->, thick] (3.0,2.5) -- (4.0,1.5);%s1-s2
	\node at (3.5,2.5) {$R_1$};
	\draw[->, thick] (5.0,1.5) -- (6.0,2.5);%s2-s4
	\node at (5.5,2.5) {$R_0$};
	\draw[->, thick] (6.5,2.0) to [out=270,in=0] (5.5,0.0) -- (3.5,0.0) to [out=180,in=270] (2.5,2.0) ;%s4-s1
	\node[above] at (4.5,0.0) {$R_2$};
	
	\end{tikzpicture}
	}
	
	\caption{Esempio di generazione random. \label{fig:BSGenRandom}}
	\end{figure*}
	
\end{itemize}

E' possibile creare altre strategie a seconda degli scopi: in questa sede, se non specificato altrimenti, si assumer� che la strategia sia sempre quella Breadth First, che consente di computare l'intero grafo degli stati.




\section{Logica per i bigrafi}
Nella precedente sezione si � vista la struttura base del model checker per i bigrafi ($MC_{big}$): il grafo degli stati. L'altro aspetto importante di ogni MC sono le propriet�: necessitiamo quindi di un linguaggio per esprimerle. Il primo problema riscontrato � stato quello riguardante la flessibilit�: come fare ad avere un unico linguaggio che astraesse dal dominio scelto e che potesse essere adatto per qualsiasi BRS? In altre parole, � necessario scegliere un linguaggio che sia flessibile e allo stesso tempo espressivo. 

La linea guida seguita per la scelta del linguaggio � stata quindi la sua \emph{universalit�}. Per esempio: si prendano gli esempi \ref{sub:exeIntro} sulla moltiplicazione e \ref{sub:networkExe} sulla rete. Sia la propriet� $p_1$ definita come ``Il risultato � il numero 8". Sia la propriet� $p_2$ definita come ``Il pacchetto � arrivato a destinazione". Si noti come i domini dei due esempi siano totalmente differenti: il nostro linguaggio deve permettere di esprimere le due propriet� $p_1$ e $p_2$, senza dover ricorrere ad altri formalismi. 

Si capisce bene come un tale linguaggio cos� generale sia molto comodo per esprimere le propriet� da fare verificare al model checker $MC_{big}$. Infatti, in questo modo si crea uno strumento generale \textbf{valido per qualsiasi BRS}.

Si � scelto di usare una logica a predicati, esprimibile attraverso il linguaggio generato da una grammatica \emph{Context Free}.

\subsection{Sintassi}
Incominciamo con il descrivere la sintassi del linguaggio.
\begin{prop}
Il linguaggio $L(G)$ per il model checker $MC_{big}$ � generato dalla grammatica $G=(V,T,P,S)$, dove:
	\begin{itemize}
		\item
		$V=\{\varphi, \sigma \}$ � l'insieme di variabili
		
		\item
		$U=\{T, \land, $\emph{`(', `)', `,' ,}$ \lnot, W, \pi , A \dots Z\}$ � l'insieme di simboli terminali
		
		\item
		$P$ � l'insieme di \emph{produzioni}, definito dalle seguenti relazioni:
		\begin{center}
		$\varphi \rightarrow T \mid \varphi \land \varphi \mid \neg \varphi \mid W_\sigma(\varphi, \varphi, \varphi) \mid \pi_\sigma$\\
		$\sigma \rightarrow A\sigma \mid \dots \mid Z\sigma \mid \varepsilon \qquad \qquad \qquad \quad$
		\end{center}
		
		\item
		$S=\{\varphi\}$ � il simbolo iniziale
	\end{itemize}
\end{prop}

Il nostro linguaggio sar� quindi definito dall'insieme 
\begin{center}
$L(M)=\{w \in U^*\ : \ S\Rightarrow_*^G w \}$.
\end{center}

Seguendo l'usuale definizione di grammatica CF, diamo ora alcuni esempi di stringhe generabili dalla grammatica G, cio� di formule appartenenti al linguaggio L(G):
\begin{itemize}
	\item
	$\varphi = W_B(T, T, T) \land \pi_C$
	\item
	$\varphi = W_B(\pi_X, \pi_Y, \pi_Z)$
	\item
	$\varphi = \lnot \pi_B \land \pi_C$
\end{itemize}


\subsection{Semantica}
Definiamo ora la semantica del linguaggio, cio� specifichiamo il significato di ogni predicato. Si vedr� che quella presentata � una logica spaziale e non temporale. Spesso nei model checker si usano logiche temporali o spazio-temporali. Nell'implementazione, si � comunque dato spazio a tali logiche, rendendo le classi flessibili. In futuro, sar� quindi possibile aggiungere una nuova logica a $MC_{big}$.

La semantica � definita come segue (ricordiamo che ogni stato $S_i$ � un bigrafo):
\begin{prop}
Siano $S$ uno stato e $\varphi$ una propriet� espressa nel linguaggio L(G). La relazione $S \models \varphi$ (lo stato S soddisfa la propriet� $\varphi$) � definita per ricorsione sulla complessit� di $\varphi$:
\begin{itemize}
	\item
	$S \models T$ sempre
	\item
	$S \models \varphi_1 \land \varphi_2 \quad \Leftrightarrow \quad \begin{cases} S \models \varphi_1 \\ S \models \varphi_2 \end{cases}$
	\item
	$S \models \lnot \varphi \quad \Leftrightarrow \quad S \not \models \varphi$
	\item
	$S \models W_\sigma(\varphi_1,\varphi_2,\varphi_3) \quad \Leftrightarrow  \quad  \begin{cases}\exists C,D \  : \  S=C \circ (\sigma \otimes id_I) \circ D \\
	C\models \varphi_1 \quad \sigma \models \varphi_2 \quad D \models \varphi_3 \end{cases}$
	\item
	$S \models \pi_\sigma \quad \Leftrightarrow \quad S \bumpeq \sigma$
\end{itemize}
\end{prop}

Si dir� che la formula $\varphi$ costituisce la \emph{politica} per il model checker.

Analizziamo ora i vari predicati. I primi tre consentono le usuali operazioni della logica proposizionale, mentre il terzo � un predicato ad-hoc per questa logica: $W_\sigma$ � detto ``Wario Predicate", e usa l'operazione di Match per controllare le tre propriet� che ha come argomento. Facciamo un esempio: sia $W_B(T,T,T)$ un Wario Predicate. Lo stato S soddisfa questo predicato ($S \models W_B(T,T,T)$) se e solo se esiste un match M di B nel bigrafo S tale che rispetti queste condizioni: il contesto del match M deve soddisfare $\varphi_1$, il redex di M deve soddisfare $\varphi_2$ mentre i parametri di M devono soddisfare $\varphi_3$. Poich� $\varphi_1=\varphi_2=\varphi_3=T$, si ha che $S \models W_B(T,T,T)$ se e solo se esiste un match di B in S. Il Wario Predicate consente quindi di isolare contesto, redex e parametri e verificare le propriet� in modo indipendente per ognuno di questi tre bigrafi.

L'ultimo predicato, $\pi_\sigma$, controlla se esiste un isomorfismo tra due bigrafi. Per esempio, lo stato $S_i$ soddisfa il predicato $\pi_A$ (in formule $S_i \models \pi_A$) se e solo se $S_i$ � isomorfo al bigrafo $A$, cio� $S_i \bumpeq A$. Questo predicato � di particolare importanza: esso funge da simbolo di uguaglianza tra bigrafi, rendendo quindi la nostra logica una \emph{logica con uguaglianza}.

Si osservino tutti e cinque i predicati: dai i primi tre � possibile derivare ogni formula della logica proposizionale. Per esempio: se si vuole esprimere la formula $\varphi_1 \lor \varphi_2$, allora si possono usare le leggi di De Morgan e scrivere $\lnot(\lnot \varphi_1 \land \lnot \varphi_2)$. Oppure, se si vuole esprimere il falso, baster� la formula $\lnot T$.
Gli ultimi due predicati sono invece propri dei bigrafi. Si osservi la loro definizione: si pu� notare che il Wario Predicate fa riferimento alla struttura interna del bigrafo consentendo infinite scomposizioni. Tramite questo predicato posso quindi \emph{isolare} qualsiasi parte del bigrafo ed esprimere propriet� su di essa. Si pu� pensare alla sua funzionalit� in questo modo: $W_\sigma$ permette di spostarci all'interno del bigrafo, scegliere una sua parte ($\sigma$) e verificare se essa soddisfa una certa propriet�.
Il predicato $\pi_\sigma$, come abbiamo gi� notato, ci consente di avere una logica con uguaglianza, permettendo quindi di aumentare la sua espressibilit�.


\section{Dettagli Implementativi}
Nel codice sorgente di questa tesi, i bigrafi e i BRS, nonch� la loro evoluzione, � stata modellata tramite \emph{JLibbig} [...], una libreria Java che tra le altre cose consente di specificare le varie regole di reazione ed eseguirle sul bigrafo. 

\subsection{Property Matcher}
\emph{JLibbig} mette a disposizione la possibilit� di assegnare ad ogni nodo delle propriet�. Per esempio, ai due router dell'esempio \ref{sub:networkExe} si possono assegnare delle stringhe (in realt� qualsiasi tipo di oggetto) che descrivano il loro nome, per esempio $R_S$ (Router Sender) e $R_R$ (Router Receiver). Inoltre, � possibile estendere la classe Matcher e creare il proprio Matcher personale: in questa sede, si era interessati a definire un matcher in cui due nodi potessero \emph{costituire} un match se e solo se avessero le stesse propriet�. Per cui si � creato il \emph{Property Matcher} che svolge questo compito. Lo stesso vale per gli outernames e innernames: due outername possono costituire un match se e solo se hanno lo stesso nome.

Il \emph{Property Matcher} � molto comodo per esprimere le propriet� per il model checker: per esempio, potremmo essere interessati a sapere quando il pacchetto con destinazione $158.110.144.31$ arrivi all'host con tale indirizzo IP. Quindi si potrebbe creare un Wario Predicate che consenta di capire quando il pacchetto e il destinatario sono dentro lo stesso dominio. Per�, senza il \emph{Property Matcher}, la presenza di pi� pacchetti creerebbe confusione. Infatti non sapremmo pi� a quale pacchetto fare riferimento. Essendo interessati \emph{solamente} al pacchetto destinato a $158.110.144.31$, ci occorre il \emph{Property Matcher}.


\subsection{Regole di Reazione con Propriet�}
In \emph{JLibbig} le regole di reazione non fanno alcun riferimento alle propriet�, il che significa che dopo l'applicazione di una regola ogni nodo coinvolto (cio� attivo) perde le sue propriet�. Si � creata quindi una classe che ne consenta il \emph{mantenimento} anche dopo lo scatto della regola. Chiaramente, il modo in cui le propriet� si devono conservare � lasciato da definire all'utente, perch� � impossibile definirlo a priori. Per esempio: prendiamo la regola di figura \ref{fig:forwardRuleTemp}, che inoltra e duplica un pacchetto.


	\begin{figure*}[th]
	\centering
	\begin{tikzpicture}
	%\draw[help lines] (0,0) grid (14,5);
	%%%   Redex   %%%
	% Sx Big
	\draw[rounded corners = 3mm, dashed] (0.0,0.0) rectangle (3.0,4.0);%Root 1
	\draw[rounded corners = 3mm, thick] (0.3,0.3) rectangle (2.7,3.7);%Domain 1
	\draw[thick] (2.0,3.0) circle [radius=0.5];%Router 1
	\draw[rounded corners = 1mm, dashed, fill=myGrey] (2.0,0.5) rectangle (2.5,1.0);%Site 0
	\node at (2.3,0.75) {0};
	%Packet
	\draw[rounded corners=1mm, thick] (0.5,1.0) rectangle (1.7,1.7);
	\draw[rounded corners = 1mm, dashed, fill=myGrey] (1.0,1.1) rectangle (1.4,1.5);%Site 2
	\node at (1.2,1.3) {2};
	%idS
	\draw[myGreen, thick] (0.6,1.7) to [out=100,in=270] (0.5,4.2);
	\draw[fill] (0.6,1.7) circle [radius=0.05];
	\node[above] at (0.4,4.2) {$id_S$};
	%idR
	\draw[rounded corners = 5mm, myGreen, thick] (1.6,1.7) -- (2.0,1.9) -- (3.2,1.8) -- (5.0,1.5) -- (6.0,1.5) -- (6.0,4.2);
	\draw[fill] (1.6,1.7) circle [radius=0.05];
	\node[above] at (6.3,4.2) {$id_R$};
	
	%Dx Big
	\draw[rounded corners = 3mm, dashed] (3.5,0.0) rectangle (6.5,4.0);%Root 2
	\draw[rounded corners = 3mm, thick] (3.8,0.3) rectangle (6.2,3.7);%Domain 2
	\draw[thick] (4.5,3.0) circle [radius=0.5];%Router 2
	\draw[rounded corners = 1mm, dashed, fill=myGrey] (4.0,0.5) rectangle (4.5,1.0);%Site 1
	\node at (4.3,0.75) {1};
	
	%LinkR
	\draw[myGreen, thick] (2.0,3.5) to [out=90,in=270] (3.15,4.5);%r1-link
	\draw[fill] (2.0,3.5) circle [radius=0.05];
	\draw[myGreen, thick] (4.5,3.5) to [out=90,in=270] (3.15,4.5);%r2-link
	\draw[fill] (4.5,3.5) circle [radius=0.05];
	\node[above] at (3.15,4.5) {link};
	%LocalS
	\draw[rounded corners = 3mm, myGreen, thick] (2.0,2.5) -- (2.0,2.1) -- (1.2,2.1) -- (1.2,4.5);
	\draw[fill] (2.0,2.5) circle [radius=0.05];
	\node[above] at (1.2,4.5) {$local_S$};
	%LocalR
	\draw[rounded corners = 3mm, myGreen, thick] (4.5,2.5) -- (4.5,2.1) -- (5.3,2.1) -- (5.3,4.5);
	\draw[fill] (4.5,2.5) circle [radius=0.05];
	\node[above] at (5.3,4.5) {$local_R$};
	
	
	\draw[->, red, very thick] (6.8,2.0) -- (7.8,2.0);
	
	%%%   Reactum   %%%
	% Sx Big
	\draw[rounded corners = 3mm, dashed] (8.0,0.0) rectangle (11.0,4.0);%Root 1
	\draw[rounded corners = 3mm, thick] (8.3,0.3) rectangle (10.7,3.7);%Domain 1
	\draw[thick] (10.0,3.0) circle [radius=0.5];%Router 1
	\draw[rounded corners = 1mm, dashed, fill=myGrey] (10.0,0.5) rectangle (10.5,1.0);%Site 0
	\node [below] at (10.3,1.0) {0};
	
	%Dx Big
	\draw[rounded corners = 3mm, dashed] (11.3, 0.0) rectangle (13.99,4.0);%Root 2
	\draw[rounded corners = 3mm, thick] (11.6,0.3) rectangle (13.6,3.7);%Domain 2
	\draw[thick] (12.5,3.0) circle [radius=0.5];%Router 2
	\draw[rounded corners = 1mm, dashed, fill=myGrey] (11.7,2.1) rectangle (12.2,2.6);%Site 1
	\node [below] at (12.0,2.6) {1};
	%Packet 1
	\draw[rounded corners=1mm, thick] (12.0,1.3) rectangle (13.2,2.0);
	\draw[rounded corners = 1mm, dashed, fill=myGrey] (12.5,1.4) rectangle (13.0,1.9);%Site 2
	\node at (12.7,1.7) {2};
	%Packet 2
	\draw[rounded corners=1mm, thick] (12.0,0.4) rectangle (13.2,1.1);
	\draw[rounded corners = 1mm, dashed, fill=myGrey] (12.5,0.5) rectangle (13.0,0.9);%Site 3
	\node at (12.7,0.7) {3};
	%idS
	\draw[rounded corners = 5mm, myGreen, thick] (12.0,2.0) -- (11.0,1.9) -- (8.5,1.5) -- (8.5,4.2);
	\draw[fill] (12.0,2.0) circle [radius=0.05];
	\draw[rounded corners = 5mm, myGreen, thick] (12.0,1.1) to [out=120,in=0] (10.0,1.75);
	\draw[fill] (12.0,1.1) circle [radius=0.05];
	\node[above] at (8.4,4.2) {$id_S$};
	%idR
	\draw[rounded corners = 5mm, myGreen, thick] (13.2,2.0) to [out=60,in=270] (13.9,4.2);
	\draw[fill] (13.2,2.0) circle [radius=0.05];
	\draw[rounded corners = 5mm, myGreen, thick] (13.2,1.1) to [out=60,in=270] (13.4,2.35);
	\draw[fill] (13.2,1.1) circle [radius=0.05];
	\node[above] at (13.59,4.0) {$id_R$};
	
	%LinkR
	\draw[myGreen, thick] (10.0,3.5) to [out=90,in=270] (11.15,4.5);%r1-link
	\draw[fill] (10.0,3.5) circle [radius=0.05];
	\draw[myGreen, thick] (12.5,3.5) to [out=90,in=270] (11.15,4.5);%r2-link
	\draw[fill] (12.5,3.5) circle [radius=0.05];
	\node[above] at (11.15,4.5) {link};
	%LocalS
	\draw[rounded corners = 3mm, myGreen, thick] (10.0,2.5) -- (10.0,2.1) -- (9.2,2.1) -- (9.2,4.5);
	\draw[fill] (10.0,2.5) circle [radius=0.05];
	\node[above] at (9.2,4.5) {$local_S$};
	%LocalR
	\draw[rounded corners = 3mm, myGreen, thick] (12.5,2.5) -- (12.5,2.1) -- (13.2,2.1) -- (13.2,4.5);
	\draw[fill] (12.5,2.5) circle [radius=0.05];
	\node[above] at (13.0,4.5) {$local_R$};
	
	\end{tikzpicture}
	\caption{Regola di inoltro tra router. \label{fig:forwardRuleTemp}}
	\end{figure*}


Secondo la definizione di regola di reazione, il redex viene sostituito dal reactum: esso per� � un \emph{nuovo} bigrafo e quindi ha nuovi nomi per tutti gli outer e inner names, e tutti i suoi nodi sono privi di propriet�. Supponiamo che prima dell'esecuzione della regola il pacchetto del redex abbia come propriet� la stringa $P_1$. Dopo l'esecuzione si vuole che il primo pacchetto del reactum abbia tutte le propriet� di $P_1$, mentre il secondo abbia la nuova propriet� $New_Packet$. E' ovvio che tale scelta � arbitraria, ed � questo il motivo per cui la definizione del modo in cui le propriet� si conservano � stata lasciata all'utente.

Infine, creando delle regole di reazione che consentano il mantenimento delle propriet�, � possibile usare il \emph{Property Matcher} anche dopo l'esecuzione di varie regole: di conseguenza lo possiamo usare anche nel model checker $MC_{big}$.





\section{Esempi}
Si vedranno ora degli esempi di formule e di come poterle usare con il model checker $MC_{big}$.

\subsection{Moltiplicazione}
Il primo esempio che si propone riprende la moltiplicazione tra numeri naturali della sottosezione \ref{sub:exeIntro}. Se rappresentiamo la moltiplicazione $x*y$, ci possiamo chiedere se il BRS funzioni correttamente con le regole che abbiamo definito e verificare che il risultato sia corretto. La propriet� da verificare � quindi questa: ``Dati due numeri x e y, il risultato della loro moltiplicazione deve essere il numero $x*y$". Vediamo ora la formula corrispondente: sappiamo che con la segnatura iniziale (vedi \ref{sub:exeIntro}) un numero naturale $n$ � rappresentato da un nodo di tipo $num$ che contiene $n$ nodi di tipo $1$. Per esempio, il numero 8 � il bigrafo di figura \ref{fig:big8}.

\begin{figure}[!htbp]
\centering
\begin{tikzpicture}
%\draw[help lines] (0,0) grid (6,5);
%Root
\draw[rounded corners=4mm,dotted, thick] (0.0,0.0) rectangle (5.0,5.0);
\node[below right] at (0.0,5.0) {0};
%Nodes
\draw[thick] (2.5,2.5) circle [radius=2.0];
\node[above] at (2.5,4.0) {num};
\draw[thick] (1.0,3.0) rectangle (1.4,3.4);%one
\node[above right] at (1.0,2.95) {1};
\draw[thick] (2.0,3.0) rectangle (2.4,3.4);%one
\node[above right] at (2.0,2.95) {1};
\draw[thick] (3.0,3.0) rectangle (3.4,3.4);%one
\node[above right] at (3.0,2.95) {1};
\draw[thick] (1.0,2.0) rectangle (1.4,2.4);%one
\node[above right] at (1.0,1.95) {1};
\draw[thick] (2.0,2.0) rectangle (2.4,2.4);%one
\node[above right] at (2.0,1.95) {1};
\draw[thick] (3.0,2.0) rectangle (3.4,2.4);%one
\node[above right] at (3.0,1.95) {1};
\draw[thick] (2.0,1.0) rectangle (2.4,1.4);%one
\node[above right] at (2.0,0.95) {1};
\draw[thick] (3.0,1.0) rectangle (3.4,1.4);%one
\node[above right] at (3.0,0.95) {1};

\end{tikzpicture}
\caption{Bigrafo per il numero 8 \label{fig:big8}}
\end{figure}

Chiamiamo $B$ il bigrafo di figura \ref{fig:big8}. La propriet� di cui sopra si pu� esprimere con la seguente formula logica: $\varphi = \pi_B$.

Diamo ora al model checker la formula $\varphi$. $MC_{big}$ incomincer� a generare il grafo degli stati con la strategia Breadth First (se non specificato altrimenti), e per ogni nuovo stato $S_i$ controller� se $MC,S_i \models \varphi$. In questo esempio, il grafo generato sar� quello di figura \ref{fig:modelCheckerMult}. Per gli stati $S_0$ e $S_1$ il model checker trover� che la propriet� non � soddisfatta, perch� nessuno di questi stati � isomorfo a B. Arrivando per� a $S_2$, la propriet� $\varphi$ sar� soddisfatta, ovvero $MC,S_2 \models \varphi$, e $MC_{big}$ ritorner� True.\\

\begin{figure*}[th]
	\centering
	\begin{tikzpicture}
	%\draw[help lines] (0,0) grid (14,5);
	%States
	\draw[thick] (0.5,0.5) circle [radius=0.5];
	\node at (0.5,0.5) {$S_0$};
	\draw[thick] (2.5,0.5) circle [radius=0.5];
	\node at (2.5,0.5) {$S_1$};
	\draw[thick, red] (4.5,0.5) circle [radius=0.5];
	\node[red] at (4.5,0.5) {$S_2$};
	
	%Edges
	\draw[->, thick] (1.0,0.5) -- (2.0,0.5);%s0-s1
	\node[above] at (1.5,0.5) {$R_0$};
	\draw[->, thick] (3.0,0.5) -- (4.0,0.5);%s1-s2
	\node[above] at (3.5,0.5) {$R_0$};
	
	\end{tikzpicture}
	\caption{Model Checker. \label{fig:modelCheckerMult}}
\end{figure*}





\subsection{Router}
Il secondo esempio riprende quello della sottosezione \ref{sub:networkExe}. Vogliamo modellare una rete con quattro domini, come quella di figura \ref{fig:moreRouters}. L'host $h_1$ vuole comunicare con $h_4$, inviando un pacchetto IP. Tra questi due host ci sono due domini, con due router ciascuno. Prendiamo in considerazione il dominio $D_2$: l'arco tra $R_{2.1}$ e $R_{2.2}$ significa che i due router sono collegati e quindi, essendo nello stesso dominio, ogni pacchetto che arriver� a $R_{2.1}$ arriver� anche a $R_{2.1}$.

\begin{figure*}[th]
	\centering
	\begin{tikzpicture}
	%\draw[help lines] (0,0) grid (14,6);
	%%%   First and Second Domains
	% Sx Big
	\draw[rounded corners = 3mm, dashed] (0.0,0.0) rectangle (3.0,4.0);%Root 1
	\draw[rounded corners = 3mm, thick] (0.3,0.3) rectangle (2.7,3.7);%Domain 1
	\node at (2.0,4.2){$D_1$};
	\draw[thick] (2.0,3.0) circle [radius=0.5];%Router 1
	\node at (2.0,3.0){$R_1$};
	\draw[rounded corners = 1mm, thick] (0.7,0.5) -- (1.5,0.5) -- (1.1,1.2) -- (0.7,0.5);%h1
	\node at (1.15,0.75) {$h_1$};
	\draw[rounded corners = 1mm, thick] (4.0,0.5) -- (4.8,0.5) -- (4.4,1.2) -- (4.0,0.5);%h2
	\node at (4.45,0.75) {$h_2$};
	%Packet
	\draw[rounded corners=1mm, thick] (1.4,1.0) rectangle (2.6,1.7);
	\draw[rounded corners = 1mm, thick, fill=myGrey] (1.6,1.1) rectangle (2.4,1.5);%tcp
	\node at (2.0,1.3) {tcp};
	%packet - idS
	\draw[myGreen, thick] (1.5,1.7) to [out=90,in=300] (0.93,2.5);
	\draw[fill] (1.5,1.7) circle [radius=0.05];
	%packet - idR
	\draw[myGreen, thick] (2.5,1.7) to [out=10,in=230] (12.43,1.7);
	\draw[fill] (2.5,1.7) circle [radius=0.05];
	
	
	%Dx Big
	\draw[rounded corners = 3mm, dashed] (3.5,0.0) rectangle (6.5,4.0);%Root 2
	\draw[rounded corners = 3mm, thick] (3.8,0.3) rectangle (6.2,3.7);%Domain 2
	\node at (4.5,4.2){$D_2$};
	\draw[thick] (4.5,3.0) circle [radius=0.5];%Router 2.1
	\node at (4.5,3.0){$R_{2.1}$};
	\draw[thick] (5.6,3.0) circle [radius=0.5];%Router 2.2
	\node at (5.6,3.0){$R_{2.2}$};
	%Down R.2.1 - R.2.2 - h2
	\draw[myGreen, thick] (4.5,2.5) to [out=270,in=270] (5.6,2.5);
	\draw[myGreen, thick] (4.4,1.2) to [out=90,in=270] (5.0,2.2);
	\draw[fill] (4.5,2.5) circle [radius=0.05];
	\draw[fill] (5.6,2.5) circle [radius=0.05];
	\draw[fill] (4.4,1.2) circle [radius=0.05];
	
	%LinkR
	\draw[myGreen, thick] (2.0,3.5) to [out=90,in=270] (3.15,4.5);%r1-link
	\draw[fill] (2.0,3.5) circle [radius=0.05];
	\draw[myGreen, thick] (4.5,3.5) to [out=90,in=270] (3.15,4.5);%r2-link
	\draw[fill] (4.5,3.5) circle [radius=0.05];
	\node[above] at (3.15,4.5) {link};
	%idS
	\node[above] at (1.2,4.5) {$158.110.3.46$};
	\draw[rounded corners = 3mm, myGreen, thick] (1.1,1.2) to [out=90,in=270] (0.6,4.5);
	\draw[fill] (1.1,1.2) circle [radius=0.05];
	
	%LocalS
	\draw[rounded corners = 3mm, myGreen, thick] (1.1,0.5) -- (1.1,0.2) --  (2.0,0.2) -- (2.0,2.5);
	\draw[fill] (2.0,2.5) circle [radius=0.05];
	\draw[fill] (1.1,0.5) circle [radius=0.05];
	
	%Link R.2.2 - R.3.1
	\draw[myGreen, thick] (5.6,3.5) to [out=90,in=270] (6.5,4.5);
	\draw[fill] (5.6,3.5) circle [radius=0.05];
	\draw[myGreen, thick] (7.9,3.5) to [out=90,in=270] (6.5,4.5);
	\draw[fill] (7.9,3.5) circle [radius=0.05];
	\node[above] at (6.5,4.5) {link};
	
	
	%%%   Third  and Fourth Domains   %%%
	% Sx Big
	\draw[rounded corners = 3mm, dashed] (7.0,0.0) rectangle (10.0,4.0);%Root 1
	\draw[rounded corners = 3mm, thick] (7.3,0.3) rectangle (9.7,3.7);%Domain 3
	\node at (8.5,4.2){$D_3$};
	\draw[thick] (7.9,3.0) circle [radius=0.5];%Router 3.1
	\node at (7.9,3.0){$R_{3.1}$};
	\draw[thick] (9.0,3.0) circle [radius=0.5];%Router 3.2
	\node at (9.0,3.0){$R_{3.2}$};
	
	%Dx Big
	\draw[rounded corners = 3mm, dashed] (10.3, 0.0) rectangle (13.3,4.0);%Root 2
	\draw[rounded corners = 3mm, thick] (10.6,0.3) rectangle (13.0,3.7);%Domain 4
	\node at (11.5,4.2){$D_4$};
	\draw[rounded corners = 1mm, thick] (12.0,0.5) -- (12.8,0.5) -- (12.4,1.3) -- (12.0,0.5);%h4
	\node at (12.4,0.75) {$h_4$};
	\draw[thick] (11.5,3.0) circle [radius=0.5];%Router 4
	\node at (11.5,3.0){$R_{4}$};
	
	%Down R.3.1-R.3.2
	\draw[myGreen, thick] (7.9,2.5) to [out=270,in=270] (9.0,2.5);
	\draw[fill] (7.9,2.5) circle [radius=0.05];
	\draw[fill] (9.0,2.5) circle [radius=0.05];
	
	%idR
	\node[above] at (12.3,4.5) {$158.110.144.31$};
	\draw[rounded corners = 3mm, myGreen, thick] (12.4,1.3) to [out=90,in=270] (13.0,4.5);
	\draw[fill] (12.4,1.3) circle [radius=0.05];
	
	%LinkR
	\draw[myGreen, thick] (9.0,3.5) to [out=90,in=270] (10.15,4.5);%r1-link
	\draw[fill] (9.0,3.5) circle [radius=0.05];
	\draw[myGreen, thick] (11.5,3.5) to [out=90,in=270] (10.15,4.5);%r2-link
	\draw[fill] (11.5,3.5) circle [radius=0.05];
	\node[above] at (10.15,4.5) {link};
	
	%LocalR
	\draw[rounded corners = 3mm, myGreen, thick] (11.5,2.5) -- (11.5,0.2) -- (12.4,0.2) -- (12.4,0.5);
	\draw[fill] (11.5,2.5) circle [radius=0.05];
	\draw[fill] (12.4,0.5) circle [radius=0.05];
	
	\end{tikzpicture}
	\caption{Bigrafo di partenza \label{fig:moreRouters}}
	\end{figure*}



Si ricordi che un router inoltra non deterministicamente ogni pacchetto verso tutte le uscite (interfacce). Questo permette di creare un compromesso tra numero di regole ed efficienza del BRS. Perci�, l'unica regola di cui abbiamo bisogno � quella di figura \ref{fig:forwardRuleBis} (si rimanda alla sottosezione \ref{sub:networkExe} per la sua descrizione).


	\begin{figure*}[th]
	\centering
	\begin{tikzpicture}
	%\draw[help lines] (0,0) grid (14,5);
	%%%   Redex   %%%
	% Sx Big
	\draw[rounded corners = 3mm, dashed] (0.0,0.0) rectangle (3.0,4.0);%Root 1
	\draw[rounded corners = 3mm, thick] (0.3,0.3) rectangle (2.7,3.7);%Domain 1
	\draw[thick] (2.0,3.0) circle [radius=0.5];%Router 1
	\draw[rounded corners = 1mm, dashed, fill=myGrey] (2.0,0.5) rectangle (2.5,1.0);%Site 0
	\node at (2.3,0.75) {0};
	%Packet
	\draw[rounded corners=1mm, thick] (0.5,1.0) rectangle (1.7,1.7);
	\draw[rounded corners = 1mm, dashed, fill=myGrey] (1.0,1.1) rectangle (1.4,1.5);%Site 2
	\node at (1.2,1.3) {2};
	%idS
	\draw[myGreen, thick] (0.6,1.7) to [out=100,in=270] (0.5,4.2);
	\draw[fill] (0.6,1.7) circle [radius=0.05];
	\node[above] at (0.4,4.2) {$id_S$};
	%idR
	\draw[rounded corners = 5mm, myGreen, thick] (1.6,1.7) -- (2.0,1.9) -- (3.2,1.8) -- (5.0,1.5) -- (6.0,1.5) -- (6.0,4.2);
	\draw[fill] (1.6,1.7) circle [radius=0.05];
	\node[above] at (6.3,4.2) {$id_R$};
	
	%Dx Big
	\draw[rounded corners = 3mm, dashed] (3.5,0.0) rectangle (6.5,4.0);%Root 2
	\draw[rounded corners = 3mm, thick] (3.8,0.3) rectangle (6.2,3.7);%Domain 2
	\draw[thick] (4.5,3.0) circle [radius=0.5];%Router 2
	\draw[rounded corners = 1mm, dashed, fill=myGrey] (4.0,0.5) rectangle (4.5,1.0);%Site 1
	\node at (4.3,0.75) {1};
	
	%LinkR
	\draw[myGreen, thick] (2.0,3.5) to [out=90,in=270] (3.15,4.5);%r1-link
	\draw[fill] (2.0,3.5) circle [radius=0.05];
	\draw[myGreen, thick] (4.5,3.5) to [out=90,in=270] (3.15,4.5);%r2-link
	\draw[fill] (4.5,3.5) circle [radius=0.05];
	\node[above] at (3.15,4.5) {link};
	%LocalS
	\draw[rounded corners = 3mm, myGreen, thick] (2.0,2.5) -- (2.0,2.1) -- (1.2,2.1) -- (1.2,4.5);
	\draw[fill] (2.0,2.5) circle [radius=0.05];
	\node[above] at (1.2,4.5) {$local_S$};
	%LocalR
	\draw[rounded corners = 3mm, myGreen, thick] (4.5,2.5) -- (4.5,2.1) -- (5.3,2.1) -- (5.3,4.5);
	\draw[fill] (4.5,2.5) circle [radius=0.05];
	\node[above] at (5.3,4.5) {$local_R$};
	
	
	\draw[->, red, very thick] (6.8,2.0) -- (7.8,2.0);
	
	%%%   Reactum   %%%
	% Sx Big
	\draw[rounded corners = 3mm, dashed] (8.0,0.0) rectangle (11.0,4.0);%Root 1
	\draw[rounded corners = 3mm, thick] (8.3,0.3) rectangle (10.7,3.7);%Domain 1
	\draw[thick] (10.0,3.0) circle [radius=0.5];%Router 1
	\draw[rounded corners = 1mm, dashed, fill=myGrey] (10.0,0.5) rectangle (10.5,1.0);%Site 0
	\node [below] at (10.3,1.0) {0};
	
	%Dx Big
	\draw[rounded corners = 3mm, dashed] (11.3, 0.0) rectangle (13.99,4.0);%Root 2
	\draw[rounded corners = 3mm, thick] (11.6,0.3) rectangle (13.6,3.7);%Domain 2
	\draw[thick] (12.5,3.0) circle [radius=0.5];%Router 2
	\draw[rounded corners = 1mm, dashed, fill=myGrey] (12.0,0.4) rectangle (12.5,0.9);%Site 1
	\node [below] at (12.3,0.9) {1};
	%Packet
	\draw[rounded corners=1mm, thick] (12.0,1.0) rectangle (13.2,1.7);
	\draw[rounded corners = 1mm, dashed, fill=myGrey] (12.5,1.1) rectangle (13.0,1.5);%Site 2
	\node at (12.7,1.3) {2};
	%idS
	\draw[rounded corners = 5mm, myGreen, thick] (12.0,1.7) -- (11.0,1.9) -- (8.5,1.5) -- (8.5,4.2);
	\draw[fill] (12.0,1.7) circle [radius=0.05];
	\node[above] at (8.4,4.2) {$id_S$};
	%idR
	\draw[rounded corners = 5mm, myGreen, thick] (13.2,1.7) to [out=60,in=270] (13.9,4.2);
	\draw[fill] (13.2,1.7) circle [radius=0.05];
	\node[above] at (13.59,4.0) {$id_R$};
	
	%LinkR
	\draw[myGreen, thick] (10.0,3.5) to [out=90,in=270] (11.15,4.5);%r1-link
	\draw[fill] (10.0,3.5) circle [radius=0.05];
	\draw[myGreen, thick] (12.5,3.5) to [out=90,in=270] (11.15,4.5);%r2-link
	\draw[fill] (12.5,3.5) circle [radius=0.05];
	\node[above] at (11.15,4.5) {link};
	%LocalS
	\draw[rounded corners = 3mm, myGreen, thick] (10.0,2.5) -- (10.0,2.1) -- (9.2,2.1) -- (9.2,4.5);
	\draw[fill] (10.0,2.5) circle [radius=0.05];
	\node[above] at (9.2,4.5) {$local_S$};
	%LocalR
	\draw[rounded corners = 3mm, myGreen, thick] (12.5,2.5) -- (12.5,2.1) -- (13.2,2.1) -- (13.2,4.5);
	\draw[fill] (12.5,2.5) circle [radius=0.05];
	\node[above] at (13.0,4.5) {$local_R$};
	
	\end{tikzpicture}
	\caption{Regola di inoltro tra router. \label{fig:forwardRuleBis}}
	\end{figure*}



Ora il problema di capire quando un pacchetto � arrivato a destinazione diventa pi� interessante. Ci sono tre host, e bisogna prestare attenzione a quale sia il corretto destinatario. 

Per esempio, non si deve fare il seguente errore: vogliamo esprimere la propriet� ``Il pacchetto � arrivato a destinazione". Nella logica descritta precedentemente, si potrebbe sbagliare e creare una formula del genere: $\varphi = W_B(T,T,T)$, dove B � il bigrafo di figura \ref{fig:wrongRightProps}.a.



\begin{figure}[!h]
\centering
\subfigure[Predicato Sbagliato]{
\begin{tikzpicture}
%\draw[help lines] (0,0) grid (14,5);
	%Root
	\draw[rounded corners=5mm, dashed] (0.0,0.0) rectangle (4.0,4.0);
	%Domain
	\draw[rounded corners=4mm, thick] (0.3,0.3) rectangle (3.7,3.7);
	%Router
	\draw[thick] (2.0,3.0) circle [radius=0.5];
	\node at (2.0,3.0) {$R$};
	%Host
	\draw[rounded corners=1mm, thick] (2.5,1.0) -- (3.5,1.0) -- (3.0,2.0) -- (2.5,1.0);
	\node at (3.05,1.3) {$h_1$};
	
	%Link Router
	\draw[myGreen, thick] (2.0,3.5) to [out=110,in=290] (2.0,4.5);
	\draw[fill] (2.0,3.5) circle [radius=0.05];
	\node[above] at (2.0,4.5){$link_R$};
	%Link idR
	\draw[rounded corners=2mm, myGreen, thick] (2.0,2.5) -- (2.0,2.2) -- (3.0,2.2) -- (3.0,4.5);
	\draw[myGreen, thick] (3.0,2.0) to [out=110,in=290] (3.0,2.35);
	\draw[fill] (2.0,2.5) circle [radius=0.05];
	\draw[fill] (3.0,2.0) circle [radius=0.05];
	\node[above] at (3.0,4.5){$id_H$};
	%Packet
	\draw[rounded corners=2mm, thick] (0.5,1.0) rectangle (2.0,1.7);
	\draw[fill] (0.7,1.7) circle [radius=0.05];
	\draw[fill] (1.8,1.7) circle [radius=0.05];
	%Site in packet
	\draw[rounded corners=1mm, fill=myGrey, dashed] (1.0,1.1) rectangle (1.5,1.6);
	\node at (1.25,1.35){0};
	%idS
	\draw[myGreen, thick] (0.7,1.75) to [out=110,in=270] (1.0,4.5);
	\node[above] at (1.0,4.5){$id_S$};
	%idR
	\draw[myGreen, thick] (1.8,1.75) to [out=60,in=270] (4.0,4.5);
	\node[above] at (4.0,4.5){$id_R$};
	%Site in domain
	\draw[rounded corners=1mm, fill=myGrey, dashed] (2.5,0.4) rectangle (3.0,0.9);
	\node at (2.75,0.70){1};

\end{tikzpicture}
}
\hspace{5mm}
\subfigure[Predicato Corretto]{
\begin{tikzpicture}
%\draw[help lines] (0,0) grid (14,5);
	%Root
	\draw[rounded corners=5mm, dashed] (0.0,0.0) rectangle (4.0,4.0);
	%Domain
	\draw[rounded corners=4mm, thick] (0.3,0.3) rectangle (3.7,3.7);
	%Router
	\draw[thick] (2.0,3.0) circle [radius=0.5];
	\node at (2.0,3.0) {$R$};
	%Host
	\draw[rounded corners=1mm, thick] (2.5,1.0) -- (3.5,1.0) -- (3.0,2.0) -- (2.5,1.0);
	\node at (3.05,1.3) {$h_1$};
	
	%Link Router
	\draw[myGreen, thick] (2.0,3.5) to [out=110,in=290] (2.0,4.5);
	\draw[fill] (2.0,3.5) circle [radius=0.05];
	\node[above] at (2.0,4.5){$link_R$};
	%Link idR
	\draw[rounded corners=2mm, myGreen, thick] (2.0,2.5) -- (2.0,2.2) -- (3.0,2.2) -- (3.0,4.5);
	\draw[myGreen, thick] (3.0,2.0) to [out=110,in=290] (3.0,2.35);
	\draw[fill] (2.0,2.5) circle [radius=0.05];
	\draw[fill] (3.0,2.0) circle [radius=0.05];
	\node[above] at (3.0,4.5){$id_R$};
	%Packet
	\draw[rounded corners=2mm, thick] (0.5,1.0) rectangle (2.0,1.7);
	\draw[fill] (0.7,1.7) circle [radius=0.05];
	\draw[fill] (1.8,1.7) circle [radius=0.05];
	%Site in packet
	\draw[rounded corners=1mm, fill=myGrey, dashed] (1.0,1.1) rectangle (1.5,1.6);
	\node at (1.25,1.35){0};
	%idS
	\draw[myGreen, thick] (0.7,1.75) to [out=110,in=270] (1.0,4.5);
	\node[above] at (1.0,4.5){$id_S$};
	%idR
	\draw[myGreen, thick] (1.8,1.75) to [out=60,in=270] (2.2,2.2);
	%Site in domain
	\draw[rounded corners=1mm, fill=myGrey, dashed] (2.5,0.4) rectangle (3.0,0.9);
	\node at (2.75,0.70){1};

\end{tikzpicture}
}
\caption{Esempi di predicati\label{fig:wrongRightProps}}
\end{figure}


Questa propriet� \textbf{non} verifica l'arrivo a destinazione di un pacchetto, infatti � soddisfatta anche esso si trova per esempio nel dominio $D_2$. Quello che non abbiamo considerato sono gli outernames: quello del pacchetto e quello dell'host devono essere lo stesso outername; solo cos� il \emph{Wario Predicate} sar� soddisfatto. Il predicato corretto � quindi quello di figura \ref{fig:wrongRightProps}.a.

Testiamo ora la formula $\varphi = W_B(T,T,T)$ sul bigrafo di partenza, cio� quello di figura \ref{fig:moreRouters}, che sar� chiamato $S_0$. Per prima cosa, il model checker controlla se in questo stato iniziale la formula sia soddisfatta, ovvero se $MC,S_0 \models \varphi$. Non essendo soddisfatta continua. Inizialmente la regola $R_0$ (in figura \ref{fig:forwardRuleBis}) trova un solo match, per cui il pacchetto viene inoltrato dal dominio $D_1$ a $D_2$, dando origine allo stato $S_1$. E' utile guardare il grafo degli stati di figura \ref{fig:stateGraphNetworkLoops} per tenere traccia di tutte le esecuzioni. 

\begin{figure}[th]
\centering
\begin{tikzpicture}
%\draw[help lines] (0,0) grid (13,5);
%Nodes
\draw[thick] (1.5,1.5) circle [radius=0.5];%s0
\node at (1.5,1.5){$S_0$};
\draw[thick] (3.5,1.5) circle [radius=0.5];%s1
\node at (3.5,1.5){$S_1$};
\draw[thick] (5.5,1.5) circle [radius=0.5];%s2
\node at (5.5,1.5){$S_2$};
\draw[thick, red] (7.5,1.5) circle [radius=0.5];%s3
\node[red] at (7.5,1.5){$S_3$};
%Edges
\draw[->, thick] (2.0,1.5) -- (3.0,1.5);%s0-s1
\draw[<-, thick] (1.5,2.0) to [out=45,in=135] (3.5,2.0);%s1-s0
\draw[->, thick] (4.0,1.5) -- (5.0,1.5);%s1-s2
\draw[<-, thick] (3.5,1.0) to [out=-45,in=-135] (5.5,1.0);%s2-s1
\draw[->, thick] (6.0,1.5) -- (7.0,1.5);%s2-s3
\draw[<-, thick] (5.5,2.0) to [out=45,in=135] (7.5,2.0);%s3-s2

\end{tikzpicture}
\caption{Grafo degli stati \label{fig:stateGraphNetworkLoops}}
\end{figure}


In $S_1$ la formula non � soddisfatta ($MC,S_1 \not\models \varphi$), quindi $MC_{big}$ continua. Dallo stato $S_1$, la regola $R_0$ trova ora due match: il primo coinvolge il router $R_{2.2}$ e mander� il pacchetto nel dominio $D_3$, mentre il secondo riguarda il router $R_{2.1}$ e mander� il pacchetto indietro verso $D_1$. Quindi il grafo degli stati avr� rispettivamente gli archi: $S_1 \to S_2$ e $S_1 \to S_0$. Il model checker ora dovr� duplicare le computazioni, cio� seguire sia il primo che il secondo arco.

Seguendo l'arco $S_1 \to S_0$, il MC trova il nodo $S_0$ grazie all'algoritmo per l'isomorfismo e riconosce che appartiene gi� al grafo degli stati. La computazione di questo ramo \textbf{viene interrotta} perch� il model checker capisce che continuando si causerebbe un'esecuzione infinita.

Seguendo l'arco $S_1 \to S_2$, si genera invece un nuovo stato $S_2$. Il model checker cerca in tutto il grafo degli stati un nodo isomorfo a $S_2$ e, se non lo trova, capisce che esso � un \emph{nuovo} stato. Dato che $MC,S_2 \not\models \varphi$, la computazione di questo ramo quindi continua. 

Applicando $R_0$ a $S_2$ il discorso � lo stesso: la computazione che segue l'arco $S_2 \to S_1$ si interrompe, mentre quella che segue $S_2 \to S_3$ continua. Si noti come la strategia \emph{Breadth First} consenta di interrompere da subito le computazioni che causano cicli infiniti: � uno dei vantaggi di questa strategia.

Infine, lo stato $S_3$ � il bigrafo in cui il pacchetto � dentro il dominio $D_4$. La regola $R_0$ trova solamente un match, ed eseguendola rimanda il pacchetto indietro verso il  dominio $D_3$, creando nel grafo degli stati l'arco $S_3 \to S_2$. Questa volta per� lo stato $S_3$ soddisfa la propriet� desiderata: il model checker trova che 
\begin{center}
$MC,S_3 \models W_B(T,T,T) \qquad$ dove B � il bigrafo di figura \ref{fig:wrongRightProps}.b
\end{center}
e quindi ritorna True.

Un'ultima osservazione: nel caso di pi� pacchetti si deve prestare attenzione ad un altro dettaglio. Se siamo interessati a verificare che il pacchetto spedito da $158.110.3.46$ a $158.110.144.31$ sia arrivato a destinazione, dobbiamo usare il Property Matcher. Definiremo quindi la formula sempre come $\varphi = W_B(T,T,T)$, ma ora il bigrafo B avr� nomi specifici nei suoi outernames, come in figura \ref{fig:propertyPredicate}.


\begin{figure}[th]
\centering
\begin{tikzpicture}
%\draw[help lines] (0,0) grid (14,5);
	%Root
	\draw[rounded corners=5mm, dashed] (0.0,0.0) rectangle (4.0,4.0);
	%Domain
	\draw[rounded corners=4mm, thick] (0.3,0.3) rectangle (3.7,3.7);
	%Router
	\draw[thick] (2.0,3.0) circle [radius=0.5];
	\node at (2.0,3.0) {$R$};
	%Host
	\draw[rounded corners=1mm, thick] (2.5,1.0) -- (3.5,1.0) -- (3.0,2.0) -- (2.5,1.0);
	\node at (3.05,1.3) {$h_1$};
	
	%Link Router
	\draw[myGreen, thick] (2.0,3.5) to [out=110,in=290] (2.0,4.5);
	\draw[fill] (2.0,3.5) circle [radius=0.05];
	\node[above] at (2.0,4.5){$link_R$};
	%Link idR
	\draw[rounded corners=2mm, myGreen, thick] (2.0,2.5) -- (2.0,2.2) -- (3.0,2.2) -- (3.0,4.5);
	\draw[myGreen, thick] (3.0,2.0) to [out=110,in=290] (3.0,2.35);
	\draw[fill] (2.0,2.5) circle [radius=0.05];
	\draw[fill] (3.0,2.0) circle [radius=0.05];
	%Packet
	\draw[rounded corners=2mm, thick] (0.5,1.0) rectangle (2.0,1.7);
	\draw[fill] (0.7,1.7) circle [radius=0.05];
	\draw[fill] (1.8,1.7) circle [radius=0.05];
	%Site in packet
	\draw[rounded corners=1mm, fill=myGrey, dashed] (1.0,1.1) rectangle (1.5,1.6);
	\node at (1.25,1.35){0};
	%idS
	\draw[myGreen, thick] (0.7,1.75) to [out=110,in=270] (1.0,4.5);
	\node[above left] at (1.0,4.5){$158.110.3.46$};
	%idR
	\draw[myGreen, thick] (1.8,1.75) to [out=60,in=270] (2.2,2.2);
	\node[above right] at (3.0,4.5){$158.110.144.31$};
	%Site in domain
	\draw[rounded corners=1mm, fill=myGrey, dashed] (2.5,0.4) rectangle (3.0,0.9);
	\node at (2.75,0.70){1};

\end{tikzpicture}
\caption{Predicato con propriet� \label{fig:propertyPredicate}}
\end{figure}

















\chapter{Casi di studio}\label{ch:examples}
In questo capitolo vedremo degli esempi che riguarderanno principalmente il model checker $MC_{big}$ e la sua logica. Si cambier� spesso dominio, mostrando come i bigrafi siano flessibili per rappresentare vari tipi di sistema.

\section{NFA}\label{sec:nfa}
In questa sezione si propone una codifica in bigrafi degli NFA (\emph{non-deterministic finite automata}). Come noto, ogni automa A denota un linguaggio L(A). Vedremo come l'implementazione di un automa con i bigrafi consenta di avere una sorta di analizzatore lessicale il cui motore interno funziona tramite BRS. In questa sede, si � costruito un modulo che accetta dall'utente una stringa $x$ e restituisce True se $x \in L(A)$, altrimenti False.


\begin{figure}[th]
\centering
\begin{tikzpicture}
%\draw[help lines] (0,0) grid (10,5);
%State q0
\draw[thick] (2.5,2.5) circle [radius=0.5];
\draw[thick] (2.5,2.5) circle [radius=0.45];
\node at (2.5,2.5){$q_0$};
%State q1
\draw[thick] (5.5,2.5) circle [radius=0.5];
\node at (5.5,2.5){$q_1$};
%Edges
\draw[->,thick] (2.5,3.0) to [out=45,in=135] (5.5,3.03);%qo-q1   a
\node at (4.0,4.0) {$a$};
\draw[<-,thick] (2.85,2.12) to [out=-20,in=-160] (5.15,2.15); %q1-q0   a
\node[above] at (4.0,2.0) {$a$};
\draw[<-,thick] (2.5,1.97) to [out=-45,in=-135] (5.5,2.0); %q1-q0   b
\node at (4.0,1.0) {$b$};
\end{tikzpicture}
\caption{Automa per il linguaggio $(a(a+b))^*$ \label{fig:automa1}}
\end{figure}


Prendiamo l'automa di figura \ref{fig:automa1}, che chiameremo A. Definiamolo formalmente:
\begin{prop}
L'automa A � una quintupla $(Q,\Sigma,\delta,q_0,F)$, dove:
\begin{itemize}
	\item
	$Q=\{q_0,q_1\}$ � l'insieme finito di stati
	\item
	$\Sigma = \{a,b\}$ � l'alfabeto di input
	\item
	$\delta : Q \times \Sigma \to Q$ � la funzione di transizione definita come:
		\begin{center}
		$\delta(q_0,a)=\{q_1\}$\\
		$\delta(q_1,a)=\{q_0\}$\\
		$\delta(q_1,b)=\{q_0\}$
		\end{center}
	\item
	$q_0$ � lo stato iniziale
	\item
	$F=\{q_0\}$ � l'insieme di stati finali
\end{itemize}
\end{prop}

A riconosce il linguaggio $L(A) = (a(a+b))^*$. 

\begin{figure}[th]
\centering
\begin{tikzpicture}
%\draw[help lines] (0,0) grid (14,7);
%Root
\draw[rounded corners=5mm, dashed] (-0.5,-0.5) rectangle (13.5,5.5);
%Edges
\node[above] at (6.5,5.5){a};
\draw[myGreen,thick] (3.0,4.5) to [out=45,in=270] (6.5,5.5);%q0-a-q1
\draw[myGreen,thick] (10.0,4.5) to [out=135,in=270] (6.5,5.5);%q1-a-q0
\node[above] at (11.5,5.5){b};
\draw[myGreen,thick] (10.0,2.5) to [out=70,in=270] (11.5,5.5);%q1-b-q0

\draw[myGreen,thick] (3.5,4.0) to [out=10,in=190] (8.0,4.0);%t0a - s1
\draw[myGreen,thick] (9.5,4.0) to [out=180,in=0] (5.0,2.5);%t1a-s0
\draw[myGreen,thick] (9.5,2.0) to [out=180,in=0] (5.0,2.5);%t1b-s0


%First State
\draw[rounded corners=5mm, thick] (0.0,0.0) rectangle (5.0,5.0);%s0
\node[above] at (0.5,5.0){$q_0$};
\draw[fill] (5.0,2.5) circle [radius=0.05];
\node[left] at (5.0,2.5){0};
%Active
\draw[thick] (1.5,1.5) circle [radius=0.5];
\node at (1.5,1.5){A};
%Final
\draw[thick] (3.5,1.5) circle [radius=0.5];
\node at (3.5,1.5){F};
%transition q0-a-q1
\draw[thick] (3.0,4.0) circle [radius=0.5];
\node at (3.0,4.0){T};
\draw[fill] (3.5,4.0) circle [radius=0.05]; \node[below right] at (3.5,4.0){1};
\draw[fill] (3.0,4.5) circle [radius=0.05]; \node[above] at (3.0,4.5){0};


%Second State
\draw[rounded corners=5mm, thick] (8.0,0.0) rectangle (13.0,5.0);%s1
\node[above] at (12.5,5.0){$q_1$};
\draw[fill] (8.0,4.0) circle [radius=0.05];
\node[right] at (8.0,4.0){0};
%transition q1-a-q0
\draw[thick] (10.0,4.0) circle [radius=0.5];
\node at (10.0,4.0){T};
\draw[fill] (9.5,4.0) circle [radius=0.05];  \node[above left] at (9.5,4.0){1};
\draw[fill] (10.0,4.5) circle [radius=0.05];  \node[above right] at (10.0,4.5){0};
%transition q1-b-q0
\draw[thick] (10.0,2.0) circle [radius=0.5];
\node at (10.0,2.0){T};
\draw[fill] (9.5,2.0) circle [radius=0.05];  \node[above left] at (9.5,2.0){1};
\draw[fill] (10.0,2.5) circle [radius=0.05];  \node[above left] at (10.0,2.5){0};

\end{tikzpicture}
\caption{Bigrafo per l'automa A \label{fig:bigAutomata1}}
\end{figure}


Trattiamo ora il problema di come rappresentare gli automi tramite i bigrafi. Per prima cosa, guardiamo la figura \ref{fig:bigAutomata1}. I due stati $q_0$ e $q_1$ sono stati modellati con dei rettangoli aventi una sola porta. Le transizioni sono i nodi di tipo T. Ogni stato $q_i$ contiene al suo interno tutte le transizioni che partono da esso: per esempio $q_0$ contiene solo un nodo T perch� la sola transizione che parte da $q_0$ � $q_0\stackrel{a}{\longrightarrow} q_1$, mentre $q_1$ contiene due nodi T, che modellano le transizioni $q_1\stackrel{a}{\longrightarrow} q_0$ e $q_1\stackrel{b}{\longrightarrow} q_0$. 

Ogni transizione � quindi modellata da un nodo di controllo T con due porte: la prima � collegata al carattere che fa scattare la transizione, mentre la seconda � collegata allo stato destinazione. Nella precedente figura, si prenda in considerazione il nodo T dentro $q_0$: esso simboleggia la transizione $q_0\stackrel{a}{\longrightarrow} q_1$ perch�:
\begin{itemize}
	\item
	T � dentro $q_0$
	\item
	la prima porta di T � collegata all'outername `a'
	\item
	la seconda porta di T � collegata a $q_1$
\end{itemize}

Lo stesso discorso vale per le altre transizioni. Si noti come l'alfabeto $\Sigma$ sia rappresentato tramite \emph{l'insieme degli outernames}.
Il nodo $A$ (Active) indica che lo stato $q_i$ che lo contiene � quello attivo, ovvero: l'automa si trova nello stato $q_i$ se e solo se $q_i$ contiene il nodo $A$.
Infine, lo stato $q_i$ � uno stato finale ($q_i \in F$) se e solo se contiene al suo interno il nodo di tipo $F$ (Final). Queste considerazioni ci portano a definire la segnatura del bigrafo:

\begin{prop}[Segnature per gli NFA]
La segnatura del bigrafo rappresentante un generico NFA � definita come segue, dove la notazione $Node:n:$ significa che il nodo $Node$ ha $n$ porte:
\begin{itemize}
	\item
	State : 1 
	\item
	Transition : 2
	\item
	ActiveState : 0 
	\item
	FinalState : 0 
	\item
	String : 0 
	\item
	Input : 3 
\end{itemize}
\end{prop}


Dalla segnatura di sopra si scopre che ci sono due nuovi controlli: \emph{String} e \emph{Input}. Il nodo di tipo \emph{String} � quello che dovr� contenere la stringa che l'utente immetter� per farla riconoscere dall'automa, mentre ogni nodo di tipo \emph{Input} � un carattere della stringa. 


\begin{figure}[th]
\centering
\begin{tikzpicture}
%\draw[help lines] (0,0) grid (13,7);
% aLinks
\draw[myGreen, thick] (1.5,2.0) to [out=90,in=270] (3.0,4.0);
\draw[myGreen, thick] (5.5,2.0) to [out=140,in=270] (3.0,4.0);
\draw[myGreen, thick] (7.5,2.0) to [out=140,in=270] (3.0,4.0);
% bLinks
\draw[myGreen, thick] (3.5,2.0) to [out=60,in=270] (6.0,4.0);
% Links between nodes
\draw[myGreen, thick] (0.3,1.5) to [out=20,in=200] (1.0,1.5);% null-a
\draw[myGreen, thick] (2.0,1.5) -- (3.0,1.5);% a-b
\draw[myGreen, thick] (4.0,1.5) -- (5.0,1.5);% b-a
\draw[myGreen, thick] (6.0,1.5) -- (7.0,1.5);% a-a
\draw[myGreen, thick] (8.0,1.5) to [out=20,in=200] (8.7,1.5);% a-null
%Root
\draw[rounded corners=5mm, dashed] (-0.5,-0.5) rectangle (9.5,3.5);
%String
\draw[rounded corners=5mm, thick] (0.0,0.0) rectangle (9.0,3.0);
%Inputs
\draw[thick] (1.5,1.5) circle [radius=0.5];%a
\node at (1.5,1.5){I};
\draw[thick] (3.5,1.5) circle [radius=0.5];%b
\node at (3.5,1.5){I};
\draw[thick] (5.5,1.5) circle [radius=0.5];%a
\node at (5.5,1.5){I};
\draw[thick] (7.5,1.5) circle [radius=0.5];%a
\node at (7.5,1.5){I};
% aPorts
\draw[fill] (1.0,1.5) circle [radius=0.05];
\draw[fill] (2.0,1.5) circle [radius=0.05];
\draw[fill] (1.5,2.0) circle [radius=0.05];
% bPorts
\draw[fill] (3.0,1.5) circle [radius=0.05];
\draw[fill] (4.0,1.5) circle [radius=0.05];
\draw[fill] (3.5,2.0) circle [radius=0.05];
% aPorts
\draw[fill] (5.0,1.5) circle [radius=0.05];
\draw[fill] (6.0,1.5) circle [radius=0.05];
\draw[fill] (5.5,2.0) circle [radius=0.05];
% aPorts
\draw[fill] (7.0,1.5) circle [radius=0.05];
\draw[fill] (8.0,1.5) circle [radius=0.05];
\draw[fill] (7.5,2.0) circle [radius=0.05];
%Alphabet
\node[above] at (3.0,4.0){a};
\node[above] at (6.0,4.0){b};
\end{tikzpicture}
\caption{Bigrafo per la stringa ``abaa" \label{fig:bigString1}}
\end{figure}

Si prenda in considerazione la figura \ref{fig:bigString1}: ogni nodo di tipo $Input$ ha tre porte, numerate da sinistra verso destra con 0, 1 e 2. La porta numero 0 � collegata al carattere precedente: per esempio, il primo carattere $a$ non � collegato a nessun altro nodo. La porta numero 2 � collegata al carattere successivo. Infine la porta numero 1 � collegata alla lettera (che � un outername) che il carattere simboleggia. Cos� facendo si crea una \emph{lista} di caratteri che forma la vera e propria stringa. Il nodo di tipo $String$ raggruppa tutti questi caratteri al suo interno. Il bigrafo di figura \ref{fig:bigString1} rappresenta dunque la stringa ``abaa". Nell'implementazione, si � costruito un modulo che accetta dall'utente una stringa e la trasforma nel bigrafo equivalente, secondo le regole appena citate.

Ora che si � definito come modellare un NFA e una sua stringa, rappresentiamo una istanza del problema. In particolare, costruiamo il bigrafo che ha come NFA l'automa di figura \ref{fig:bigAutomata1} e come stringa di input il bigrafo di figura \ref{fig:bigString1}. Il risultato � il bigrafo di figura \ref{fig:bigAutomataProblem}.


\begin{figure}[th]
\centering
\begin{tikzpicture}[scale=0.7]
%\draw[help lines] (0,0) grid (14,14);
%Root
\draw[rounded corners=5mm, dashed] (-0.5,-0.5) rectangle (13.5,9.5);
%Edges
\draw[myGreen,thick] (3.0,4.5) to [out=90,in=270] (1.0,7.0) to [out=90,in=220] (5.0,10.0);%q0-a-q1
\draw[myGreen,thick] (10.0,4.5) to [out=130,in=-20] (2.375,5.5);%q1-a-q0
\draw[myGreen,thick] (10.0,2.5) to [out=30,in=270] (13.0,6.0) to [out=90,in=270] (8.0,10.0);%q1-b-q0

\draw[myGreen,thick] (3.5,4.0) to [out=10,in=190] (8.0,4.0);%t0a - s1
\draw[myGreen,thick] (9.5,4.0) to [out=180,in=0] (5.0,2.5);%t1a-s0
\draw[myGreen,thick] (9.5,2.0) to [out=180,in=0] (5.0,2.5);%t1b-s0
% aLinks
\draw[myGreen, thick] (3.5,8.0) to [out=90,in=270] (5.0,10.0);
\draw[myGreen, thick] (7.5,8.0) to [out=140,in=270] (5.0,10.0);
\draw[myGreen, thick] (9.5,8.0) to [out=140,in=270] (5.0,10.0);
% bLinks
\draw[myGreen, thick] (5.5,8.0) to [out=60,in=270] (8.0,10.0);
% Links between nodes
\draw[myGreen, thick] (2.3,7.5) to [out=20,in=200] (3.0,7.5);% null-a
\draw[myGreen, thick] (4.0,7.5) -- (5.0,7.5);% a-b
\draw[myGreen, thick] (6.0,7.5) -- (7.0,7.5);% b-a
\draw[myGreen, thick] (8.0,7.5) -- (9.0,7.5);% a-a
\draw[myGreen, thick] (10.0,7.5) to [out=20,in=200] (10.7,7.5);% a-null


%First State
\draw[rounded corners=5mm, thick] (0.0,0.0) rectangle (5.0,5.0);%s0
\node[above] at (0.5,5.0){$q_0$};
\draw[fill] (5.0,2.5) circle [radius=0.05];
%\node[left] at (5.0,2.5){0};
%Active
\draw[thick] (1.5,1.5) circle [radius=0.5];
\node at (1.5,1.5){A};
%Final
\draw[thick] (3.5,1.5) circle [radius=0.5];
\node at (3.5,1.5){F};
%transition q0-a-q1
\draw[thick] (3.0,4.0) circle [radius=0.5];
\node at (3.0,4.0){T};
%\draw[fill] (3.5,4.0) circle [radius=0.05]; \node[below right] at (3.5,4.0){1};
%\draw[fill] (3.0,4.5) circle [radius=0.05]; \node[above left] at (3.0,4.5){0};


%Second State
\draw[rounded corners=5mm, thick] (8.0,0.0) rectangle (13.0,5.0);%s1
\node[above] at (12.5,5.0){$q_1$};
\draw[fill] (8.0,4.0) circle [radius=0.05];
%\node[right] at (8.0,4.0){0};
%transition q1-a-q0
\draw[thick] (10.0,4.0) circle [radius=0.5];
\node at (10.0,4.0){T};
%\draw[fill] (9.5,4.0) circle [radius=0.05];  \node[above left] at (9.5,4.0){1};
%\draw[fill] (10.0,4.5) circle [radius=0.05];  \node[above right] at (10.0,4.5){0};
%transition q1-b-q0
\draw[thick] (10.0,2.0) circle [radius=0.5];
\node at (10.0,2.0){T};
%\draw[fill] (9.5,2.0) circle [radius=0.05];  \node[above left] at (9.5,2.0){1};
%\draw[fill] (10.0,2.5) circle [radius=0.05];  \node[above left] at (10.0,2.5){0};


%%%    String   %%%
%String
\draw[rounded corners=5mm, thick] (2.0,6.0) rectangle (11.0,9.0);
%Inputs
\draw[thick] (3.5,7.5) circle [radius=0.5];%a
\node at (3.5,7.5){I};
\draw[thick] (5.5,7.5) circle [radius=0.5];%b
\node at (5.5,7.5){I};
\draw[thick] (7.5,7.5) circle [radius=0.5];%a
\node at (7.5,7.5){I};
\draw[thick] (9.5,7.5) circle [radius=0.5];%a
\node at (9.5,7.5){I};
% aPorts
\draw[fill] (3.0,7.5) circle [radius=0.05];
\draw[fill] (4.0,7.5) circle [radius=0.05];
\draw[fill] (3.5,8.0) circle [radius=0.05];
% bPorts
\draw[fill] (5.0,7.5) circle [radius=0.05];
\draw[fill] (6.0,7.5) circle [radius=0.05];
\draw[fill] (5.5,8.0) circle [radius=0.05];
% aPorts
\draw[fill] (7.0,7.5) circle [radius=0.05];
\draw[fill] (8.0,7.5) circle [radius=0.05];
\draw[fill] (7.5,8.0) circle [radius=0.05];
% aPorts
\draw[fill] (9.0,7.5) circle [radius=0.05];
\draw[fill] (10.0,7.5) circle [radius=0.05];
\draw[fill] (9.5,8.0) circle [radius=0.05];
%Alphabet
\node[above] at (5.0,10.0){a};
\node[above] at (8.0,10.0){b};
\end{tikzpicture}
\caption{Istanza del problema: la stringa ``abaa" viene accettata dall'automa A? \label{fig:bigAutomataProblem}}
\end{figure}




Vediamo ora come il bigrafo si pu� evolvere. L'idea principale � di far ``consumare" al sistema un carattere alla volta. Si veda la regola della figura \ref{fig:reactionRuleAutomata} che rappresenta la regola di reazione $R_0$: se il primo carattere della stringa � la lettera $\sigma$ e se lo stato attivo possiede una transizione tramite la lettera $\sigma$, allora si elimina tale carattere dal bigrafo e si sposta il nodo \emph{Active} nello stato destinazione.



\begin{figure}[th]
\centering
\begin{tikzpicture}
%\draw[help lines] (0,0) grid (14,8);
%%%   Redex   %%%
%Root
\draw[rounded corners=5mm, dashed] (0.0,0.0) rectangle (6.0,6.0);
%State s0
\draw[rounded corners=5mm,thick] (0.2,0.2) rectangle (3.0,4.0);
\draw[fill] (1.0,4.0) circle [radius=0.05]; % 1 port
%Site 0
\draw[rounded corners=1mm, fill=myGrey, dashed] (0.5,0.5) rectangle (1.0,1.0);
\node at (0.75,0.75){0};
%Active
\draw[thick] (2.0,1.0) circle [radius=0.5];
\node at (2.0,1.0){A};
%Transition
\draw[thick] (2.0,3.0) circle [radius=0.5];
\node at (2.0,3.0){T};
\draw[fill] (2.0,3.5) circle [radius=0.05]; % 1 port
\draw[fill] (2.5,3.0) circle [radius=0.05]; % 2 port
%String
\draw[rounded corners=2mm, thick] (2.0,4.3) rectangle (5.0,5.8);
%Site 2
\draw[rounded corners=1mm, fill=myGrey, dashed] (4.4,4.4) rectangle (4.9,4.9);
\node at (4.6,4.6){2};
%Input
\draw[thick] (3.5,5.0) circle [radius=0.5];
\draw[fill] (3.0,5.0) circle [radius=0.05]; % 1 port
\draw[fill] (3.5,5.5) circle [radius=0.05]; % 2 port
\draw[fill] (4.0,5.0) circle [radius=0.05]; % 3 port
\node at (3.5,5.0){I};
%State s1
\draw[rounded corners=5mm,thick] (3.2,0.2) rectangle (5.8,4.0);
\draw[fill] (5.0,4.0) circle [radius=0.05]; % 1 port
%Site 1
\draw[rounded corners=1mm, fill=myGrey, dashed] (3.5,0.5) rectangle (4.0,1.0);
\node at (3.75,0.75){1};
%Outers
\node[above] at (1.0,6.5){x};
\node[above] at (2.0,6.5){$\sigma$};
\node[above] at (4.0,6.5){y};
\node[above] at (5.5,6.5){z};
%Links
\draw[myGreen,thick] (1.0,4.0) to [out=110,in=250] (1.0,6.5);%s0-x
\draw[myGreen,thick] (2.0,3.5) to [out=110,in=250] (2.0,6.5);%t-sigma
\draw[myGreen,thick] (3.5,5.5) to [out=180,in=270] (1.8,5.8);%i-sigma
\draw[myGreen,thick] (2.3,5.0) to [out=20,in=200] (3.0,5.0);%null-i
\draw[myGreen,thick] (4.0,5.0) to [out=20,in=270] (4.0,6.5);%i-y
\draw[myGreen,thick] (2.5,3.0) to [out=0,in=290] (5.6,5.1);%t-z
\draw[myGreen,thick] (5.0,4.0) to [out=50,in=290] (5.5,6.5);%s1-z


\draw[->, very thick, red] (6.2,3.0) -- (7.8,3.0);


%%%   Reactum   %%%
%Root
\draw[rounded corners=5mm, dashed] (8.0,0.0) rectangle (14.0,6.0);
%State s0
\draw[rounded corners=5mm,thick] (8.2,0.2) rectangle (11.0,4.0);
\draw[fill] (9.0,4.0) circle [radius=0.05]; % 1 port
%Site 0
\draw[rounded corners=1mm, fill=myGrey, dashed] (8.5,0.5) rectangle (9.0,1.0);
\node at (8.75,0.75){0};
%State s1
\draw[rounded corners=5mm,thick] (11.2,0.2) rectangle (13.8,4.0);
\draw[fill] (13.0,4.0) circle [radius=0.05]; % 1 port
%Site 1
\draw[rounded corners=1mm, fill=myGrey, dashed] (11.5,0.5) rectangle (12.0,1.0);
\node at (11.75,0.75){1};
%Outers
\node[above] at (9.0,6.5){x};
\node[above] at (10.0,6.5){$\sigma$};
\node[above] at (12.0,6.5){y};
\node[above] at (13.5,6.5){z};
%Transition
\draw[thick] (10.0,3.0) circle [radius=0.5];
\node at (10.0,3.0){T};
\draw[fill] (10.0,3.5) circle [radius=0.05]; % 1 port
\draw[fill] (10.5,3.0) circle [radius=0.05]; % 2 port
%Active
\draw[thick] (13.0,1.0) circle [radius=0.5];
\node at (13.0,1.0){A};
%String
\draw[rounded corners=2mm, thick] (10.0,4.3) rectangle (13.0,5.8);
%Site 2
\draw[rounded corners=1mm, fill=myGrey, dashed] (12.4,4.4) rectangle (12.9,4.9);
\node at (12.6,4.6){2};
%Links
\draw[myGreen,thick] (9.0,4.0) to [out=110,in=250] (9.0,6.5);%s0-x
\draw[myGreen,thick] (10.0,3.5) to [out=110,in=250] (10.0,6.5);%t-sigma
\draw[myGreen,thick] (10.5,3.0) to [out=0,in=290] (13.5,4.835);%t-z
\draw[myGreen,thick] (13.0,4.0) to [out=50,in=290] (13.5,6.5);%s1-z

\end{tikzpicture}
\caption{Regola di reazione $R_0$ \label{fig:reactionRuleAutomata}}
\end{figure}


La regola $R_0$ � molto intuitiva: se per esempio prendiamo in considerazione il bigrafo di figura \ref{fig:bigAutomataProblem}, allora applicando la regola si ha che il primo carattere ($a$) viene eliminato ed il nodo $A$ passa dentro lo stato $q_1$. 
Quindi il BRS per questo problema � definito da una sola regola ($R_0$), che itera finch� trova un match nel bigrafo, cio� si ferma solo quando il nodo di tipo $String$ \textbf{non} contiene pi� nessun nodo $Input$.

Ora che abbiamo creato il BRS, � possibile usare il model checker $MC_{big}$ definendo la formula che esso andr� a verificare. Nella teoria degli automi, vale la seguente preposizione:

\begin{prop}
Una stringa x viene accettata se e solo se alla fine di essa l'automa si trova in uno stato finale.
\end{prop}

Nella nostra segnatura, tutti gli stati finali ($q_F \in F$) vengono distinti tramite un nodo $F$ all'interno di essi: essendo passivi, nessuna regola di reazione pu� modificarli. Per cui la propriet� da verificare sar� la seguente: ``Il nodo di tipo $String$ non deve contenere nessun altro nodo e il nodo $A$ e il nodo $F$ si devono trovare dentro lo stesso stato $S$". Nella logica di $MC_{big}$, questo si traduce nella formula:
\begin{center}
$\varphi = \wario_B(T,T,T)$
\end{center}
dove B � il bigrafo di figura \ref{fig:bigAimNFA}.

\begin{figure}[th]
\centering
\begin{tikzpicture}
%\draw[help lines] (0,0) grid (13,8);
%Root
\draw[rounded corners=5mm, dashed] (0.0,0.0) rectangle (6.0,6.0);
%State s0
\draw[rounded corners=5mm,thick] (0.2,0.2) rectangle (3.0,4.0);
\draw[fill] (1.0,4.0) circle [radius=0.05]; % 1 port
\node at (3.5,0.5){State};
%Site 0
\draw[rounded corners=1mm, fill=myGrey, dashed] (0.5,0.5) rectangle (1.0,1.0);
\node at (0.75,0.75){0};
%Active
\draw[thick] (2.0,1.0) circle [radius=0.5];
\node at (2.0,1.0){A};
%Final
\draw[thick] (2.0,3.0) circle [radius=0.5];
\node at (2.0,3.0){F};
%String
\draw[rounded corners=2mm, thick] (2.0,4.3) rectangle (5.0,5.8);
\node at (5.0,4.0){String};
%Links
\draw[myGreen,thick] (1.0,4.0) to [out=110,in=250] (1.0,6.5);%s0-x
\node[above] at (1.0,6.5){x};
\end{tikzpicture}
\caption{Bigrafo B \label{fig:bigAimNFA}}
\end{figure}


Si noti la semplicit� di quest'ultima formula: il nodo di tipo $String$ deve essere vuoto, il che significa che tutti i caratteri sono stati ``consumati" dall'automa. Inoltre, il nodo attivo e il nodo finale si devono trovare nello stesso stato, assicurando che l'automa dopo aver letto tutta la stringa � finito in uno stato finale. Quindi possiamo affermare che:
\begin{prop}\label{prop:strWario}
Una stringa x viene accettata dall'automa A se e solo se \\$MC,S_i \models \varphi$ per qualche $i$, dove $\varphi = \wario_B(T,T,T)$.
\end{prop}

Infine, si noti come l'automa modellato tramite bigrafi sia \emph{non deterministico}: le transizioni sono modellate tramite regole di reazione che scattano dopo aver trovato un match. Il processo di matching per� � per natura non deterministico, il che vuol dire che se nel bigrafo di partenza la regola $R_0$ trova pi� di un match, allora sceglie uno dei due in maniera non deterministica. Si prenda il bigrafo di figura \ref{fig:bigAutomataProblem}: se si aggiunge nel nodo $q_0$ un'altra transizione collegata all'outername `a' che porta ad un terzo stato $q_2$, allora l'automa diventa non deterministico, perch� la regola $R_0$ trover� sempre due match nel bigrafo e ne sceglier� uno in maniera casuale.
Riassumendo: nella versione bigrafica, gli automi non deterministici si distinguono da quelli deterministici \emph{solamente} perch� nei primi esiste almeno un nodo che ha due o pi� archi uscenti con la stessa etichetta.\\


Prendiamo l'istanza del problema in figura \ref{fig:bigAutomataProblem}. Seguiremo una traccia d'esecuzione, cio� faremo tutti i passi che fa $MC_{big}$ per verificare la propriet� desiderata. Vedremo come il grafo degli stati sar� molto semplice: questo spesso � un indice di buona progettazione. Vuol dire che il sistema � stato modellato correttamente, scegliendo poche regole ed evitando di costruire regole ad-hoc per casi particolari. Il nostro BRS � formato da una sola regola, il che evita per esempio che in uno stato $S_i$ venga applicata la regola sbagliata creando nodi inutili nel grafo degli stati.

In figura \ref{fig:MCbigTrace}, c'� la traccia d'esecuzione di $MC_{big}$, mentre in \ref{fig:BSGAutomata} c'� il corrispondente grafo degli stati. \\


\begin{figure*}[h]
\centering
\scalebox{.7}{
\subfigure[Primo Passo]{
\begin{tikzpicture}[scale=0.5]
%\draw[help lines] (0,0) grid (14,14);
%Root
\draw[rounded corners=5mm, dashed] (-0.5,-0.5) rectangle (13.5,9.5);
%Edges
\draw[myGreen,thick] (3.0,4.5) to [out=90,in=270] (1.0,7.0) to [out=90,in=220] (5.0,10.0);%q0-a-q1
\draw[myGreen,thick] (10.0,4.5) to [out=130,in=-20] (2.375,5.5);%q1-a-q0
\draw[myGreen,thick] (10.0,2.5) to [out=30,in=270] (13.0,6.0) to [out=90,in=270] (8.0,10.0);%q1-b-q0

\draw[myGreen,thick] (3.5,4.0) to [out=10,in=190] (8.0,4.0);%t0a - s1
\draw[myGreen,thick] (9.5,4.0) to [out=180,in=0] (5.0,2.5);%t1a-s0
\draw[myGreen,thick] (9.5,2.0) to [out=180,in=0] (5.0,2.5);%t1b-s0
% aLinks
\draw[myGreen, thick] (3.5,8.0) to [out=90,in=270] (5.0,10.0);
\draw[myGreen, thick] (7.5,8.0) to [out=140,in=270] (5.0,10.0);
\draw[myGreen, thick] (9.5,8.0) to [out=140,in=270] (5.0,10.0);
% bLinks
\draw[myGreen, thick] (5.5,8.0) to [out=60,in=270] (8.0,10.0);
% Links between nodes
\draw[myGreen, thick] (2.3,7.5) to [out=20,in=200] (3.0,7.5);% null-a
\draw[myGreen, thick] (4.0,7.5) -- (5.0,7.5);% a-b
\draw[myGreen, thick] (6.0,7.5) -- (7.0,7.5);% b-a
\draw[myGreen, thick] (8.0,7.5) -- (9.0,7.5);% a-a
\draw[myGreen, thick] (10.0,7.5) to [out=20,in=200] (10.7,7.5);% a-null


%First State
\draw[rounded corners=5mm, thick] (0.0,0.0) rectangle (5.0,5.0);%s0
\node[above] at (0.5,5.0){$q_0$};
\draw[fill] (5.0,2.5) circle [radius=0.05];
%\node[left] at (5.0,2.5){0};
%Active
\draw[thick] (1.5,1.5) circle [radius=0.5];
\node at (1.5,1.5){A};
%Final
\draw[thick] (3.5,1.5) circle [radius=0.5];
\node at (3.5,1.5){F};
%transition q0-a-q1
\draw[thick] (3.0,4.0) circle [radius=0.5];
\node at (3.0,4.0){T};
%\draw[fill] (3.5,4.0) circle [radius=0.05]; \node[below right] at (3.5,4.0){1};
%\draw[fill] (3.0,4.5) circle [radius=0.05]; \node[above left] at (3.0,4.5){0};


%Second State
\draw[rounded corners=5mm, thick] (8.0,0.0) rectangle (13.0,5.0);%s1
\node[above] at (12.5,5.0){$q_1$};
\draw[fill] (8.0,4.0) circle [radius=0.05];
%\node[right] at (8.0,4.0){0};
%transition q1-a-q0
\draw[thick] (10.0,4.0) circle [radius=0.5];
\node at (10.0,4.0){T};
%\draw[fill] (9.5,4.0) circle [radius=0.05];  \node[above left] at (9.5,4.0){1};
%\draw[fill] (10.0,4.5) circle [radius=0.05];  \node[above right] at (10.0,4.5){0};
%transition q1-b-q0
\draw[thick] (10.0,2.0) circle [radius=0.5];
\node at (10.0,2.0){T};
%\draw[fill] (9.5,2.0) circle [radius=0.05];  \node[above left] at (9.5,2.0){1};
%\draw[fill] (10.0,2.5) circle [radius=0.05];  \node[above left] at (10.0,2.5){0};


%%%    String   %%%
%String
\draw[rounded corners=5mm, thick] (2.0,6.0) rectangle (11.0,9.0);
%Inputs
\draw[thick] (3.5,7.5) circle [radius=0.5];%a
\node at (3.5,7.5){I};
\draw[thick] (5.5,7.5) circle [radius=0.5];%b
\node at (5.5,7.5){I};
\draw[thick] (7.5,7.5) circle [radius=0.5];%a
\node at (7.5,7.5){I};
\draw[thick] (9.5,7.5) circle [radius=0.5];%a
\node at (9.5,7.5){I};
% aPorts
\draw[fill] (3.0,7.5) circle [radius=0.05];
\draw[fill] (4.0,7.5) circle [radius=0.05];
\draw[fill] (3.5,8.0) circle [radius=0.05];
% bPorts
\draw[fill] (5.0,7.5) circle [radius=0.05];
\draw[fill] (6.0,7.5) circle [radius=0.05];
\draw[fill] (5.5,8.0) circle [radius=0.05];
% aPorts
\draw[fill] (7.0,7.5) circle [radius=0.05];
\draw[fill] (8.0,7.5) circle [radius=0.05];
\draw[fill] (7.5,8.0) circle [radius=0.05];
% aPorts
\draw[fill] (9.0,7.5) circle [radius=0.05];
\draw[fill] (10.0,7.5) circle [radius=0.05];
\draw[fill] (9.5,8.0) circle [radius=0.05];
%Alphabet
\node[above] at (5.0,10.0){a};
\node[above] at (8.0,10.0){b};
\end{tikzpicture}
}

\hspace{3mm}

\subfigure[Secondo Passo]{
\begin{tikzpicture}[scale=0.5]
%\draw[help lines] (0,0) grid (14,14);
%Root
\draw[rounded corners=5mm, dashed] (-0.5,-0.5) rectangle (13.5,9.5);
%Edges
\draw[myGreen,thick] (3.0,4.5) to [out=90,in=270] (1.0,7.0) to [out=90,in=220] (5.0,10.0);%q0-a-q1
\draw[myGreen,thick] (10.0,4.5) to [out=130,in=-20] (2.375,5.5);%q1-a-q0
\draw[myGreen,thick] (10.0,2.5) to [out=30,in=270] (13.0,6.0) to [out=90,in=270] (8.0,10.0);%q1-b-q0

\draw[myGreen,thick] (3.5,4.0) to [out=10,in=190] (8.0,4.0);%t0a - s1
\draw[myGreen,thick] (9.5,4.0) to [out=180,in=0] (5.0,2.5);%t1a-s0
\draw[myGreen,thick] (9.5,2.0) to [out=180,in=0] (5.0,2.5);%t1b-s0
% aLinks
\draw[myGreen, thick] (7.5,8.0) to [out=140,in=270] (5.0,10.0);
\draw[myGreen, thick] (9.5,8.0) to [out=140,in=270] (5.0,10.0);
% bLinks
\draw[myGreen, thick] (5.5,8.0) to [out=60,in=270] (8.0,10.0);
% Links between nodes
\draw[myGreen, thick] (4.0,7.5) -- (5.0,7.5);% a-b
\draw[myGreen, thick] (6.0,7.5) -- (7.0,7.5);% b-a
\draw[myGreen, thick] (8.0,7.5) -- (9.0,7.5);% a-a
\draw[myGreen, thick] (10.0,7.5) to [out=20,in=200] (10.7,7.5);% a-null


%First State
\draw[rounded corners=5mm, thick] (0.0,0.0) rectangle (5.0,5.0);%s0
\node[above] at (0.5,5.0){$q_0$};
\draw[fill] (5.0,2.5) circle [radius=0.05];
%\node[left] at (5.0,2.5){0};
%Active
\draw[thick] (12.0,1.5) circle [radius=0.5];
\node at (12.0,1.5){A};
%Final
\draw[thick] (3.5,1.5) circle [radius=0.5];
\node at (3.5,1.5){F};
%transition q0-a-q1
\draw[thick] (3.0,4.0) circle [radius=0.5];
\node at (3.0,4.0){T};
%\draw[fill] (3.5,4.0) circle [radius=0.05]; \node[below right] at (3.5,4.0){1};
%\draw[fill] (3.0,4.5) circle [radius=0.05]; \node[above left] at (3.0,4.5){0};


%Second State
\draw[rounded corners=5mm, thick] (8.0,0.0) rectangle (13.0,5.0);%s1
\node[above] at (12.5,5.0){$q_1$};
\draw[fill] (8.0,4.0) circle [radius=0.05];
%\node[right] at (8.0,4.0){0};
%transition q1-a-q0
\draw[thick] (10.0,4.0) circle [radius=0.5];
\node at (10.0,4.0){T};
%\draw[fill] (9.5,4.0) circle [radius=0.05];  \node[above left] at (9.5,4.0){1};
%\draw[fill] (10.0,4.5) circle [radius=0.05];  \node[above right] at (10.0,4.5){0};
%transition q1-b-q0
\draw[thick] (10.0,2.0) circle [radius=0.5];
\node at (10.0,2.0){T};
%\draw[fill] (9.5,2.0) circle [radius=0.05];  \node[above left] at (9.5,2.0){1};
%\draw[fill] (10.0,2.5) circle [radius=0.05];  \node[above left] at (10.0,2.5){0};


%%%    String   %%%
%String
\draw[rounded corners=5mm, thick] (2.0,6.0) rectangle (11.0,9.0);
%Inputs
\draw[thick] (5.5,7.5) circle [radius=0.5];%b
\node at (5.5,7.5){I};
\draw[thick] (7.5,7.5) circle [radius=0.5];%a
\node at (7.5,7.5){I};
\draw[thick] (9.5,7.5) circle [radius=0.5];%a
\node at (9.5,7.5){I};
% bPorts
\draw[fill] (5.0,7.5) circle [radius=0.05];
\draw[fill] (6.0,7.5) circle [radius=0.05];
\draw[fill] (5.5,8.0) circle [radius=0.05];
% aPorts
\draw[fill] (7.0,7.5) circle [radius=0.05];
\draw[fill] (8.0,7.5) circle [radius=0.05];
\draw[fill] (7.5,8.0) circle [radius=0.05];
% aPorts
\draw[fill] (9.0,7.5) circle [radius=0.05];
\draw[fill] (10.0,7.5) circle [radius=0.05];
\draw[fill] (9.5,8.0) circle [radius=0.05];
%Alphabet
\node[above] at (5.0,10.0){a};
\node[above] at (8.0,10.0){b};
\end{tikzpicture}
}
}
\end{figure*}


\begin{figure*}[h]
\centering
\scalebox{.7}{
\subfigure[Terzo Passo]{
\begin{tikzpicture}[scale=0.5]
%\draw[help lines] (0,0) grid (14,14);
%Root
\draw[rounded corners=5mm, dashed] (-0.5,-0.5) rectangle (13.5,9.5);
%Edges
\draw[myGreen,thick] (3.0,4.5) to [out=90,in=270] (1.0,7.0) to [out=90,in=220] (5.0,10.0);%q0-a-q1
\draw[myGreen,thick] (10.0,4.5) to [out=130,in=-20] (2.375,5.5);%q1-a-q0
\draw[myGreen,thick] (10.0,2.5) to [out=30,in=270] (13.0,6.0) to [out=90,in=270] (8.0,10.0);%q1-b-q0

\draw[myGreen,thick] (3.5,4.0) to [out=10,in=190] (8.0,4.0);%t0a - s1
\draw[myGreen,thick] (9.5,4.0) to [out=180,in=0] (5.0,2.5);%t1a-s0
\draw[myGreen,thick] (9.5,2.0) to [out=180,in=0] (5.0,2.5);%t1b-s0
% aLinks
\draw[myGreen, thick] (7.5,8.0) to [out=140,in=270] (5.0,10.0);
\draw[myGreen, thick] (9.5,8.0) to [out=140,in=270] (5.0,10.0);
% Links between nodes
\draw[myGreen, thick] (6.0,7.5) -- (7.0,7.5);% b-a
\draw[myGreen, thick] (8.0,7.5) -- (9.0,7.5);% a-a
\draw[myGreen, thick] (10.0,7.5) to [out=20,in=200] (10.7,7.5);% a-null


%First State
\draw[rounded corners=5mm, thick] (0.0,0.0) rectangle (5.0,5.0);%s0
\node[above] at (0.5,5.0){$q_0$};
\draw[fill] (5.0,2.5) circle [radius=0.05];
%\node[left] at (5.0,2.5){0};
%Active
\draw[thick] (1.5,1.5) circle [radius=0.5];
\node at (1.5,1.5){A};
%Final
\draw[thick] (3.5,1.5) circle [radius=0.5];
\node at (3.5,1.5){F};
%transition q0-a-q1
\draw[thick] (3.0,4.0) circle [radius=0.5];
\node at (3.0,4.0){T};
%\draw[fill] (3.5,4.0) circle [radius=0.05]; \node[below right] at (3.5,4.0){1};
%\draw[fill] (3.0,4.5) circle [radius=0.05]; \node[above left] at (3.0,4.5){0};


%Second State
\draw[rounded corners=5mm, thick] (8.0,0.0) rectangle (13.0,5.0);%s1
\node[above] at (12.5,5.0){$q_1$};
\draw[fill] (8.0,4.0) circle [radius=0.05];
%\node[right] at (8.0,4.0){0};
%transition q1-a-q0
\draw[thick] (10.0,4.0) circle [radius=0.5];
\node at (10.0,4.0){T};
%\draw[fill] (9.5,4.0) circle [radius=0.05];  \node[above left] at (9.5,4.0){1};
%\draw[fill] (10.0,4.5) circle [radius=0.05];  \node[above right] at (10.0,4.5){0};
%transition q1-b-q0
\draw[thick] (10.0,2.0) circle [radius=0.5];
\node at (10.0,2.0){T};
%\draw[fill] (9.5,2.0) circle [radius=0.05];  \node[above left] at (9.5,2.0){1};
%\draw[fill] (10.0,2.5) circle [radius=0.05];  \node[above left] at (10.0,2.5){0};


%%%    String   %%%
%String
\draw[rounded corners=5mm, thick] (2.0,6.0) rectangle (11.0,9.0);
%Inputs
\draw[thick] (7.5,7.5) circle [radius=0.5];%a
\node at (7.5,7.5){I};
\draw[thick] (9.5,7.5) circle [radius=0.5];%a
\node at (9.5,7.5){I};
% aPorts
\draw[fill] (7.0,7.5) circle [radius=0.05];
\draw[fill] (8.0,7.5) circle [radius=0.05];
\draw[fill] (7.5,8.0) circle [radius=0.05];
% aPorts
\draw[fill] (9.0,7.5) circle [radius=0.05];
\draw[fill] (10.0,7.5) circle [radius=0.05];
\draw[fill] (9.5,8.0) circle [radius=0.05];
%Alphabet
\node[above] at (5.0,10.0){a};
\node[above] at (8.0,10.0){b};
\end{tikzpicture}
}

\hspace{5mm}

\subfigure[Quarto Passo]{
\begin{tikzpicture}[scale=0.5]
%\draw[help lines] (0,0) grid (14,14);
%Root
\draw[rounded corners=5mm, dashed] (-0.5,-0.5) rectangle (13.5,9.5);
%Edges
\draw[myGreen,thick] (3.0,4.5) to [out=90,in=270] (1.0,7.0) to [out=90,in=220] (5.0,10.0);%q0-a-q1
\draw[myGreen,thick] (10.0,4.5) to [out=130,in=-20] (2.375,5.5);%q1-a-q0
\draw[myGreen,thick] (10.0,2.5) to [out=30,in=270] (13.0,6.0) to [out=90,in=270] (8.0,10.0);%q1-b-q0

\draw[myGreen,thick] (3.5,4.0) to [out=10,in=190] (8.0,4.0);%t0a - s1
\draw[myGreen,thick] (9.5,4.0) to [out=180,in=0] (5.0,2.5);%t1a-s0
\draw[myGreen,thick] (9.5,2.0) to [out=180,in=0] (5.0,2.5);%t1b-s0
% aLinks
\draw[myGreen, thick] (9.5,8.0) to [out=140,in=270] (5.0,10.0);
% Links between nodes
\draw[myGreen, thick] (8.0,7.5) -- (9.0,7.5);% a-a
\draw[myGreen, thick] (10.0,7.5) to [out=20,in=200] (10.7,7.5);% a-null


%First State
\draw[rounded corners=5mm, thick] (0.0,0.0) rectangle (5.0,5.0);%s0
\node[above] at (0.5,5.0){$q_0$};
\draw[fill] (5.0,2.5) circle [radius=0.05];
%\node[left] at (5.0,2.5){0};
%Active
\draw[thick] (12.0,1.5) circle [radius=0.5];
\node at (12.0,1.5){A};
%Final
\draw[thick] (3.5,1.5) circle [radius=0.5];
\node at (3.5,1.5){F};
%transition q0-a-q1
\draw[thick] (3.0,4.0) circle [radius=0.5];
\node at (3.0,4.0){T};
%\draw[fill] (3.5,4.0) circle [radius=0.05]; \node[below right] at (3.5,4.0){1};
%\draw[fill] (3.0,4.5) circle [radius=0.05]; \node[above left] at (3.0,4.5){0};


%Second State
\draw[rounded corners=5mm, thick] (8.0,0.0) rectangle (13.0,5.0);%s1
\node[above] at (12.5,5.0){$q_1$};
\draw[fill] (8.0,4.0) circle [radius=0.05];
%\node[right] at (8.0,4.0){0};
%transition q1-a-q0
\draw[thick] (10.0,4.0) circle [radius=0.5];
\node at (10.0,4.0){T};
%\draw[fill] (9.5,4.0) circle [radius=0.05];  \node[above left] at (9.5,4.0){1};
%\draw[fill] (10.0,4.5) circle [radius=0.05];  \node[above right] at (10.0,4.5){0};
%transition q1-b-q0
\draw[thick] (10.0,2.0) circle [radius=0.5];
\node at (10.0,2.0){T};
%\draw[fill] (9.5,2.0) circle [radius=0.05];  \node[above left] at (9.5,2.0){1};
%\draw[fill] (10.0,2.5) circle [radius=0.05];  \node[above left] at (10.0,2.5){0};


%%%    String   %%%
%String
\draw[rounded corners=5mm, thick] (2.0,6.0) rectangle (11.0,9.0);
%Inputs
\draw[thick] (9.5,7.5) circle [radius=0.5];%a
\node at (9.5,7.5){I};
% aPorts
\draw[fill] (9.0,7.5) circle [radius=0.05];
\draw[fill] (10.0,7.5) circle [radius=0.05];
\draw[fill] (9.5,8.0) circle [radius=0.05];
%Alphabet
\node[above] at (5.0,10.0){a};
\node[above] at (8.0,10.0){b};
\end{tikzpicture}
}
}
\end{figure*}

\begin{figure*}[h]
\centering
\scalebox{.7}{
\subfigure[Quinto Passo]{
\begin{tikzpicture}[scale=0.5]
%\draw[help lines] (0,0) grid (14,14);
%Root
\draw[rounded corners=5mm, dashed] (-0.5,-0.5) rectangle (13.5,9.5);
%Edges
\draw[myGreen,thick] (3.0,4.5) to [out=90,in=270] (1.0,7.0) to [out=90,in=220] (5.0,10.0);%q0-a-q1
\draw[myGreen,thick] (10.0,4.5) to [out=130,in=-20] (2.375,5.5);%q1-a-q0
\draw[myGreen,thick] (10.0,2.5) to [out=30,in=270] (13.0,6.0) to [out=90,in=270] (8.0,10.0);%q1-b-q0

\draw[myGreen,thick] (3.5,4.0) to [out=10,in=190] (8.0,4.0);%t0a - s1
\draw[myGreen,thick] (9.5,4.0) to [out=180,in=0] (5.0,2.5);%t1a-s0
\draw[myGreen,thick] (9.5,2.0) to [out=180,in=0] (5.0,2.5);%t1b-s0

%First State
\draw[rounded corners=5mm, thick] (0.0,0.0) rectangle (5.0,5.0);%s0
\node[above] at (0.5,5.0){$q_0$};
\draw[fill] (5.0,2.5) circle [radius=0.05];
%\node[left] at (5.0,2.5){0};
%Active
\draw[thick] (1.5,1.5) circle [radius=0.5];
\node at (1.5,1.5){A};
%Final
\draw[thick] (3.5,1.5) circle [radius=0.5];
\node at (3.5,1.5){F};
%transition q0-a-q1
\draw[thick] (3.0,4.0) circle [radius=0.5];
\node at (3.0,4.0){T};
%\draw[fill] (3.5,4.0) circle [radius=0.05]; \node[below right] at (3.5,4.0){1};
%\draw[fill] (3.0,4.5) circle [radius=0.05]; \node[above left] at (3.0,4.5){0};


%Second State
\draw[rounded corners=5mm, thick] (8.0,0.0) rectangle (13.0,5.0);%s1
\node[above] at (12.5,5.0){$q_1$};
\draw[fill] (8.0,4.0) circle [radius=0.05];
%\node[right] at (8.0,4.0){0};
%transition q1-a-q0
\draw[thick] (10.0,4.0) circle [radius=0.5];
\node at (10.0,4.0){T};
%\draw[fill] (9.5,4.0) circle [radius=0.05];  \node[above left] at (9.5,4.0){1};
%\draw[fill] (10.0,4.5) circle [radius=0.05];  \node[above right] at (10.0,4.5){0};
%transition q1-b-q0
\draw[thick] (10.0,2.0) circle [radius=0.5];
\node at (10.0,2.0){T};
%\draw[fill] (9.5,2.0) circle [radius=0.05];  \node[above left] at (9.5,2.0){1};
%\draw[fill] (10.0,2.5) circle [radius=0.05];  \node[above left] at (10.0,2.5){0};


%%%    String   %%%
%String
\draw[rounded corners=5mm, thick] (2.0,6.0) rectangle (11.0,9.0);
%Alphabet
\node[above] at (5.0,10.0){a};
\node[above] at (8.0,10.0){b};
\end{tikzpicture}
}
}
\caption{Traccia d'esecuzione di $MC_{big}$ \label{fig:MCbigTrace}}
\end{figure*}




Si noti come l'ultimo stato $S_4$ soddisfi la formula $\varphi$:  il Wario Predicate \\$\wario_B(T,T,T)$ � soddisfatto dallo stato $S_4$, in formule $MC,S_4 \models \wario_B(T,T,T)$, perch� il model checker trova in $S_4$ un match del bigrafo B. Poich� tutti gli argomenti del Wario Predicate sono True, basta che questo match esista perch� il predicato sia soddisfatto. Quindi, la proposizione \ref{prop:strWario} � rispettata.




\begin{figure}[th]
\centering
\begin{tikzpicture}
%\draw[help lines] (0,0) grid (13,5);
%Nodes
\draw[thick] (1.5,1.5) circle [radius=0.5];%s0
\node at (1.5,1.5){$S_0$};
\draw[thick] (3.5,1.5) circle [radius=0.5];%s1
\node at (3.5,1.5){$S_1$};
\draw[thick] (5.5,1.5) circle [radius=0.5];%s2
\node at (5.5,1.5){$S_2$};
\draw[thick] (7.5,1.5) circle [radius=0.5];%s3
\node at (7.5,1.5){$S_3$};
\draw[thick, red] (9.5,1.5) circle [radius=0.5];%s4
\node[red] at (9.5,1.5){$S_4$};
%Edges
\draw[->, thick] (2.0,1.5) -- (3.0,1.5);%s0-s1
\draw[->, thick] (4.0,1.5) -- (5.0,1.5);%s1-s2
\draw[->, thick] (6.0,1.5) -- (7.0,1.5);%s2-s3
\draw[->, thick] (8.0,1.5) -- (9.0,1.5);%s3-s4
\end{tikzpicture}

\caption{Grafo degli stati \label{fig:BSGAutomata}}
\end{figure}


\subsection{Implementazione}
Si � costruito un modulo che accetta dall'utente una stringa e costruisce il rispettivo bigrafo. Se vogliamo sapere se una data stringa � riconosciuta dal \emph{NFA}, basta innanzitutto estendere la classe \emph{NFA} costruendo il proprio automa, e in seguito inserire una stringa appena il sistema lo chiede. Ora la stringa verr� processata e trasformata in bigrafo. Per esempio, se vogliamo risolvere il problema di figura \ref{fig:bigAutomataProblem}, allora dobbiamo estendere la classe $NFA$ costruendo il bigrafo di figura \ref{fig:bigAutomata1} ed infine inserire la stringa ``abaa":
\begin{lstlisting}
Insert the string: 
abaa
Does this NFA recognize the string "abaa"?	YES
\end{lstlisting}
Appena inseriamo la stringa, il software crea il bigrafo rappresentante l'$NFA$ che abbiamo precedentemente costruito, il quale � affiancato ad un altro bigrafo che contiene la stringa, proprio come in figura \ref{fig:bigAutomataProblem}. In questo caso, l'automa riconosce correttamente la parola. Se invece scegliamo una stringa che non appartiene al suo linguaggio, allora il sistema non la riconosce:
\begin{lstlisting}
Insert the string: 
aaba
Does this NFA recognize the string "aaba"?	NO
\end{lstlisting}




Questo esempio � molto importante perch� mostra due aspetti fondamentali:
\begin{itemize}
	\item
	il primo � la flessibilit� dei bigrafi: tramite una sola regola di reazione si � riusciti a modellare un NFA. Inoltre, si presti attenzione al grafo degli stati: � molto semplice e lineare. Come gi� scritto, questo � un indizio di buona progettazione, perch� significa che abbiamo creato una sola regola per tutti i casi possibili. Se avessimo creato una regola per casi particolari, allora ci sarebbero state diramazioni del grafo che avrebbero portato a vicoli ciechi, cio� rami in cui la foglia non rispetta la propriet� $\varphi$.
	\item
	il secondo � la generalit� della logica per $MC_{big}$: si � espressa una propriet� di uno specifico dominio usando la logica generale creata per il model checker. In questo esempio si � usato il Wario Predicate, che permette di spostarsi all'interno del bigrafo e di verificare se una sua parte esiste e rispetta determinate propriet�. Nel NFA ci � bastato verificare che esistesse la parte denotata dal bigrafo B.
\end{itemize}

In questo esempio si � potuto apprezzare la comodit� del model checker e della sua logica per verificare una generica propriet�, che rappresentano quindi uno strumento generale adatto per ogni BRS.





%%%%%%%%%%%%%%%%%%%%%%%%%%%%%%%%%%
\section{Problema dei filosofi a cena}
In questa sezione si fornisce una codifica in bigrafi del problema dei filosofi a cena, introdotto da Dijkstra nel 1965 per esporre un problema di concorrenza tra processi paralleli.

\begin{define}[Filosofi a cena]
Cinque filosofi sono seduti a cena ad una tavola rotonda. Ogni filosofo ha davanti il piatto in cui mangiare e due forchette, una a destra e l'altra a sinistra: per cui nel tavolo sono presenti cinque filosofi, cinque piatti e cinque forchette. Ogni filosofo alterna periodi in cui mangia ad altri in cui pensa. Per mangiare, ha bisogno di entrambe le forchette, ma deve prenderle una per volta. Quando ha finito di mangiare, lascia le forchette e continua a pensare.

Si progetti un algoritmo che eviti \emph{deadlock} o \emph{starvation}.
\end{define}

Il problema chiede di progettare un algoritmo che eviti queste due situazioni:
\begin{itemize}
	\item
	ogni filosofo ha una forchetta e aspetta l'altra dal suo vicino: la situazione si trova in uno stato di stallo (\emph{deadlock})
	\item
	una parte di filosofi riesce a mangiare e pensare ripetute volte, a discapito di un'altra parte che non riesce mai a mangiare perch� non ha mai due forchette, morendo di inedia (\emph{starvation}).
\end{itemize}

Il problema � una metafora, dove i filosofi sono processi paralleli in un calcolatore: il momento in cui devono prendere le forchette corrisponde alla lettura dei dati, che possono per l'appunto essere condivisi tra pi� processi come le forchette; il momento in cui un filosofo mangia corrisponde al momento in cui un processo consuma i dati che ha appena recuperato. Si vedranno due strategie: la prima causer� una situazione di deadlock, mentre la seconda lo eviter� e sar� una soluzione al problema.

\begin{figure}[th]
\centering
\begin{tikzpicture}
%\draw[help lines] (0,0) grid (13,8);
%Root
\draw[rounded corners=5mm,dashed] (0.0,0.0) rectangle (12.0,8.0);
%Philosophers
\draw[thick] (6.0,6.0) circle [radius=0.5];%P1
\node at (6.0,6.0) {$P_1$};
\draw[thick] (3.0,4.0) circle [radius=0.5];%P2
\node at (3.0,4.0) {$P_2$};
\draw[thick] (4.0,1.0) circle [radius=0.5];%P3
\node at (4.0,1.0) {$P_3$};
\draw[thick] (8.0,1.0) circle [radius=0.5];%P4
\node at (8.0,1.0) {$P_4$};
\draw[thick] (9.0,4.0) circle [radius=0.5];%P5
\node at (9.0,4.0) {$P_5$};
%Forks
\draw[rounded corners=1mm, thick] (4.0,5.0) -- (4.5,5.0) -- (4.25,5.8) -- (4.0,5.0);%F2
\draw[rounded corners=1mm, thick] (2.0,2.0) -- (2.5,2.0) -- (2.25,2.8) -- (2.0,2.0);%F3
\draw[rounded corners=1mm, thick] (5.75,1.0) -- (6.25,1.0) -- (6.0,1.8) -- (5.75,1.0);%F4
\draw[rounded corners=1mm, thick] (10.0,2.0) -- (10.5,2.0) -- (10.25,2.8) -- (10.0,2.0);%F5
\draw[rounded corners=1mm, thick] (8.0,5.0) -- (8.5,5.0) -- (8.25,5.8) -- (8.0,5.0);%F1
%%%   Links   %%%
%F1
\draw[myGreen, thick] (6.5,6.0) to [out=90,in=270] (8.0,7.0);
\draw[myGreen, thick] (8.25,5.7) to [out=90,in=270] (8.0,7.0);
\draw[myGreen, thick] (9.0,4.5) to [out=90,in=270] (8.0,7.0);
\draw[fill] (6.5,6.0) circle [radius=0.05]; \node[right] at (6.5,6.0){0};
\draw[fill] (8.25,5.7) circle [radius=0.05];
\draw[fill] (9.0,4.5) circle [radius=0.05];
\node[above] at (8.0,7.0) {$F_1$};
%F2
\draw[myGreen, thick] (5.5,6.0) to [out=90,in=270] (3.0,7.0);
\draw[myGreen, thick] (4.25,5.7) to [out=90,in=270] (3.0,7.0);
\draw[myGreen, thick] (3.0,4.5) to [out=110,in=290] (3.0,7.0);
\draw[fill] (5.5,6.0) circle [radius=0.05]; \node[left] at (5.5,6.0){1};
\draw[fill] (4.25,5.7) circle [radius=0.05];
\draw[fill] (3.0,4.5) circle [radius=0.05];
\node[above] at (3.0,7.0) {$F_2$};
%F3
\draw[myGreen, thick] (3.0,3.5) to [out=270,in=270] (1.0,4.0);
\draw[myGreen, thick] (2.25,2.7) to [out=130,in=270] (1.0,4.0);
\draw[myGreen, thick] (3.5,1.0) to [out=180,in=270] (1.0,4.0);
\draw[fill] (3.0,3.5) circle [radius=0.05];
\draw[fill] (2.25,2.7) circle [radius=0.05];
\draw[fill] (3.5,1.0) circle [radius=0.05];
\node[above] at (1.0,4.0) {$F_3$};
%F4
\draw[myGreen, thick] (4.5,1.0) to [out=0,in=270] (6.0,3.0);
\draw[myGreen, thick] (6.0,1.7) to [out=90,in=270] (6.0,3.0);
\draw[myGreen, thick] (7.5,1.0) to [out=180,in=270] (6.0,3.0);
\draw[fill] (4.5,1.0) circle [radius=0.05];
\draw[fill] (6.0,1.7) circle [radius=0.05];
\draw[fill] (7.5,1.0) circle [radius=0.05];
\node[above] at (6.0,3.0) {$F_4$};
%F5
\draw[myGreen, thick] (8.5,1.0) to [out=0,in=290] (11.0,4.0);
\draw[myGreen, thick] (10.25,2.8) to [out=90,in=270] (11.0,4.0);
\draw[myGreen, thick] (9.0,3.5) to [out=270,in=270] (11.0,4.0);
\draw[fill] (8.5,1.0) circle [radius=0.05];
\draw[fill] (10.25,2.8) circle [radius=0.05];
\draw[fill] (9.0,3.5) circle [radius=0.05];
\node[above] at (11.0,4.0) {$F_5$};

\end{tikzpicture}
\caption{Codifica in bigrafi di un'istanza del problema \label{fig:diningPhil}}
\end{figure}



Per prima cosa, si definir� come il problema pu� essere tradotto in bigrafi. Si consideri la figura \ref{fig:diningPhil}: ogni filosofo � rappresentato da un nodo circolare con due porte; si immagini che ogni filosofo sia rivolto verso il centro del tavolo: si ha che la porta a sinistra \emph{dal punto di vista del filosofo} rappresenta la mano sinistra e la chiameremo \emph{porta 0}, mentre la porta a destra rappresenta la mano destra (\emph{porta 1}). Per esempio: la mano sinistra le filosofo $P_1$ � quella collegata a $F_1$, mentre la mano destra � quella collegata a $F_2$.

Le forchette sono rappresentate come nodi triangolari e sono identificate tramite un outername, per esempio $F_4$. Modelliamo il fatto che la forchetta sinistra di un filosofo sia $F_i$ collegando la sua porta 0 all'outername $F_i$. Per esempio: il filosofo $P_4$ ha come forchetta sinistra $F_4$ perch� la sua porta 0 � collegata a questo outername, mentre come forchetta destra $F_5$ perch� la sua porta 1 punta a $F_5$.

Inizialmente tutte le forchette sono posizionate sul tavolo. Modelliamo il fatto che un filosofo $P_i$ abbiamo preso la forchetta $F_k$ spostando quest'ultima all'interno di $P_i$.

\subsection{Prima strategia}
Tramite la prima strategia, che causer� situazioni di stallo, vogliamo fare vedere come il model checker $MC_{big}$ riesca ad individuare un deadlock. Essa prevede che ogni filosofo, per riuscire a mangiare, debba prendere prima la forchetta sinistra e poi quella destra, e le rimetta in ordine sul tavolo (prima la sinistra e poi la destra). Questa strategia causa un deadlock perch� $P_i$ prende prima la forchetta $F_i$ per $i \in \{1,2,3,4,5\}$, e quindi ogni filosofo $P_i$ aspetta che $P_{i+1}$ liberi la forchetta causando uno stallo.


\begin{figure}[th]
\centering
\subfigure[Prima Regola]{
\begin{tikzpicture}
%\draw[help lines] (0,0) grid (13,6);
%%%   Redex   %%%
\draw[rounded corners=3mm, dashed] (0.0,0.0) rectangle (4.0,3.0);
%Nodes
\draw[thick] (2.7,1.5) circle [radius=0.8];%Phil
\draw[rounded corners=1mm, thick] (0.8,0.5) -- (1.2,0.5) -- (1.0,1.3) -- (0.8,0.5);%Fork
%Outers
\node[above] at (1.0,3.2){LF};
\node[above] at (3.5,3.2){RF};
%Link
\draw[myGreen,thick] (1.9,1.5) to [out=180,in=270] (1.0,3.2);
\draw[myGreen,thick] (3.5,1.5) to [out=0,in=270] (3.5,3.2);
\draw[myGreen,thick] (1.0,1.2) to [out=90,in=270] (1.0,3.2);
\draw[fill] (1.9,1.5) circle [radius=0.05];
\draw[fill] (3.5,1.5) circle [radius=0.05];
\draw[fill] (1.0,1.2) circle [radius=0.05];

\draw[->, red, thick] (4.1,1.5) -- (4.9,1.5);

%%%   Reactum   %%%
\draw[rounded corners=3mm, dashed] (5.0,0.0) rectangle (9.0,3.0);
%Nodes
\draw[thick] (7.7,1.5) circle [radius=0.8];%Phil
\draw[rounded corners=1mm, thick] (7.6,1.0) -- (8.0,1.0) -- (7.8,1.8) -- (7.6,1.0);%Fork
%Outers
\node[above] at (6.0,3.2){LF};
\node[above] at (8.5,3.2){RF};
%Link
\draw[myGreen,thick] (6.9,1.5) to [out=180,in=270] (6.0,3.2);
\draw[myGreen,thick] (8.5,1.5) to [out=0,in=270] (8.5,3.2);
\draw[myGreen,thick] (7.8,1.7) to [out=90,in=270] (6.0,3.2);
\draw[fill] (6.9,1.5) circle [radius=0.05];
\draw[fill] (8.5,1.5) circle [radius=0.05];
\draw[fill] (7.8,1.7) circle [radius=0.05];

\end{tikzpicture}
}

\hspace{3mm}

\subfigure[Seconda Regola]{
\begin{tikzpicture}
%\draw[help lines] (0,0) grid (13,6);
%%%   Redex   %%%
\draw[rounded corners=3mm, dashed] (0.0,0.0) rectangle (4.0,3.0);
%Nodes
\draw[thick] (1.5,1.5) circle [radius=1.0];%Phil
\draw[rounded corners=1mm, thick] (0.8,1.0) -- (1.2,1.0) -- (1.0,1.8) -- (0.8,1.0);%Right Fork
\draw[rounded corners=1mm, thick] (3.0,0.2) -- (3.4,0.2) -- (3.2,1.0) -- (3.0,0.2);%Left Fork
%Outers
\node[above] at (1.0,3.2){LF};
\node[above] at (3.5,3.2){RF};
%Link
\draw[myGreen,thick] (0.5,1.5) to [out=180,in=270] (1.0,3.2);
\draw[myGreen,thick] (2.5,1.5) to [out=0,in=270] (3.5,3.2);
\draw[myGreen,thick] (1.0,1.7) to [out=90,in=270] (1.0,3.2);
\draw[myGreen,thick] (3.2,0.9) to [out=90,in=270] (3.5,3.2);
\draw[fill] (0.5,1.5) circle [radius=0.05];%left port
\draw[fill] (2.5,1.5) circle [radius=0.05];%right port
\draw[fill] (1.0,1.7) circle [radius=0.05];%inner fork
\draw[fill] (3.2,0.9) circle [radius=0.05];%outer fork

\draw[->, red, thick] (4.1,1.5) -- (4.9,1.5);

%%%   Reactum   %%%
\draw[rounded corners=3mm, dashed] (5.0,0.0) rectangle (9.0,3.0);
%Nodes
\draw[thick] (6.5,1.5) circle [radius=1.0];%Phil
\draw[rounded corners=1mm, thick] (5.8,1.0) -- (6.2,1.0) -- (6.0,1.8) -- (5.8,1.0);%Right Fork
\draw[rounded corners=1mm, thick] (6.8,1.0) -- (7.2,1.0) -- (7.0,1.8) -- (6.8,1.0);%Left Fork
%Outers
\node[above] at (6.0,3.2){LF};
\node[above] at (8.5,3.2){RF};
%Link
\draw[myGreen,thick] (5.5,1.5) to [out=180,in=270] (6.0,3.2);
\draw[myGreen,thick] (7.5,1.5) to [out=0,in=270] (8.5,3.2);
\draw[myGreen,thick] (6.0,1.7) to [out=90,in=270] (6.0,3.2);
\draw[myGreen,thick] (7.0,1.7) to [out=90,in=270] (8.5,3.2);
\draw[fill] (5.5,1.5) circle [radius=0.05];%left port
\draw[fill] (7.5,1.5) circle [radius=0.05];%right port
\draw[fill] (6.0,1.7) circle [radius=0.05];%inner fork
\draw[fill] (7.0,1.7) circle [radius=0.05];%inner right fork

\end{tikzpicture}
}
\caption{Regole per prendere le forchette \label{fig:takeFork}}
\end{figure}



In figura \ref{fig:takeFork}, ci sono le regole che consentono ad un filosofo di prendere le forchette. Si noti come si debba prendere prima la forchetta sinistra e poi la destra: la seconda regola infatti scatta se e solo se il filosofo possiede gi� la forchetta sinistra, che a sua volta ha potuto prendere se e solo se non possedeva ancora nessuna forchetta (nella regola \ref{fig:takeFork}.a non c'� nessun sito dentro il nodo del filosofo).





\begin{figure}[th]
\centering
\subfigure[Prima Regola]{
\begin{tikzpicture}
%\draw[help lines] (0,0) grid (13,6);
%%%   Redex   %%%
\draw[rounded corners=3mm, dashed] (0.0,0.0) rectangle (4.0,3.0);
%Nodes
\draw[thick] (1.5,1.5) circle [radius=1.0];%Phil
\draw[rounded corners=1mm, thick] (0.8,1.0) -- (1.2,1.0) -- (1.0,1.8) -- (0.8,1.0);%Right Fork
\draw[rounded corners=1mm, thick] (1.8,1.0) -- (2.2,1.0) -- (2.0,1.8) -- (1.8,1.0);%Left Fork
%Outers
\node[above] at (1.0,3.2){LF};
\node[above] at (3.5,3.2){RF};
%Link
\draw[myGreen,thick] (0.5,1.5) to [out=180,in=270] (1.0,3.2);
\draw[myGreen,thick] (2.5,1.5) to [out=0,in=270] (3.5,3.2);
\draw[myGreen,thick] (1.0,1.7) to [out=90,in=270] (1.0,3.2);
\draw[myGreen,thick] (2.0,1.7) to [out=90,in=270] (3.5,3.2);
\draw[fill] (0.5,1.5) circle [radius=0.05];%left port
\draw[fill] (2.5,1.5) circle [radius=0.05];%right port
\draw[fill] (1.0,1.7) circle [radius=0.05];%inner fork
\draw[fill] (2.0,1.7) circle [radius=0.05];%inner right fork

\draw[->, red, thick] (4.1,1.5) -- (4.9,1.5);

%%%   Reactum   %%%
\draw[rounded corners=3mm, dashed] (5.0,0.0) rectangle (9.0,3.0);
%Nodes
\draw[thick] (7.5,1.5) circle [radius=1.0];%Phil
\draw[rounded corners=1mm, thick] (5.8,0.2) -- (6.2,0.2) -- (6.0,1.0) -- (5.8,0.2);%Right Fork
\draw[rounded corners=1mm, thick] (7.8,1.0) -- (8.2,1.0) -- (8.0,1.8) -- (7.8,1.0);%Left Fork
%Outers
\node[above] at (6.0,3.2){LF};
\node[above] at (8.5,3.2){RF};
%Link
\draw[myGreen,thick] (6.5,1.5) to [out=180,in=270] (6.0,3.2);
\draw[myGreen,thick] (8.5,1.5) to [out=0,in=270] (8.5,3.2);
\draw[myGreen,thick] (6.0,0.9) to [out=110,in=270] (6.0,3.2);
\draw[myGreen,thick] (8.0,1.7) to [out=90,in=270] (8.5,3.2);
\draw[fill] (6.5,1.5) circle [radius=0.05];%left port
\draw[fill] (8.5,1.5) circle [radius=0.05];%right port
\draw[fill] (6.0,0.9) circle [radius=0.05];%outer fork
\draw[fill] (8.0,1.7) circle [radius=0.05];%inner fork
\end{tikzpicture}
}

\hspace{3mm}

\subfigure[Seconda Regola]{
\begin{tikzpicture}
%\draw[help lines] (0,0) grid (13,6);
%%%   Redex   %%%
\draw[rounded corners=3mm, dashed] (0.0,0.0) rectangle (4.0,3.0);
%Nodes
\draw[thick] (1.5,1.5) circle [radius=1.0];%Phil
\draw[rounded corners=1mm, thick] (1.8,1.0) -- (2.2,1.0) -- (2.0,1.8) -- (1.8,1.0);%Left Fork
%Outers
\node[above] at (1.0,3.2){LF};
\node[above] at (3.5,3.2){RF};
%Link
\draw[myGreen,thick] (0.5,1.5) to [out=180,in=270] (1.0,3.2);
\draw[myGreen,thick] (2.5,1.5) to [out=0,in=270] (3.5,3.2);
\draw[myGreen,thick] (2.0,1.7) to [out=90,in=270] (3.5,3.2);
\draw[fill] (0.5,1.5) circle [radius=0.05];%left port
\draw[fill] (2.5,1.5) circle [radius=0.05];%right port
\draw[fill] (2.0,1.7) circle [radius=0.05];%inner right fork

\draw[->, red, thick] (4.1,1.5) -- (4.9,1.5);

%%%   Reactum   %%%
\draw[rounded corners=3mm, dashed] (5.0,0.0) rectangle (9.0,3.0);
%Nodes
\draw[thick] (6.5,1.5) circle [radius=1.0];%Phil
\draw[rounded corners=1mm, thick] (7.8,0.2) -- (8.2,0.2) -- (8.0,1.0) -- (7.8,0.2);%Left Fork
%Outers
\node[above] at (6.0,3.2){LF};
\node[above] at (8.5,3.2){RF};
%Link
\draw[myGreen,thick] (5.5,1.5) to [out=180,in=270] (6.0,3.2);
\draw[myGreen,thick] (7.5,1.5) to [out=0,in=270] (8.5,3.2);
\draw[myGreen,thick] (8.0,0.9) to [out=60,in=270] (8.5,3.2);
\draw[fill] (5.5,1.5) circle [radius=0.05];%left port
\draw[fill] (7.5,1.5) circle [radius=0.05];%right port
\draw[fill] (8.0,0.9) circle [radius=0.05];%inner right fork

\end{tikzpicture}
}
\caption{Regole per lasciare le forchette \label{fig:dropFork}}
\end{figure}



Le regole per lasciare le forchette sono similari: quando un filosofo ha entrambe le forchette significa che ha mangiato e, grazie alla regola \ref{fig:dropFork}.a, lascia prima la forchetta a sinistra, come vuole la nostra strategia. Infine, tramite \ref{fig:dropFork}.b, lascia la forchetta destra.

Queste quattro regole andranno a formare il BRS per il nostro problema. Si noti come un BRS di questo tipo sia particolarmente adatto per il problema della cena tra filosofi: quest'ultimo � infatti un problema di sincronizzazione tra processi paralleli che viene modellato perfettamente dalle regole non deterministiche del BRS. Ogni filosofo decide autonomamente quando incominciare a mangiare: per cui pu� essere che incominci il terzo filosofo cos� come il primo. Questa � la stessa situazione in cui operano le regole: per esempio, in figura \ref{fig:diningPhil} la regola \ref{fig:takeFork}.a trova cinque match e ne sceglie in modo non deterministico uno solo. In altre parole, il BRS modella bene il caso reale in cui i processi decidono di leggere dati autonomamente, senza che ci sia alcun ordine tra di loro.

Ora si user� $MC_{big}$ per capire se con questa strategia c'� pericolo di deadlock. In questo caso, la propriet� da fare verificare al model checker � una disgiunzione tra due  \emph{IsoProperty}:
\begin{center}
$\varphi =  \pi_G \lor \pi_{G'}$
\end{center}
dove $G$ � il bigrafo di figura \ref{fig:diningAim1} e $G'$ quello di figura \ref{fig:diningAim2}. Nel primo, ogni filosofo $P_i$ contiene solamente la forchetta $F_i$, mentre nel secondo ogni filosofo $P_i$ contiene solamente $F_{i+1}$.

Dato che la grammatica di $MC_{big}$ non contiene il simbolo $\lor$ per la disgiunzione, si � usata l'equivalenza di De Morgan, avendo quindi che:
\begin{center}
$\varphi =  \pi_G \lor \pi_{G'} = \lnot(\lnot \pi_G \land \lnot \pi_{G'})$
\end{center}
Infatti, nell'implementazione le righe corrispondenti alla creazione di questa propriet� sono queste riportate di seguito:
\begin{lstlisting}
Predicate p1 = new IsoPredicate(getAim1(n));
Predicate p2 = new IsoPredicate(getAim2(n));
Predicate notP1 = new NotPredicate(p1);
Predicate notP2 = new NotPredicate(p2);
Predicate andNP1NP2 = new AndPredicate(notP1,notP2);
Predicate prop = new NotPredicate(andNP1NP2);
\end{lstlisting}
da cui si vede bene come si sia usata la legge di De Morgan.

\begin{figure}[!htbp]
\centering
\begin{tikzpicture}[scale=0.88]
%\draw[help lines] (0,0) grid (13,8);
%Root
\draw[rounded corners=5mm,dashed] (0.0,0.0) rectangle (12.0,8.0);
%Philosophers
\draw[thick] (6.0,6.0) circle [radius=0.5];%P1
\node at (6.0,5.25) {$P_1$};
\draw[thick] (3.0,4.0) circle [radius=0.5];%P2
\node at (3.0,3.0) {$P_2$};
\draw[thick] (4.0,1.0) circle [radius=0.5];%P3
\node at (4.0,0.25) {$P_3$};
\draw[thick] (8.0,1.0) circle [radius=0.5];%P4
\node at (8.0,0.25) {$P_4$};
\draw[thick] (9.0,4.0) circle [radius=0.5];%P5
\node at (9.0,3.0) {$P_5$};
%Forks
\draw[rounded corners=1mm, thick] (2.8,3.8) -- (3.2,3.8) -- (3.0,4.4) -- (2.8,3.8);%F2
\draw[rounded corners=1mm, thick] (3.8,0.8) -- (4.2,0.8) -- (4.0,1.4) -- (3.8,0.8);%F3
\draw[rounded corners=1mm, thick] (7.8,0.8) -- (8.2,0.8) -- (8.0,1.4) -- (7.8,0.8);%F4
\draw[rounded corners=1mm, thick] (8.8,3.8) -- (9.2,3.8) -- (9.0,4.4) -- (8.8,3.8);%F5
\draw[rounded corners=1mm, thick] (5.8,5.8) -- (6.2,5.8) -- (6.0,6.4) -- (5.8,5.8);%F1
%%%   Links   %%%
%F1
\draw[myGreen, thick] (6.5,6.0) to [out=90,in=270] (8.0,7.0);
\draw[myGreen, thick] (6.0,6.3) to [out=90,in=270] (8.0,7.0);
\draw[myGreen, thick] (9.0,4.5) to [out=90,in=270] (8.0,7.0);
\draw[fill] (6.5,6.0) circle [radius=0.05]; \node[right] at (6.5,6.0){0};
\draw[fill] (6.0,6.3) circle [radius=0.05];
\draw[fill] (9.0,4.5) circle [radius=0.05];
\node[above] at (8.0,7.0) {$F_1$};
%F2
\draw[myGreen, thick] (5.5,6.0) to [out=90,in=270] (3.0,7.0);
\draw[myGreen, thick] (3.0,4.3) to [out=150,in=270] (3.0,7.0);
\draw[myGreen, thick] (3.0,4.5) to [out=110,in=290] (3.0,7.0);
\draw[fill] (5.5,6.0) circle [radius=0.05]; \node[left] at (5.5,6.0){1};
\draw[fill] (3.0,4.3) circle [radius=0.05];
\draw[fill] (3.0,4.5) circle [radius=0.05];
\node[above] at (3.0,7.0) {$F_2$};
%F3
\draw[myGreen, thick] (3.0,3.5) to [out=270,in=270] (1.0,4.0);
\draw[myGreen, thick] (4.0,1.3) to [out=130,in=270] (1.0,4.0);
\draw[myGreen, thick] (3.5,1.0) to [out=180,in=270] (1.0,4.0);
\draw[fill] (3.0,3.5) circle [radius=0.05];
\draw[fill] (4.0,1.3) circle [radius=0.05];
\draw[fill] (3.5,1.0) circle [radius=0.05];
\node[above] at (1.0,4.0) {$F_3$};
%F4
\draw[myGreen, thick] (4.5,1.0) to [out=0,in=270] (6.0,3.0);
\draw[myGreen, thick] (8.0,1.3) to [out=90,in=270] (6.0,3.0);
\draw[myGreen, thick] (7.5,1.0) to [out=180,in=270] (6.0,3.0);
\draw[fill] (4.5,1.0) circle [radius=0.05];
\draw[fill] (8.0,1.3) circle [radius=0.05];
\draw[fill] (7.5,1.0) circle [radius=0.05];
\node[above] at (6.0,3.0) {$F_4$};
%F5
\draw[myGreen, thick] (8.5,1.0) to [out=0,in=290] (11.0,4.0);
\draw[myGreen, thick] (9.0,4.3) to [out=40,in=270] (11.0,4.0);
\draw[myGreen, thick] (9.0,3.5) to [out=270,in=270] (11.0,4.0);
\draw[fill] (8.5,1.0) circle [radius=0.05];
\draw[fill] (9.0,4.3) circle [radius=0.05];
\draw[fill] (9.0,3.5) circle [radius=0.05];
\node[above] at (11.0,4.0) {$F_5$};

\end{tikzpicture}
\caption{Bigrafo $G$: prima propriet� di stallo \label{fig:diningAim1}}
\end{figure}




\begin{figure}[!htbp]
\centering
\begin{tikzpicture}[scale=0.88]
%\draw[help lines] (0,0) grid (13,8);
%Root
\draw[rounded corners=5mm,dashed] (0.0,0.0) rectangle (12.0,8.0);
%Philosophers
\draw[thick] (6.0,6.0) circle [radius=0.5];%P1
\node at (6.0,5.25) {$P_1$};
\draw[thick] (3.0,4.0) circle [radius=0.5];%P2
\node at (3.0,3.0) {$P_2$};
\draw[thick] (4.0,1.0) circle [radius=0.5];%P3
\node at (4.0,0.25) {$P_3$};
\draw[thick] (8.0,1.0) circle [radius=0.5];%P4
\node at (8.0,0.25) {$P_4$};
\draw[thick] (9.0,4.0) circle [radius=0.5];%P5
\node at (9.0,3.0) {$P_5$};
%Forks
\draw[rounded corners=1mm, thick] (2.8,3.8) -- (3.2,3.8) -- (3.0,4.4) -- (2.8,3.8);%F2
\draw[rounded corners=1mm, thick] (3.8,0.8) -- (4.2,0.8) -- (4.0,1.4) -- (3.8,0.8);%F3
\draw[rounded corners=1mm, thick] (7.8,0.8) -- (8.2,0.8) -- (8.0,1.4) -- (7.8,0.8);%F4
\draw[rounded corners=1mm, thick] (8.8,3.8) -- (9.2,3.8) -- (9.0,4.4) -- (8.8,3.8);%F5
\draw[rounded corners=1mm, thick] (5.8,5.8) -- (6.2,5.8) -- (6.0,6.4) -- (5.8,5.8);%F1
%%%   Links   %%%
%F1
\draw[myGreen, thick] (6.5,6.0) to [out=90,in=270] (8.0,7.0);
\draw[myGreen, thick] (6.0,6.3) to [out=90,in=270] (3.0,7.0);
\draw[myGreen, thick] (9.0,4.5) to [out=90,in=270] (8.0,7.0);
\draw[fill] (6.5,6.0) circle [radius=0.05]; \node[right] at (6.5,6.0){0};
\draw[fill] (6.0,6.3) circle [radius=0.05];
\draw[fill] (9.0,4.5) circle [radius=0.05];
\node[above] at (8.0,7.0) {$F_1$};
%F2
\draw[myGreen, thick] (5.5,6.0) to [out=90,in=270] (3.0,7.0);
\draw[myGreen, thick] (3.0,4.3) to [out=150,in=270] (1.0,4.0);
\draw[myGreen, thick] (3.0,4.5) to [out=110,in=290] (3.0,7.0);
\draw[fill] (5.5,6.0) circle [radius=0.05]; \node[left] at (5.5,6.0){1};
\draw[fill] (3.0,4.3) circle [radius=0.05];
\draw[fill] (3.0,4.5) circle [radius=0.05];
\node[above] at (3.0,7.0) {$F_2$};
%F3
\draw[myGreen, thick] (3.0,3.5) to [out=270,in=270] (1.0,4.0);
\draw[myGreen, thick] (4.0,1.3) to [out=90,in=270] (6.0,3.0);
\draw[myGreen, thick] (3.5,1.0) to [out=180,in=270] (1.0,4.0);
\draw[fill] (3.0,3.5) circle [radius=0.05];
\draw[fill] (4.0,1.3) circle [radius=0.05];
\draw[fill] (3.5,1.0) circle [radius=0.05];
\node[above] at (1.0,4.0) {$F_3$};
%F4
\draw[myGreen, thick] (4.5,1.0) to [out=0,in=270] (6.0,3.0);
\draw[myGreen, thick] (8.0,1.3) to [out=90,in=270] (11.0,4.0);
\draw[myGreen, thick] (7.5,1.0) to [out=180,in=270] (6.0,3.0);
\draw[fill] (4.5,1.0) circle [radius=0.05];
\draw[fill] (8.0,1.3) circle [radius=0.05];
\draw[fill] (7.5,1.0) circle [radius=0.05];
\node[above] at (6.0,3.0) {$F_4$};
%F5
\draw[myGreen, thick] (8.5,1.0) to [out=0,in=290] (11.0,4.0);
\draw[myGreen, thick] (9.0,4.3) to [out=20,in=300] (8.0,7.0);
\draw[myGreen, thick] (9.0,3.5) to [out=270,in=270] (11.0,4.0);
\draw[fill] (8.5,1.0) circle [radius=0.05];
\draw[fill] (9.0,4.3) circle [radius=0.05];
\draw[fill] (9.0,3.5) circle [radius=0.05];
\node[above] at (11.0,4.0) {$F_5$};

\end{tikzpicture}
\caption{Bigrafo $G'$: seconda propriet� di stallo \label{fig:diningAim2}}
\end{figure}


$MC_{big}$ incomincia a computare l'intero grafo degli stati: esso � \textbf{teoricamente infinito} perch� le quattro regole viste prima possono essere applicate un numero arbitrario di volte; per esempio, posso applicarle una dopo l'altra sempre sul filosofo $P_1$, facendolo mangiare e pensare all'infinito. Nella prossima sottosezione, vedremo invece che nella pratica il grafo degli stati � finito, pur rappresentando processi infiniti. 

Con la strategia di questa sottosezione � impossibile trovarsi nella situazione di figura \ref{fig:diningAim2}. Quando $MC_{big}$ trova la situazione di stallo di figura \ref{fig:diningAim1}, il model checker smette di generare il grafo degli stati e ritorna False, perch� si � causato un deadlock:
\begin{prop}
$MC_{big}$ ritorna False se e solo se esiste almeno una situazione di deadlock.
\end{prop}

Per convincersi che il bigrafo B rappresenti una situazione di deadlock, si noti come su di esso non si possa applicare pi� nessuna delle quattro regole: poich� la propriet� � una \emph{IsoProperty}, $MC_{big}$ cerca un nodo che sia uguale a B, cio� a cui non si possano applicare pi� regole. Un tale nodo del grafo degli stati si chiama \textbf{stato finale}.

Si noti come ad ogni passo ogni regola trovi molti match: il grafo degli stati sar� molto grande. Si consideri la figura \ref{fig:diningPhil}: la regola \ref{fig:takeFork}.a pu� scattare su ognuno dei 5 filosofi; per esempio, ipotizziamo che scatti sul primo filosofo $P_1$. Ora, le possibilit� sono molteplici:
\begin{itemize}
	\item
	$P_1$ pu� prendere anche la forchetta destra: quindi scatta la regola \ref{fig:takeFork}.b
	\item
	uno qualsiasi degli altri quattro filosofi pu� prendere la propria forchetta sinistra: quindi statta la regola \ref{fig:takeFork}.a
\end{itemize}
per un totale di 6 stati possibili. Si vede bene come l'\textbf{esplosione combinatoria} faccia aumentare in modo esponenziale il numero di nodi del grafo degli stati. \\ 

\begin{lstlisting}
Insert the number of philosophers: 
5



STRATEGY: Every philosopher takes first the left fork.
Are deadlocks avoided?	NO
Number of states of the Model Checker:	93
Resolution Time:	4.786207234 seconds
\end{lstlisting}

Quello mostrato qui sopra � il risultato dell'esecuzione del software: si � implementato un modulo che costruisce il bigrafo con un numero di filosofi scelto dall'utente. In questo caso si � scelto di avere cinque filosofi come in figura \ref{fig:diningPhil}. In seguito, il sistema informa quale strategia si sta adottando: per ora ci interessa solamente la prima strategia. Nella riga seguente il software ritorna il risultato (cio� se c'� o non c'� un deadlock), ed infine il numero di nodi del grafo degli stati.

Per verificare che la strategia in uso causa deadlock, con solamente cinque filosofi il model checker ha creato un grafo degli stati con 93 nodi. Questo � il risultato dell'esplosione combinatoria, che causa appunto un aumento esponenziale del numero di nodi che il model checker deve controllare.

In questo esercizio si � usato l'isomorfismo con propriet� (\emph{Property Isomorphism}), il cui funzionamento, che � poco differente dal normale algoritmo, viene riportato qui sotto:

\begin{prop}[Property Isomorphism]
Due nodi costituiscono un match nel Property Isomorphism se e solo se lo fanno nel normale Isomorphism e hanno le stesse propriet�.
\end{prop}

\begin{figure}[!htbp]
\centering
\subfigure[Primo Bigrafo]{
\begin{tikzpicture}[scale=0.5]
%\draw[help lines] (0,0) grid (13,8);
%Root
\draw[rounded corners=5mm,dashed] (0.0,0.0) rectangle (12.0,8.0);
%Philosophers
\draw[thick] (6.0,6.0) circle [radius=0.5];%P1
\draw[thick] (3.0,4.0) circle [radius=0.5];%P2
\node at (3.0,4.0) {$P_2$};
\draw[thick] (4.0,1.0) circle [radius=0.5];%P3
\node at (4.0,1.0) {$P_3$};
\draw[thick] (8.0,1.0) circle [radius=0.5];%P4
\node at (8.0,1.0) {$P_4$};
\draw[thick] (9.0,4.0) circle [radius=0.5];%P5
\node at (9.0,4.0) {$P_5$};
%Forks
\draw[rounded corners=1mm, thick] (4.0,5.0) -- (4.5,5.0) -- (4.25,5.8) -- (4.0,5.0);%F2
\draw[rounded corners=1mm, thick] (2.0,2.0) -- (2.5,2.0) -- (2.25,2.8) -- (2.0,2.0);%F3
\draw[rounded corners=1mm, thick] (5.75,1.0) -- (6.25,1.0) -- (6.0,1.8) -- (5.75,1.0);%F4
\draw[rounded corners=1mm, thick] (10.0,2.0) -- (10.5,2.0) -- (10.25,2.8) -- (10.0,2.0);%F5
\draw[rounded corners=1mm, thick] (5.8,5.8) -- (6.2,5.8) -- (6.0,6.4) -- (5.8,5.8);%F1
%%%   Links   %%%
%F1
\draw[myGreen, thick] (6.5,6.0) to [out=90,in=270] (8.0,7.0);
\draw[myGreen, thick] (6.0,6.4) to [out=90,in=270] (8.0,7.0);
\draw[myGreen, thick] (9.0,4.5) to [out=90,in=270] (8.0,7.0);
\draw[fill] (6.5,6.0) circle [radius=0.05]; \node[right] at (6.5,6.0){0};
\draw[fill] (6.0,6.4) circle [radius=0.05];
\draw[fill] (9.0,4.5) circle [radius=0.05];
\node[above] at (8.0,7.0) {$F_1$};
%F2
\draw[myGreen, thick] (5.5,6.0) to [out=90,in=270] (3.0,7.0);
\draw[myGreen, thick] (4.25,5.7) to [out=90,in=270] (3.0,7.0);
\draw[myGreen, thick] (3.0,4.5) to [out=110,in=290] (3.0,7.0);
\draw[fill] (5.5,6.0) circle [radius=0.05]; \node[left] at (5.5,6.0){1};
\draw[fill] (4.25,5.7) circle [radius=0.05];
\draw[fill] (3.0,4.5) circle [radius=0.05];
\node[above] at (3.0,7.0) {$F_2$};
%F3
\draw[myGreen, thick] (3.0,3.5) to [out=270,in=270] (1.0,4.0);
\draw[myGreen, thick] (2.25,2.7) to [out=130,in=270] (1.0,4.0);
\draw[myGreen, thick] (3.5,1.0) to [out=180,in=270] (1.0,4.0);
\draw[fill] (3.0,3.5) circle [radius=0.05];
\draw[fill] (2.25,2.7) circle [radius=0.05];
\draw[fill] (3.5,1.0) circle [radius=0.05];
\node[above] at (1.0,4.0) {$F_3$};
%F4
\draw[myGreen, thick] (4.5,1.0) to [out=0,in=270] (6.0,3.0);
\draw[myGreen, thick] (6.0,1.7) to [out=90,in=270] (6.0,3.0);
\draw[myGreen, thick] (7.5,1.0) to [out=180,in=270] (6.0,3.0);
\draw[fill] (4.5,1.0) circle [radius=0.05];
\draw[fill] (6.0,1.7) circle [radius=0.05];
\draw[fill] (7.5,1.0) circle [radius=0.05];
\node[above] at (6.0,3.0) {$F_4$};
%F5
\draw[myGreen, thick] (8.5,1.0) to [out=0,in=290] (11.0,4.0);
\draw[myGreen, thick] (10.25,2.8) to [out=90,in=270] (11.0,4.0);
\draw[myGreen, thick] (9.0,3.5) to [out=270,in=270] (11.0,4.0);
\draw[fill] (8.5,1.0) circle [radius=0.05];
\draw[fill] (10.25,2.8) circle [radius=0.05];
\draw[fill] (9.0,3.5) circle [radius=0.05];
\node[above] at (11.0,4.0) {$F_5$};
\end{tikzpicture}
}

\hspace{1mm}

\subfigure[Secondo Bigrafo]{
\begin{tikzpicture}[scale=0.5]
%\draw[help lines] (0,0) grid (13,8);
%Root
\draw[rounded corners=5mm,dashed] (0.0,0.0) rectangle (12.0,8.0);
%Philosophers
\draw[thick] (6.0,6.0) circle [radius=0.5];%P1
\node at (6.0,6.0) {$P_1$};
\draw[thick] (3.0,4.0) circle [radius=0.5];%P2
\draw[thick] (4.0,1.0) circle [radius=0.5];%P3
\node at (4.0,1.0) {$P_3$};
\draw[thick] (8.0,1.0) circle [radius=0.5];%P4
\node at (8.0,1.0) {$P_4$};
\draw[thick] (9.0,4.0) circle [radius=0.5];%P5
\node at (9.0,4.0) {$P_5$};
%Forks
\draw[rounded corners=1mm, thick] (2.8,3.8) -- (3.2,3.8) -- (3.0,4.4) -- (2.8,3.8);%F2
\draw[rounded corners=1mm, thick] (2.0,2.0) -- (2.5,2.0) -- (2.25,2.8) -- (2.0,2.0);%F3
\draw[rounded corners=1mm, thick] (5.75,1.0) -- (6.25,1.0) -- (6.0,1.8) -- (5.75,1.0);%F4
\draw[rounded corners=1mm, thick] (10.0,2.0) -- (10.5,2.0) -- (10.25,2.8) -- (10.0,2.0);%F5
\draw[rounded corners=1mm, thick] (8.0,5.0) -- (8.5,5.0) -- (8.25,5.8) -- (8.0,5.0);%F1
%%%   Links   %%%
%F1
\draw[myGreen, thick] (6.5,6.0) to [out=90,in=270] (8.0,7.0);
\draw[myGreen, thick] (8.25,5.7) to [out=90,in=270] (8.0,7.0);
\draw[myGreen, thick] (9.0,4.5) to [out=90,in=270] (8.0,7.0);
\draw[fill] (6.5,6.0) circle [radius=0.05]; \node[right] at (6.5,6.0){0};
\draw[fill] (8.25,5.7) circle [radius=0.05];
\draw[fill] (9.0,4.5) circle [radius=0.05];
\node[above] at (8.0,7.0) {$F_1$};
%F2
\draw[myGreen, thick] (5.5,6.0) to [out=90,in=270] (3.0,7.0);
\draw[myGreen, thick] (3.0,4.3) to [out=150,in=270] (3.0,7.0);
\draw[myGreen, thick] (3.0,4.5) to [out=110,in=290] (3.0,7.0);
\draw[fill] (5.5,6.0) circle [radius=0.05]; \node[left] at (5.5,6.0){1};
\draw[fill] (3.0,4.3) circle [radius=0.05];
\draw[fill] (3.0,4.5) circle [radius=0.05];
\node[above] at (3.0,7.0) {$F_2$};
%F3
\draw[myGreen, thick] (3.0,3.5) to [out=270,in=270] (1.0,4.0);
\draw[myGreen, thick] (2.25,2.7) to [out=130,in=270] (1.0,4.0);
\draw[myGreen, thick] (3.5,1.0) to [out=180,in=270] (1.0,4.0);
\draw[fill] (3.0,3.5) circle [radius=0.05];
\draw[fill] (2.25,2.7) circle [radius=0.05];
\draw[fill] (3.5,1.0) circle [radius=0.05];
\node[above] at (1.0,4.0) {$F_3$};
%F4
\draw[myGreen, thick] (4.5,1.0) to [out=0,in=270] (6.0,3.0);
\draw[myGreen, thick] (6.0,1.7) to [out=90,in=270] (6.0,3.0);
\draw[myGreen, thick] (7.5,1.0) to [out=180,in=270] (6.0,3.0);
\draw[fill] (4.5,1.0) circle [radius=0.05];
\draw[fill] (6.0,1.7) circle [radius=0.05];
\draw[fill] (7.5,1.0) circle [radius=0.05];
\node[above] at (6.0,3.0) {$F_4$};
%F5
\draw[myGreen, thick] (8.5,1.0) to [out=0,in=290] (11.0,4.0);
\draw[myGreen, thick] (10.25,2.8) to [out=90,in=270] (11.0,4.0);
\draw[myGreen, thick] (9.0,3.5) to [out=270,in=270] (11.0,4.0);
\draw[fill] (8.5,1.0) circle [radius=0.05];
\draw[fill] (10.25,2.8) circle [radius=0.05];
\draw[fill] (9.0,3.5) circle [radius=0.05];
\node[above] at (11.0,4.0) {$F_5$};
\end{tikzpicture}
}
\caption{Due bigrafi isomorfi \label{fig:isoDinings}}
\end{figure}


Nella libreria \emph{JLibbig}, c'� la possibilit� di allegare ad ogni nodo delle propriet�. Si sono quindi dati degli identificativi per ogni filosofo ed ogni forchetta. In seguito, si sono usate le regole con propriet� introdotte nella sottosezione \ref{sec:dettImpl} per manterle anche dopo l'esecuzione delle quattro regole viste prima. 

La creazione del \emph{Property Isomorphism} � stato indispensabile: senza di esso il model checker avrebbe riconosciuto come uguali i due bigrafi in figura \ref{fig:isoDinings}, causando un comportamento anomalo nella verifica delle propriet�. 

Dalla figura \ref{fig:isoDinings} si capisce immediatamente che senza il \emph{Property Isomorphism} il grafo degli stati sarebbe molto pi� piccolo, perch� non si esplorerebbero tutte le possibilit�. Se $S_0$ � lo stato iniziale, allora ipotizziamo che $S_1$ sia lo stato in cui $P_1$ prende la sua forchetta sinistra $F_1$. Ora per�, da $S_0$ si pu� anche applicare la regola sul filosofo $P_2$, che prender� la forchetta $F_2$ dando vita allo stato $S_2$. Senza l'uso del \emph{Property Isomorphism}, il model checker trover� che $S_1 = S_2$, memorizzando solamente il primo. Quindi il model checker non distingue tra le due situazioni, creando un grafo degli stati \textbf{incompleto}.


\subsection{Seconda strategia}
Vediamo ora un modo per evitare i deadlock: si enumerino tutte le forchette presenti sul tavolo (nel nostro esempio consideriamo il pedice di $F_i$); per mangiare, ogni filosofo deve prendere per prima la forchetta con indice minore. Una traccia d'esecuzione basata sulla figura \ref{fig:diningPhil} potrebbe essere questa: tutti i filosofi $P_i$ ($i \in \{1,2,3,4\}$) prenderanno la forchetta $F_i$ tranne l'ultimo ($P_5$), che dovr� prendere $F_1$, che ha indice minore; questa forchetta per� � gi� occupata da $P_1$ e quindi $P_5$ dovr� aspettare, lasciando libera $F_5$ che potr� essere utilizzata da $F_4$. Quest'ultimo riuscir� a mangiare e dopo un po' lascer� libera la forchetta $F_4$, che verr� presa da $P_3$, e cos� via. Infine, quando si liberer� $F_1$, anche il filosofo $P_5$ riuscir� a mangiare. Con questa soluzione, si evitano quindi sia i deadlock sia le situazioni di starvation.

Le regole sono uguali a quelle della strategia precedente con un'eccezione: si noti come l'ultimo filosofo debba prendere la prima forchetta (quella con indice minore) con la mano destra, mentre la seconda con la sinistra. Quindi si creano altre quattro regole ad-hoc per l'ultimo filosofo: deve prendere per prima la forchetta destra e per seconda quella sinistra, a differenza di tutti gli altri. Lo stesso vale per posare le forchette, per un totale di quattro nuove regole. La nuova strategia � quindi identica alla prima, modulo l'aggiunta di queste quattro regole. Il nodo finale viene identificato tramite un controllo $L$ (Last) al suo interno.

Il BRS per questa strategia ha otto perci� regole. La propriet� da verificare resta la stessa: $\varphi = \pi_B$. Si vedr� per� che ora $MC_{big}$ non riuscir� ad identificare nessuna situazione di deadlock. 

Prima per� si deve precisare un'aspetto importante: si sarebbe tentati di pensare che in assenza di deadlock il model checker continui all'infinito a generare stati. Infatti, se al bigrafo di partenza, che chiameremo $B_1$, applichiamo le quattro regole sempre al filosofo $P_1$, allora quest'ultimo (in ordine):
\begin{itemize}
	\item
	prende la forchetta sinistra
	\item
	prende la forchetta destra
	\item
	posa la forchetta sinistra
	\item
	posa la forchetta destra
\end{itemize}
Dopo la quarta regola, si ritorna ad uno stato uguale al primo, che chiameremo $B_2$, e nulla vieta che queste quattro regole continuino ad eseguire all'infinito. Tramite il \emph{Property Isomorphism} si riesce ad evitare questa situazione: questo algoritmo riesce infatti a verificare che $B1$ e $B_2$ sono uguali, evitando quindi di tornare ad eseguire tutte le regole. Grazie ad esso, si � riusciti ad avere \textbf{un grafo degli stati finito per un processo infinito}.

Riprendendo quanto detto nella sottosezione precedente sull'\emph{Property Isomorphism}, si ha che esso porta i seguenti benefici:
\begin{itemize}
	\item
	il model checker computa un grafo degli stati \textbf{completo}, senza dimenticare nessuna possibile situazione
	\item
	il model checker verifica le propriet� di un processo infinito su un grafo degli stati \textbf{finito}.
\end{itemize}

$MC_{big}$ riesce quindi a computare l'intero grafo degli stati e, dato che per ogni suo nodo $S_i$ si ha che $MC,S_i \not \models \varphi$, ritorna True, a significare che la tecnica � priva di situazioni di deadlock o starvation.

\begin{prop}
$MC_{big}$ ritorna True se e solo se ogni stato non rappresenta una situazione di deadlock.
\end{prop}

\begin{lstlisting}
Insert the number of philosophers: 
5



STRATEGY: all the forks are enumerated. Every philosopher takes first the fork with the lower index.
Are deadlocks avoided?	YES
Number of states of the Model Checker:	189
Resolution Time:	14.129189577 seconds
\end{lstlisting}


Il risultato ora � quello della seconda strategia. Si noti come l'assenza di deadlock causi un aumento del numero di stati, che ora ha raggiunto il valore 189. Aumentando il numero di filosofi ci si rende conto dell'andamento esponenziale dell'esplosione. Infine si prendano in considerazione i tempi di risoluzione del problema: $MC_{big}$ impiega 20 minuti per capire che questa strategia � priva di deadlock.

\begin{lstlisting}
Insert the number of philosophers: 
7



STRATEGY: Every philosopher takes first the left fork.
Are deadlocks avoided?	NO
Number of states of the Model Checker:	801
Resolution Time:	214.26274781 seconds



STRATEGY: all the forks are enumerated. Every philosopher takes first the fork with the lower index.
Are deadlocks avoided?	YES
Number of states of the Model Checker:	1701
Resolution Time:	1210.612988337 seconds
\end{lstlisting}




%%%%%%%%%%%%%%%%%%%%%%%%%%

\section{Politiche di sicurezza}\label{sec:secureBuilding}
In quest'ultimo esempio vedremo come tramite i bigrafi ed il model checker $MC_{big}$ si possano testare delle politiche di sicurezza. In particolare, faremo riferimento alla seguente situazione: un edificio appartenente ad un'azienda contiene varie stanze, ognuna con un computer al suo interno. L'azienda ha importanti file segreti (token) da mantenere al sicuro dentro ogni computer. Ha anche vari dipendenti, che possono entrare liberamente in ogni stanza e collegarsi ai vari computer. Uno di questi dipendenti si chiama Alice.
Lei ed il suo complice Bob vogliono rubare un segreto dell'azienda, seguendo questo piano:

\begin{itemize}
	\item
	Alice, che essendo una dipendente pu� collegarsi ad un computer, ruba il file segreto e lo trasferisce sul proprio smartphone.
	\item
	Alice esce dalla stanza, pur rimanendo all'interno dell'azienda
	\item
	Alice tramite il suo smartphone stabilisce un collegamento con quello di Bob, che si trova al di fuori dell'azienda, e trasferisce il file segreto, che ora � di dominio pubblico
\end{itemize}

Il nostro compito � quello di stabilire una politica di sicurezza che eviti a qualsiasi token dell'azienda di diventare pubblico, in questo caso di diventare in possesso di Bob. Si vedranno quindi due politiche: la prima consentir� a Bob di ottenere il token, mentre la seconda lo eviter� assicurando la sicurezza dell'azienda. 

Tutte le possibili situazioni in cui Alice e Bob possono agire per rubare i file segreti verranno calcolate dal model checker $MC_{big}$, che alla fine dell'esecuzione dir� se la politica scelta sar� sicura o meno. Infine, si � costruito un modulo che offre all'utente la possibilit� di visitare il grafo degli stati, consentendogli di ripercorrere tutte le azioni che hanno consentito a Alice e Bob di ottenere il file segreto.


\subsection{Segnatura}
Incominciamo con il definire la segnatura del bigrafo che modella il problema. Si consideri la figura \ref{fig:bigSafety}: il nodo pi� grande � ``\emph{Building}" e rappresenta l'edificio dell'azienda. In questo esempio, sono presenti due stanze, ognuna con un computer all'interno. Ogni computer contiene un ``\emph{Token}", che � uno dei file segreti dell'azienda. 

I nodi per Alice e Bob sono modellati tramite due circonferenze, che per� si trovano sotto radici diverse: Alice � dentro l'azienda, in quanto � una dipendente e pu� entrare a tutti gli effetti sia dentro il ``\emph{Building}" sia dentro le ``\emph{Room}". Invece Bob si trova al di fuori dell'azienda, perch� non � un dipendente e quindi il suo accesso � vietato. Ai fini della verifica da parte del model checker, tutti i dipendenti verranno considerati con cattive intenzioni, ovvero: ogni dipendente sar� modellato da un nodo di controllo ``\emph{Alice}". Allo stesso modo, ogni persona al di fuori dell'azienda verr� modellata con un nodo di controllo ``\emph{Bob}".
Sia Alice che Bob possiedono uno smartphone, che useranno per trasferire il ``\emph{Token}".


\begin{figure}[th]
\centering
\begin{tikzpicture}
%\draw[help lines] (0,0) grid (13,10);
%Edges
\draw[-|, myGreen, thick] (6.0,1.5) to [out=90,in=45] (5.0,1.5);%bob port
\draw[-|, myGreen, thick] (3.0,4.0) to [out=130,in=270] (2.5,4.5);%Alice port
\draw[-|, myGreen, thick] (3.5,6.0) to [out=130,in=300] (3.5,7.0);%computer1 port
\draw[-|, myGreen, thick] (8.5,6.0) to [out=130,in=300] (8.5,7.0);%computer2 port

%Root
\draw[rounded corners=5mm, dashed] (0.0,0.0) rectangle (13.0,10.0);
%Building
\draw[rounded corners=5mm, thick] (1.0,2.0) rectangle (12.0,9.0);
\draw[rounded corners=5mm, thick] (1.05,2.05) rectangle (11.95,8.95);
\node[above right] at (1.0,9.0) {Building};
%Room 1
\draw[rounded corners=3mm, thick] (2.0,5.0) rectangle (6.0,8.0);
\node[above right] at (2.0,8.0) {$Room_1$};
%Computer 1
\draw[rounded corners=3mm, thick] (3.0,5.2) rectangle (5.0,6.0);
\node[above right] at (3.5,6.0) {Computer};
\draw[fill] (3.5,6.0) circle [radius=0.05];
%Token 1
\draw[rounded corners=1mm, thick] (3.7,5.3) -- (4.3,5.3) -- (4.0,5.9) -- (3.7,5.3);
\draw[red] (4.0,5.6) to [out=270,in=90] (6.0,4.2);
\node at (6.0,4.0){Token};
%Room 2
\draw[rounded corners=3mm, thick] (7.0,5.0) rectangle (11.0,8.0);
\node[above right] at (7.0,8.0) {$Room_2$};
%Computer 2
\draw[rounded corners=3mm, thick] (8.0,5.2) rectangle (10.0,6.0);
\node[above right] at (8.5,6.0) {Computer};
\draw[fill] (8.5,6.0) circle [radius=0.05];
%Token 2
\draw[rounded corners=1mm, thick] (8.7,5.3) -- (9.3,5.3) -- (9.0,5.9) -- (8.7,5.3);
\draw[red] (9.0,5.6) to [out=270,in=90] (6.2,4.2);
%Alice
\draw[thick] (3.0,3.5) circle [radius=0.8];
\node[below] at (3.0,2.7) {Alice};
% Phone A
\draw[rounded corners=1mm, thick] (2.8,3.0) rectangle (3.2,4.0);
\draw[red] (3.0,3.2) to [out=180,in=90] (1.8,1.3);
\node at (2.0,1.0){Smartphone};
\draw[fill] (3.0,4.0) circle [radius=0.05];
%Bob
\draw[thick, red] (6.0,1.0) circle [radius=0.8];
\node[right] at (6.8,1.0) {Bob};
% Phone B
\draw[rounded corners=1mm, thick] (5.8,0.5) rectangle (6.2,1.5);
\draw[red] (6.0,1.0) to [out=180,in=50] (3.2,1.2);
\draw[fill] (6.0,1.5) circle [radius=0.05];
\end{tikzpicture}
\caption{Bigrafo per il problema della sicurezza dell'azienda \label{fig:bigSafety}}
\end{figure}

Ogni smartphone ed ogni computer possiedono una porta per consentire il collegamento: uno smartphone si pu� collegare ad un altro smartphone oppure ad un computer. Stabilito il collegamento, � possibile trasferire il Token. Per distinguere il ``\emph{Building}" dalle ``\emph{Room}", si � disegnato il primo tramite doppie linee.  


\subsection{Prima politica}
Vediamo ora la prima politica, che come gi� anticipato \textbf{non} assicurer� la sicurezza dell'azienda. Nei BRS, una politica si traduce in un insieme di regole di reazione, che descrivono come ogni nodo deve comportarsi. Per esempio, una regola potrebbe coinvolgere una ``\emph{Room}", obbligandola a chiedere un badge ad ogni dipendente che vuole entrare.



\begin{figure}[!htbp]
\centering
\begin{tikzpicture}
%\draw[help lines] (0,0) grid (14,10);
%%%   Redex   %%%
\draw[rounded corners=3mm, dashed] (0.0,0.0) rectangle (6.0,4.0);
%Room
\draw[rounded corners=2mm, thick] (2.5,0.5) rectangle (5.5,3.5);
%Site 0
\draw[fill=myGrey, dashed] (2.8,0.8) rectangle (3.2,1.2);
\node at (3.0,1.0){0};
%Alice
\draw[thick] (1.0,1.0) circle [radius=0.8];
%Site 1
\draw[fill=myGrey, dashed] (0.8,0.8) rectangle (1.2,1.2);
\node at (1.0,1.0){1};

\draw[->, red, thick] (6.2,2.0) -- (7.8,2.0);

%%%   Reactum   %%%
\draw[rounded corners=3mm, dashed] (8.0,0.0) rectangle (14.0,4.0);
%Room
\draw[rounded corners=2mm, thick] (10.5,0.5) rectangle (13.5,3.5);
%Site 0
\draw[fill=myGrey, dashed] (10.8,0.8) rectangle (11.2,1.2);
\node at (11.0,1.0){0};
%Alice
\draw[thick] (12.0,2.0) circle [radius=0.8];
%Site 1
\draw[fill=myGrey, dashed] (11.8,1.8) rectangle (12.2,2.2);
\node at (12.0,2.0){1};


\end{tikzpicture}
\caption{regola per l'entrata in una stanza ($R_0$)\label{fig:enterRoom}}
\end{figure}





\begin{figure}[!htbp]
\centering
\begin{tikzpicture}
%\draw[help lines] (0,0) grid (14,10);
%%%   Redex   %%%
\draw[rounded corners=3mm, dashed] (0.0,0.0) rectangle (6.0,4.0);
%Room
\draw[rounded corners=2mm, thick] (2.5,0.5) rectangle (5.5,3.5);
%Site 0
\draw[fill=myGrey, dashed] (2.8,0.8) rectangle (3.2,1.2);
\node at (3.0,1.0){0};
%Alice
\draw[thick] (4.0,2.0) circle [radius=0.8];
%Site 1
\draw[fill=myGrey, dashed] (3.8,1.8) rectangle (4.2,2.2);
\node at (4.0,2.0){1};

\draw[->, red, thick] (6.2,2.0) -- (7.8,2.0);

%%%   Reactum   %%%
\draw[rounded corners=3mm, dashed] (8.0,0.0) rectangle (14.0,4.0);
%Room
\draw[rounded corners=2mm, thick] (10.5,0.5) rectangle (13.5,3.5);
%Site 0
\draw[fill=myGrey, dashed] (10.8,0.8) rectangle (11.2,1.2);
\node at (11.0,1.0){0};
%Alice
\draw[thick] (9.0,1.0) circle [radius=0.8];
%Site 1
\draw[fill=myGrey, dashed] (8.8,0.8) rectangle (9.2,1.2);
\node at (9.0,1.0){1};

\end{tikzpicture}
\caption{regola per l'uscita da una stanza ($R_1$)\label{fig:leaveRoom}}
\end{figure}



Le regole $R_0$ e $R_1$ modellano rispettivamente l'entrata e l'uscita da una stanza all'interno dell'azienda. Si noti come ogni dipendente, ovvero ogni nodo circolare all'interno dell'azienda, possa entrare liberamente in ogni stanza. La politica di sicurezza � quindi \textbf{minima}, perch� non vengono fatti controlli e ogni dipendente pu� accedere al computer: in particolare, pu� collegarlo al suo smartphone e trasferire il Token. Se togliessimo il sito numero 1, allora si potrebbe entrare nella stanza solamente senza oggetti pericolosi: in altre parole, ogni dipendente non dovrebbe avere niente con s� per poter entrare in una stanza.



\begin{figure}[!htbp]
\centering
\begin{tikzpicture}
%\draw[help lines] (0,0) grid (14,10);
%%%   Redex   %%%
%Links
\draw[myGreen,thick] (2.7,2.0) to [out=60,in=270] (2.5,4.3);%x
\draw[myGreen,thick] (4.25,2.5) to [out=120,in=270] (4.5,4.3);%y
\draw[rounded corners=3mm, dashed] (0.0,0.0) rectangle (6.0,4.0);
%Room
\draw[rounded corners=2mm, thick] (0.5,0.5) rectangle (5.5,3.5);
%Computer
\draw[rounded corners=2mm, thick] (1.0,1.0) rectangle (3.0,2.0);
\draw[fill] (2.7,2.0) circle [radius=0.05];
%Token
\draw[rounded corners=1mm, thick] (1.7,1.1) -- (2.3,1.1) -- (2.0,1.9) -- (1.7,1.1);
%Alice
\draw[thick] (4.3,2.0) circle [radius=0.8];
%Alice phone
\draw[rounded corners=1mm, thick] (4.0,1.5) rectangle (4.5,2.5);
\draw[fill] (4.25,2.5) circle [radius=0.05];
%Outers
\node at (2.5,4.5){x};
\node at (4.5,4.5){y};

\draw[->, red, thick] (6.2,2.0) -- (7.8,2.0);

%%%   Reactum   %%%
%Links
\draw[myGreen,thick] (10.7,2.0) to [out=60,in=270] (10.5,4.3);%x
\draw[myGreen,thick] (12.25,2.5) to [out=120,in=270] (10.8,3.0);
\draw[rounded corners=3mm, dashed] (8.0,0.0) rectangle (14.0,4.0);
%Room
\draw[rounded corners=2mm, thick] (8.5,0.5) rectangle (13.5,3.5);
%Computer
\draw[rounded corners=2mm, thick] (9.0,1.0) rectangle (11.0,2.0);
\draw[fill] (10.7,2.0) circle [radius=0.05];
%Token
\draw[rounded corners=1mm, thick] (9.7,1.1) -- (10.3,1.1) -- (10.0,1.9) -- (9.7,1.1);
%Alice
\draw[thick] (12.3,2.0) circle [radius=0.8];
%Alice phone
\draw[rounded corners=1mm, thick] (12.0,1.5) rectangle (12.5,2.5);
\draw[fill] (12.25,2.5) circle [radius=0.05];
%Outers
\node at (10.5,4.5){x};
\node at (12.5,4.5){y};
\end{tikzpicture}
\caption{regola per il collegamento ad un computer ($R_2$)\label{fig:compConnect}}
\end{figure}








\begin{figure}[!htbp]
\centering
\begin{tikzpicture}
%\draw[help lines] (0,0) grid (14,10);
%%%   Redex   %%%
%Links
\draw[myGreen, thick] (2.0,1.3) to [out=90,in=270] (3.0,4.3);
\draw[myGreen, thick] (4.5,3.0) to [out=90,in=270] (3.0,4.3);
%Root 1
\draw[rounded corners=3mm, dashed] (0.0,0.0) rectangle (3.0,4.0);
%Computer
\draw[rounded corners=2mm, thick] (0.3,0.3) rectangle (2.7,1.3);
\draw[fill] (2.0,1.3) circle [radius=0.05];
%Token
\draw[rounded corners=1mm, thick] (1.2,0.4) -- (1.8,0.4) -- (1.5,1.2) -- (1.2,0.4);
%Root 2
\draw[rounded corners=3mm, dashed] (3.3,0.0) rectangle (6.0,4.0);
%Phone
\draw[rounded corners=3mm, thick] (4.0,1.0) rectangle (5.0,3.0);
\draw[fill] (4.5,3.0) circle [radius=0.05];
%Outers
\node at (3.0,4.5) {x};

\draw[->, red, thick] (6.2,2.0) -- (7.8,2.0);

%%%   Reactum   %%%
%Links
\draw[myGreen, thick] (10.0,1.3) to [out=90,in=270] (11.0,4.3);
%Root 1
\draw[rounded corners=3mm, dashed] (8.0,0.0) rectangle (11.0,4.0);
%Computer
\draw[rounded corners=2mm, thick] (8.3,0.3) rectangle (10.7,1.3);
\draw[fill] (10.0,1.3) circle [radius=0.05];
%Token
\draw[rounded corners=1mm, thick] (12.2,1.4) -- (12.8,1.4) -- (12.5,2.2) -- (12.2,1.4);
%Root 2
\draw[rounded corners=3mm, dashed] (11.3,0.0) rectangle (14.0,4.0);
%Phone
\draw[rounded corners=3mm, thick] (12.0,1.0) rectangle (13.0,3.0);
\draw[fill] (12.5,3.0) circle [radius=0.05];
%Outers
\node at (11.0,4.5) {x};
\end{tikzpicture}
\caption{regola per il trasferimento di un token da un computer ad uno smartphone ($R_3$)\label{fig:transfer_comp}}
\end{figure}


Le regola $R_2$ e $R_3$ trattano il trasferimento di un Token dal computer ad uno smartphone: la prima stabilisce una connessione mentre la seconda esegue lo spostamento del file segreto. Si noti come lo smartphone del dipendente non debba contenere nessun altro token: se Alice ha gi� rubato un file segreto, allora il prossimo computer a cui si connette se ne accorger� e non le dar� il file segreto al suo interno.\\




\begin{figure}[th]
\centering
\begin{tikzpicture}
%\draw[help lines] (0,0) grid (14,10);
%%%   Redex   %%%
%Root 1
\draw[rounded corners=3mm, dashed] (0.0,0.0) rectangle (3.0,4.0);
%Phone
\draw[rounded corners=3mm, thick] (1.0,1.0) rectangle (2.0,3.0);
\draw[fill] (1.5,3.0) circle [radius=0.05];
%Token
\draw[rounded corners=1mm, thick] (1.2,1.4) -- (1.8,1.4) -- (1.5,2.2) -- (1.2,1.4);
%Root 2
\draw[rounded corners=3mm, dashed] (3.3,0.0) rectangle (6.0,4.0);
%Phone
\draw[rounded corners=3mm, thick] (4.0,1.0) rectangle (5.0,3.0);
\draw[fill] (4.5,3.0) circle [radius=0.05];
%Site
\draw[rounded corners=1mm,fill=myGrey,dashed] (4.1,2.2) rectangle (4.6,2.7);
\node at (4.35,2.45) {0};
%Outers
\node at (3.0,4.5) {x};

\draw[->, red, thick] (6.2,2.0) -- (7.8,2.0);

%%%   Reactum   %%%
%Root 1
\draw[rounded corners=3mm, dashed] (8.0,0.0) rectangle (11.0,4.0);
%Links
\draw[myGreen, thick] (9.5,3.0) to [out=90,in=270] (11.0,4.3);
\draw[myGreen, thick] (12.5,3.0) to [out=90,in=270] (11.0,4.3);
%Phone
\draw[rounded corners=3mm, thick] (9.0,1.0) rectangle (10.0,3.0);
\draw[fill] (9.5,3.0) circle [radius=0.05];
%Token
\draw[rounded corners=1mm, thick] (9.2,1.4) -- (9.8,1.4) -- (9.5,2.2) -- (9.2,1.4);
%Root 2
\draw[rounded corners=3mm, dashed] (11.3,0.0) rectangle (14.0,4.0);
%Phone
\draw[rounded corners=3mm, thick] (12.0,1.0) rectangle (13.0,3.0);
\draw[fill] (12.5,3.0) circle [radius=0.05];
%Site
\draw[rounded corners=1mm,fill=myGrey,dashed] (12.1,2.2) rectangle (12.6,2.7);
\node at (12.35,2.45) {0};
%Outers
\node at (11.0,4.5) {x};


\end{tikzpicture}
\caption{regola per iniziare una connessione tra due smartphones ($R_4$)\label{fig:call}}
\end{figure}






\begin{figure}[th]
\centering
\begin{tikzpicture}
%\draw[help lines] (0,0) grid (14,10);
%%%   Redex   %%%
%Root 1
\draw[rounded corners=3mm, dashed] (0.0,0.0) rectangle (3.0,4.0);
%Links
\draw[myGreen, thick] (1.5,3.0) to [out=90,in=270] (3.0,4.5);
\draw[myGreen, thick] (4.5,3.0) to [out=90,in=270] (3.0,4.5);
%Phone
\draw[rounded corners=3mm, thick] (1.0,1.0) rectangle (2.0,3.0);
\draw[fill] (1.5,3.0) circle [radius=0.05];
%Token
\draw[rounded corners=1mm, thick] (1.2,1.4) -- (1.8,1.4) -- (1.5,2.2) -- (1.2,1.4);
%Root 2
\draw[rounded corners=3mm, dashed] (3.3,0.0) rectangle (6.0,4.0);
%Phone
\draw[rounded corners=3mm, thick] (4.0,1.0) rectangle (5.0,3.0);
\draw[fill] (4.5,3.0) circle [radius=0.05];
%Site
\draw[rounded corners=1mm,fill=myGrey,dashed] (4.1,2.2) rectangle (4.6,2.7);
\node at (4.35,2.45) {0};
%Outers
\node at (3.0,4.5) {x};

\draw[->, red, thick] (6.2,2.0) -- (7.8,2.0);

%%%   Reactum   %%%
%Root 1
\draw[rounded corners=3mm, dashed] (8.0,0.0) rectangle (11.0,4.0);
%Phone
\draw[rounded corners=3mm, thick] (9.0,1.0) rectangle (10.0,3.0);
\draw[fill] (9.5,3.0) circle [radius=0.05];
%Token
\draw[rounded corners=1mm, thick] (12.2,1.4) -- (12.8,1.4) -- (12.5,2.2) -- (12.2,1.4);
%Root 2
\draw[rounded corners=3mm, dashed] (11.3,0.0) rectangle (14.0,4.0);
%Phone
\draw[rounded corners=3mm, thick] (12.0,1.0) rectangle (13.0,3.0);
\draw[fill] (12.5,3.0) circle [radius=0.05];
%Site
\draw[rounded corners=1mm,fill=myGrey,dashed] (12.1,2.2) rectangle (12.6,2.7);
\node at (12.35,2.45) {0};
%Outers
\node at (11.0,4.5) {x};


\end{tikzpicture}
\caption{regola per trasferire un Token fra due smartphones ($R_5$)\label{fig:transf_token}}
\end{figure}



Infine, le regole $R_4$ e $R_5$ consentono allo smarthphone di Alice di stabilire una connessione con quello di Bob e di trasferire il Token, che ora si trova al di fuori dell'azienda. Alice pu� chiamare solo se ha ottenuto il Token, come suggerisce il redex della regola $R_4$. Si noti come, una volta ottenuto il Token, Alice possa connettersi a Bob anche dall'interno dell'azienda.\\


Vediamo ora l'applicazione delle regole sul bigrafo di figura \ref{fig:bigSafety}, che chiameremo $S_0$. Applicando in ordine le regole $R_0$, $R_2$ ed $R_3$ al bigrafo $S_0$ si ottiene il bigrafo di figura \ref{fig:threeRules}: Alice � entrata nella prima stanza, ha collegato il suo smartphone al computer ed ha trasferito un file segreto dell'azienda dal computer allo smartphone. Dato che abbiamo applicato tre regole, chiameremo questo bigrafo $S_3$.



\begin{figure}[th]
\centering
\begin{tikzpicture}
%\draw[help lines] (0,0) grid (13,10);
%Edges
\draw[-|, myGreen, thick] (6.0,1.5) to [out=90,in=45] (5.0,1.5);%bob port
\draw[-|, myGreen, thick] (5.0,7.5) to [out=90,in=45] (4.5,8.5);%alice port
\draw[-|, myGreen, thick] (3.5,6.0) to [out=130,in=300] (3.5,7.0);%computer1 port
\draw[-|, myGreen, thick] (8.5,6.0) to [out=130,in=300] (8.5,7.0);%computer2 port

%Root
\draw[rounded corners=5mm, dashed] (0.0,0.0) rectangle (13.0,10.0);
%Building
\draw[rounded corners=5mm, thick] (1.0,2.0) rectangle (12.0,9.0);
\draw[rounded corners=5mm, thick] (1.05,2.05) rectangle (11.95,8.95);
%Room 1
\draw[rounded corners=3mm, thick] (2.0,5.0) rectangle (6.0,8.0);
%Computer 1
\draw[rounded corners=3mm, thick] (3.0,5.2) rectangle (5.0,6.0);
\draw[fill] (3.5,6.0) circle [radius=0.05];
%Token 1
\draw[rounded corners=1mm, thick] (4.7,6.6) -- (5.3,6.6) -- (5.0,7.2) -- (4.7,6.6);
%Room 2
\draw[rounded corners=3mm, thick] (7.0,5.0) rectangle (11.0,8.0);
%Computer 2
\draw[rounded corners=3mm, thick] (8.0,5.2) rectangle (10.0,6.0);
\draw[fill] (8.5,6.0) circle [radius=0.05];
%Token 2
\draw[rounded corners=1mm, thick] (8.7,5.3) -- (9.3,5.3) -- (9.0,5.9) -- (8.7,5.3);
%Alice
\draw[thick] (5.0,7.0) circle [radius=0.8];
% Phone A
\draw[rounded corners=1mm, thick] (4.6,6.5) rectangle (5.4,7.5);
\draw[fill] (5.0,7.5) circle [radius=0.05];
%Bob
\draw[thick, red] (6.0,1.0) circle [radius=0.8];
% Phone B
\draw[rounded corners=1mm, thick] (5.8,0.5) rectangle (6.2,1.5);
\draw[fill] (6.0,1.5) circle [radius=0.05];


\end{tikzpicture}
\caption{Applicazione delle regola $R_0$,$R_2$ e $R_3$ \label{fig:threeRules}}
\end{figure}





\begin{figure}[!htbp]
\centering
\begin{tikzpicture}
%\draw[help lines] (0,0) grid (13,10);
%Edges
\draw[-|, myGreen, thick] (6.0,1.5) to [out=90,in=45] (5.0,1.5);%bob port
\draw[-|, myGreen, thick] (3.0,4.0) to [out=130,in=270] (2.5,4.5);%Alice port
\draw[-|, myGreen, thick] (3.5,6.0) to [out=130,in=300] (3.5,7.0);%computer1 port
\draw[-|, myGreen, thick] (8.5,6.0) to [out=130,in=300] (8.5,7.0);%computer2 port

%Root
\draw[rounded corners=5mm, dashed] (0.0,0.0) rectangle (13.0,10.0);
%Building
\draw[rounded corners=5mm, thick] (1.0,2.0) rectangle (12.0,9.0);
\draw[rounded corners=5mm, thick] (1.05,2.05) rectangle (11.95,8.95);
%Room 1
\draw[rounded corners=3mm, thick] (2.0,5.0) rectangle (6.0,8.0);
%Computer 1
\draw[rounded corners=3mm, thick] (3.0,5.2) rectangle (5.0,6.0);
\draw[fill] (3.5,6.0) circle [radius=0.05];
%Token 1
\draw[rounded corners=1mm, thick] (5.7,0.6) -- (6.3,0.6) -- (6.0,1.3) -- (5.7,0.6);
%Room 2
\draw[rounded corners=3mm, thick] (7.0,5.0) rectangle (11.0,8.0);
%Computer 2
\draw[rounded corners=3mm, thick] (8.0,5.2) rectangle (10.0,6.0);
\draw[fill] (8.5,6.0) circle [radius=0.05];
%Token 2
\draw[rounded corners=1mm, thick] (8.7,5.3) -- (9.3,5.3) -- (9.0,5.9) -- (8.7,5.3);
%Alice
\draw[thick] (3.0,3.5) circle [radius=0.8];
% Phone A
\draw[rounded corners=1mm, thick] (2.8,3.0) rectangle (3.2,4.0);
\draw[fill] (3.0,4.0) circle [radius=0.05];
%Bob
\draw[thick, red] (6.0,1.0) circle [radius=0.8];
% Phone B
\draw[rounded corners=1mm, thick] (5.6,0.5) rectangle (6.4,1.5);
\draw[fill] (6.0,1.5) circle [radius=0.05];
\end{tikzpicture}
\caption{Applicazione delle regole $R_1$,$R_4$ e $R_5$ \label{fig:threeLastRules}}
\end{figure}




In figura \ref{fig:threeLastRules}, � raffigurato il bigrafo $S_6$, ottenuto tramite l'applicazione delle regole $R_1$,$R_4$ e $R_5$ al bigrafo $S_3$. In ordine: Alice � uscita dalla stanza con il Token, ha stabilito una connessione con Bob ed ha trasferito il file segreto. Si noti come non fosse necessario che Alice uscisse dalla stanza per chiamare Bob. Questa politica di sicurezza � quindi fallace, perch� consente a Bob o a qualsiasi persona esterna di ottenere un file segreto dell'azienda.\\

In molti casi l'azienda non ha un edificio semplice come quello di figura \ref{fig:bigSafety}, e quindi c'� bisogno di uno strumento automatico, che affronti il problema anche quando nell'edificio sono presenti numerose stanze, computers e soprattutto numerosi dipendenti. Per questi motivi si � usato $MC_{big}$ per la verifica: esso computa l'intero grafo degli stati, cio� considera tutte le possibili situazioni in cui l'azienda si pu� trovare, ed appena trova una situazione di pericolo ritorna False, a significare che la politica adottata dall'azienda non � sicura.
\begin{prop}
Il model checker $MC_{big}$ ritorna True se e solo se non c'� \emph{nessuna} situazione di pericolo.
\end{prop}
La propriet� da far verificare al model checker � molto semplice:
\begin{center}
$\varphi = \wario_B(T,T,T)$
\end{center}
dove B � il bigrafo di figura \ref{fig:bigBSafety}. Quindi la precedente proposizione, la possiamo tradurre in:
\begin{prop}\label{prop:MCsafety}
Il model checker $MC_{big}$ ritorna False $\Leftrightarrow \exists S_i$ t.c. $MC,S_i \models \wario_B(T,T,T)$. 
Equivalentemente, ritorna True  $\Leftrightarrow \forall S_i$ $(MC,S_i \not \models \wario_B(T,T,T) )$.
\end{prop}



\begin{figure}[!htbp]
\centering
\begin{tikzpicture}
%\draw[help lines] (0,0) grid (13,10);
%Root
\draw[rounded corners=3mm, dashed] (0.0,0.0) rectangle (5.0,5.0);
%Outers
\node at (3.0,5.5){x};
%Link
\draw[myGreen,thick] (2.5,4.0) to [out=90,in=270] (3.0,5.3);
%Bob
\draw[thick, red] (2.5,2.5) circle [radius=2.0];
%Phone
\draw[rounded corners=3mm, thick] (1.8,1.0) rectangle (3.2,4.0);
\draw[fill] (2.5,4.0) circle [radius=0.05];
%Site
\draw[rounded corners=1mm,fill=myGrey,dashed] (2.2,1.2) rectangle (2.7,1.7);
\node at (2.45,1.45) {0};
%Token
\draw[rounded corners=1mm,thick] (2.1,2.0) -- (2.9,2.0) -- (2.5,2.8) -- (2.1,2.0); 



\end{tikzpicture}
\caption{Applicazione delle regole $R_1$,$R_4$ e $R_5$ \label{fig:bigBSafety}}
\end{figure}





\subsection{Seconda strategia}
Si vedr� ora una strategia che, tramite l'introduzione di due nuove regole, riesce a garantire la sicurezza dell'azienda. Le regole sono le stesse del caso precedente, eccetto $R_0$ e $R_1$, che ora sono state sostituite da $secureR_0$ e $secureR_1$, in figura \ref{fig:secureEnterRoom} e \ref{fig:secureLeaveRoom}.


\begin{figure}[!htbp]
\centering
\begin{tikzpicture}
%\draw[help lines] (0,0) grid (14,10);
%%%   Redex   %%%
%Outers
\node at (1.5,4.5){x};
%Links
\draw[myGreen,thick] (1.0,1.5) to [out=90,in=270] (1.5,4.3);
%Root
\draw[rounded corners=3mm, dashed] (0.0,0.0) rectangle (6.0,4.0);
%Room
\draw[rounded corners=2mm, thick] (2.5,0.5) rectangle (5.5,3.5);
%Site 0
\draw[fill=myGrey, dashed] (2.8,0.8) rectangle (3.2,1.2);
\node at (3.0,1.0){0};
%Alice
\draw[thick] (1.0,1.0) circle [radius=0.8];
%Phone 
\draw[rounded corners=2mm, thick] (0.6,0.5) rectangle (1.4,1.5);
\draw[fill] (1.0,1.5) circle [radius=0.05];
%Site
\draw[rounded corners=1mm,fill=myGrey,dashed] (0.8,0.8) rectangle (1.2,1.2);
\node at (1.0,1.0){1};

\draw[->, red, thick] (6.2,2.0) -- (7.8,2.0);

%%%   Reactum   %%%
\draw[rounded corners=3mm, dashed] (8.0,0.0) rectangle (14.0,4.0);
%Room
\draw[rounded corners=2mm, thick] (10.5,0.5) rectangle (13.5,3.5);
%Site 0
\draw[fill=myGrey, dashed] (10.8,0.8) rectangle (11.2,1.2);
\node at (11.0,1.0){0};
%Alice
\draw[thick] (12.0,2.0) circle [radius=0.8];
\end{tikzpicture}
\caption{regola per l'entrata sicura in una stanza ($secureR_0$)\label{fig:secureEnterRoom}}
\end{figure}





\begin{figure}[!htbp]
\centering
\begin{tikzpicture}
%\draw[help lines] (0,0) grid (14,10);
%%%   Redex   %%%
\draw[rounded corners=3mm, dashed] (0.0,0.0) rectangle (6.0,4.0);
%Room
\draw[rounded corners=2mm, thick] (2.5,0.5) rectangle (5.5,3.5);
%Site 0
\draw[fill=myGrey, dashed] (2.8,0.8) rectangle (3.2,1.2);
\node at (3.0,1.0){0};
%Alice
\draw[thick] (4.0,2.0) circle [radius=0.8];

\draw[->, red, thick] (6.2,2.0) -- (7.8,2.0);

%%%   Reactum   %%%
\draw[rounded corners=3mm, dashed] (8.0,0.0) rectangle (14.0,4.0);
%Room
\draw[rounded corners=2mm, thick] (10.5,0.5) rectangle (13.5,3.5);
%Site 0
\draw[fill=myGrey, dashed] (10.8,0.8) rectangle (11.2,1.2);
\node at (11.0,1.0){0};
%Alice
\draw[thick] (9.0,1.0) circle [radius=0.8];
%Phone
\draw[rounded corners=2mm, thick] (8.7,0.3) rectangle (9.3,1.5);
\draw[fill] (9.0,1.5) circle [radius=0.05];

\end{tikzpicture}
\caption{regola per l'uscita sicura da una stanza ($secureR_1$)\label{fig:secureLeaveRoom}}
\end{figure}




Ora Alice non pu� pi� trasferire il Token sul suo smartphone, e di conseguenza le uniche due regole applicabili sono $secureR_0$ e $secureR_1$. La propriet� che $MC_{big}$ dovr� verificare sar� sempre $\varphi = \wario_B(T,T,T)$: ora nessuno stato $S_i$ del grafo sar� tale che $MC,S_i \models \varphi$, per cui, data la proposizione \ref{prop:MCsafety}, il model checker ritorner� True. Questa politica rende sicura l'azienda.

\subsection{Implementazione}
Si sono costruite delle classi flessibili, in modo che l'utente possa definire il proprio edificio (per quanto grande esso sia) senza ridefinire le regole di reazione della politica di sicurezza. Per esempio, in \cite{BRSAnalysis} sono presenti le classi astratte \emph{Building} e \emph{SecureBuilding}, che modellano rispettivamente la prima e la seconda strategia. L'edificio rappresentato nelle figure precedenti � stato costruito nelle due classi \emph{MyBuilding} e \emph{MySecureBuilding}, che estendono le prime due, in modo da poter simulare entrambe.

Se modelliamo il problema di figura \ref{fig:bigSafety} e lanciamo l'esecuzione, allora il sistema avr� il seguente output in cui informa che la prima politica non � sicura mentre la seconda lo �:
\begin{lstlisting}
1. First building: the policy does not ensure the safety
Are the tokens inside the building safe?	NO



2. Second building: the policy ensures the safety
Are the tokens inside the building safe?	YES
\end{lstlisting}
Un'importante caratteristica di questo esempio, � il modulo che consente di poter navigare nel grafo degli stati: per esempio, se il titolare dell'azienda vuole sapere quali sono stati i singoli passi che hanno eseguito Alice e Bob per rubare il file segreto, allora pu� navigare nel grafo scegliendo di volta in volta che regola di reazione applicare.
\begin{lstlisting}
Do you want to visit the first graph?	(yes,no)
yes
40 - 0
----- Printing Bigraph Root -----
edge E_412{ N_411:Phone; }
edge E_418{ N_417:Phone; }
edge E_414{ N_413:Computer; }
edge E_416{ N_415:Computer; }
root 0 {}
   node N_40D:Bob {}
      node N_411:Phone { port0: Edge E_412;}
   node N_40C:Building {}
      node N_40F:Room {}
         node N_415:Computer { port0: Edge E_416;}
            node N_41A:Token {}
      node N_40E:Room {}
         node N_413:Computer { port0: Edge E_414;}
            node N_419:Token {}
      node N_410:Alice {}
         node N_417:Phone { port0: Edge E_418;}
----- Done Printing Root ------

Choose a branch: 
0- enter_room
1- enter_room
\end{lstlisting}
Nell'output di cui sopra, il sistema chiede se si vuole visitare il grafo. Nel caso di risposta affermativa, stampa il primo nodo ($S_0$) del grafo degli stati da cui � partita l'esecuzione e ci chiede quale regola vogliamo applicare. In questo caso, Alice pu� entrare nella prima o nella seconda stanza: se scriviamo $0$ allora Alice, entrer� nella prima stanza e l'output sar�:
\begin{lstlisting}
----- Printing Bigraph enter_room -----
edge E_4CB{ N_4CA:Phone; }
edge E_4E3{ N_4E2:Computer; }
edge E_4D4{ N_4D3:Computer; }
edge E_4E5{ N_4E4:Phone; }
root 0 {}
   node N_4C7:Building {}
      node N_4CE:Room {}
         node N_4D3:Computer { port0: Edge E_4D4;}
            node N_4D9:Token {}
      node N_4DC:Room {}
         node N_4DD:Alice {}
            node N_4E4:Phone { port0: Edge E_4E5;}
         node N_4E2:Computer { port0: Edge E_4E3;}
            node N_4E6:Token {}
   node N_4C6:Bob {}
      node N_4CA:Phone { port0: Edge E_4CB;}
----- Done Printing enter_room ------

Choose a branch: 
0- comp_connect
1- leave_room
\end{lstlisting}
Alice � entrata nella stanza, e ora pu� semplicemente lasciarla ($1$) o connettersi al computer($0$).

In questo modo, si consente all'utente di navigare all'interno del grafo degli stati e di poter vedere gli effetti di ogni regola di reazione. Quindi, il model checker $MC_{big}$ � ora pi� informativo: non dice solamente se le propriet� sono rispettate o meno, ma fornisce anche una feature per seguire dei percorsi sul grafo.\\



Quest'ultimo esempio mostra particolarmente bene con i bigrafi ed i BRS siano flessibili per rappresentare qualsiasi dominio e la loro utilit� in casi reali come questo. $MC_{big}$ e la sua logica, implementati per questa tesi, sono quindi di aiuto sia per bigrafi che modellano un formalismo (come quello per gli NFA) sia per i bigrafi \emph{domain specific} (come quest'ultimo esempio), rendendo l'isomorfismo tra bigrafi, il model checker e la sua relativa logica degli strumenti generali adatti ad ogni dominio.










% others chapters(through include) and/or parts
% \include{...}
% \part{...}
% ...

\backmatter
\conclusions\label{ch:concl}
asd
asd
asd

% here ...

% if you don't have an appendix just delete (comment) the following 3 lines
\appendix
\chapter{Notazione dei vari capitoli}\label{ch:appA}

Sono riportate le principali notazioni delle varie sezioni.

\section{Notazione della sezione \ref{sec:formalBigraphs}}
Il pi� delle volte si tratter� un intero come l'insieme di tutti i suoi precedenti, cos� che se si scriver� $m$ in realt� si intende $\{0, \dots, m-1 \}$. \\
Scriviamo $S \# T$ per dire che i due insiemi S e T sono disgiunti. Scriviamo $S \uplus T$ per descrivere l'unione dei due insiemi S e T dopo che sappiamo o assumiamo che
siano disgiunti.\\
Se $f$ ha dominio S e $S' \subset S$, allora denotiamo con $f \upharpoonleft S'$ la restrizione di f su S'. Per due funzioni f e g con dominio disgiunto S e T, scriviamo 
$f \uplus g$ per la funzione con dominio $S \uplus T$ tale che $(f \uplus g)\upharpoonleft S = f$ e $(f \uplus g)\upharpoonleft T = g$.  



\section{Notazione della sezione \ref{sec:algebra} }
I \emph{posti} di $G: \langle m, X \rangle \to \langle n, Y \rangle$ sono i suoi siti $m$, i suoi nodi e le sue radici $n$.\\
I \emph{punti} di $G$ sono le sue porte e i suoi inner names X. \\
I \emph{link} di $G$ sono i suoi archi (edge) e i suoi outer names Y. \\
Gli edges sono \emph{link chiusi} mentre gli outer names sono \emph{link aperti}. Un punto � detto \emph{aperto} se il suo punto � aperto, altrimenti � detto chiuso. G �
detto \emph{aperto} se tutti i suoi link sono aperti (ovvero se non ha edges).\\
Un place senza figli o un link senza punti viene detto \emph{idle}. Due places con lo stesso genitore o due link con lo stesso punto vengono detti \emph{fratelli}.\\
Se un' interfaccia $I = \langle m, X \rangle$ ha $X = \emptyset$, allora scriveremo I come $m$; se $m = 0$ o $m = 1$, scriveremo $I$ come $X$ e $\langle X \rangle$
rispettivamente.\\
Si definisce l'unit� $\varepsilon$ come $\varepsilon = \langle 0, \emptyset \rangle$.\\
L'unico bigrafo con supporto vuoto in $\varepsilon \to I$ viene scritto come $I$.
Un bigrafo $g: \varepsilon \to I$, con dominio $\varepsilon$, viene detto $ground$. Useremo lettere minuscole per i bigrafi ground e li scriveremo come $g:I$.


\section{Notazione della sezione \ref{sec:operazioniDerivate}}
Spesso si omette '$\cdots \otimes id_I$' in una composizione del tipo
$(F \otimes id_I) \circ G$, dove $F$ non possiede un'interfaccia sufficiente per comporre con $G$. Si scriver� dunque,
qualora non presenti ambiguit�, $F \circ G$.
Dato un linking $\lambda : Y \to Z$, potremmo volerlo applicare ad un bigrafo $G$ con interfaccia esterna $\left<m,X\right>$
avente meno nomi, i.e. $Y = X \uplus X^\prime$. Scriveremo allora $\lambda \circ G$ per indicare
$(id_m \otimes \lambda) \circ (G \otimes X^\prime)$, quando $m$ e $X^\prime$ possono essere capiti dal contesto.













\backmatter %Fine numerazione pagine

\bibliography{biblio}
\bibliographystyle{plain_ita}
% \printindex%use makeindex to generate the index
\end{document}
